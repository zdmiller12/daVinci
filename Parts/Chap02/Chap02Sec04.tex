\section{Science and Systems Science}\index{Science and Systems Science}

The significant accumulation of scientific knowledge, which began in the eighteenth century and rapidly expanded in the twentieth, made it necessary to classify what was discovered into scientific disciplines. Science began its separation from philosophy almost two centuries ago. It then proliferated into more than 100 distinct disciplines. A relatively recent unifying development is the idea that systems have general characteristics, independent of the area of science to which they belong. In this section, the evolution of a science of systems is presented through an examination of cybernetics, general systems theory, and systemology.

\subsection{General Systems Theory}\index{General Systems Theory}

An even broader unifying concept than cybernetic evolved during the late 1940’s. It was the idea that basic principle common to all systems could be found that go beyond the concept of control and self-regulation. A unifying principle for science and a common ground for interdisciplinary relationships needed in the study of complex systems were being sought. Ludwig von Bertalanffy used the phrase general systems theory around 1950 to describe this endeavor. A related contribution was made by Kenneth Boulding.

General systems theory is concerned with developing a systematic framework for describing general relationships in the natural and the human-made world. The need for a general theory of systems arises out of the problem of communication among various disciplines. Although the scientific method brings similarity between the methods of approach, the results are often difficult to communicate across disciplinary boundaries. Concepts and hypotheses formulated in one area seldom carry over to another, where they could lead to significant forward progress.

One approach to an orderly framework is the structuring of a hierarchy of levels of complexity for individual systems studied in the various fields of inquiry. A hierarchy of levels can lead to a systematic approach to systems that has broad application. Boulding suggested such a hierarchy. It begins with the simplest level and proceeds to increasingly complex levels that usually incorporate the capabilities of all the previous levels, summarized approximately as follows:

\begin{enumerate}
\item The level of static structure or frameworks, ranging from the pattern of the atom to the anatomy of an animal to a map of the earth to the geography of the universe.
\item The level of the simple dynamic system, or clockworks, adding predetermined, necessary motions, such as the pulley, the steam engine, and the solar system.
\item The level of the thermostat or cybernetic system, adding the transmission and interpretation of information.
\item The level of the cell, the open system where life begins to be evident, adding self-maintenance of structure in the midst of a through put of material.
\item The level of the plant, adding a genetic-societal structure with differentiated and mutually dependent parts, “blueprinted” growth, and primitive information receptors.
\item The level of the animal, adding mobility, teleological behavior, and self-awareness using specialized information receptors, a nervous system, and a brain with a knowledge structure.
\item The level of the human, adding self-consciousness, the ability to produce, absorb, and interpret symbols; and understanding of time, relationship, and history.
\item The level of social organization, adding roles, communication channels, the content and meaning of messages, value systems, transcription of image into historical record, art, music, poetry, and complex human emotion.
\item The level of the transcendental system, adding the ultimates and absolutes and unknowables.
\end{enumerate}

The first level in Boulding’s hierarchy is the most pervasive. Static systems are everywhere, and this category provides a basis for analysis and synthesis of systems at higher levels. Dynamic systems with predetermined outcomes are predominant in the natural sciences. At higher levels, cybernetic models are available, mostly in closed-loop form. Open systems are currently receiving scientific attention, but modeling difficulties arise regarding their self-regulating properties. Beyond this level, there is little systematic knowledge available. However, general systems theory provides science with a useful framework within which each specialized discipline may contribute. It allows scientists to compare concepts and similar findings, with its greatest benefit being that of communication across disciplines.

\subsection{Systemology and Synthesis}\index{Systemology and Synthesis}

The science of systems or their information is called systemology. Problems and problem complexes faced by humankind are not organized along disciplinary lines. A new organization of scientific and professional effort based on the common attributes and characteristics of problems will likely accelerate beneficial progress. As systems science is promulgated by the formation and acceptance of interdisciplines, humankind will benefit from systemology and systems thinking.

Disciplines in science and the humanities developed largely by what society permitted scientists and humanists to investigate. Areas that provided the least challenge to cultural, social, and moral beliefs were given priority. The survival of science was also of concern in the progress of certain disciplines. However, recent developments have added to the acceptance of a scientific approach in most areas. Much credit for this can be given to the recent respectability of interdisciplinary inquiry. One of the most important contributions of systemology is that it offers a single vocabulary and a unified set of concepts applicable to many types of systems.

During the 1940s, scientists of established reputation in their respective fields accepted the challenge of attempting to understand a number of common processes in military operations. Their team effort was called operations research, and the focus of their attention was the optimization of operational military systems. After the war, this interdisciplinary area began to take on the attributes of a discipline and a profession. Today a body of systematic knowledge exists for both military and commercial operations. But operations research is not the only science of systems available today. Cybernetics, general systems research, organizational and policy sciences, management science, and the information sciences are others.

Formation of interdisciplines began in the middle of the last century and has brough about an evolutionary synthesis of knowledge. This has occurred not only within science but also between science and technology and between science and the humanities. The forward progress of systemology in the study of large-scale complex systems requires a synthesis of science and the humanities as well as a synthesis of science and technology. Synthesis, sometimes referred to as an interdisciplinary discipline, is the central activity of people often considered to be synthesists.

The science community must build public understanding of and appreciation for science and evidence-based thinking. We must show that evidence-based thinking leads to more reliable policies to create jobs, maintain a healthy environment and improve teaching.

With more than 120,000 members across the world, the American Association for the Advancement of Science is uniquely positioned to bring together voices and ideas from a diverse range of disciplines and backgrounds to lead the charge. But we need your help.

With your support, we will expand our efforts to speak up and draw people — the public and policymakers — to the idea that science is relevant to their lives and can inform their decisions. We will provide training, tools, and resources for scientists to communicate their research and find opportunities to connect with loc

\subsection{Emergence}\index{Emergence}

\subsection{Ethical Principles From the Basics of Science}\index{Ethical Principles From the Basics of Science}

The principles that form the basis of every rational discussion, that is, of every discussion in the search for truth, are in the main ethical principles. I should like to state three such principles:

\begin{enumerate}
\item The principle of fallibility: perhaps I am wrong and perhaps you are right. But we could easily both be wrong.
\item The principle of rational discussion: we want to try, as impersonally as possible, to weight up our reasons for and against a theory; a theory that is definite and criticizable.
\item The principle of approximation to the truth: we can nearly always come closer to the truth in a discussion which avoids personal attacks. It can help us to achieve a better understanding; even in those cases where we do not reach an agreement.
\end{enumerate}

It is worth noting that these three principles are both epistemological and ethical principles. For they imply, among other things, toleration: if I hope to learn from you, and I want to learn in the interest of truth, then I have not only to tolerate you but also to recognize you as a potential equal; the potential unity and willingness to discuss matters rationally. Of importance also is the principle that we can learn much from a discussion, even when it does not lead to agreement: a discussion can help us by shedding light upon some of our errors.

You raise an interesting question, one which has caused argument and confusion over many years. Considering the advanced state of systems engineering, systems thinking, etc., this state of affairs might be considered curious, at the very least.

The problem with defining “systems” is that various disciplines perceive “system” from their own perspectives, and may not always define “system” in an entirely general way. For instance, engineers today often describe complicated artefacts as systems, whereas 20-30 years ago they described them as ‘equipments.’  The artefacts haven’t changed, but somehow the soubriquet ‘system’ has taken over, whether appropriate or not.

Aerospace engineers might describe an aircraft as a system, or even a system of systems, while others might describe the aircraft as a sophisticated artefact, a tool, for the use of a pilot and crew. Together, aircraft and crew comprise a sociotechnical system. Why?  Because only when together, man and machine, is the flying machine complete. Neither man nor machine can fly without the other.

So, an important element of any definition of ‘system’ is the notion of completeness. A system is a complete something that, usually, performs some function(s). A gambling system is a means and method for winning at gambling, but it is only a system if all the moves, all the tactics, are in place. If only one is absent – no system.

Another aspect of ‘system’ in general is organization. By definition, a system is organized. It is not enough for a system to comprise many parts: these parts have to be interconnected, interacting and organized. A system’s “degree of organization” can be measured as entropy. The lower the entropy of a system, the more of its internal energy can be converted for useful external work. (Second Law of Thermodynamics)

Some systems are purposeful: humans as systems can be purposeful. Are all systems purposeful? Apparently not. The solar system is organized, low configuration entropy, etc. – we cannot avow that it is complete – yet it appears to have no explicit purpose, outside of the transcendental, that is. So, purpose need not form part of any definition … nor does the solar system perform any notable function: it just `is.’ So, is function essential?

Within systems there are observed to be levels of organization. For living things, the smallest element that can be described as living in the cell. All living things are made from cells. However, the cells are organized, grouped into tissues; tissues into organs; organs into organ systems; organ systems into organisms (species); species into populations; populations into communities, communities into ecosystem, ecosystems into biomes; biomes into the biosphere. So, an empiricist might say, with some justification, that the definition of “system” places it necessarily in such a hierarchy of organization.

Reductionists, OTOH, seek to explain high-level phenomena in more fundamental, low level explanations. Some argue that the relation between high and low – supervenience - may not allow for reduction. High level properties may be irreducible and may represent new, emergent properties that are more than the sum of their parts. The simple, limited behavior of individual ants results in the sophisticated organization of a colony.

So, another aspect of `system’ is that some systems may exhibit emergent properties, properties/behaviors of the whole that are not evident in the properties/behaviors of the parts. Some pundits seek to divide emergence into `weak’ and `strong,’ where weak is predictable, calculable, and not, therefore, entirely justifiable as emergence. Should emergence appear in the definition of `system?’  Possibly, if only because many natural systems do exhibit emergence (e.g., the Hymenoptera), and some (complex?) manmade systems also exhibit emergence.

Lastly, a system has be open, i.e., it exchanges energy, information and substance with other systems and with its environment. Were a system truly closed, we would be unaware of it.