\section{Scientific Theory and Laws}\index{Scientific Theory and Laws}

\subsection{What is Scientific Theory?}\index{What is Scientific Theory?}

Imagine for a moment that you are omniscient. Endowed with such knowledge, you would completely understand how the world ``works.''  You would completely understand how light works, how molecules and atoms work, how genetics work, how tectonic plates work, and how the universe came into existence. There would be nothing about the social or natural worlds that you would not understand in its entirety.

Where you endowed with such omniscience, you would have no use whatsoever for ``science.''  You would have no need to study the world in a patient and systematic way; because you would already possess all the knowledge about the world that ``science'' could ever hope to yield. Science would not only bore you to tears; it would appear to be an imperfect and dreadfully tedious means to arrive at the knowledge you already possess.

Unfortunately, however, no human being possesses omniscience. We are born into the world without knowledge about how light works, how tectonic plates work, how atom work, and how the universe came into existence. We also lack perfect knowledge about how capitalism and socialism work, how democracy and monarchy work, and how price controls work.

Our uncertainty about how the social and natural worlds work restricts our ability to act. Our uncertainty about how tectonic plates work restricts our ability to predict and control earthquakes. Our uncertainty about how light works restricts our ability to harness it for our own purposes. And our uncertainty about how monarchy and democracy work restricts our ability to construct political and economic systems that are best suited to our nature. This list could be extended ad infinitum.

We are not without means to overcome our uncertainty about how the world works, however. We are not, like the brute animals, doomed to struggle for our existence in a world that we will never understand or be able to harness for our own purposes. We have reason and memory at our disposal, which, with the aid of our senses, allow us to examine the world and learn how its elements ``work.'' These fantastic mental abilities afford us the means to investigate the world in the hope of overcoming at least a small part of our natural ignorance and uncertainty.

Our fantastic mental abilities do not, however, automatically yield to us infallible knowledge about how the world works. We can misinterpret what is going on, and we can reason unsoundly. Our senses can fail us, and our thinking can become clouded, biased, or myopic. In addition, the world we seek to understand is so fantastically large and complicated, and our time so very scarce, that each of us is severely limited in the amount of knowledge we can individually acquire about how the world works.

Hence, only by working with an learning from other men can we as individual hope to learn more than a tiny fraction about how the world works. By working with and learning from other men, we can take advantage of an intellectual division of labor that allows individuals to investigate very specific aspects of the world and then share the fruits of their investigations with the rest of humanity. This specialization and exchange of ideas allows men to economize their scarce time, learn more about the world than they otherwise could as isolated individual, and serves as a check on the fallible reasoning of each individual.

The concept of ``science'' in the Western world has been to connect a community of individuals who are committed to study the world in a specialized, systematic, and intersubjective verifiable way. Ideally, this scientific community accumulates knowledge about how the world works as individuals learn from the specialized investigations of their colleagues and build on them, and as the scientific community critiques and refines their theories through time.

The process by which the scientific community investigates the world is not a magic or automatic path to enlightenment or omniscience, however. The theories that are fashionable in the scientific community at any specific moment may or may not accurately describe the actual working of the world. Communities of individual scholars, like the individual scholars themselves, can fall victim to intellectual error. They can misinterpret what is going on, and they can reason unsoundly. Their senses can fail them, and their thinking can become clouded, biased, or myopic.

The critical and inexorable problem that the community of scholars faces, therefore, is knowing whether the theories it currently embraces accurately and completely describe the workings of the world. This uncertainty about the accuracy of their scientific theories stems, once again, from the fact that no member of the scientific community is omniscient. No member of the scientific community is in a position to say with certainty that any theory does or does not accurately describe the workings of the world.

If even one member of the scientific community were omniscient, it would be possible to appeal to that member as an objective assessor of scientific theories. In that case, the omniscient assessor would not trouble himself with describing the world using the clumsy word ``theory,'' however. He would say ``the world works thusly,'' or ``the world does not work thusly.'' If such a person existed, moreover, the practice of ``science'' would cease altogether, because certain knowledge about the world could be obtained from the omniscience person without the need to tediously and imperfectly study the world ``scientifically.''

Because the scientific community does not count omniscient members among its number, its members have developed a ``scientific method'' to try to deal with their uncertainty about their theories. The ``scientific method,'' which consists of developing hypotheses and ``testing'' those hypotheses against empirical experience, does not provide the scientific community with certain knowledge, however. It merely serves a rather low hurdle that assists in weeding out what most scientists would consider implausible, unverifiable, and silly theories.

A theory’s ability to clear this low hurdle by no means can be interpreted as ``verifying'' a theory, or ``proving'' its truth, however, because alternative theories could always be imagined that would also be consistent with the empirical ``facts.''  The scientific method does not provide the scientific community with a means to determine which theory, if any, out of the limitless set of alternative theories that could be dreamed up to explain the same empirical phenomena is ``correct.''  Nor does the scientific method provide the scientific community with a means to know for certain that its members are not misinterpreting the empirical evidence. Only an omniscient being could know these things for certain.

Because empirical evidence doesn’t not ``speak for itself,'' and because scientists are not omniscient (and thus cannot know if they are ``correctly'' interpreting empirical evidence), scientists can never know for certain if their theories correctly describe physical reality. This means that any theory that relies on the interpretation of empirical evidence can never be more than a subjective statement of belief about how a part of the world works, based on some empirical evidence.

This definition is unavoidable, because no scientist is in an omniscient position to know for certain whether he has interpreted empirical evidence correctly, or whether his theory is the ``correct'' one out of the infinite set of alternative theories that could be imagined to explain a given phenomenon.

This is not to say that scientific theories rely on the interpretation of empirical evidence are useless or meaningless, simply because they are subjective statements of belief. Nor does it imply that all empirically derived scientific theories are equally plausible, or that they all must be deemed “equal” in some other way, simply because they are all subjective statements of belief about how the world works. On the contrary, a theory that relies on empirical evidence is nothing more than an ``expert opinion'' about how a part of the world works, but it can nevertheless be useful - sometimes amazingly useful, in fact – even when it is known to be ``incorrect'' in some respects (e.g., Newtonian physics). Moreover, individual are free to evaluate the plausibility of scientific theories on their own, which means that they are free to accord some empirical theories more plausibility than others.

The fact that scientific theories are subjective statements of belief does mean, however, that the scientist who claims that his empirically derived theory is a ``fact'' or ``undeniably certain'' does not understand the limitations of his method. He is deluding himself – and anyone else who believes his claims – if he thinks he is able to ``prove'' his empirically derived theory to be ``irrefutably true.''  Only an omniscient being could possibly know for certain that a specific theory out of the infinite set of alternative theories that could be imagined to explain a given phenomenon is ``correct.''  But, again, an omniscient being would not bother with the clumsy and inefficient methods of science. He would merely say, ``the world works thusly,'' or ``the world does not work thusly.''  He certainly would not bother ``testing'' his ideas against empirical experience, because he would already know the outcome. Hence, the fact that the scientist bother to ``test'' his theories and hypotheses reveals his lack of omniscience, and it also reveals, a fortiori, his inability to know for certain whether he is interpreting empirical evidence ``correctly.''

In order to move beyond making subjective statements of belief about how parts of the world work the scientist would either need to become omniscient himself or consult someone who is omniscient, or else he would need to move beyond gathering and interpreting empirical evidence. Because the former options are, presumably, not open to him, the scientist’s only viable option is to discover “facts” about the world, or parts of the world, that cannot possibly be thought to be false, and which are not open to misinterpretation. In other words, the scientist would have to transform himself from an empiricist into a ``rationalist'' who was concerned to discover fundamental truths about the world (i.e., a priori truths about the world) and elucidate them by means of a deductive and rationalistic method. Only then would the scientist be in a position to say that he has found “facts” about parts of the world that are ``indisputably true.''

By dogmatically endorsing the ``scientific method'' as the only means to acquire knowledge about the world, the empirically minded scientist tacitly admits that it is possible to discover fundamental truths about the world without going out and ``testing'' them. For, the proposition ``all hypotheses and theories must be `tested’ against empirical experience'' purports to be objectively and universally true, yet the proposition itself has not and can never be ``tested.''  Therefore, the proposition is self-contradictory and thus false, a fact that establishes that it is indeed possible to discover irrefutable and demonstrable truths about the world without going out and testing them.

Thus, absolute certainty in science cannot be acquired by means of the “scientific method” and the collection and interpretation of empirical evidence. For beings that lack omniscience, collection and interpretation of empirical evidence can only yield imperfect and subjective beliefs about how the world “works.”  Instead, absolute certainly in science can only be acquired by discovering propositions about the world that can be known to be true a priori – propositions that cannot be thought to be false.

This observation, in a nutshell, forms the foundation and is the great strength of the Austrian School of economics, which stands virtually alone in the contemporary would as a bastion for thinkers who are unsatisfied with imperfect and subjective approaches to science.

System thinking is not enough. We must also engage in systems feeling and making. We must start during conceptual design and continue through the multiple years of integration and maintenance of operational systems.

SysML helps us make the evidence that system thinking has occurred. SysML must be augmented by ways of communicating intended system feeling, particularly for systems that include humans as active components. Example attributes might be engages, influences, inspires, learns, etc. as well as trusted, trustworthy, alert, etc.

While designing a system model we must ensure that these necessary and sufficient emotions and judgements are in play and evolving in the right direction(s).

An extension to SysML is not likely, rather associated stories and a set of key metrics may be appropriate. Azad Madni and others such as IDEO and other design laboratories have shown us examples of such stories. Tom Love champions performing conceptual design sans words, a flow of cartoons only.

How shall we evolve an ontology for the Feeling aspect of systems as an adjunct to SysML, particularly for highly autonomous systems that emulate human emotions? Jack.

\subsection{Acknowledging Scientific Risk}\index{Acknowledging Scientific Risk}

The knowledge that is built buy science is always open to question and revision. No scientific idea is ever once-and-for-all ``proved.''  Why not?  Well, science is constantly seeking new evidence, which could reveal problems with our current understandings. Ideas that we fully accept today may be rejected or modified considering new evidence discovered tomorrow. For example, up until 1938, paleontologists accepted the idea that coelacanths (an ancient fish) went extinct at the time that they last appear in the fossil record – about 80 million years ago. But that year, a live coelacanth was discovered off the coast of South Africa, causing scientists to revise their ideas and begin to investigate how this animal survives in the deep sea.

Even though they are subject to change, scientific ideas are reliable. The ideas that have gained scientific acceptance have done so because they are supported by many lines of evidence. These scientific explanations continually generate expectations that hold true, allowing us to figure out how entities in the natural world are likely to behave (e.g., how likely it is that a child will inherit a particular genetic disease) and how we can harness that understanding to solve problems (e.g., how electricity, wire, glass, and various compounds can be fashioned into a working light bulb). For example, scientific understandings of motion and gases allow us to build airplanes that reliably get us from one airport to the next. Though the knowledge used to design airplanes is technically provisional, time and time again, that knowledge has allowed us to produce airplanes that fly. We have good reason to trust scientific ideas: they work!

Systems science - systemology (Greco. Systema, logos) or systems theory is an interdisciplinary field that studies the nature of systems - from simple to complex - in nature, society, and science itself. The field aims to develop interdisciplinary foundations that are applicable in a variety of areas, such as engineering, biology, medicine, and social sciences.

Systems science covers formal sciences such as complex systems, cybernetics, dynamical systems theory, and systems theory, and applications in the field of the natural and social sciences and engineering, such a control theory, operations research, social systems theory, systems biology, system dynamics, human factors, systems ecology, systems engineering and systems psychology. Themes commonly stressed in system science are (a) holistic view, (b) interaction between a system and its embedding environment, and (c)complex (often subtle) trajectories of dynamic behavior that sometimes are stable (and thus reinforcing), while at various `boundary conditions’ can become wildly unstable (and thus destructive). Concerns about Earth-scale biosphere/geosphere dynamics is an example of the nature of problems in which systems science seeks to contribute meaningful insights.

Since the emergence of general systems research in the 1950s, systems thinking and systems science have developed into many theoretical frameworks.
Systems notes of Henk Bikker, TU Delft, 1991

A theory is almost never proven, though a few theories do become scientific laws. One example would be the laws of conservation of energy, which is the first law of thermodynamics. Dr. Linda Boland, a neurobiologist and chairperson of the biology department at the University of Richmond, Virginia told Live Science that this is her favorite scientific law. ``This is one that guides much of my research on cellular electrical activity and it states that energy cannot be created nor destroyed, only changed in form. This law continually reminds me of the many forms of energy,'' she said.

Laws are generally considered to be without exception, though some laws have been modified over time after further testing found discrepancies. This does not mean theories are not meaningful. For a hypothesis to become a theory, rigorous testing must occur, typically across multiple disciplines by separate groups of scientists. Saying something is ``just a theory'' is a layperson’s term that has no relationship to science. To most people a theory is a hunch. In science a theory is the framework for observations and facts, Jaime Tanner, a professor of biology at Marlboro College, told Life Science.

Debate on whether `scientific consensus’ is to be given special weight seems to turn on the two-sides of the injunction from Richard Feynman:

“When someone says science teaches such and such, he is using the word incorrectly. Science doesn’t teach it; experience teaches it. If they say to you science has shown such and such, you might ask, “How does science show it – how did the scientists find out – how, what, where?”  Not science has shown, but this experiment, this effect, has shown. And you have as much right as anyone else, upon hearing about the experiments (but we must listen to all the evidence), to judge whether a reusable conclusion has been arrived at.
    • (Feynman, “The Pleasure of Finding things out”, page 187, emphasis added)
    
Is there any dispute that both aspects are required for the integrity of the scientific process?

\begin{enumerate}
\item Everyone has the right to judge for themselves whether `a reusable conclusion’ has been arrived at, \textit{but}
\item This must be done after listening to all the evidence, including what, how, and where scientists obtained the evidence
\end{enumerate}

I have not studied this topic in detail, but from a common-sense point of view, it does seem that many people (groups) want to declare their beliefs to be true, with these beliefs often coming from what they wish to be true based on ulterior motives or to conform with a specific narrative.

Kant wins. Marx wins. Everything has become politicized. Death of reason and dialogue. Victory to mindless rhetoric. Say it enough and people will believe it is true.

A person’s right to believe as he/she wishes does not imply an obligation on anyone else to accept that belief, no matter how many people choose to accept it. A group claiming it has consensus on a viewpoint cannot claim truth any more than a jury declaring “not guilty” proves the defendant is innocent of the act. A claim of consensus is not a form of higher or moral authority over others. It is not an act of reason. It is an act of intellectual treason.

Remember, it was consensus that all swans were white. It was consensus that the heavens revolved around the Earth, while simultaneously the world was flat. Until, of course, that those pesky non-believers and intellectual saboteurs revealed the errors by having the guts to find black swans and circumnavigate the Earth.

My lay viewpoint is that we would do well to eliminate the concept of “consensus” from our consciousness when speaking of science. Consensus does not imply truth.

We might also wish to recall Warfield’s definition of Clan-Think when touting consensus.

This makes interesting demands on education.

RE: “the practice of science has achieved considerable success by setting goals of sharing information to enable independent replication, debate, and testing of predictions – that is, valuing objective evidence over personal opinion or ideology or theology.”

Is there some other mode of the practice of science  that does not require independent replication, debate, and testing of predictions?

I ask this because it seems you are saying that humans are also practicing science when a sufficient number simply agree on some assertion without performing independent replication, debate, and testing.

Are you saying that replication, debate, and testing of predictions does not have to be done physically but can be done conceptually AND does not have to be done independently but is even better if done collegially?
Seems to me if you eliminated the concept of consensus from the practice of science you would be left with “independent replication, debate, and testing of predictions.”

If this would not be ‘science’ as practiced for the past few hundred years, then viola! The scientific method triumphs by detecting and highlighting that which is false science.

Please tell us what you think of Cornell Prof. Derek Cabrera’s rule that (if I am representing his claim accurately) we should reserve the label ‘theory’ for an idea that has been independently tested and we should use the label ‘hypothesis’ for an idea that has not yet been – even if highly popular among those who have awarded themselves the label scientist.

Could it be that a model of an intended system as produced by a system engineering activity is a hypothesis waiting to be vetted by those who develop and deploy actual systems and measure their effects?  Then, if updated to a sufficient degree of fidelity to the measurable system does that model become a theory?