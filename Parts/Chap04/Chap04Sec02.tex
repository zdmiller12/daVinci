\section{Concepts of Value and Utility}\index{Concepts of Value and Utility}

In economics, the term value designates the worth that a person attaches to a good or service. Thus, the value of an item is not inherent in the item but is inherent in the regard a person or people have for it. Value should not be confused with the cost or the price of an item in economic analysis. There may be little relation between the value a person ascribes to an article and the cost of providing it or the price that is asked for it.

\subsection{The Power to Satisfy Wants}\index{The Power to Satisfy Wants}

An accepted economic definition of the term utility is the power to satisfy human wants. The utility that an item has for an individual is determined subjectively. Thus, the utility of an item, like its value, is not inherent in the article itself but is inherent in the regard that a person has for it. Value and utility are closely related in the economic sense. The utility than an object has for a person is the satisfaction he derives from its use. Value is an appraisal of utility in terms of a medium of exchange.

In ordinary circumstances a large variety of goods and services is available to an individual. The utility that available items may have in the mind of a prospective user may be expected to be such that his desire for them will range from abhorrence, through indifference, to intense desire. His evaluation of the utility of various items is not ordinarily constant but may be expected to change with time. Each person also possesses either goods or services that he may offer to exchange. These have the utility for the person himself that he regards them to have. These same goods and possible services may also be desired by others, who may ascribe to them very different utilities. The possibility for exchange exists when each of two persons possesses goods or services desired by the other.

Most things that have utility for an individual are physically manifested. This is readily apparent in such physical objects as a house, an automobile, or a pair of shoes. The situation may be true as well with less tangible things. One can enjoy a television program because light waves impinge on the retina and sound waves strike the ear. Even friendship is realized through the senses and, therefore, has its physical aspects.
    
\subsection{The Creation of Utility}\index{The Creation of Utility}

The creation of utility is achieved through a change in the physical environment. For example, the consumer utility of raw steak can be increased by altering its physical condition by an appropriate application of heat. In the area of producer utilities, the machining of a bar of steel to produce a shaft for a rolling mill is an example of creating activity is to determine how the physical environment may be altered to create the most utility for the least cost.

Carl Menger, an early Austrian Economist, published Principles of Economics that challenged the fundamental premises of the classical economists, from Adam Smith through David Ricardo to John Stuart Mill. Menger argued that the labor theory of value (apparently adopted by Alan) was flawed in presuming that the value of goods was determined by the relative quantities of labor that had been expended in their production. Others argued that value if added justified pricing changes.

Menger formulated a subjective theory of value, reasoning that value originates in the mind of an evaluator (this cannot be Alan; it must be Wolt). Labor, like raw materials and other resources, derives value from the value of the goods it can produce. From this starting point, Menger outlined a theory of the value of goods and factors of production, and a theory of the limits of exchange and the formation of prices.

Böhm-Bawerk, mentor of Ludwig von Mises, came across Menger’s book soon after its publication. Both immediately saw the significance of this new subjective approach for the development of economic theory.

\subsection{The Subjective Theory of Value}\index{The Subjective Theory of Value}

Carl Menger, an early Austrian Economist, published Principles of Economics that challenged the fundamental premises of the classical economists, from Adam Smith through David Ricardo to John Stuart Mill. Menger argued that the labor theory of value (apparently adopted by Alan) was flawed in presuming that the value of goods was determined by the relative quantities of labor that had been expended in their production.

Menger formulated a subjective theory of value, reasoning that value originates in the mind of an evaluator (this cannot be Alan; it must be Wolt). Labor, like raw materials and other resources, derives value from the value of the goods it can produce. From this starting point, Menger outlined a theory of the value of goods and factors of production, and a theory of the limits of exchange and the formation of prices.

Böhm-Bawerk, mentor of Ludwig von Mises, came across Menger’s book soon after its publication. Both immediately saw the significance of this new subjective approach for the development of economic theory.
