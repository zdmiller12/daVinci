\section{General Understanding of Systems}\index{General Understanding of Systems}\label{sec:generalUnderstandingSystems}

A system is a set of interconnected elements which form a whole and show properties which are properties of the whole rather than of the individual elements. This definition is valid for a cell, an organism, a society, or a galaxy. Joanna Macy says that a system is less a thing than a pattern – a pattern of organization. It consists of a dynamic flow of interactions which is non-summative, irreducible, and integrated at a new level of organization permitted by the interdependence of its part. The word “system” derives from the Greek “synhistanai” which means “to place together.”

From a cognitive perspective, systems thinking integrates analysis and synthesis. Natural science has been primarily reductionistic, studying the components of systems and using quantitative empirical verification. Human science, as a response to the use of a positivistic methods for studying human phenomena, has embraced more holistic approaches, studying social phenomena through qualitative means to create meaning. Systems thinking bridges these two approaches by using both analysis and synthesis to create knowledge and understanding and integrating an ethical perspective.

Analysis answers the ‘what’ and ‘how’ questions while synthesis answers the ‘why’ and ‘what for’ questions. By combining analysis and synthesis, systems thinking creates a rich inquiring platform for approaches such a social systems design, developed by Bela H. Banathy, and evolutionary systems design, as Alexander Laszlo and myself have developed to include a deeper understanding of a system in its larger context as well as a vision of the future for co-creating ethical innovations for sustainability. Just like the first image of Earth from outer space had a huge impact on our ability to see the unity of our planet, systems thinking is a way of seeing ourselves as part of larger interconnected systems.

Systems thinking is a gateway to seeing interconnections. Once we see a new reality, we cannot go back and ignore it. More importantly, that “seen” has an emotional connection, beautifully captured in the statement by Rusty Schweickart after his experience of seeing his home planet from space. – Oliver W. Holmes

What are the emotions evoked by perceiving for the first time the unity, interconnectedness, and relatedness of a system?  What are the feelings evoked by perceiving and experiencing disconnection and isolation?  Humberto Maturana says that “emotions are fundamental to what happens in all our doings” and yet, bringing up emotions in a scientific or business conversation is in many cases considered irrelevant, inappropriate or simply uncomfortable. Following Maturana’s views, I would say that the simplified answer to my two questions on the emotions evoked by unity and disconnection are love and fear, correspondingly. Love is the only emotion that expands intelligence, creativity and vision; it is the emotion that enables autonomy and responsibility. Maturana defines love as “relational behaviors through which another (a person, being, or thing) arises as a legitimate other in coexistence with oneself.”  Only in a context of safety, respect and freedom to be and create (that is, a context of love) can people be relaxed and find the conditions conductive to engage in higher intelligent behaviors that uses their brain neocortex. Learning, collaboration and creativity happen when we are able to function from a consciousness capable of including a world centric awareness of “all of us”, as Ken Wilber puts it. On the contrary, in a situation of stress, insecurity, or any other manifestation of ear, we are conditions to react more instinctually and to operate from our reptilian brain to fulfil more rudimentary needs linked to survival.

The first image of our whole Earth from space created a sense of awe and beauty. From space, we can see (and feel) its wholeness: there are no political lines dividing our national territories, there is only one whole system. However, from our terrestrial and regionally bounded experiences, we can feel that the neighbor tribe is sufficiently different and threatening to be considered an enemy.

Systems being involves embodying a new consciousness, an expanded sense of self, a recognition that we cannot survive alone, that a future that works for humanity needs also to work for other species and the planet. It involves empathy and love for the greater human family and for all our relationships – plants and animals, earth and sky, ancestors and descendants, and the many peoples and beings that inhabit our Earth. This is the wisdom of many indigenous cultures around the world, this is part of the heritage that we have forgotten, and we are in the process of recovering.

Systems being and systems living brings it all together: linking head, heart, and hands. The expression of systems being is an integration of our full human capacities. It involves rationality with reverence to the mystery of life, listening beyond words, sensing with our whole being, and expressing our authentic self in every moment of our life. The journey from systems thinking to systems being is a transformative learning process of expansion of consciousness – from awareness to embodiment.

Kathia Laszlo, Ph.D., directs Saybrook University’s program in Leadership of Sustainable Systems. This post is an excerpt from the plenary presentation “Beyond Systems Thinking: The role of beauty and love in the transformation of our world” by Dr. Kathia Laszlo at the 55th Meeting of the International Society for the Systems Sciences at the University of Hull, U.K., on July 21, 2011.

The several ways to think of and define a system include:
\begin{enumerate}
\item A system is composed of parts.
\item All the parts of a system must be related (directly or indirectly), else there are really two more distinct systems.
\item A system is encapsulated; it has a boundary.
\end{enumerate}

A system is an assemblage or combination of functionally related elements or parts forming a unitary whole, such as a river system or a transportation system. Not every set of items, facts, methods, or procedures is a system. A random group of items in a room would constitute a set with definite relationships between the items, but it would not qualify as a system because of the absence of functional relationships. This book deal primarily with systems that include physical elements and have useful purposes, including systems associated with all kinds of products, structures, and services, as well as those that consist of a coordinated body of methods or a complex scheme or plan of procedure.

\subsection{System Scope From a Boundary}\index{System Scope From a Boundary}\label{subsec:systemScopeFromBoundary}

We have said that a system has a boundary that encloses the parts of the system. Once we have set the system boundary, all of the above properties and characteristics are ``objective'', independent of human values or intentionality. But how do we reconcile this claimed objectivity with the apparent need for a human choice - presumably a subjective choice - of the system boundary?  There are two possible answers:

\begin{enumerate}
\item Some systems have a very clear physical boundary and are loosely coupled from the rest of the universe. For example, the solar system, the Earth, an airplane, and animal
\item Sometimes we are interested in a particular ``property of interest'' - the earth’s energy balance, the profitability of a business, the viability of a community, the operational effectiveness of a military system. In such cases it is usually convenient to define the boundary of the ``system of interest'' as the physical boundary of the ``system''
\end{enumerate}

Once we have established the ``property of interest'', the ``system of interest'', and corresponding system boundary can be determined by finding the set of parts and relationships that are necessary and sufficient to account for the property or properties of interest. (This I believed to be a key and novel insight – but Ashby said something similar in 1956!)

Often the system of interest and corresponding boundary will be different depending on the property of interest: for example, whether the question is ``what are we contracted to deliver'', “what is needed to achieve our purpose or goal”?  This is the difference between a ``product systems engineering'', a ``service systems engineering'', and a ``capability systems engineering'' viewpoint.

There are usually at least two boundaries we need to consider: the ``responsibility boundary'' that encloses the new or modified system (or part of a system) that we are responsible for and have control over; and the ``analysis boundary'' which includes everything else we need to consider to understand what the system of interest will actually do when deployed into the real environment.

So the key role of systems thinking as it is defined in this paper is to establish the purpose and value of the system of interest. Once the purpose and value are decided – or perhaps it is better to use the word “chosen”, because this is a human choice – the appropriate system boundary can be determined. If the wrong boundary is chosen for a system engineering effort, the wrong design choices will be made, purpose will not be achieved, and value will not be delivered. So, the skill and process for relating and aligning purpose, value and system boundary is the key to the success of a systems approach. This brings us to systems thinking.

18 century concept from the ``Encyclopedie de D. DIDEROT'':
\begin{enumerate}
\item SYSTEM (metaphysical) is nothing else than the provision of the various parts of an art or a science in a state where they support themselves all mutually, and where the last ones are explained by the first ones. Those which return reason of the others are called principles; and the system is more perfect if the principles are small in number: it is even wishing to reduce them only one. Because just as in a clock, there is a principal spring on which all the others depend, in all the systems there is also a first principle to which the various parts are subordinate to the other which make it up
\item SYSTEM (philosophy) generally means assembly or sequence of principles and conclusions; or everything and the whole of a theory of which the various parts are dependent between them, follow and depend the ones on the others
\item SYSTEM (astronomy) is the assumption of a certain arrangement of the various part which make the universe; according to this assumption the astronomers explain all the phenomena or appearances of the celestial bodies
\end{enumerate}
    
\subsection{System Descriptions and Elements}\index{System Descriptions and Elements}\label{subsec:systemDescriptionsElements}

Thus far, the search for nan acceptable system description has not ended. A valiant effort within the professional society almost ended with agreement.1 The effort was destined to be quite generic, but . . . 

Systems Function. When designing as system, the objective(s) or purpose(s) of the system must be explicitly defined and understood so that system components may be engineered to provide the desired function(s), such as a desired output for each given set of inputs. Once defined, the objective(s) or purpose(s) make possible the derivation of measures of effectiveness indicating how well the system performs. Achieving the intended purpose(s) of a human-made system and defining its measures of effectiveness are usually challenging tasks.

The purposeful action performed by a system is its function. A common system function is that of altering material, energy, or information. This alteration embraces input, process, and output. Some examples are the materials processing in a manufacturing system or a digestive system, the conversion of coal to electricity in a power plant system, and the information processing in a computer system or a customer service system.

Systems that alter material, energy, or information are composed of structural components, operating components, and flow components. Structural components are the static parts; operating components are the parts that perform the processing; and flow components are the material, energy, or information being altered. A motive force must be present to provide the alteration within the restrictions set by structural and operating components.

System components have attributes that determine the component’s contribution to the system’s function. Examples of component attributes include the color of an automobile (a characteristic), the strength of a steel beam (a quality), the number and arrangement of bridge piers (a configuration), the capacitance of an electrical circuit (a power), the maximum speed permitted by the governor of a turbine (a constraint), and whether or not a person is talking on the telephone (a state). An example of a system-level attribute is the length of runway required by an aircraft for takeoff and landing. The runway length requirement is determined by the attributes and relationships of the aircraft as a component and by the configuration attributes of the air transportation system.

A single relationship exists between two and only two components based on their attributes. The two components are directly connected in some way, though they are not necessarily physically adjacent. In a system with more than two components, at least one of the components in the relationship also has at least one relationship with some other component. Each component in a relationship provides something that the other component needs so that it can contribute to the system’s function. In order to form a relationship of maximum effectiveness, the attributes of each component must be engineered so that the collaborative functioning of the two components is optimized.

Relationships that are functionally necessary to both components may be characterized as first order. An example is symbiosis, the association of two unlike organisms for the benefit of each other. Second-order relationships, called synergistic, are those that are complementary and add to system performance. Redundancy in a system exists when duplicate components are present for the purpose of assuring continuation of the system function in case of component failure.

System Elements. Systems are composed of components, attributes, and relationships. These are described as follows:
\begin{enumerate}
\item Components are the parts of a system
\item Attributes are the properties (characteristics, configuration, qualities, powers, constraints, and state) of the components and of the system as a whole
\item Relationships between pairs of linked components are the result of engineering the attributes of both components so that the pair operates together effectively in contributing to the system’s purpose(s)
\end{enumerate}

The state is the situation (condition and location) at a point in time of the system, or of a system component, with regard to its attributes and relationships. The situation of a system may change over time in only certain ways, as in the on or off state of an electrical switching system. A connected series of changes in the state over time comprise a behavior. The set of all behaviors with their relative sequence and timing comprise the process. The process of a component may control the process of another component.

A system is a set of interrelated components functioning together toward some common objective(s) or purpose(s). The set of components meets the following requirements:
\begin{enumerate}
\item The properties and behavior of each component of the set have an effect on the properties and behavior of the set as a whole
\item The properties and behavior of each component of the set depend on the properties and behavior of at least one other component in the set
\item Each possible subset of components meets the two requirements listed above; the components cannot be divided into independent subsets
\end{enumerate}

The previous requirements ensure that the set of components constituting a system always has some property, or behavior pattern, that cannot be exhibited by any of its subsets acting alone. A system is more than the sum of its component parts. However, the components of a system may themselves be systems, and every system may be part of a larger system in a hierarchy.

When designing a system, the objective(s) or purpose(s) of the system must be explicitly defined and understood so that system components may be engineered to provide the desired function(s), such as a desired output for each given set of inputs. Once defined, the objective(s) or purpose(s) make possible the derivation of measures of effectiveness indicating how well the system performs. Achieving the intended purpose(s) of a human-made system and defining its measures of effectiveness are usually challenging tasks.

The purposeful action performed by a system is its function. A common system function is that of altering material, energy, or information. This alteration embraces input, process, and output. Some examples are the materials processing in a manufacturing system or a digestive system, the conversion of coal to electricity in a power plant system, and the information processing in a computer system or a customer service system.

Systems that alter material, energy, or information are composed of structural components, operating components, and flow components. Structural components are the static parts; operating components are the parts that perform the processing; and flow components are the material, energy, or information being altered. A motive force must be present to provide the alteration within the restriction set by structural and operating components.

System components have attributes that determine the component’s contribution to the system’s function. Example of component attributes include the color of an automobile (a characteristic), the strength of a steel beam (a quality), the number and arrangement of bridge piers (a configuration), the capacitance of an electrical circuit (a power), the maximum speed permitted by the governor of a turbine (a constraint), and whether or not a person is talking on the telephone (a state). An example of a system-level attribute is the length of runway required by an aircraft for takeoff and landing. The runway length requirement is determined by the attributes and relationships of the aircraft as a component and by the configuration attributes of the air transportation system.

A single relationship exists between two and only two components based on their attributes. The two components are directly connected in some way, though they are not necessarily physically adjacent. In a system with more than two components, at least one of the components in the relationship also has at least one relationship with some other component. Each component in a relationship provides something that the other component needs so that it can contribute to the system’s function. In order to form a relationship of maximum effectiveness, the attributes of each component must e engineered so that the collaborative functioning of the two components is optimized.

Relationships that are functionally necessary to both components may be characterized as first order. An example is symbiosis, the association of two unlike organisms for the benefit of each other. Second-order relationships, called synergistic, are those that are complementary and add to system performance. Redundancy in a system exists when duplicate components are present for the purpose of assuring continuation of the system function in case of component failure.

Systems and Subsystems. The definition of a system is not complete without consideration of its position in the hierarchy of systems. Every system is made up of components, and many components can be broken down into smaller components. If two hierarchical levels are involved in a given system, the lower is conveniently called a subsystem. For example, in an air transportation system, the aircraft, control tower, and terminals are subsystems. Equipment, people, and software are components. The designation of system, subsystem, and component are relative, because the system at one level in the hierarchy is the subsystem or component at another.

In any particular situation, it is important to define the system under consideration by specifying its limits, boundaries, or scope. Everything that remains outside the boundaries of the system is considered to be the environment. However, no system is completely isolated from its environment. Material, energy, and/or information must often pass through the boundaries as inputs to the system. In reverse, material, energy, and/or information that pass from the system to the environment are called outputs. That which enters the system in one form and leaves the system in another form is usually called throughput.

The total system, at whatever level in the hierarchy, consists of all components, attributes and relationships needed to accomplish one or more objectives. Each system has objective(s) providing purpose(s) for which all system components, attributes, and relationships have been organized. Constraints placed on the system limit its operation and define the boundary within which it is intended to operate. Similarly, the system places boundaries and constraints on its subsystems.

An example of a total system is a fire department. The subsystems of this “fire control system” are the building, the fire engines, the firefighters with personal equipment, the communication equipment, and maintenance facilities. Each of these subsystems has several contributing components. At each level in the hierarchy, the description must include all components, all attributes, and all relationships.

Systems thinking and the systems viewpoint looks at a system from the top down rather than from the bottom up. Attention is first directed to the system as a black box that interacts with its environment. Next, attention is focused on how the smaller black boxes (subsystems) combine to achieve the system objective(s). The lowest level of concern is then with individual components.

The process of bringing systems into being and of improving systems already in existence, in a holistic way, is receiving increasing attention. By bounding the total system for study purposes, the systems engineer or analyst will be more likely to obtain a satisfactory result. The focus on systems, subsystems, and components in a hierarchy forces consideration of the pertinent functional relationships. Components and attributes are important, but only to the end that the purpose of the whole system is achieved through the functional relationships linking them.

\subsection{Classification of Systems}\index{Classification of Systems}

Beyond the classification of systems that emanates from the partitioning of our world into N. HM and HJMod, are classifications of a subordinate nature. Systems may be classified for convenience and to provide insight into their wide range. In this section, classification will be accomplished by several dichotomies conceptually contrasting system similarities and dissimilarities. Descriptions are given of physical and conceptual systems, static and dynamic systems, and closed and open systems.

Physical and Conceptual Systems. Physical systems are those that manifest themselves in physical form. They are composed of real components and may be contrasted with conceptual systems, where symbols represent the attributes of components. Ideas, plans, concepts, and hypotheses are examples of conceptual systems.

A physical system consumes physical space, whereas conceptual systems are organizations of ideas. One type of conceptual system is the set of plans and specifications for a physical system before it is actually brought into being. A proposed physical system may be simulated in the abstract by a mathematical or another conceptual model. Conceptual systems often play an essential role in the operation of physical systems in the real world.

The totality of elements encompassed by all components, attributes, and relationships focused on a given result employ a process that determines the state changes (behaviors) of a system. A process may be mental (thinking, planning, and learning), mental-motor (writing, drawing, and testing), or mechanical (operating, functioning, and producing). Processes exist in both physical and conceptual systems.

Process occurs at many different levels within systems. The subordinate process essential for the operation of a total system is provided by the subsystem. The subsystem may, in turn, depend on more detailed subsystems. System complexity is the feature that defines the number of subsystems present and, consequently, the number of processes involved. A system may be bounded for the purpose of study at any process or subsystem level. Also, related systems that are normally analyzed individually may be studied as a group, and the group is often called a system-of-systems (SOS).

Static and Dynamic Systems.  Another system dichotomy is the distinction of static and dynamic types. A static system is one whose states do not change because it has structural components but no operating or flow components, as exemplified by a bridge. A dynamic system exhibits behaviors because it combines structural components with operating and/or flow components. An example is a school, combining a building, students, teachers, books, curricula, and knowledge.

A dynamic conception of the universe has become a necessity. Yet, a general definition of a system as an ongoing process is incomplete. Many systems would not be included under this broad definition because they lack operating and flow components. A highway system is static, yet it contains the system of elements of components, attributes, and functional relationships.

A system is static only in a limited frame of reference. A bridge system is constructed, maintained, and altered over a period of time. This is a dynamic process conducted by a construction subsystem operating on a flow of construction materials. A structural engineer must view the bridge’s members as operating components that expand and contract as they experience temperature changes.

A static system serves a useful purpose only as a component or subsystem of a dynamic system. For example, a static bridge is part of a dynamic system with an overpass operating component processing a traffic flow component and with an underpass component handling water or traffic flow.

Systems may be characterized as having random properties. In almost all systems in both the natural and human-made categories, the inputs, process, and output can only be described in statistical terms. Uncertainty often occurs in both the number of inputs and the distribution of these inputs over time. For example, it is difficult to predict exactly the number of passengers who will check in for a flight, or the exact time each will arrive at the airport. However, because these factors can be described in terms of probability distributions, system operation may be considered probabilistic in its behavior.

For centuries, humans viewed the universe of phenomena as immutable and unchanging. People habitually thought in terms of certainties and constants. The substitution of a process-oriented description for the static description of the universe, and the idea that almost anything can be improved, distinguishes modern science and engineering from earlier thinking.

Closed and Open Systems. A closed system is one that does not interact significantly with its environment. The environment provides only a context for the system. Closed systems usually exhibit the characteristic of equilibrium resulting from internal rigidity that maintains the system in spide of influences from the environment. An example is the chemical equilibrium eventually reached in a closed vessel when various reactants are mixed together, provided that the reaction does not increase pressure to the point that the vessel explodes. The reaction and pressure can be predicted from a set of initial conditions. Closed systems involve deterministic interactions, with a one-to-one correspondence between initial and final states. There are relatively few closed systems in the natural and the human-made world.

An open system allows information, energy, and matter to cross its boundaries. Open systems interact with their environment, examples being plants, ecological systems, and business organizations. They exhibit the characteristics of steady state, wherein a dynamic interaction of system elements adjusts to changes in the environment. Because of this steady state, open systems are self-regulatory and often self-adaptive.

It is not always easy to classify a system as either open or closed. Open systems are typical of those that have come into being by natural processes. Human-made systems have both open and closed characteristics. They may reproduce natural conditions not manageable in the natural world. They are closed when designed for invariant input and statistically predictable output, as in the case of a spacecraft in flight.
Both closed and open systems exhibit the property of entropy. Entropy is defined here as the degree of disorganization in a system and is analogous to the use of the term in thermodynamics. In the thermodynamic usage, entropy is the energy unavailable for work resulting from energy transformation from one form to another.

In systems, increased entropy means increased disorganization. A decrease in entropy occurs as order occurs. Life represents a transition from disorder to order. Atoms of carbon, hydrogen, oxygen, and other elements become arranged in a complex and orderly fashion to produce a living organism. A conscious decrease in entropy must occur to create a human-made system. All human0made systems, from the most primitive to the most complex, consume entropy because they involve the creation of more orderly states from less orderly states.

Scientific System Classifications. Science systems and the application of science systems thinking has been grouped into three categories based on the techniques used to tackle a system:
\begin{enumerate}
\item Hard systems – involving simulations, often using computers and the techniques of operation research/management science. Useful for problems that can justifiably be quantified. However, it cannot easily consider unquantifiable variables (opinions, culture, politics, etc.), and may treat people as being passive, rather than having complex motivations
\item Soft systems – For systems that cannot easily be quantified, especially those involving people holding multiple and conflicting frames of reference. Useful for understanding motivations, viewpoints, and interactions and addressing qualitative as well as quantitative dimensions of problem situations. Soft systems are a field that utilizes foundation methodological work developed by Peter Checkland, Brian Wilson and their colleagues at Lancaster University Morphological analysis is a complementary method for structuring and analyzing non-quantifiable problem complexes
\item Evolutionary systems – Bela H. Banathy developed a methodology that is applicable to the design of complex social systems. This technique integrates critical systems inquiry with soft systems methodology. Evolutionary systems, like dynamic systems are understood as open, complex systems, but with the capacity to evolve over time. Banathy uniquely integrated the interdisciplinary perspectives of systems research (including chaos, complexity, cybernetics), cultural anthropology, evolutionary theory, and others
\end{enumerate}

The systems thinking approach incorporates several tenets:
\begin{enumerate}
\item Interdependence of objects and their attributes – independent elements can never constitute a system.
\item Holism – emergent properties not possible to detect by analysis should be possible to define by a holistic approach.
\item Goal seeking – systemic interaction must result in some goal or final state.
\item Inputs and Outputs – in a closed system inputs are determined once and constant; in an open system additional inputs are admitted from the environment.
\item Transformation of inputs into outputs – this is the process by which the goals are obtained.
\item Entropy – the amount of disorder or randomness present in any system.
\item Regulation – a method of feedback is necessary for the system to operate predictably.
\item Hierarchy – complex wholes are made up of smaller subsystems.
\item Differentiation – specialized units perform specialized functions.
\item Equifinality – alternative ways of attaining the same objectives (convergence).
\item Multifinality – attaining alternative objectives from the same inputs (divergence).
\end{enumerate}

A treatise on systems thinking ought to address many issues including:

\begin{enumerate}
\item Encapsulation of a system in space and/or time
\item Active and passive systems (or structures)
\item Transformation by an activity system of inputs into outputs
\item Persistent and transient systems
\item Evolution, the effects of time passing, the life histories of systems and their parts.
\item Design and designers.
\end{enumerate}