\chapter{ENGINEERING AND SYSTEMS ENGINEERING}\label{chap:5}

Activities of analysis and design are not ends in themselves but are a means for satisfying human wants. Thus, modern engineering has two aspects. One aspect concerns itself with the materials and forces of nature; the other is concerned with the needs of people.

According to the definition of engineering adopted by the Accreditation Board for Engineering and Technology (ABET), ``Engineering is the profession in which a knowledge of the mathematical and natural sciences gained by study, experience, and practice is applied with judgment to develop ways to utilize economically the materials and forces of nature for the benefit of mankind.'' This definition is understood herein to encompass both systems and products, with the product often being a structure or service.

As unnecessary as it might seem, this chapter is intended to establish engineering as the foundation for systems engineering. Alternative equivalent routes to the study and practice of systems engineering are acknowledged. By, by accepting the principle that the alteration of physical factors is key to human want and purpose satisfaction, we are compelled to stay close to classical physics and its derivatives.

\section{Relationship of Science and Engineering}\index{Relationship of Science and Engineering}

Engineering is not science, but an application of science. It is an art composed of skill and ingenuity in adapting knowledge to the uses of humanity. The Accreditation Board for Engineering and Technology has adopted the definition:

Engineering is the profession in which a knowledge of the mathematical and natural sciences gained by study, experience, and practice is applied with judgment to develop ways to utilize, economically, the materials and forces of nature for the benefit of mankind.

This, like most other accepted definitions, emphasizes the applied nature of engineering.

The role of the scientist is to add to mankind’s accumulated body of systematic knowledge and to discover universal laws of behavior. The role of the engineer is to apply this knowledge to situations to produce products and services. To the engineer, knowledge is not an end in itself but is the raw material from which he or she fashions structures, systems, products, and services. Thus, engineering involves the determination of the combination of materials, forces, information, and human factors that will yield a desired result. Engineering activities are rarely carried out for the satisfaction that may be derived from them directly. With few exceptions, their use is confined to satisfying human wants.

Modern civilization depends to a large degree upon engineering. Most products and services used to facilitate work, communication, transportation, and national defense and to furnish sustenance, shelter, and health are directly or indirectly a result of engineering activity. Engineering has also been instrumental in providing leisure time for pursuing and enjoying culture. Through the development of instant communication and rapid transportation, engineering has provided the means for both cultural and economic improvement of humanity.

Science is the foundation upon which the engineer builds toward the advancement of mankind. With the continued development of science and the worldwide application of engineering, the general standard of living may be expected to improve and further increase the demand for those things that contribute to people’s love for the comfortable and beautiful. The fact that these human wants may be expected increasingly to engage the attention of engineers is, in part, the basis for the incorporation of humanistic and social considerations in engineering curricula. An understanding of these fields is essential as engineers seek solutions to the complex socioecological problems of today.

5.1.1 Historical Perspectives of Science and Engineering

Historically, science and engineering were two separate cultures that did not interact. Only in the past two or three centuries has a scientific foundation been explicitly built and used for engineering. A description of how the relationship between science and engineering develops as part of the maturation process is given for engineering disciplines such as mechanical and chemical engineering, by Shaw (1990) and Finch (1951). The summary of this process provided here is from Hybertson (2009).

The first two parts of this section address broader perspectives on science and engineering in general. The last three parts discuss perspectives on SE and SS.

On 7 Nov 2013, at 1846, Hillary Sillitto hsillitto@googlemail.com wrote:

Second, someone said ``SE is not E.''  An interesting philosophical discussion, but so politically fraught as not to be useful, indeed actively damaging to our cause: the claim makes many enemies and no friends. SE is ``a kind of'' E and goes beyond the boundaries of the technology-focused kind of “E” into the realms of the deliciously ambiguous other meaning of ``E'', ``to cunningly arrange...'' and ‘twas I who said it. Moreover, I believe this inappropriate notion that SE is E, to be the main issue restraining the advancement of SE. Nor is the clear distinction between SE and E a political bogyman – on the contrary, it may be a political trump card. And the hackneyed idea of calling upon the ambiguous employment of “E” is surely unworthy. We all know what engineering is - even the French.

Consider a very direct analogy between SE and E, on the one hand, with Architectural Design and Civil Engineering on the other. The Norman Fosters of this world are not engineers, but instead are creative, innovative designers of sophisticated buildings and complexes – often in competition. They employ firms of civil design and construction engineers to undertake building work, frequently supervised by the architects.

It would be naïve to imagine that such international architects are of the same discipline as civil engineers; they are clearly not. And architects are concerned with the conception and design of whole buildings, whole complexes, and in their future context and environment, into which their putative solution must fit in full harmony. Architects must surely know enough about civil engineering not to design something that cannot be built - but they are not civil engineers per se. They may not even make their own fly-thu simulations and models of future buildings/complexes. OTOH, civil engineers would never have designed the Sydney Opera House. So, should it be with SEs and Es. Sytems investigation/conception/CONOPS/architecting/etc., are different from engineering: the more complex the problem and its solution, the more different - SE must be free to be creative and innovative, so long as it stays within the bounds of that which can be engineered.

So, may I reiterate: SE is not E – they are different, in that they are different disciplines, with different tenets, principles, methods, practices and, yes, different sciences. Yet, they are connected – as Architects are connected with Civil Engineers – in that SAs and SEs conceive, design and architect the whole, while Es make the products and artefacts. Cheers, Derek H.

The initial phase is ad hoc practice, in which problems are solved by talented amateurs and virtuosos. The solutions of these craftsmen then gradually move into the folklore, become routine, and begin limited production. This gradually leads to commercial-scale production. The problems of large-scale production then stimulate the development of a supporting science that defines the principles that explain the behavior that practitioners have already observed and exploited. As the science matures, it finally gets out ahead of current practice in the sense that it formulates principles that can be used to improve current practice: more predictability, fewer errors, more precise safety margins that allow less over-design and therefore lower cost. At this point, people can be educated in both the theory and practice of the field, and the field is now a mature discipline.

As both science and engineering mature, they contribute to each other. Engineering builds thing, the properties of which science needs to analyze and understand to enable more efficient and economical engineering. It should be noted that not all supporting science is stimulated by or results from problems in an engineering discipline. Often a science already exists and is found to be helpful to an engineering discipline.

How are (or can or should) the systems sciences and systems engineering (be) related?  For the web conference for the INCOSE (International Council of Systems Engineers) Complex Systems Working Group on November 22, 2010, I decided to present a personal perspective on linkages. The ideas were essentially in two parts with:

\begin{enumerate}
\item The systems movement as a system of ideas including
\item The systems science community as some individuals, some organizations, and some publications; and
\item Ten frames to guide thinking and discussion about changes in society, economics and technology in the 21st century (based on Ing (2011)); and
\item John N. Warfield’s ``A Challenge for Systems Engineers: To Evolve towards Systems Science,'' published in INCOSE Insight (2007).
\end{enumerate}

The first point reflects my view of the breadth and diversity of the system sciences. The second point reviewed some challenges presented by John N. Warfield, who was both a pioneer in the systems engineering community and a luminary in the systems sciences community. As a guide for the web conference, I provided a context map.

\subsection{Science, Engineering, and Systems Engineering}\index{Science, Engineering, and Systems Engineering}

Excerpts from Systems Engineering, Alberto Sols, Pp21-23

Several recognized scholars and historians have written extensively on science and engineering, from its meaning and origin to its current status [Leonard, 1989; van Doren, 1991; Checkland, 1993; McClellan and Dorn, 1999; Gribbin, 2002; Bryson, 2004; Priestley, 2014]. It is not the purpose of this chapter to present a thorough review of science and engineering, but just to set a basic framework for the better understanding of the advent of systems engineering as a discipline in the second half of the 20th century.

        1.1. Science

Long gone are the days when our ancestors would look terrified at a thunderstorm, at an erupting volcano or at so many other natural phenomena. In their inability to come up with an explanation they resorted to gods and mythology to explain what in one way or another caught their attention or shocked them.

Science is one of the major activities of the human mind. It can be defined as the knowledge gained from the physical and material world through systematic study, observation and experimentation. The word ``science'' comes from the Latin word scientia, which means knowledge. Science as we know it originated about 2500 years ago in the ancient Greece. Thales was a pre-Socratic Greek philosopher who lived in Miletus (Asia Minor) circa 624 to 546 B.C.E. He is known as the first true scientist as, according to available records, he was the first person to try to explain natural phenomena without reference to gods and mythology. Thales tried to develop testable and verifiable explanations about the universe and the behavior of things. The Greeks succeeded first by discovering nature, willing as they were to consider the world as a natural system, governed by natural laws, thus leaving the gods out. The Greeks were interested primarily in knowledge and understanding, and only secondarily in practical usefulness. This focus on knowledge per se is what we would call today basic research.

Science involves structured and unbiased observations, from which understanding of the observed phenomena is gained. The conducted observations enable the formulation of general laws or rules; further observations will result in those laws either being re-confirmed, rejected, or in need of amendments. That is the process of the so-called scientific method, which has been described by a number of authors [Carey, 2010; Gauch, 2012]. The scientific method is the body of techniques for investigating phenomena and gaining new knowledge, as well as for validating or amending what we already know. Knowledge is gained through the recording and analysis of observable, measurable, empirical and reproducible evidence, subject to the laws of reasoning. The scientific method is thus the procedure by which knowledge is generated in empirical studies. The main advantage of the scientific method is that it is unprejudiced as theories are accepted based on the result of tests and experiments that anyone can reproduce. The scientific method is divided into six steps, as depicted in figure 1.1. The first step is conducted somehow informally and noc necessarily in a structured manner. Second, a hypothesis is constructed and it will allow the prediction of future observations. Third, an experiment is designed to test the validity of the hypothesis. Fourth, the experiment is run and data are collected. Fifth, based on the gathered evidence the hypothesis is approved or rejected in total or in part. If the hypothesis is not approved then it is necessary to go back to the second step, and to reformulate the hypothesis and/or redesign the experiment, as needed. In any case, the sixth and final step is to report the results. This last step is essential because it enables the needed repeatability of the experiments, which is the cornerstone of the scientific method.

Research enables the advancement of science. In the words of Hungarian-born and Nobel Prize winner Albert Szent-Gyorgyi, to do research is to see what everybody has seen and to thing what nobody has thought. Research is frequently divided into two categories: basic research and applied research; the former is performed without thought of practical ends, seeking to widen the understanding of the phenomena of a certain field, whereas the latter refers to scientific studies and activities that seek to solve practical problems or to achieve specific goals. Basic research is the true pacemaker of technological progress. If basic research aims at extending the boundaries of fundamental understanding, applied research is always directed toward some identified societal problem or need.

Systems Engineering from? Systems engineering (SE) is an interdisciplinary field of engineering, that focuses on the development and organization of complex systems. It is the “art and science of creating whole solutions to complex problems,” for example: signal processing systems, control systems and communication system, or other forms of high-level modelling and design in specific fields of engineering.

Systems Methodologies. There are several types of Systems Methodologies, that is, disciplines for analysis of systems. For example:

\begin{itemize}
\item Soft systems methodology (SSM): in the field of organizational studies is an approach to organizational process modelling, and it can be used both for general problem solving and in the management of change. It was developed in England by academics at the University of Lancaster Systems Department through a ten-year Action Research programme.
\item System development methodology (SDM) in the field of IT development is a general term applied to a variety of structured, organized processes for developing information technology and embedded software systems.
\end{itemize}
    
\subsection{Synergies Between Systems Science and Engineering}\index{Synergies Between Systems Science and Engineering}

The purpose of this section is to describe some of the commonalities, complementary roles, and potential synergies between systems science and systems engineering. Two general themes are prominent:

\begin{enumerate}
\item The essence of the relationships between SS and SE, the stable relation and potential synergies between them.
\item The changes needed in SE and SS to address emerging challenges and opportunities in the systems arena.
\end{enumerate}

Theme 1: As human endeavors, both fields will continue to evolve, as will their potential connections to each other. Nevertheless, the essence of the relationship between these fields is expected to be relatively stable. The intent is that this paper will mature over time in its understanding of the essence of the relationship and will evolve to reflect the evolution of the fields and to indicate future directions for both fields.

Theme 2: Both SE and SS face emerging challenges and opportunities, and the authors believe that both need to change to adequately address the challenges and exploit the opportunities. This paper will provide a rationale for those beliefs and outline a possible path forward.

There have been movements recently to more closely align systems science and systems engineering organizations, both to understand and leverage the synergies and to mutually foster the needed changes.

Following this introduction, Section 2 of the paper describes specific organizations which represent significant members and practitioners of each area: the International Society for the Systems Sciences (ISSS: www.isss.org), the International Federation for Systems Research (IFSR: www.ifsr.org ), and the International Council on Systems Engineering (INCOSE: www.incose.org ). Section 3 addresses some of the histories of systems science and systems engineering. Section 4discusses emerging changes and challenges for system engineering, and Section 5 discusses emerging changes and challenges for systems science. Section 6 explores some of the interdependencies and synergies between systems science and systems engineering, including ways in which science and engineering (i.e., understanding and acting) more generally might be brought closer together. Section 7 offers some very preliminary conclusions.

%------------------------------------------------

\section{Engineering Basis for Systems Engineering}\index{Engineering Basis for Systems Engineering}

From a cognitive perspective, systems thinking integrates analysis and synthesis. Natural science has been primarily reductionistic, studying the components of systems and using quantitative empirical verification. Human science, as a response to the use of positivistic methods for studying human phenomena, has embraced more holistic approaches, studying social phenomena through qualitative means to create meaning. Systems thinking bridges these two approaches by using both analysis and synthesis to create knowledge and understanding and integrating an ethical perspective. Analysis answers the `what’ and `how’ questions while synthesis answers the rich inquiring platform for approaches such as social systems design, developed by Bela H. Banathy, and evolutionary systems design, as Alexander Laszlo and myself have developed to include a deeper understanding of a system in its larger context as well as a vision of the future for co-creating ethical innovations for sustainability.

Accordingly, the overall purpose of the chapters in Part II is to impart an in-depth understanding of system design as a process; a process that is greater than the sum of its artificial categories as identified by chapter titles. To achieve this process purpose, it is necessary to introduce some terminology and notation somewhat prematurely; that is, before the more complete treatment provided in subsequent chapters.

Systems Engineering and the Scientific Method – The Scientific Method is a fundamental basis for scientific development. Like SE it is not a one-size-fits-all cookbook method and requires intelligent tailoring of fundamental principles to achieve desired outcomes. The Scientific Method is learned early in STEM education and applied across multiple courses that follow. Systems Engineering, however, is just as fundamental to engineering of even the most fundamental concepts. The proposed research task would develop a simple and easy way to teach the concept of SE based on the scientific method model that all STEM students can also learn and internalize with the same ubiquity as the Scientific Method. This task will require revising current internet sites that tout an ``Engineering Method'' that bears little resemblance to how engineering is practiced except on very simple projects. FROM: Chaput, et al.

\subsection{Systems Engineering: Understanding the Differences}\index{Systems Engineering: Understanding the Differences}

Fundamental to the application of systems engineering is an understanding of life-cycle thinking, illustrated for the product in. The product life cycle begins with the identification of a need and extends through conceptual and preliminary design, detail design and development, production or construction, distribution, utilization, support, phase-out, and disposal. The life-cycle phases are classified as acquisition and utilization to recognize producer and customer activities.

Figure 5.1 Here

There is a need to properly define the ``intellectual foundations of systems engineering;'' we need to look beyond systems engineering to do this. This paper presents a new framework for understanding and integrating the distinct and complementary contributions of systems science, systems thinking, and systems engineering (in both domain-independent and domain-specific forms) to an ``integrated systems approach.'' The key step is to properly separate out and show the relationship between the triad of: systems science as an objective ``science of systems;'' systems engineering as ``creating, adjusting, and configuring systems for a purpose.'' None of these is a subset of another; all must be considered as distinct through interdependent subjects. They can be so considered if they are correctly defined and bounded. The key conclusion of the analysis in that the ``correct'' choice of system boundary for a purpose depends on the property of interest. This choice seems to belong in the domain of ``systems thinking,'' which thus provides a key input to ``systems engineering.'' In practice in many systems businesses, the role of “systems architect” or ``systems engineer'' integrates the skills of systems science, systems thinking and systems engineering which are therefore essential competencies for the role.

This starts with the premise that there is a ``systems science,'' or at least a ``theory of systems,'' that defines systems and system properties. These properties of a system - which can be either natural or human-made, and which can exhibit a greater or lesser degree of adaptive behavior - are independent of how and why it was created. The nature of such systems is independent of human intentionality, and we see similar or identical system properties and patterns recurring across different domains of application.

We consider systems engineering as those activities relating to the purposeful creation or adaptation, adjustment, and operation, of systems. From this perspective, the core function of systems engineering is ``making choices about how to create and adjust a new system or modify an existing on the better to achieve a purpose.''

So, if systems science is independent of human intentionality, and systems engineering makes choices about creating or modifying systems ``for a purpose,'' where do we establish purpose?  If we consider systems thinking to be about ``understanding systems in a human context,'' we can then view it as the activity that looks at systems through a lens of human intentionality and establishes the ``purpose'' and ``value'' that drive systems engineering. This triad approach, treating systems science, systems thinking and systems engineering as equal partners in an integrated systems approach, allows us to resolve paradoxes that have troubled practitioners for some time.

A systems approach in a domain will apply these general principles within the context of existing knowledge about the domain. This may include an understanding of domain problems, constraints, risks and opportunities; and the best to tackle issues as we approach is better efficiency based on risk-aware replication of known practices and proven design rules. Potential disadvantages are blindness to cross-domain opportunities and issues, and a risk of the ``wrong-problem syndrome,'' solving the problems that interest domain experts rather than what is needed to resolve the problem situation. Draft Thales Copyright 2011 12 Jan 2011

Background. There has long been a divide between systems practitioners concerned with ``hard'' systems - often involving software and complex technologies - and ``soft systems,'' concerned with social systems and human understanding of systems and human response to complex situations. Both sets of practice seek an underpinning theory or science of systems. However, the relationship of systems science to systems thinking and systems engineering is a process-oriented approach that developed out of US DoD and NASA practices in the 1960s and onwards. Peter Checkland developed ``soft systems methodology'' as a way of introducing systems thinking and systemic understanding into complex organizational problem situations. Derek Hitchins in a number of publications between 1990 and 2010 has presented a set of definitions that allow us to characterize ``systems'' and systemic behavior. There is increasing interest in ``complex adaptive systems.'' Recently Bristol University has sought to unify the ``Systems engineering'' and ``systems thinking'' communities using the principle ``hard systems exist within soft systems.'' This view, while useful for integrating the domains of ``systems engineers'' and ``systems thinkers,'' does not account the fact that many systems - all natural ones and many man-made ones as well - exist and evolve independent of human intentionality. Jack Ring’s ``System value cycle'' describes an integrated system approach that addresses a “community situation” focusing in turn on ``value,'' ``purpose,'' and ``system.'' But Lawson’s ``system coupling model'' shows systems as being configured from available assets in response to a ``problem situation.'' There is a renewed interest within INCOSE - often regarded as a purely ``hard systems'' and process-oriented community in systems science as a theoretical underpinning for systems engineering and in the use of ontologies and model based systems engineering methods to improve the rigor and predictability of systems engineering endeavor.

Theory of systems
Applying systems approach to a domain
“whole systems thinking”, “understanding systems in a human context”
Establish human interest and intentionality wrt systems
Making choices about how to create and adjust a new system or modify an existing one the better to achieve a purpose.
Purpose and value
Stakeholder alignment
Properties of interest;
Appropriate boundary
Effect of proposed changes

Success in systems engineering derives from the realization that the design activity requires a <emphasis>team</emphasis> approach. As one proceeds from conceptual design to the subsequent phases of the life cycle, the actual team ``make-up'' will vary in terms of the specific expertise required and the number of project personnel assigned. Early in the conceptual and preliminary design phases, there is a need for a few highly qualified individuals with broad technical knowledge who understand the customer’s requirements and the user’s operational environment, the major functional elements of the system and their interface relationships, and the general process for bringing a system into being. These key individuals are those who understand and believe in the systems approach and know when to call on the appropriate disciplinary expertise for assistance. The objective is to ensure the early consideration of all of the ``design-for'' requirements described in 

As the design progresses, the need for representation from the various design disciplines increases. Referring to (which is an extension of the concept in ), a systems engineering implementation goal is to ensure the proper integration of the design disciplines as appropriate to the need. Depending on the project, there may be relatively few individuals assigned, or there may be hundreds involved. Further, some of the expertise desired may be located within the same physical facility, whereas other members of the project team may be remotely located in supplier organizations (both locally and internationally). In a system-of-systems (SOS) context, the project team may need to include individuals representing major “interfacing” systems.

A major goal in the implementation of the system engineering process is first to understand system requirements and the expectations of the customer and then to provide the technical guidance necessary to ensure that the ultimate system configuration will meet the need. Realization of this goal depends on providing the right personnel and material resources at the right location and in a timely manner. Such resources may include a combination of the following:	

\begin{enumerate}
\item Engineering technical expertise (e.g., aeronautical engineers, civil engineers, electrical engineers, mechanical engineers, software engineers, reliability engineers, logistics engineers, and environmental engineers)
\item Engineering technical support (e.g., technicians, component part specialists, computer programmers, model builders, drafting personnel, test technicians, and data analysts)
\item Nontechnical support (e.g., marketing, purchasing and procurement, contracts, budgeting and accounting, industrial relations, manufacturing personnel, and logistics supply chain specialists). 
\end{enumerate}

Whatever the case, the objective of systems engineering is to promote the “team” approach and to create the proper working environment for the necessary ongoing communications and exchange of information on a continuing day-to-day basis. illustrates this essential integration function, and the “system engineer” must be knowledgeable of the various disciplines, their respective objectives, and when to integrate these requirements into the overall design process. The organization for systems engineering is discussed further in

%------------------------------------------------

\section{Engineering for the Systems Age}\index{Engineering for the Systems Age}

In the Systems Age, successful accomplishment of engineering objectives usually requires a combination of technical specialties and expertise. Engineering in the Systems Age must be a team activity where various individuals involved are cognizant of the important relationships between specialties and between economic factors, ecological factors, political factors, and societal factors. The engineering decisions of today require consideration of these factors in the early stage of system design and development, and the results of such decisions have a definite impact on these factors. Conversely, these factors usually impose constraints on the design process. Thus, technical expertise must include not only the basic knowledge of individual specialty fields of engineering but also knowledge of the context of the system being brought into being.

Although relatively small products, such as a wireless communication device, an electrical household appliance, or even an automobile, may employ a limited number of direct engineering personnel and supporting resources, there are many large-scale systems that require the combined input of specialists representing a wide variety of engineering and related disciplines. An example is that of a ground-based mass-transit system.

Civil engineers are required for the layout and/or design of the railway, tunnels, bridges, and facilities. Electrical engineers are involved in the design and provision of automatic controls, traction power, substations for power distribution, automatic fare collection, digital data systems, and so on. Mechanical engineers are necessary in the design of passenger vehicles and related mechanical equipment. Architectural engineers provide design support for the construction of passenger terminals. Reliability and maintainability engineers are involved in the design for system availability and the incorporation of supportability characteristics. Industrial engineers deal with the production and utilization aspects of passenger vehicles and human components. Test engineers evaluate the system to ensure that all performance, effectiveness, and system support requirements are met. Engineers in the planning and marketing areas are required to keep the public informed, to explain the technical aspects of the system, and to gather and incorporate public input. General systems engineers are required to ensure that all aspects of the system are properly integrated and function together as a single entity.

Although the preceding example is not all-inclusive, it is evident that many different disciplines are needed. In fact, there are some large projects, such as the development of a new aircraft, where the number of engineers needed to perform engineering functions is in the thousands. In addition, the different engineering types often range in the hundreds. These engineers, forming a part of a large organization, must not only be able to communicate with each other but must also be conversant with such interface areas as purchasing, accounting, personnel, and legal.

Another major factor associated with large projects is that considerable system development, production, evaluation, and support are often accomplished at supplier (sometimes known as subcontractor) facilities located throughout the world. Often there is a prime producer or contractor who is ultimately responsible for the development and production of the total system as an entity, and there are numerous suppliers providing different system components. Thus, much of the project work and many of the associated engineering functions may be accomplished at dispersed locations, often worldwide.

\subsection{Systems Engineering Definition}\index{Systems Engineering Definition}

To this day, there is no commonly accepted definition of Systems Engineering (SE) in the literature. Almost a half-century ago, Hendrick W. Bode, writing on ``The Systems Approach'' in Applied Science-Technological Progress, said that ``It seems natural to begin the discussion with an immediate formal definition of Systems Engineering. However, Systems Engineering is an amorphous, slippery subject that does not lend itself to such formal, didactic treatment. One does much better with a broader, more loose-jointed approach. Some writers have, in fact, sidestepped the issue by saying that Systems Engineering is what systems engineers do.''

The definition of system engineering and the systems approach is usually based on the background and experience of the individual or the performing organization. The variations are evident from the following five published definitions:

\begin{itemize}
\item “An interdisciplinary approach and means to enable the realization of successful systems.”
\item “An interdisciplinary approach encompassing the entire technical effort to evolve into and verify an integrated and life-cycle balanced set of systems people, product, and process solutions that satisfy customer needs. Systems engineering encompasses (a) the technical efforts related to the development, manufacturing, verification, deployment, operations, support, disposal of, and user training for, system products and processes; (b) the definition and management of the system configuration; (c) the translation of the system definition into work breakdown structures; and (d) development of information for management decision making.”
\item “The application of scientific and engineering efforts to (a) transform an operational need into a description of system performance parameters and a system configuration through the use of an iterative process of definition, synthesis, analysis, design, test, and evaluation; (b) integrate related technical parameters and ensure compatibility of all physical, functional, and program interfaces in a manner that optimizes the total system definition and design; and (c) integrate reliability, maintainability, safety, survivability, human engineering, and other such factors into the total engineering effort to meet cost, schedule, supportability, and technical performance objectives.”
\item “An interdisciplinary collaborative approach to derive, evolve, and verify a life-cycle balanced system solution which satisfies customer expectations and meets public acceptability.”
\item “An approach to translate operational needs and requirements into operationally suitable blocks of systems. The approach shall consist of a top-down, iterative process of requirements analysis, functional analysis and allocation, design synthesis and verification, and system analysis and control. Systems engineering shall permeate design, manufacturing, test and evaluation, and support of the product. Systems engineering principles shall influence the balance between performance, risk, cost, and schedule.”
\end{itemize}

Although the definitions vary, there are some common threads. Basically, systems engineering is good engineering with special areas of emphasis. Some of these are following:

\begin{enumerate}
\item A top-down approach that views the system as a whole. Although engineering activities in the past have adequately covered the design of various system components (representing a bottom-up approach), the necessary overview and understanding of how these components effectively perform together is frequently overlooked.
\item A life-cycle orientation that addresses all phases to include system design and development, production and/or construction, distribution, operation, maintenance and support, retirement, phase-out, and disposal. Emphasis in the past has been placed primarily on design and system acquisition activities, with little (if any) consideration given to their impact on production, operations, maintenance, support, and disposal. If one is to adequately identify risks associated with the up-front decision-making process, then such decisions must be based on life-cycle considerations.
\item A better and more complete effort is required regarding the initial definition of system requirements, relating these requirements to specific design criteria, and the follow-on analysis effort to ensure the effectiveness of early decision making in the design process. The true system requirements need to be well defined and specified and the traceability of these requirements from the system level downward needs to be visible. In the past, the early “front-end” analysis as applied to many new systems has been minimal. The lack of defining an early “baseline” has resulted in greater individual design efforts downstream.
\item An interdisciplinary or team approach throughout the system design and development process to ensure that all design objectives are addressed in an effective and efficient manner. This requires a complete understanding of many different design disciplines and their interrelationships, together with the methods, techniques, and tools that can be applied to facilitate implementation of the system engineering process.
\end{enumerate}

Systems engineering is not a traditional engineering discipline in the same sense as civil engineering, electrical engineering, industrial engineering, mechanical engineering, reliability engineering, or any of the other engineering specialties. It should not be organized in a similar manner, nor does the implementation of systems engineering (or its methods) require extensive resources. However, a well-planned and highly disciplined approach must be followed. The systems engineering process involves the use of appropriate technologies and management principles in a synergetic manner. Its application requires synthesis and focus on process, along with a new “thought process” that is compatible with the needs of the Systems Age.

\subsection{Promulgating Systems Engineering Within the Engineering Profession}\index{Promulgating Systems Engineering Within the Engineering Profession}

From its modest beginning more than a half-century ago, Systems Engineering is now gaining international recognition as an effective technology based interdisciplinary process for bringing human-made systems into being, and for improving systems already in being. Certain desirable academic and professional attributes are coming into clear view. Others require further study, development, testing, and implementation.

This section summarizes the heritage from which Systems Engineering entered the 21st century. Several emerging attributes of Systems Engineering education and professional practice are addressed. These include the necessary but not sufficient academic and professional activities of technical societies, degree programs and program accreditation, certification and licensing, knowledge generation and publications, recognition and honors, and considerations regarding maturity. Special attention is directed to those attributes that should be developed further to enable Systems Engineering to serve society will in this century.

Conceptually sound system design derives from focusing on what the system is intended to do before determining what the system is, with form following function. This focus is most effective when based on essential design dependent parameters, recognizing the concurrent life-cycle factors of production, support, maintenance, phase-out, and disposal. It invokes integrating and iterating synthesis, analysis, and evaluation. These considerations are germane to system and product design when embedded within the systems engineering process. The purpose of this presentation is to provide an overview of the embedded relationship of design dependent parameters as key controllables in the effective, and orderly process of bringing cost-effective systems, products, structures, and services (the human-made world) into being.  Draft Thales Copyright 2011 12 Jan 2011 Page 1 of 8
    
%------------------------------------------------

\section{Organizations for Advancing Systems Thinking}\index{Organizations for Advancing Systems Thinking}

The content of this chapter is anchored by the domains of systems science and systems engineering, beginning with the former and ending with the latter. Accordingly, it is important to recognize that at least one professional organization exists for each domain. For systems science, there is the International Society for the System Sciences (ISSS), originally named the ``Society for General Systems Research.'' ISSS was established at the 1956 meeting of the American Association for the Advancement of Science under the leadership of biologist Ludwig von Bertalanffy, economist Kenneth Boulding, mathematician-biologist Anatol Rapoport, neurophysiologist Ralph Gerard, psychologist James Miller, and anthropologist Margaret Mead.

The founders of the International Society for the System Sciences felt strongly that the systematic (holistic) aspect of reality was being overlooked or downgraded by the conventional disciplines, which emphasize specialization and reductionist approaches to science. They stressed the need for more general principles and theories and sought to create a professional organization that would transcend the tendency toward fragmentation in the scientific enterprise. The reader interested in exploring the field of systems science and learning more about the work of the International Society for System Sciences, may visit the ISSS website at http://www.isss.org.

Technology, human-made, and human-modified systems comprise the core of this chapter, with science and systems science as the foundation. It is the engineered system that is the justification for almost every concept, process, method, and model presented in this textbook. Accordingly, and in-depth understanding of the engineered system (through a focused definition and description) is not part of this chapter. It is deferred for the opening of Chapter 6, where the paradigm for bringing systems into being is treated at a high level. The purpose is to clarify the distinction between systems that are engineered and systems that exist naturally.

The most prominent professional organization for systems engineering is the International Council on Systems Engineering (INCOSE). Originally named the ``National Council on Systems Engineering,'' NCOSE was chartered in 1991 in the United States; it has now expanded worldwide to become the leading society to develop, nurture, and enhance the interdisciplinary approach and means to enable the realization of successful systems. INCOSE has strong and enduring ties with industry, academia, and government to achieve the following goals:

\begin{enumerate}
\item Provide a focal point for the dissemination of systems engineering knowledge
\item Promote collaboration in systems engineering education and research
\item Assure the establishment of professional standards for integrity in the practice of systems engineering
\item Improve the professional status of persons engaged in the practice of systems engineering
\item Ecourage governmental and industrial support for research and educational programs that will improve the systems engineering process and its practice. 
\end{enumerate}

An expanded view of systems engineering, as promulgated through the International Council of Systems Engineering, as well as a window into a wealth of information about this relatively new engineering interdisciplines may be obtained at http://www.incose.org.

The ISSS was established in 1956, originally the Society for General Systems Research, an affiliate of the American Association for the Advancement of Science. It is often associated with the work of Ludwig von Bertalanffy and the efforts to develop a General System Theory, but its members and leaders have spanned a wide range of professional backgrounds and practice areas. The international Federation for Systems Research (IFSR: www.ifsr.org ), founded in 1981, is a federation of systems organizations around the world (including the ISSS and since 2011 even INCOSE). Those member organizations cover a spectrum of geographic regions and of specific systems orientations (e.g. cybernetics). INCOSE, founded in 1991, has a membership of nearly 8000 systems engineers, most prominently from aerospace and defense industries in the US, but also spanning many other countries and other domains.

In 2010, INCOSE began an internal working group focused on a systems science, whose charter is to “promote the advancement and understanding of Systems Science and its application to SE” (http://www.incose.org/practice/techactivities/wg/syssciwg/ ). The stated objectives were to: “1) encourage advancement of systems science principles and concepts as they apply to systems engineering, 2) promote awareness of systems science as a foundation for systems engineering, and 3) highlight linkages between systems science theories and empirical practices of systems engineering.”

In concert with this same intent, INCOSE and ISSS began formally inviting representatives as guests to each other’s meetings, in an exchange of ideas and interests. In 2011, ISSS and INCOSE formally signed a memorandum of understanding, based on the following principles:

\begin{enumerate}
\item The ISSS and INCOSE agree to a relationship for mutual benefit, to be reconfirmed every three years. The purpose of the relationship is to further the practices and knowledge jointly in systems sciences and systems engineering.
\item INCOSE members are interested in gaining foundational knowledge in systems science concepts, methods and tools that may be applied in the practice of systems engineering.
\item ISSS members are interested in seeing systems theories applied in practice, and further developing approaches on practical problems in systems engineering, based on the rich legacy of research in the systems sciences and on the feedback from systems engineering experience in applying the theories.
\end{enumerate}

Most scientific and professional societies in the United States interact and collaborate with cognizant but independent honor societies. The cognizant honor society for systems engineering is the Omega Alpha Association (OAA), emerging under the motto “Think About the End Before the Beginning.”  Chartered in 2006 as an international honor association, OAA has the overarching objective of advancing the systems engineering process and its professional practice in service to humankind. Among subordinate objectives are opportunities to (1) inculate a greater appreciation within the engineering profession that every human being decision shapes the human-made world and determines its impact upon the natural world and upon people; (2) advance system design and development morphology through a better comprehension and adaptation of the da Vinci philosophy of thinking about the end before the beginning; that is, determining what designed entities are intended to do before specifying what the entities are and concentrating on the provision of functionality, capability, or a solution before designing the entities per se; and (3) encourage excellence in systems engineering education and research through collaboration with academic institutions and professional societies to evolve robust policies and procedures for recognizing superb academic programs and students. The OAA website, http://www.omegalpha.org, provides information about OAA goals and objectives, as well as the OAA vision for recognizing and advancing excellence in systems engineering, particularly in academia.

%------------------------------------------------

\section{Grand Challenges for Engineering}\index{Grand Challenges for Engineering}

%------------------------------------------------

\section{Systems Engineering Education}\index{Systems Engineering Education}

Engineering education has been subjected to in-depth study every decade or so, beginning with the Mann Report in 1918. The most recent and authoritative study was conducted by the National Academy of Engineering (NAE) and published in 2005 under the title Educating the Engineer of 2020.
Although acknowledging that certain basics of engineering will not change, this NAE report concluded that the explosion of knowledge, the global economy, and the way engineers will work will reflect on ongoing evolution that began to gain momentum at the end of the twentieth century. The report gives three overarching trends to be reckoned with the engineering educators, interacting with engineering leaders in government and industry:

\begin{enumerate}
\item The economy in which we work will be strongly influenced by the global marketplace for engineering services, evidenced by the outsourcing of engineering jobs, a growing need for interdisciplinary and system-based approaches, demands for new paradigms of customization, and an increasingly international talent pool.
\item The steady integration of technology in our public infrastructures and lives will call for more involvement by engineers in the setting of public policy and for participation in the civic arena.
\item The external forces in society, the economy, and the professional environment will all challenge the stability of the engineering workforce and affect our ability to attract the most talented individuals to an engineering career.
\end{enumerate}

Continuing technological advances have created in increasing demand for engineers in most fields. But certain engineering and technical specialties will be merged or become obsolete with time. There will always be a demand for engineers who can synthesize and adapt to changes. The astute engineer should be able to detect trends and plan for satisfactory transitions by acquiring knowledge to broaden his or her capability. To help obtain the knowledge needed for adaptation is one aim of this systems engineering textbook.

\subsection{Systems Engineering}\index{Systems Engineering}

From its modest beginnings more than a half-century ago, systems engineering (SE) is emerging as an effective technologically based interdisciplinary process for bringing systems and services into being. While the primary focus is nominally on the entities themselves, systems engineering is inherently oriented to considering “the end before the beginning”. It concentrates on what the entities are intended to do before determining what the entities are, with form following function.

Systems engineering is concerned with the engineering of human-made systems and with systems analysis. In the first case, emphasis is one the process of bringing systems into being, beginning with the identification of a need or deficiency and extending through requirements determination, functional analysis and allocation, design synthesis and evaluation, design validation, deployment and distribution, operation and support, sustainment, and phase-out and disposal. In the second case, focus is on the improvement of systems already in being. By utilizing the iterative process of analysis, evaluation, modification, and feedback, most systems now in existence can be improved in their operational effectiveness, delivered service quality, user affordability, product and environmental sustainability, and stakeholder satisfaction. The systems approach is increasingly being considered by forward-looking private and public organizations and enterprises. It is applicable to most types of systems, encompassing the human activity domains of communication, defense, education, healthcare, manufacturing, transportation, and other named in the National Academy of Engineering compilation of Grand Challenges.

\subsection{Operations Research and Management Science}\index{Operations Research and Management Science}

Operations research (OR) and the management sciences (MS), provide a body of systematic knowledge embracing models, modeling, and simulation approaching for performing systems analysis (SA). Applicable to operations and management as the name implies, OR/MS has been found to be necessary but not sufficient for Engineering Economy to be rigorously linked to operations. Accordingly, both OR and MS will be recognized in the sections that follow as primary enablers for EE@SL to provide more opportunities to “think about the end before the beginning”.

Systems and Systems Science

Sillito, H. G. (2012). Integrating systems science, systems thinking, and systems engineering: Understanding the differences and exploiting the synergies. Proceedings of INCOSE International Symposium, Rome, July 2012.

One score and five years ago, our beloved Council (on SE, now international) was founded by visionary Charter Members to serve society through Systems Engineering, and to do so in collaboration with the disciplines (domains) of engineering. Our INCOSE of today has embraced and subsumed that vision, held also by then President Brian Mar. Brian recognized and promulgated domain centric systems engineering without the acronym DCSE, believing that SE is best developed within the domains of engineering. I supported that founding position.

It was not until INCOSE began seeking inroads into the venerable engineering organizations of ABET and ASEE that we felt compelled to explain our intent. I credit colleague Dennis Buede and my co-author Elizabeth McCrae for the inspiration leading to the acronym pair that now includes systems centric systems engineering (SCSE), ad developed in our seminal 2005 INCOSE paper. The non-threatening and peacekeeping benefit of DCSE/SCSE partitioning has proven its value over time in our relations with the engineering profession.

Not all SE degree programs are administered through the classical departmental structure of the host institution. Although most undergraduate programs are classically organized, the following variants will be found:

\begin{enumerate}
\item There are instances where an academic administrative unit will be the home for more than one-degree program or major; e.g., Systems Engineering and Industrial Engineering. The department name may or may not subsume the names of all degree programs.
\item There are instances where the institution will offer both a SE Centric (SEC) and a Domain Centric SE (DCSE) program; e.g., Systems Engineering and Manufacturing Systems Engineering. The DCSE program may be administered in an interdepartmental mode, whereas the SEC program will usually be administered within a department.
\item In those instances where an institution offers a SEC program at the basic and advanced levels, all are usually administered within a department. This is also true for DCSE programs, except that the SE component may not exist at all degree levels.
\end{enumerate}

The above variants are mentioned to emphasize that one must be aware of the administrative and organizational home for a degree program of interest.

The focus in this paper is always on the degree program itself. In discussing the basic and advanced level programs in the SEC and SCSE categories, this program focus will be strengthened by recognizing that Systems Engineering is broad in nature. It cannot be viewed in the same context as the traditional engineering disciplines. This notwithstanding, many domains of engineering are seeking a better technological balance by adopting systems thinking. This is the primary reason for the rapid growth in the number of engineering domains adding a systems component to their programs.

%------------------------------------------------

\section{Value of Engineering to Society}\index{Value of Engineering to Society}

An understanding of the interrelationships among the factors identified in Figures 5.2 and 5.3 is essential if the full benefit of systems engineering is to be realized. There is a need to ensure that the applicable engineering disciplines responsible for the design of individual system elements are properly integrated. This need extends to the proper implementation of concurrent engineering to address the life cycles for the product and for the supporting capabilities of production, support, and phase-out, as is illustrated in Figure 6.2. A good communication network, with local-area and wide-area capability, must also be in place and available to all critical project personnel. This is a particular challenge when essential project personnel are located remotely, often worldwide.

Successful implementation of the systems engineering process depends not only on the availability and application required to accomplish the overall objective. Although the steps described in Section 6.3 may be specified for a given program, successful implementation (and the benefits to be derived) will not be realized unless the proper organizational environment is established that will encourage it to happen. There have been numerous instances where a project organization included a “systems engineering” function but where the impact on design has been almost nonexistent, resulting in objectives not being met.

Although some of the benefits associate with application of the concepts and principle of systems engineering have been provided through this chapter, it may be helpful to provide a compact summary for reference. Accordingly, application of the systems engineering process can lead to the following benefits:

Reduction in the cost of system design and development, production and/or construction, system operation and support, system retirement and material disposal; hence, a reduction in life-cycle cost should occur. Often it is perceived that the implantation of systems engineering will increase the cost of system acquisition. Although there may be a few more steps to perform during the early (conceptual and preliminary) system design phases, this investment could significantly reduce the requirements in the integration, test, and evaluation efforts accomplished late in the detail design and development phase. The bottom-up approach involved in making the system work can be simplified if a holistic engineering effort is initiated from the beginning. In addition, experience indicates that the early emphasis on systems engineering can result in a cost savings later in the production, operations and support, and retirement phases of the life cycle.

Reduction in system acquisition time (or the time from the initial identification of a customer need to the delivery of a system to the customer). Evaluation of all feasible alternative approaches to design early in the life cycle (with the support of available design aids such as the use of CAD technology) should help to promote greater design maturity earlier. Changes can be incorporated at an early stage before the design is “fixed” and costlier to modify. Further, the results should enable a reduction in the time that it takes for final system integration, tests, and evaluation.

More visibility and a reduction in the risks associated with the design decision-making process. Increased visibility is provided through viewing the system from a long-term and life-cycle perspective. The long-term impacts because of early design decisions and “cause-and-effect” relationships can be assessed at an early stage. This should cause a reduction in potential risks, resulting in greater customer satisfaction.

Without the proper organizational emphasis from top management, the establishment of an environment that will allow for creativity and innovation, a leadership style that will promote a “team” approach to design, and so on, implementation of the concepts and the methodologies described herein may not occur. Thus, systems engineering must be implemented in terms of both technology and management. This joint implementation of systems engineering is the responsibility of systems engineering management. Part V of this textbook, consisting of two chapters, is devoted to this important activity.

The NAE Grand Challenges FROM ASEE PAPER.

%------------------------------------------------

\section{Summary and Extensions}\index{Summary and Extensions}

In this chapter, some system definitions and systems science concepts were presented to provide a basis for the study of systems engineering and analysis. They include definitions of system characteristics, a classification of systems into various types, consideration of the current state of systems science, and a discussion of the transition to the Systems Age. Finally, the chapter presents technology and the nature and role of engineering in the Systems Age and ends with a number of commonly accepted definitions of systems engineering.</para>

Upon completion of , the reader will have obtained essential insight into systems and systems thinking, with an orientation toward systems engineering and analysis. The system definitions, classifications, and concepts presented in this chapter are intended to impart a general understanding about the following:

\begin{enumerate}
\item System classifications, similarities, and dissimilarities
\item The fundamental distinction between natural and human-made systems
\item The elements of a system and the position of the system in the hierarchy of systems
\item The domain of systems science, with consideration of cybernetics, general systems Theory, and systemology
\item Technology as the progenitor for the creation of technical systems, recognizing its impact on the natural world
\item The transition from the machine or industrial age to the Systems Age, with recognition of its impact upon people and society
\item System complexity and scope and the demands these factors make on engineering in the Systems Age
\item The range of contemporary definitions of systems engineering used within the profession
\end{enumerate}

Although this book focuses on the engineering of systems and on systems analysis, it would not be intellectually prudent to begin the discussion at that level. Upon examination, it is evident that both the engineering and the analysis aspects of the focus are directed to systems. Accordingly, this chapter is devoted to helping the reader gain essential insight into systems in general, and systems thinking in particular, with orientation toward the engineering and analysis of technical systems.

System definitions, a discussion of system elements, and a high-level classification of systems provide an opening panorama. It is here that a consideration of the origin of systems provides an orientation to natural and human-made domains as an overarching dichotomy. The importance of this dichotomy cannot be overemphasized in the study and application of systems engineering and analysis. It is the suggested frame of reference for considering and understanding the interface and impact of the human-made world on the natural world and on humans.

Individuals interested in obtaining as in-depth appreciation for this interface and the mitigation of environmental impacts are encouraged to read T.E. Graedel and B.R. Allenby, Industrial Ecology, 2nd ed., Prentice Hall, 2003. Also of contemporary interest is the issue of sustainability treated as part of an integrated approach to sustainable engineering by P. Stasinopoulos, M.H. Smith, K. Hargroves, and C. Desha, Whole System Design, Earthscan Publishing, 2009. These works are recommended as an extension to this chapter (as well as Chapter 16), because they illuminate and address the sensitive interface between the natural and the human-made.

%------------------------------------------------

%% SEA CHAPTER 5 - DETAIL DESIGN AND DEVELOPMENT
% SEA Question Location in \label{sea-Chapter#-Problem#}
\begin{exercises}
    \begin{exercise}
    \label{sea-5-1}
        What are the basic differences between conceptual design, preliminary system design, and detailed design and development? Are these stages of design applicable to the acquisition of all systems? Explain.
    \end{exercise}
    \begin{solution}
    \end{solution}
    
    \begin{exercise}
    \label{sea-5-2}
        Design constitutes a team effort. Explain why. What constitutes the make-up of the design team? How can this be accomplished? How does systems engineering fit in to the process?
    \end{exercise}
    \begin{solution}
    \end{solution}
    
    \begin{exercise}
    \label{sea-5-3}
        Briefly describe the role of systems engineering in the overall design process as it is described in Chapters 3, 4, and 5.
    \end{exercise}
    \begin{solution}
    \end{solution}
    
    \begin{exercise}
    \label{sea-5-4}
        Refer to Figure 5.1. What are some of the advantages of the concurrent approach in design? Identify some of the problems that could occur in its implementation.
    \end{exercise}
    \begin{solution}
    \end{solution}
    
    \begin{exercise}
    \label{sea-5-5}
        Refer to Figures 4.4 and 4.8 (Chapter 4). As the systems engineering manager on a given program, what steps would you take to ensure that the proper integration of requirements occurs across the three life cycles (hardware, software, human) from the beginning?
    \end{exercise}
    \begin{solution}
    \end{solution}
    
    \begin{exercise}
    \label{sea-5-6}
        As a designer, one of your tasks constitutes the selection of a component to fulfill a specific design objective. What priorities would you consider in the selection process (if any)?
    \end{exercise}
    \begin{solution}
    \end{solution}
    
    \begin{exercise}
    \label{sea-5-7}
        Why are design standards (as applied to component parts and processes) important?
    \end{exercise}
    \begin{solution}
    \end{solution}
    
    \begin{exercise}
    \label{sea-5-8}
        Why are engineering documentation and the establishment of a design database necessary?
    \end{exercise}
    \begin{solution}
    \end{solution}
    
    \begin{exercise}
    \label{sea-5-9}
        Refer to Figure 5.5. When accomplishing the necessary trade-offs, there may be some confusion as to which of the three options to pursue. Describe what information is required as an input in order to evolve into a “clear-cut” approach.
    \end{exercise}
    \begin{solution}
    \end{solution}
    
    \begin{exercise}
    \label{sea-5-10}
        Describe how the application of CAD, CAM, and CAS tools can facilitate the system design process. Identify some benefits. Address some of the problems that could occur in the event of misapplication.
    \end{exercise}
    \begin{solution}
    \end{solution}
    
    \begin{exercise}
    \label{sea-5-11}
        How can CAD, CAM, and CAS tools be applied to validate the design? Provide an example of two.
    \end{exercise}
    \begin{solution}
    \end{solution}
    
    \begin{exercise}
    \label{sea-5-12}
        What is the purpose of developing a physical model of the system, or an element thereof, early in the system design process?
    \end{exercise}
    \begin{solution}
    \end{solution}
    
    \begin{exercise}
    \label{sea-5-13}
        What are some of the differences between a mock-up, an engineering model, and a prototype?
    \end{exercise}
    \begin{solution}
    \end{solution}
    
    \begin{exercise}
    \label{sea-5-14}
        Select a system (or an element of a system) of your choice and develop a design review checklist that you can use for evaluation purposes. (Refer to Figure 5.8 and Appendix B.)
    \end{exercise}
    \begin{solution}
    \end{solution}
    
    \begin{exercise}
    \label{sea-5-15}
        What are some of the benefits that can be acquired through implementation of a formal design review process?
    \end{exercise}
    \begin{solution}
    \end{solution}
    
    \begin{exercise}
    \label{sea-5-16}
        Refer to Figure 5.9. The predicted LCC value for the system, at the time of a system design review, is around $500K, which is well above the $420K design-to requirement. What steps would you take to ensure that the ultimate requirement will be met at (or before) the critical design review? Be specific.
    \end{exercise}
    \begin{solution}
    \end{solution}
    
    \begin{exercise}
    \label{sea-5-17}
        Refer to Figure 5.10. As a systems engineering manager, how would you ensure that all of the TPM requirements are being properly “tracked”?
    \end{exercise}
    \begin{solution}
    \end{solution}
    
    \begin{exercise}
    \label{sea-5-18}
        What determines whether or not a given design review has been successful?
    \end{exercise}
    \begin{solution}
    \end{solution}
    
    \begin{exercise}
    \label{sea-5-19}
        In evaluating the feasibility of an ECP, what considerations need to be addressed?
    \end{exercise}
    \begin{solution}
    \end{solution}
    
    \begin{exercise}
    \label{sea-5-20}
        Assume that an ECP has been approved by the CCB. What steps need to be taken in implementing the proposed change?
    \end{exercise}
    \begin{solution}
    \end{solution}
    
    \begin{exercise}
    \label{sea-5-21}
        What is configuration management? When can it be implemented? Why is it important?
    \end{exercise}
    \begin{solution}
    \end{solution}
    
    \begin{exercise}
    \label{sea-5-22}
        Why is baseline management so important in the implementation of the systems engineering process?
    \end{exercise}
    \begin{solution}
    \end{solution}
\end{exercises}
% SKIPPED