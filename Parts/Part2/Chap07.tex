\chapter{MODELS AND MODELING IN SYSTEMS ENGINEERING}\label{chap:7}

The utilization of models has become essential in almost all disciplines, especially the applied sciences and engineering disciplines of. Early examples of models include diagrams of structures and schematics to plan battlefield movements, while modern examples cover a variety of applications for indirect experimentation ranging from mathematical manipulation through complex computer simulations.
Beyond classical models, recent software, analytical, and technology advancements have positioned modeling in engineering with promises to reduce cost, improve performance, and make collaboration more effective. While modeling has a rich history, wide application in engineering, and profound potential benefits, the realization from advanced and innovative modeling techniques depends entirely on awareness, application, and adoption by technology-based enterprises.
But, there are fundamental differences between models used in science and engineering. Science is concerned with the natural world, whereas engineering is concerned primarily with the human-made world. Science uses models to gain an understanding of the way things are in the natural world. Engineering uses models of the human-made world in an attempt to achieve what humans want. The validated models of science are used in engineering to establish bounds for engineered systems and to improve the products of such systems.

%------------------------------------------------

\section{Models and Indirect Experimentation}\index{Models and Indirect Experimentation}

The only accurate representation of reality is reality itself. Accordingly, all representations or abstractions of reality offered for various purposes are properly designated as models of reality. Fortunately, indirect experimentation may be accomplished through models. Direct experimentation would require manipulating reality, often an impossibility.
The use of models is ubiquitous in engineering, but data on model use and user sophistication in engineering-driven firms is remarkably sparse. Furthermore, some predict a significant change in engineering workflow through model-based system engineering (MBSE). Furthermore, MBSE has been trending over the last decade, arguing for a central model to serve as a means of coordinating system design. Despite benefits that modeling provides, the prevalence of effective modeling in enterprises remains an open question.
</para></para></section></section><para>When used as a noun, the word “model” implies representation. An aeronautical engineer may construct a wind-tunnel model of a possible configuration for a proposed aircraft type. An architect might represent a proposed building with a scale model of the building. An industrial engineer may use templates to represent a proposed layout of equipment in a factory. The word “model” may also be used as an adjective, carrying with it the implication of ideal. From this motivation comes the desire for optimizing operations through decision models.

\subsection{Classical Categories of Models}\index{Classical Categories of Models}

Models are designed to represent a system under study, by an idealized example of reality, to explain the essential relationships involved. They can be classified by distinguishing physical, analogue, schematic, and mathematical types. Physical models look like what they represent, analogue models behave like the original, schematic models graphically describe a situation or process, and mathematical models symbolize the principles of a situation being studied. These model types are used available for use in systems engineering and analysis.</para>
<section id="ch07lev3sec1"><title id="ch07lev3sec1.title">Physical Models. </title><para><inst></inst>Physical models are geometric equivalents, either as miniatures, enlargements, or duplicates made to the same scale. Globes are one example. They are used to demonstrate the shape and orientation of continents, water bodies, and other geographic features of the earth. A model of the solar system is used to demonstrate the orientation of the sun and planets in space. A model of an atomic structure would be similar in appearance but at the other extreme in dimensional reproduction. Each of these models represents reality and is used for demonstration.</para>
<para>Some physical models are used in the simulation process. An aeronautical engineer may test a specific tail assembly design with a model airplane in a wind tunnel. A pilot plant might be built by a chemical engineer to test a new chemical process for the purpose of locating operational difficulties before full-scale production. An environmental chamber is often used to create conditions anticipated for a component under test.</para>
<para>The use of templates in plant layout is an example of experimentation with a physical model. Templates are either two- or three-dimensional replicas of machinery and equipment that are moved about on a scale-model area. The relationship of distance is important, and the templates are manipulated until a desirable layout is obtained. Such factors as noise generation, vibration, and lighting are also important but are not a part of the experimentation and must be considered separately.</para></section>
<section id="ch07lev3sec2"><title id="ch07lev3sec2.title">Analogue Models.</title><para><inst></inst>Analogue comes from the Greek word <emphasis>analogia</emphasis>, which means proportion. This explains the concept of an <emphasis>analogue model</emphasis>; the focus is on similarity in relations. Analogues are usually meaningless from the visual standpoint.</para>
<para>Analogue models can be physical in nature, such as where electric circuits are used to represent mechanical systems, hydraulic systems, or even economic systems. Analogue computers use electronic components to model power distribution systems, chemical processes, and the dynamic loading of structures. The analogue is represented by physical elements. When a digital computer is used as a model for a system, the analogue is more abstract. It is represented by symbols in the computer program and not by the physical structure of the computer components.</para>
<para>The analogue may be a partial subsystem, or it may be an almost complete representation of the system under study. For example, the tail assembly design being tested in a wind tunnel may be complete in detail but incomplete in the properties being studied. The wind tunnel test may examine only the aerodynamic properties and not the structural, weight, or cost characteristics of the assembly. From this it is evident that only those features of an analogue model that serve to describe reality should be considered. These models, like other types, suffer from certain inadequacies.</para></section>
<section id="ch07lev3sec3"><title id="ch07lev3sec3.title">Schematic Models.</title><para><inst></inst>A schematic model is developed by reducing a state or event to a chart or diagram. The schematic model may or may not look like the real-world situation it represents. It is usually possible to achieve a much better understanding of the real-world system described by the model through use of an explicit coding process employed in the construction of the model. The execution of a football play may be diagrammed on a game board with a simple code. It is the idealized aspect of this schematic model that permits this insight into the football play.</para>
<para>An organization chart is a common schematic model. It is a representation of the state of formal relationships existing between various members of the organization. A human–machine chart is another example of a schematic model. It is a model of an event, that is, the time-varying interaction of one or more people and one or more machines over a complete work cycle. A flow process chart is a schematic model that describes the order or occurrence of several events that constitute an objective, such as the assembly of an automobile from a multitude of component parts.</para>
<para>In each case, the value of the schematic model lies in its ability to describe the essential aspects of the existing situation. It does not include all extraneous actions and relationships but rather concentrates on a single facet. Thus, the schematic model is not in itself a solution but only facilitates a solution. After the model has been carefully analyzed, a proposed solution can be defined, tested, and implemented.</para></section>
<section id="ch07lev3sec4"><title id="ch07lev3sec4.title">Mathematical Models.</title><para><inst></inst>A mathematical model employs the language of mathematics and, like other models, may be a description and then an explanation of the system it represents. Although its symbols may be more difficult to comprehend than verbal symbols, they do provide a much higher degree of abstraction and precision in their application. Because of the logic it incorporates, a mathematical model may be manipulated in accordance with established mathematical procedures.</para>
<para>Almost all mathematical models are used either to predict or to control. Such laws as Boyle’s law, Ohm’s law, and Newton’s laws of motion are formulated mathematically and may be used to predict certain outcomes when dealing with physical phenomena. Outcomes of alternative courses of action may also be predicted if a measure of evaluation is adopted. For example, a linear programming model may predict the profit associated with various production quantities within a multiproduct process. Mathematical models may be used to control an inventory. In quality control, a mathematical model may be employed to monitor the proportion of defects that will be accepted from a supplier. Such models maintain control over a state of reality.</para>
<para>Mathematical models directed to the study of systems differ from those traditionally used in the physical sciences in two important ways. First, because the system being studied usually involves social and economic factors, these models must often incorporate probabilistic elements to explain their random behavior. Second, mathematical models formulated to explain existing or planned operations incorporate two classes of variables: those under the control of a decision maker and those not directly under control. The objective is to select values for controllable variables so that some measure of effectiveness is optimized. Thus, these models are of great benefit in systems engineering and systems analysis.

\subsection{Direct and Indirect Experimentation}\index{Direct and Indirect Experimentation}

In direct experimentation, the object, state, or event, and/or the environment are subject to manipulation, and the results are observed. For example, a family might rearrange the furniture in their living room by this method. They move the furniture and observe the results. This process may then be repeated with a second move and perhaps a third, until all logical alternatives have been exhausted. Eventually, one such move is subjectively judged best; the furniture is returned to this position, and the experiment is completed. Direct experimentation, such as this, may be applied to the rearrangement of equipment in a factory. Such a procedure is time-consuming, disruptive, and costly. Hence, simulation or indirect experimentation is employed with templates representing the equipment to be moved.</para>
<para>Direct experimentation in aircraft design would involve constructing a full-scale prototype that would be flight tested under real conditions. Although this is an essential step in the evolution of a new design, it would be costly as the first step. The usual procedure is evaluating several proposed configurations by building a model of each and then testing in a wind tunnel. This is the process of indirect experimentation or <emphasis>simulation</emphasis>. It is extensively used in situations where direct experimentation is not economically feasible.</para>
In systems analysis, indirect experimentation is affected through the formulation and manipulation of decision models. This makes it possible to determine how changes in those aspects of the system under control of the decision maker affect the modeled system. Indirect experimentation enables the systems analyst to evaluate the probable outcome of a given decision without changing the operational system itself. In effect, indirect experimentation in the study of operations provides a means for making quantitative information available to the decision maker without disturbing the operations under his or her control.

\subsection{Simulation Through Indirect Experimentation}\index{Simulation Through Indirect Experimentation}

Models and their manipulation (the process of simulation) are useful tools in systems analysis. A <emphasis>model</emphasis> may be used as a representation of a system to be brought into being, or to analyze a system already in being. Experimental investigation using a model yields design or operational decisions in less time and at less cost than direct manipulation of the system itself. This is particularly true when it is not possible to manipulate reality because the system is not yet in existence, or when manipulation is costly and disruptive as with complex human-made systems.</para>
	<para>Models and the process of simulation provide a convenient means of obtaining information about a system being designed or a system in being. In component design, it is customary and feasible to build several prototypes, test them, and then modify the design based on the test results. This is often not possible in systems engineering because of the cost involved and the length of time required over the system life cycle. A major part of the design process requires decisions based on a model of the system rather than decisions derived from the system that does not yet exist. But then, reality cannot be called a model.
	In most design and operational situations, the objective sought is the optimization of an effectiveness or performance measure. Rarely, if ever, can this be done by direct experimentation with a system under development or a system in being. Also, there is no available theory by which the best model for a given system simulation can be selected. The choice of an appropriate model is determined as much by the experience of the systems analyst as the system itself.</para>
<para>The primary use of simulation in systems engineering is to explore the effects of alternative system characteristics on system performance without actually producing and testing each candidate system. Most models used will fit the classification given earlier, and many will be mathematical. The type used will depend on the questions to be answered. In some instances, simple schematic diagrams will suffice. In others, mathematical or probabilistic representations will be needed. In many cases, simulation with the aid of an analogue or digital computer will be required.</para>
<para>In most systems engineering and analysis undertakings, several models are usually formulated. These models form a hierarchy ranging from considerable aggregation to extreme detail. At the start of a systems project, knowledge of the system is sketchy and general. As the design progresses, this knowledge becomes more detailed and, consequently, the models used for simulation should be detailed

%------------------------------------------------

\section{Process and Decision Model Categories}\index{Process and Decision Model Categories}

Just as there is a fundamental difference at the science and engineering levels, there is a significant difference between models for the systems engineering process itself (as in bringing into being) and the system life-cycle level (as in evaluating design and operations). The difference is so profound and significantly important, that the material cannot be developed and conveyed effectively if the categories are presented together within the same chapter..
Accordingly, this section will serve the unique role of exposing the totality of topics and applications needed to convey understanding about models and modeling as it is needed in systems engineering and systems analysis. On purpose, this chapter in coordination with the next one on process phases, will end the aspect of modeling pertaining to process. Then, all of Part III (five chapters) on systems thinking and analysis will be devoted to decision models for design and operations; models of a mathematical kind. The role of design decision models is to bring forward into the system design phase insight about the predicted and estimated operational outcomes as projected from mutually exclusive design alternatives. But in both cases, the thinking pertaining to direct and indirect experimentation is fully applicable.

\subsection{Process Models for System Realization}\index{Process Models for System Realization}

The broadest classification of models for Systems Engineering derives from a high-level partition of the life cycle into design and operations; more specifically into models for process versus models for decision analysis.
	The first graphical model-like representation of the time-oriented unfolding of the developing system appeared in Figure 5.1. This is a process model exhibiting a left-to-right orientation of high-level system life-cycle phases. Variations and elaborations have appeared down through the years to convey thinking judged to be of importance. 
The basic models for the process of systems engineering are, understandably, life cycle oriented. Figure 5.1 is the most elementary of these. It is properly classified as a schematic model. But it provides the basis for two expanded elaborations.
An example conveying thinking of importance comes into view by partitioning Figure 5.1 into its concurrent life cycles. The overall system model is actually made up of four concurrent life cycles progressing in parallel, as is illustrated in. This conceptualization is one way of representing concurrent (or simultaneous) engineering<emphasis> </emphasis>.

Figure 7.1 Here (was 2.2) 

Modeling Concurrent Engineering. <para>System life-cycle engineering goes beyond the product life cycle viewed in isolation. It must simultaneously embrace the life cycle of the production or construction subsystem, the life cycle of the maintenance and support subsystem, and the life cycle for retirement, phase out, reuse, and disposal as another subsystem. The overall system is made up of four concurrent life cycles progressing in parallel, as is illustrated in. This conceptualization is the basis for concurrent engineering1
<para>The need for the product comes into focus first. This recognition initiates conceptual design to meet the need. Then, during conceptual design of the product, consideration should simultaneously be given to its production. This gives rise to a parallel life cycle for a production and/or construction capability. Many producer-related activities are needed to prepare for the production of a product, whether the production capability is a manufacturing plant, construction contractors, or a service activity.</para>
<para>Also shown in Figure 7.1 is another life cycle of considerable importance that is often neglected until product and production design is completed. This is the life cycle for support activities, including the maintenance, logistic support, and technical skills needed to service the product during use, to support the production capability during its duty cycle, and to maintain the viability of the entire system. Logistic, maintenance, technical support, and regeneration requirements planning should begin during product conceptual design in a coordinated manner.</para>
<para>As each of the life cycles is considered, design features should be integrated to facilitate phase out, regeneration, or retirement having minimal impact on interrelated systems. For example, attention to end-of-life recyclability, reusability, and disposability will contribute to environmental sustainability. Also, the system should be made ready for regeneration by anticipating and addressing changes in requirements, such as increases in complexity, incorporation of planned new technology, likely new regulations, market expansion, and others.
<para>In addition, the interactions between the product and system and any related systems should begin receiving compatibility attention during conceptual design to minimize the need for product and system redesign. Whether the interrelated system is a companion product sold by the same company, an environmental system that may be degraded, or a computer system on which a software product runs, the relationship with the product and system under development must be engineered concurrently.</para></section>
<section id=”ch02lev1sec2” label=”2.2”><title id=”ch02lev1sec2.title”><inst> </title><para>Experience over many decades indicates that a properly functioning system that is effective and economically competitive cannot be achieved through efforts applied largely after it comes into being. Accordingly, it is essential that anticipated outcomes during, as well as after, system utilization be considered during the early stages of design and development. Responsibility for <emphasis>life-cycle engineering,</emphasis> largely neglected in the past, must become the central engineering focus.</para><section id="ch02lev2sec6" label="2.2.2"><title id="ch02lev2sec6.title"><inst> </title>
	<para>Design within the system life-cycle context differs from design in the ordinary sense. Life-cycle-guided design is simultaneously responsive to customer needs (i.e., to requirements expressed in functional terms) and to life-cycle outcomes. Design should not only transform a need into a system configuration but should also ensure the design’s compatibility with related physical and functional requirements. Further, it should consider operational outcomes expressed as producibility, reliability, maintainability, usability, supportability, serviceability, disposability, sustainability, and others, in addition to performance, effectiveness, and affordability.</para>
<para>A detailed presentation of the elaborate technological activities and interactions that must be integrated over the system life cycle process is given in. The progression is iterative from left to right and not serial in nature, as might be inferred.

Figure 7.2 Here (was 2.3)

<para>Although the level of activity and detail may vary, the life-cycle functions described and illustrated are generic. They are applicable whenever a new need or changed requirement is identified, with the process being common to large as well as small-scale systems. It is essential that this process be implemented completely at an appropriate level of detail not only in the engineering of new systems but also in the re-engineering of existing or legacy systems.</para>
<para>Major technical activities performed during the design, production or construction, utilization, support, and phase-out phases of the life cycle are highlighted in. These are initiated when a new need is identified. A planning function is followed by conceptual, preliminary, and detail design activities. Producing and/or constructing the system are the function that completes the acquisition phase. System operation and support functions occur during the utilization phase of the life cycle. Phase-out and disposal are important final functions of utilization to be considered as part of design for the life cycle.
The numbered blocks in “map” and elaborate on the phases of the life cycles depicted in as follows:</para>
<para>The communication and coordination needed to design and develop the product, the production capability, the system support capability, and the relationships with interrelated systems—so that they traverse the life cycle together seamlessly—is not easy to accomplish. Progress in this area is facilitated by technologies that make more timely acquisition and use of design information possible. Computer-Aided Design (CAD) technology with internet/intranet connectivity enables a geographically dispersed multidiscipline team to collaborate effectively on complex physical designs.</para>
<para>Concern for the entire life cycle is particularly strong within the U.S. Department of Defense (DOD) and its non-U.S. counterparts. This may be attributed to the fact that acquired defense systems are owned, operated, and maintained by the DOD. This is unlike the situation most often encountered in the private sector, where the consumer or user is usually not the producer. Those private firms serving as defense contractors are obliged to design and develop in accordance with DOD directives, specifications, and standards. Because the DOD is the customer and also the user of the resulting system, considerable DOD intervention occurs during the acquisition phase.
<para>Many firms that produce for private-sector markets have chosen to design with the life cycle in mind. For example, design for energy efficiency is now common in appliances such as water heaters and air conditioners. Fuel efficiency is a required design characteristic for automobiles. Some truck manufacturers promise that life cycle maintenance costs will be within stated limits. These developments are commendable, but they do not go far enough. When the producer is not the consumer, it is less likely that potential operational problems will be addressed during development. Undesirable outcomes too often end up as problems for the user of the product instead of the producer.</para></section></section>
<para>The elaborate model of Figure 7.4 is not intended to emphasize any particular model, such as the “waterfall” model, the “spiral” model, the “vee” model, or equivalent. These well-known process models are illustrated and briefly described in , with related references to the literature found in 
	<para>The systems engineering process, and the steps illustrated in, is developed and presented and described in more detail in <link olinkend="part02" preference="0">the next chapter</inst></xref></link> of this textbook on SE process phases. The overarching objective is to describe a <emphasis>process</emphasis> model (as a frame of reference) that should be “tailored” to the specific program need.</para>

Insert Process Models Here

\subsection{Decision Models for System Evaluation}\index{Decision Models for System Evaluation}

Figure 7.3 Here (was3.26)

<section id="ch04lev2sec5" label="4.6.2"><title id="ch04lev2sec5.title"><inst>7.5.17</inst>Analytical Models and Modeling
</title>
<para>The design evaluation process may be further facilitated through the use of various analytical models, methods, and tools in support of the Macro-CAD objective. A model, in this context, is a simplified representation of the real world that abstracts features of the situation relative to the problem being analyzed. It is a tool employed by an analyst to assess the likely consequences of various alternative courses of action being examined. The model must be adapted to the problem at hand and the output must be oriented to the selected evaluation criteria. The model, in itself, is not the decision maker but is a tool that provides the necessary data in a timely manner in support of the decision-making process.</para>
<para>The extensiveness of the <emphasis>model</emphasis> will depend on the nature of the problem, the number of variables, input parameter relationships, number of alternatives being evaluated, and the complexity of operation. The ultimate objective in the selection and development of a model is simplicity and usefulness. The model used should incorporate the following features:</para>
<orderedlist numeration="arabic" spacing="normal" inheritnum="ignore" continuation="restarts"><listitem><inst>	1.	</inst><para>The model should represent the dynamics of the system configuration being evaluated in a way that is simple enough to understand and manipulate, and yet close enough to the operating reality to yield successful results.</para></listitem>
<listitem><inst>	2.	</inst><para>The model should highlight those factors that are most relevant to the problem at hand and suppress (with discretion) those that are not as important.</para></listitem>
<listitem><inst>	3.	</inst><para>The model should be comprehensive, by including <emphasis>all</emphasis> relevant factors, and be reliable in terms of repeatability of results.</para></listitem>
<listitem><inst>	4.	</inst><para>Model design should be simple enough to allow for timely implementation in problem solving. Unless the tool can be utilized in a timely and efficient manner by the analyst (or the manager), it is of little value. If the model is large and highly complex, it may be appropriate to develop a series of models where the output of one can be tied to the input of another. Also, it may be desirable to evaluate a specific element of the system independent of other elements.</para></listitem>
<listitem><inst>	5.	</inst><para>Model design should incorporate provisions for ease of modification or expansion to permit the evaluation of additional factors as required. Successful model development often includes a series of trials before the overall objective is met. Initial attempts may suggest information gaps, which are not immediately apparent and consequently may suggest beneficial changes.</para></listitem></orderedlist>
<para>The use of mathematical models offers significant benefits. In terms of system application, several considerations exist—operational considerations, design considerations, product/construction considerations, testing considerations, logistic support considerations, and recycling and disposal considerations. There are many interrelated elements that must be integrated as a system and not treated on an individual basis. The mathematical model makes it possible to deal with the problem as an entity and allows consideration of all major variables of the problem on a simultaneous basis. More specifically:</para>
<orderedlist numeration="arabic" spacing="normal" inheritnum="ignore" continuation="restarts"><listitem><inst>	1.	</inst><para>The mathematical model will uncover relations between the various aspects of a problem that are not apparent in the verbal description.</para></listitem>
<listitem><inst>	2.	</inst><para>The mathematical model enables a comparison of <emphasis>many</emphasis> possible solutions and aids in selecting the best among them rapidly and efficiently.</para></listitem>
<listitem><inst>	3.	</inst><para>The mathematical model often explains situations that have been left unexplained in the past by indicating cause-and-effect relationships.</para></listitem>
<listitem><inst>	4.	</inst><para>The mathematical model readily indicates the type of data that should be collected to deal with the problem in a quantitative manner.</para></listitem>
<listitem><inst>	5.	</inst><para>The mathematical model facilitates the prediction of future events, such as effectiveness factors, reliability and maintainability parameters, logistics requirements, and so on.</para></listitem>
<listitem><inst>	6.	</inst><para>The mathematical model aids in identifying areas of risk and uncertainty.</para></listitem></orderedlist>
<para>When analyzing a problem in terms of selecting a mathematical model for evaluation purposes, it is desirable to first investigate the tools that are currently available. If a model already exists and is proven, then it may be feasible to adopt that model. However, extreme care must be exercised to relate the right technique with the problem being addressed and to apply it to the depth necessary to provide the sensitivity required in arriving at a solution. Improper application may not provide the results desired, and the consequence may be costly.</para>
<para>Conversely, it might be necessary to construct a new model. In accomplishing this task, one should generate a comprehensive list of system parameters that will describe the situation being simulated. Next, it is necessary to develop a matrix showing parameter relationships, each parameter being analyzed with respect to every other parameter to determine the magnitude of relationship. Model input–output factors and parameter feedback relationships must be established. The model is constructed by combining the various factors and then testing it for validity. Testing is difficult to do because the problems addressed primarily deal with actions in the future that are impossible to verify. However, it may be possible to select a known system or equipment item that has been in existence for several years and exercise the model using established parameters. Data and relationships are known and can be compared with historical experience. In any event, the analyst might attempt to answer the following questions: <emphasis>Can the model describe known facts and situations sufficiently well? When major input parameters are varied, do the results remain consistent and are they realistic? Relative to system application, is the model sensitive to changes in operational requirements, design, production/construction, and logistics and maintenance support? Can cause-and-effect relationships be established?</emphasis></para>
<para>Models and appropriate systems analysis methods are presented further in <link olinkend="part03" preference="0">Part <xref olinkend="part03" label="III"><inst>III</inst></xref></link>. Trade-off and system optimization relies on these methods and their application. A fundamental knowledge of probability and statistics, economic analysis methods, modeling and optimization, simulation, queuing theory, control techniques, and the other analytical techniques is essential to accomplishing effective systems analysis.

\subsection{Analysis Models for Systems Engineering}\index{Analysis Models for Systems Engineering}

%------------------------------------------------

\section{Modeling System Functions}\index{Modeling System Functions}

Once the operational analysis has been concluded, the next step is to perform the functional analysis, which transforms the system requirements into the architecture of the system. In an interative process, the requirements are analyzed with the focus on the whats, and not on the hows. The analysis enables the identification of the functionalities that the system is to render in order to fulfill the requirements, and this allows the identification of the elements needed in the system to deliver those identified functionalities (Soles).

\subsection{Functional Analysis and Allocation}\index{Functional Analysis and Allocation}

A critical step in implementing the systems engineering process is the accomplishment of the functional analysis and the definition of the system in “functional” terms. Functions are initially identified as part of defining the need and the basic requirements for the system (, Block 0.1). System operational requirements and the maintenance concept are defined, and the functional analysis is expanded to establish a functional baseline, from which the resource requirements for the system are identified; that is, equipment, software, people, facilities, data, the various elements of maintenance and support, and so on. The functional analysis is initiated during the conceptual design phase and is described in <link olinkend="ch03lev1sec7" preference="0">Section <xref olinkend="ch03lev1sec7" label="3.7"><inst>3.7</inst></xref></link>. As design and development continues, the functional analysis is accomplished to a greater depth, to the subsystem level and below, during the preliminary system design phase, as described in <link olinkend="ch04lev1sec3" preference="0">Section <xref olinkend="ch04lev1sec3" label="4.3"><inst>4.3</inst></xref></link>. This appendix provides guidance as to the detailed steps involved in accomplishing a functional analysis and in the development of functional flow block diagrams (FFBDs).</para>
<para>Functional analysis includes the process of translating top-level system requirements into specific qualitative and quantitative design-to requirements. Given an identified need for a system, supported by the definition of system operational requirements and the maintenance concept, it is necessary to translate this information into meaningful design criteria. This translation task constitutes an iterative process of breaking down system-level requirements into successive levels of detail; a convenient mechanism for communicating this information is through the various levels of FFBDs.</para>

</title>
<para>An essential activity in early conceptual and preliminary design is the development of a <emphasis>functional</emphasis> description of the system to serve as a basis for identification of the resources necessary for the system to accomplish its mission. A <emphasis>function</emphasis> refers to a specific or discrete action (or series of actions) that is necessary to achieve a given objective; that is, an operation that the system must perform, or a maintenance action that is necessary to restore a faulty system to operational use. Such actions may ultimately be accomplished through the use of equipment, software, people, facilities, data, or various combinations thereof. However, at this point in the life cycle, the objective is to specify the <emphasis>whats</emphasis> and not the <emphasis>hows</emphasis>; that is, <emphasis>what</emphasis> needs to be accomplished versus <emphasis>how</emphasis> it is to be done.</para>
<para>The <emphasis>functional analysis</emphasis> is an iterative process of translating system requirements into detailed design criteria and the subsequent identification of the resources required for system operation and support. It includes breaking requirements at the system level down to the subsystem, and as far down the hierarchical structure as necessary to identify input design criteria and/or constraints for the various elements of the system. The purpose is to develop the top-level <emphasis>system architecture</emphasis>, which deals with both “requirements” and “structure.”</para>
<para>Referring to the iterative process in , the functional analysis actually commences (in a broad context) in conceptual design as part of the problem definition and needs analysis task (refer to ). Subsequently, “operational” and “maintenance” functions are identified, leading to the development of top-level systems requirements, as described in . The purpose of the “functional analysis” is to present an overall integrated and composite description of the system’s <emphasis>functional architecture</emphasis>, to establish a functional baseline for all subsequent design and support activities, and to provide a foundation from which all physical resource requirements are identified and justified; that is, the system’s <emphasis>physical architecture</emphasis>. A continuation of the functional analysis at the subsystem level and below is presented in , and the specific mechanics for the development of functional flow block diagrams (FFBDs) are discussed further in <link olinkend="app01" preference="0">Appendix <xref olinkend="app01" label="A"><inst>A</inst></xref></link>.</para>

Functional Flow Block Diagram. </title><para>Accomplishment of the functional analysis is facilitated through the use of <emphasis>functional flow block diagrams</emphasis> (FFBDs). The preparation of these diagrams may be accomplished through the application of any one of a number of graphical methods, including the Integrated DEFinition (IDEF) modeling method, the Behavioral Diagram method, the N-Squared Charting method, and so on. Although the graphical presentations are different, the ultimate objectives are similar. The approach assumed here is illustrated in 
, a simplified flow diagram with some decomposition is shown. Top-level functions are broken down into second-level functions, second-level functions into third-level functions, and so on, down to the level necessary to adequately describe the system and its various elements in functional terms to show the various applicable functional interface relationships and to identify the resources needed for functional implementation. Block numbers are used to show sequential and parallel relationships, initially for the purpose of providing top-down “traceability” of requirements, and later as a bottom-up “traceability” and justification of the physical resources necessary to accomplish these functions.</para>
 shows an expansion of a functional flow block diagram (FFBD), identifying a partial top level of activity, a breakdown of Function 4.0 into a top “operational” flow, and a breakdown of Function 4.2 into a top “maintenance” flow in the event that this function does not “perform” as required. Note that the words in each block are “action-oriented.” Each block represents some operational or maintenance support function that must be performed for the system to accomplish its designated mission, and there are performance measures (i.e., metrics) associated with each block that are allocated from the top. In addition, each block can be expanded (through further downward iteration) and then evaluated in terms of inputs, outputs, controls and/or constraints, and enabling mechanisms. Basically, the “mechanisms” lead to the identification of the physical resources necessary to accomplish the function, evolving from the <emphasis>whats</emphasis> to the <emphasis>hows</emphasis>. The identification of the appropriate resources in terms of equipment needs, software, people, facilities, information, data, and so on, is a result of one or more trade-off studies leading to a preferred approach as to <emphasis>how best</emphasis> to accomplish a given function.</para>
 shows an expansion of one of the functions identified in the overall basic community health care infrastructure illustrated in . The objective here is to show traceability from the definition of operational requirements in <link linkend="ch03lev1sec4" preference="0" type="backward">Section <xref linkend="ch03lev1sec4" label="3.4"><inst>8.1.4</inst></xref></link> by selecting a specific functional block from one of the example illustrations; that is, the “Community Hospital” in Illustration 5.</para>
<para>The functional analysis evolves through a series of steps illustrated in. Initially, there is a need to accomplish one or more functions . Through the definition and system operational requirements () and the maintenance and support concept (<link linkend="ch03lev1sec5" preference="0" type="backward">Section <xref linkend="ch03lev1sec5" label="3.5"><inst>8.1.5</inst></xref></link>), the required functions are further delineated, with the functional analysis providing an overall description of system requirements; that is, <emphasis>functional architecture</emphasis>. FFBDs are developed for the primary purpose of structuring these requirements by illustrating organizational and functional interfaces.</para>
<para>As one progresses through the functional analysis, and particularly when developing a new system that is within a higher-level SOS structure, there may be functions identified as “common” and shared with a different system. For example, a “transmitter” or “receiver” function may be common for two different and separate communication systems; a “power supply” function may provide the necessary power for more than one system; an “imaging center” may provide the necessary medical diagnostic services for more than one hospital and/or doctor’s office complex (refer to ), and so on. illustrates the application of several “common functions” as part of the functional breakdown for three different systems; that is, Systems <emphasis>A</emphasis>, <emphasis>B</emphasis>, and <emphasis>C.</emphasis></para>
<para>The formal functional analysis is fully initiated during the latter stages of conceptual design, and is intended to enable the completion of the system design and development process in a comprehensive and logical manner. More specifically, the functional approach helps to ensure the following:</para>
<orderedlist numeration="arabic" spacing="normal" inheritnum="ignore" continuation="restarts"><listitem><inst>	1.	</inst><para>All facets of system design and development, production, operation, support, and phase-out are considered (i.e., all significant activities within the system life cycle).</para></listitem>
<listitem><inst>	2.	</inst><para>All elements of the system are fully recognized and defined (i.e., prime equipment, spare/repair parts, test and support equipment, facilities, personnel, data, and software).</para></listitem>
<listitem><inst>	3.	</inst><para>A means is provided for relating system packaging concepts and support requirements to specific system functions (i.e., satisfying the requirements of good functional design).</para></listitem>
<listitem><inst>	4.	</inst><para>The proper sequences of activity and design relationships are established along with critical design interfaces.</para></listitem></orderedlist>
<para>Finally, it should be emphasized that the functional analysis provides the baseline from which reliability requirements (refer to and the reliability “model” in), maintainability requirements, human factors requirements, and supportability requirements are determined. Referring to, the <emphasis>functional baseline</emphasis> leads to the <emphasis>allocated baseline</emphasis>, which leads to the <emphasis>product baseline</emphasis>.</para></section>

Functional Allocation. </title><para>Given a top-level description of the system through the functional analysis, the next step is to break the system down into elements (or components) by <emphasis>partitioning</emphasis>. The challenge is to identify and group closely related functions into packages employing a common resource (e.g., an equipment item or a software package) and to accomplish multiple functions to the extent possible. Although it may be relatively easy to identify individual functional requirements and associated resources on an independent basis, this process may turn out to be rather costly when it comes to packaging system components, weight, size, and so on. The questions are as follows: <emphasis>What hardware or software (or other) can be selected that will perform multiple functions reliably, effectively, and efficiently? How can new functional requirements be added without adding new physical elements to the system structure?</emphasis></para>
<para>The partitioning of the system into elements is evolutionary in nature. Common functions may be grouped or combined to provide a system packaging scheme, with the following objectives in mind:</para>
<orderedlist numeration="arabic" spacing="normal" inheritnum="ignore" continuation="restarts"><listitem><inst>	1.	</inst><para>System elements may be grouped by geographical location, a common environment, or by similar types of items (e.g., equipment, software, data packages, etc.) that have similar functions.</para></listitem>
<listitem><inst>	2.	</inst><para>Individual system packages should be as independent as possible with a minimum of “interaction effects” with other packages. A design objective is to enable the removal and replacement of a given package without having to remove and replace other packages in the process, or requiring an extensive amount of alignment and adjustment as a result.</para></listitem>
<listitem><inst>	3.	</inst><para>In breaking a system down into subsystems, select a configuration in which the “communications” between the various different subsystems is minimized. In other words, whereas the subsystem’s <emphasis>internal</emphasis> complexity may be high, the <emphasis>external</emphasis> complexity should be low. Breaking the system down into packages requiring high rates of information exchange between these packages should be avoided.</para></listitem>
<listitem><inst>	4.	</inst><para>An objective is to pursue an <emphasis>open-architecture</emphasis> approach in system design. This includes the application of common and standard modules with well-defined standard interfaces, grouped in such a way as to allow for system upgrade modifications without destroying the overall functionality of the system.</para></listitem></orderedlist>

<para role="continued">An overall design objective is to break the system down into elements such that only a few critical events can influence or change the inner workings of the various packages that make up the system architecture.</para>
<para>Through the process of partitioning and functional packaging, trade-off studies are conducted in evaluating the different design approaches that can be followed in responding to a given functional requirement. It may be appropriate to perform a designated function through the use of equipment, software, people, facilities, data, and/or various combinations thereof. The proper mix is established, and the result may take the form of a system structure similar to the example presented in. With this structure representing the proposed system “make-up,” the next step is to determine the design-to requirements for each of the system elements; that is, system operator, Equipment 123, Unit <emphasis>B</emphasis>, computer resources, and facilities.</para>
<para>Referring to (and ), specific quantitative design-to requirements have been established at the top level for the <emphasis>system</emphasis>. These TPMs, which evolved from the definition of operational requirements and the maintenance support concept, must be <emphasis>allocated</emphasis> or <emphasis>apportioned</emphasis> down to the appropriate subsystems or the elements that make up the system. For instance, given an operational availability Ao requirement of 0.985 for the system, <emphasis>what design requirements should be specified for the equipment, software, facility, and operator such that, when combined, they will meet the availability requirements for the overall system?</emphasis> In other words, to guarantee an ultimate system design configuration that will meet all customer (user) requirements, there must be a top-down allocation of design criteria from the beginning; that is, during the latter stages of conceptual design. The allocation process starts at this point and continues with the appropriate subsystems and below as required. presents a case-study example of allocation from the system to next level below, and the process is expanded for the allocation of reliability requirements in , maintainability requirements in and so on.

\subsection{Functional Flow Block Diagrams}\index{Functional Flow Block Diagrams}

Functional flow block diagrams (FFBDs) are models developed to describe the system and its elements in functional terms. These diagrams reflect both operational and support activities as they occur throughout the system life cycle, and they are structured in a manner that illustrates the hierarchical aspects of the system (see . Some of the key features of the overall functional flow process are noted as follows:</para>
<orderedlist numeration="arabic" spacing="normal" inheritnum="ignore" continuation="restarts"><listitem><inst>1.	</inst><para>The functional block diagram approach should include coverage of all activities throughout the system life cycle, and the method of presentation should reflect proper activity sequences and interface interrelationships.</para></listitem>
<listitem><inst>2.	</inst><para>The information included within the functional blocks should be concerned with <emphasis>what</emphasis> is required before looking at <emphasis>how</emphasis> it should be accomplished.</para></listitem>
<listitem><inst>3.	</inst><para>The process should be flexible to allow for expansion if additional definition is required or reduction if too much detail is presented. The objective is to progressively and systematically work down to the level where resources can be identified with how a task should be accomplished (refer to 
<para role="continued">In the development of functional flow diagrams, some degree of standardization is necessary (for communication) in defining the system. Thus, certain basic practices and symbols should be used, whenever possible, in the physical layout of functional diagrams. The paragraphs below provide some guidance in this direction.</para>
<orderedlist numeration="arabic" spacing="normal" inheritnum="ignore" continuation="restarts"><listitem><inst>1.	</inst><para><emphasis>Function block.</emphasis> Each separate function in a functional diagram should be presented in a single box enclosed by a solid line. Blocks used for reference to other flows should be indicated as partially enclosed boxes labeled “Ref.” Each function may be as gross or detailed as required by the level of functional diagram on which it appears, but it should stand for a definite, finite, discrete action to be accomplished by equipment, personnel, facilities, software, or any combination thereof. Questionable or tentative functions should be enclosed in dotted blocks.</para></listitem>
<listitem><inst>2.	</inst><para><emphasis>Function numbering.</emphasis> Functions identified in the functional flow diagrams at each level should be numbered in a manner which preserves the continuity of functions and provides information with respect to function origin throughout the system. Functions on the top-level functional diagram should be numbered 1.0, 2.0, 3.0, and so on. Functions which further indenture these top functions should contain the same parent identifier and should be coded at the next decimal level for each indenture. For example, the first indenture of function 3.0 would be 3.1, the second 3.1.1, the third 3.1.1.1, and so on. For expansion of a higher-level function within a particular level of indenture, a numerical sequence should be used to preserve the continuity of the function. For example, if more than one function is required to amplify function 3.0 at the first level of indenture, the sequence should be 3.1, 3.2, 3.3, ..., 3.<emphasis>n</emphasis>. For expansion of function 3.3 at the second level, the numbering shall be 3.3.1, 3.3.2, ..., 3.3.<emphasis>n</emphasis>. Where several levels of indentures appear in a single functional diagram, the same pattern should be maintained. While the basic ground rule should be to maintain a minimum level of indentures in any one particular flow, it may become necessary to include several levels to preserve the continuity of functions and to minimize the number of flows required to functionally depict the system.</para></listitem>
<listitem><inst>3.	</inst><para><emphasis>Functional reference.</emphasis> Each functional diagram should contain a reference to its next higher functional diagram through the use of a reference block. For example, function 4.3 should be shown as a reference block in the case where the functions 4.3.1, 4.3.2, . . . , 4.3.<emphasis>n</emphasis>, and so on, are being used to expand function 4.3. Reference blocks shall also be used to indicate interfacing functions as appropriate.</para></listitem>
<listitem><inst>4.	</inst><para><emphasis>Flow connection.</emphasis> Lines connecting functions should indicate only the functional flow and should not represent either a lapse in time or any intermediate activity. Vertical and horizontal lines between blocks should indicate that all functions so interrelated must be performed in either a parallel or a series sequence. Diagonal lines may be used to indicate alternative sequences (cases where alternative paths lead to the next function in the sequence).</para></listitem>
<listitem><inst>	5.	</inst><para><emphasis>Flow directions.</emphasis> Functional diagrams should be laid out so that the functional flow is generally from left to right and the reverse flow, in the case of a feedback functional loop, from right to left. Primary input lines should enter the function block from the left side; the primary output, or go line, should exit from the right, and the no-go line should exit from the bottom of the box.</para></listitem>
<listitem><inst>	6.	</inst><para><emphasis>Summing gates.</emphasis> A circle should be used to depict a summing gate. As in the case of functional blocks, lines should enter or exit the summing gate as appropriate. The summing gate is used to indicate the convergence, divergence, parallel, or alternative functional paths and is annotated with the term AND or OR. The term AND is used to indicate that parallel functions leading into the gate must be accomplished before proceeding to the next function, or that paths emerging from the AND gate must be accomplished after the preceding functions. The term OR is used to indicate that any of the several alternative paths (alternative functions) converge to, or diverge from, the OR gate. The OR gate thus indicates that alternative paths may lead or follow a particular function.</para></listitem>
<listitem><inst>	7.	</inst><para><emphasis>Go and no-go paths</emphasis>. The symbols G and <inlineequation id="app01ie01"><inlinemediaobject><textobject role="xpressmath"></textobject></inlinemediaobject></inlineequation> are used to indicate go and no-go paths, respectively. The symbols are entered adjacent to the lines leaving a particular function to indicate alternative functional paths.</para></listitem>
<listitem><inst>	8.	</inst><para><emphasis>Numbering procedure for changes</emphasis>. Additions of functions to existing data should be accomplished by locating the new function in its correct position without regard to sequence of numbering. The new function should be numbered using the first unused number at the level of indenture appropriate for the new function.

\subsection{Some Example Application}\index{Some Example Application}

With the objective of illustrating how some of these general guidelines are employed, are included to present a few simple applications.</para>

<para role="continued">Although these sample block diagrams do not cover the selected systems entirely, it is hoped that the material is presented in enough detail to provide an appropriate level of guidance for the development of functional block diagrams.

%------------------------------------------------

\section{Modeling Concurrent Engineering}\index{Modeling Concurrent Engineering}

%------------------------------------------------

\section{Functional Models and Analysis}\index{Functional Models and Analysis}

%------------------------------------------------

\section{Methods for Modeling Requirements}\index{Methods for Modeling Requirements}


The Requirements Framework

Transforming the need or opportunity into a system that satisfies it is achieved by achieved by applying the systems engineering framework. Nevertheless, there is no single model. Several models have been proposed, each one with its own pros and cons. The systems engineer needs to know the different roads or methods available, their advantages and drawbacks, and select in each case the most appropriate one. Each of the systems engineering models or frameworks can potentially deliver the desired results (namely, the design of the required system); in any case, what is important is not to lose sight of the goal sought, and to rely on timely feedback in order to alter the course of action when and as needed. The systems process has been described by a number of authors [Blanchard, 1991; Sage, 2000; Kossiakoff and Sweet, 2003; Sage and Rouse, 2009].

ABSTRACT: Systems and software qualities (SQs) are also known as non-functional requirements (NFRs). Where functional requirements (FRs) specify what a system should do, the NFRs specify how well the system should do them. Many of them, such as Reliability, Availability, Maintainability, Usability, Affordability, Interoperability, and Adaptability, are often called “ilities,” but not to the exclusion of other SQs such as Safety, Security, Resilience, Robustness, Accuracy, and Speed.
In 2012, the US Department of Defense (DoD) identified seven Critical Technology Areas needing emphasis in its technology investments. One of them was called Engineered Resilient Systems (ERS). The SERC sponsor, the DoD Undersecretary for Systems Engineering, and the lead ERS research organization, the Army Engineering Research Center (ERC), held two workshops to explore what research was being addressed, and how the SERC could complement it. It turned out that the existing ERS research underway was primarily directed at field testing, supercomputer modeling, and resilient design of physical systems, and that the SERC could best complement this research by addressing the resilient design of cyber-physical-human (CPH) systems, Some of the SERC universities were performing such research, such as AFIT, Georgia Tech, MIT, NPS, Penn State, USC, U. Virginia, and Wayne State. These universities have been addressing aspects of this research area as a team since 2013.
Initially, the team found a veritable quagmire of SQ definitions and relationships. For example, looking up “resilience” in Wikipedia, the team found over 20 different definitions of “resilience,” with over 10 different definitions of a system’s post-resilient state. The leading standard in the area, ISC/IEC 25010, had weak and inconsistent definitions of the qualities. For example, it defined Reliability with respect to the satisfaction of a system’s functional requirements, but not its quality requirements. Some of the SERC universities had developed partial ontologies of the SQs, and exploration of alternative ontology structures identified found one that addressed not only the inter-quality relationships, but also their sources of value variation.
BIO:Dr. Barry Boehm received his B.A. degree from Harvard in 1957, and his M.S. and Ph.D. degrees from UCLA in 1961 and 1964, all in Mathematics. He has also received honorary Sc.D. in Computer Science from the U. of Massachusetts in 2000 and in Software Engineering from the Chinese Academy of Sciences in 2011.

\subsection{Functional and Non\-Functional Requirements}\index{Functional and Non\-Functional Requirements}

A second way for classifying requirements is by dividing them into functional and non-functional ones. Functional requirements are those that express what the system is to be able to do or perform, whereas non-functional are those that define overall qualities or attributes of the system of or a relevant part of it (like a subsystem). Functional requirements describe what the system is to do, while the non-functional requirements set conditions or constraints on how the functional ones are to be implemented. For example, a functional requirement for a cellular phone may be the hours of autonomy of its battery; a non-functional requirement may be one stating its expected useful life (a reliability-related requirement). One thing is what the system can perform (the functional requirements). Typical non-functional requirements are related to safety, security, availability, reliability, maintainability, supportability, interfaces, system acceptance, and the like. Non-functional requirements can be divided into the following three groups:
    a. Product requirements. These specify characteristics or properties that a system, or a part of it, has to have. For example, if the need is to have a means for transporting a certain cargo and the solution is going to be a special truck, a couple of non-functional requirements could be for the truck to have a certain operational availability and for its power pack (engine and transmission) to have a certain mission reliability.
    b. Process requirements. These relate to the processes associated with the design, development, operation and/or support of the system. Examples could be meeting a certain standard, using a given tool, or following an indicated method or procedure. For example, in the Eurofighter program the Air Forces of the four nations involved (Germany, Italy, Spain, and the United Kingdom) decided to exchange information electronically and consequently new logistics information systems had to be developed. It was decided that those systems would be developed in compliance with standard AECMA 2000M International Specification for Material Management, which regulates the exchange of material support data in electronic form.
    c. External requirements. These are derived from the environment (in the broadest sense of the term) in which the system will be deployed. For example, the need to interface and work together with other systems is an external requirement, as it is also the need for a certain fault tolerance capability if the environment in which the system is intended to operate will make access for maintenance very difficult and/or expensive.
    
\subsection{Characteristics of Requirements}\index{Characteristics of Requirements}

There are a number of essential characteristics or abilities that the requirements are to have in order for them to truly constitute the needed solid foundation on which to design and develop the system desired. Each individual requirement has to have certain abilities that will be somewhat different depending on whether they are stakeholder or system requirements. Furthermore, the entire set of requirements needs to meet some characteristics, too. In a nutshell, requirements have their own requirements. Too many projects derail or end up in serious trouble because requirements have not been well defined and fail to translate in an accurate manner the perceived need or opportunity. Such distortion in the translation contaminates all subsequent steps in the process. The likelihood of having the requirements well written and formulated from the beginning increases if the systems engineer has a clear understanding of the needed characteristics of the requirements, of each one individually and of the entire set of them.

Characteristics of Individual Requirements. Each individual stakeholder or system requirement is to be:
Needed. Seems trivial, but it is not. Every requirement poses constraints and demands on the system and consequently has potential impacts on aspects, such as performance or cost. It is therefore important to ensure that every requirement is really needed and to avoid the mistake of requesting things that are not needed. Customers and other stakeholders are generally little aware of the broad and deep implications of their requirements and, if they had better visibility of those consequences, they would probably waive a number of them. Unneeded requirements will take a toll on the project, be it in terms of performance, cost and/or schedule. For the set of requirements, more is not necessarily better. Actually, the effect of unnecessary requirements can be as bad, or even worse, than that of missed requirements. The need or opportunity is to be accurately translated into requirements can be as bad or even worse, than that of missed requirements, and that is what really matters. A requirement that is not needed reduces the solutions space, being it then more difficult to optimize the solution for the customer and the stakeholders.
Atomic. Only one needed functionality or property per requirement; that is, requirements are not to be compounded. If several aspects need to be covered, then it is best to split them and address each one in a separate requirement. This implies that the use of conjunctions (and, or, and the like) is to be avoided. It is in general more difficult to verify the fulfillment of non-atomic requirements and problems arise easily when there is evidence that part of a requirement has been met, while others have not. This complicates the verification effort, which is highly facilitated when requirements are atomic.
Unique. Each needed system characteristic or functionality is to be stated only once, in a single requirement. When things are repeated in several requirements nothing positive is added and there is a large risk of conflict if the different wordings do not really convey the same message. Even if perfectly consistent, duplications or multiple repetitions make it more complicated for the designer to grasp the entire picture. Later, if a change to a requirement is proposed and accepted, it would be necessary to amend all others expressing the same need so that their wordings remain aligned. Consistency problems will easily surface if requirements are not unique.
Positive. The requirement has to state what functionality is needed and not what is not to be present. Negative statements are more likely to be misinterpreted, especially when there are several negative statements embedded in the requirement. Positive statements are always easier to understand.
Objective. A requirement is not to be open to interpretation. Fuzziness is to be avoided; subjectivity is not desirable as it only increases the chances for reworks and disputes. Furthermore, the likelihood of different interpretations by different parties may lead to a design that eventually will not satisfy the user; that is, subjective requirements increase the risk of eventual problems in the validation of the solution and should thus be avoided.
    a. Understandable. Goes without saying, but the purpose of a requirement is to translate the perception of a need or opportunity in such a way that it can be communicated to, and understood by, the recipient party. In some instances the recipient party may be the stakeholders themselves, who are sent the requirements seeking confirmation or amendment as necessary. In some other cases, the recipient party may be the designers, who will take the requirements to perform the functional analysis. If a requirement is not understood, it does not serve any purpose.
    b. Correct. Apparently another obvious thing, but frequently overlooked. Requirements are to depict needed functionalities or properties of the system. Should a requirement be incorrect and demand something that is not needed, it will imply either unnecessary costs to the end user, or pose a conflict with other truly-needed requirements, or complicate unnecessarily the design by introducing gratuitous or superfluous constraints. A requirement that is not correct displaces the solution space, increasing thus the risk of not fulfilling the needs of expectations of customer and stakeholders.
    c. Concise. Requirements should state simply and directly what is needed, avoiding superfluous details and unnecessary information that would make it more difficult to be understood.
    d. Traceable. It is essential to know who is behind each requirement for a number of reasons. It may be necessary to seek confirmation to the wording of a requirement of to its interpretation, or it could be the case that a change is needed and the originator and owner of the requirement is to be approached, or that evidence on the verification of the fulfillment of the requirement is to be furnished. Knowing which stakeholder originated which requirement is thus a sine qua non condition. Traceability is to be done forward and backward, from stakeholder to actually physical characteristic in the system and back.
    e. Prioritizable. Not all requirements need to have the same relevance or importance to the end customer and to the rest of stakeholders. In principle there should not be any conflicts among the requirements, but even if that is the case there is always the possibility that some requirements will only be completely fulfilled at the expense of others not being so. That means that requirements should be prioritized; that is, that they should be classified or arranged as per their importance in order to facilitate the decisions to be taken whenever unsolvable conflicts show up. It is typical to classify requirements in three categories: ‘must’, ‘desirable’, and ‘nice to have’. The first group includes those that are absolutely mandatory. The second are the requirements that should definitely be met unless they pose a conflict with one of the ‘must’ group. Finally the third group integrates those other requirements that are desirable but only as long as they do not jeopardize the fulfillment of any other requirement in the two other groups. In any case the requirements priority taxonomy or classification, in whatever number of levels, has to be decided for each particular project and agreed upon with the stakeholders.
    f. Solution-independent. The requirements are to be stated without any prejudice or bias as to what the solution might be. The important thing is that they reflect what functionalities and characteristics are required, not how they will be implemented and rendered by the system. A common mistake is to state requirements for a certain type of solution simply because the customer is familiar with it. Doing this will constitute an unnecessary constraint that will reduce the freedom of the systems engineer to find the best solution to the problem at stake.
In addition to the above, each system requirement is to be:
    a. Concept-dependent. Once the list of stakeholder requirements is considered to the complete and is validated in terms of reflecting the perceived need or opportunity, the next step in the systems engineering framework is to identify all (or as many as possible) ways for solving it. Each potential solution is a design concept and, out of the many that are identified, the best one will be selected. All stakeholder requirements will then be translated to the chosen design concept in a one-to-many relationship as portrayed in figure 5.3.
    b. Verifiable. There is no point in stating a requirement if its fulfillment cannot be objectively confirmed, and yet many projects are plagued with requirements for which an objective verification is not specified. Whenever a functionality is required by using a standard concept or metric there is no further need to state how its fulfillment will be verified. During pre-production some models or prototypes may enable physical measurements. During serial production other more accurate measurements will be feasible, and during the operational life the true characteristics of the real system, once fielded and in operation by the user, may be measured. The relationship of system requirements and verification methods is thus, in general, one-to-many (refer again to figure 5.2).
    c. Feasible. Upfront, it may be difficult to know if a particular requirement is attainable or not. To further complicate things it could be that some requirements can only be achieved if others are not, at least to their full extent. Such case would mean a certain lack of coherence or consistency, a property that the entire set of requirements is to have, as explained in detail in the next subsection. Yet, it is highly desirable that all accepted requirements be considered, in principle, achievable; otherwise, they should be reconsidered. Given the subjectivity of this characteristic, it is essential to conduct a thorough risk management program that considers the effect in the end system of the potential lack of achievement of certain requirements (for example, due to a certain achievement of certain technology not reaching a needed level of maturity, resulting in a certain requirement being partially unattainable).
    d. Modifiable. Although system requirements are to be feasible, it may be that the design effort eventually proves the contrary. The appropriate thing is to go back to the stakeholder that generated the stakeholder requirement from which the system requirement was derived, in order to modify it as appropriate. This means that desirably all stakeholders formulate their requirements with some latitude for amendments or modification, should it prove necessary.
Figure 5.5 summarizes the characteristics that the individual requirements need to have, whether at the stakeholder level or at the system level. Failure of the requirements to meet these characteristics will disrupt the subsequent design and development effort. The needs perceived by the user are always legitimate, but too frequently they are wrongly stated in form of stakeholder requirements. As important as detecting a need or opportunity is to translate it into the right set of well-written stakeholder requirements.
(Figure 5.5. Characteristics of the individual requirements.)
	Ill-defined stakeholder and/or system requirements almost inevitably spell out the lack of customer satisfaction with the design and developed system.

Characteristics of the Set of Requirements. 	If each requirement is to meet a number of characteristics of abilities, as described in the previous section, the entire set is to be:
    a. Complete. The identified need or the perceived opportunity is to be fully explained through the set of requirements. If any required functionalities are not reflected, then the system will most likely not truly fulfill its purpose. This has three implications. First, it means that all stakeholders are to be considered because if any one is not considered and involved, it will then follow that all requirements coming from it will be missed. Second, it is essential that each stakeholder addresses its specific needs or demands regarding the system and expresses it in a complete set of requirements. The requirements coming from the different stakeholders will have to be merged; redundancies will have to be eliminated and potential conflicts will have to be sorted out. Third and final, every stakeholder requirement will have to be completely translated into the appropriate array of system requirements for the selected design concept. Failure to do so will mean that the input received by the designers will not truly reflect what the end users and the rest of the stakeholders need.
    b. Coherent. As stated in the previous characteristic, no conflicts or contradictions are to be present in the set of requirements at any level (stakeholder or system), which is the same as to say that the set has to be consistent.
    c. Structured. A well-organized set of requirements will be easier to understand than a poorly-structured one. As there will be requirements covering many areas or aspects, it is mandatory that they are organized in a logical manner that facilitates their understanding and the grasping of the global picture.
    d. Non-redundant. As it was also mentioned when addressing the need for the set of requirements to be complete, it is clear that it should not contain any repetitions. Each aspect is to be covered just once.

%------------------------------------------------

\section{Technical Performance Measures}\index{Technical Performance Measures}

As defined in, <emphasis>technical performance measures</emphasis> (TPMs) are quantitative values (estimated, predicted, and/or measured) that describe system performance. TPMs are measures of the attributes and/or characteristics that are inherent within the design. <emphasis>Design dependent parameters</emphasis> (DDPs) are quantified and are the bases for the determination of TPMs. During conceptual design, these values are best determined by parametric methods.

%------------------------------------------------

\section{Systems Engineering Process Models}\index{Systems Engineering Process Models}

Although there is general agreement regarding the principles and objectives of systems engineering, its actual implementation will vary from one system and engineering endeavor to the next. The process approach and steps used will depend on the nature of the system application and the backgrounds and experiences of the individuals on the team.
	To establish a common frame of reference for improving communication and understanding, it is important that a “baseline” be defined that describes the systems engineering process, along with the essential life-cycle phases and steps within that process. Augmenting this common frame of reference are top-down and bottom-up approaches. And, there are other process models that have attracted various degrees of attention. Each of these topics is presented in this section.</para>
 illustrates the major life-cycle process phases and selected milestones for a generic system. This is the “model” that will serve as a frame of reference for material presented in subsequent chapters. Included are the basic steps in the systems engineering process (i.e., requirements analysis, functional analysis and allocation, synthesis, trade-off studies, design evaluation, and so on).

<para>A newly identified need, or an evolving need, reveals a new system requirement. If a decision is made to seek a solution for the need, then a decision is needed whether to consider other needs in designing the solution. Based on an initial determination regarding the scope of needs, the basic phases of conceptual design and onward through system retirement and phase-out are then applicable, as described in the paragraphs that follow. The scope of needs may contract or expand, but the scope should be stabilized as early as possible during conceptual design, preferably based on an evaluation of value and cost by the customer.</para>
<para>Program phases described in are not intended to convey specific tasks, or time periods, or levels of funding, or numbers of iterations. Individual program requirements will vary from one application to the next. The figure exhibits an overall <emphasis>process</emphasis> that needs to be followed during system acquisition and deployment. Regardless of the type, size, and complexity of the system, there is a conceptual design requirement (i.e., to include requirements analysis), a preliminary design requirement, and so on. Also, to ensure maximum effectiveness, the concepts presented in must be properly “tailored” to the particular system application being addressed.</para>
shows the basic steps in the systems engineering process to be iterative in nature, providing a top-down definition of the system, and then proceeding down to the subsystem level (and below as necessary). Focused on the needs, and beginning with conceptual design, the completion of Block 0.2 defines the system in <emphasis>functional</emphasis> terms (having identified the “whats” from a requirements perspective). These “whats” are translated into an applicable set of “hows” through the iterative process of functional partitioning and requirements allocation, together with conceptual design synthesis, analysis, and evaluation. This conceptual design phase is where the initial configuration of the system (or system architecture) is defined.</para>
<para>During preliminary design, completion of Block 1.1 defines the system in <emphasis>refined functional</emphasis> terms providing a top-down definition of subsystems with preparation for moving down to the component level. Here the “whats” are extracted from (provided by) the conceptual design phase. These “whats” are translated into an applicable set of “hows” through the iterative process of functional partitioning and requirements allocation, together with preliminary design synthesis, analysis, and evaluation. This preliminary design phase is where the initial configuration of subsystems (or subsystem architecture) is defined.</para>
<para>Blocks 1.1–1.7 are an evolution from Blocks 0.1–0.8, Blocks 6.1–6.5 are an evolution from Blocks 1.1–1.7, and Blocks 3.1–3.6 are an evolution from Blocks 6.1–6.5. The overall process reflected in the figure constitutes an evolutionary design and development process. With appropriate feedback and design refinement provisions incorporated, the process should eventually converge to a successful design. The functional definition of the system, its subsystems, and its components serves as the baseline for the identification of resource requirements for production and then operational use (i.e., hardware, software, people, facilities, data, elements of support, or a combination thereof).</para></section>
</title>
	<para>The systems engineering process, and the steps illustrated in 7.X is now developed and presented and described in more detail. The overarching objective is to describe a <emphasis>process</emphasis> (as a frame of reference) that should be “tailored” to the specific program need.</para>
<para> 7.X7.X is not intended to emphasize any particular model, such as the “waterfall” model, the “spiral” model, the “vee” model, or equivalent. These well-known process models are illustrated and briefly described in, with related references to the literature found in
Figure 7.5 sheet 1 and sheet 2 HERE</para></section></section>

THEN DO MBSE - Use the NSF Designers Unification

%------------------------------------------------

\section{Systems Engineering Process Models}\index{Systems Engineering Process Models}

System Life-Cycle Models

<para>Experience over many decades indicates that a properly functioning system that is effective and economically competitive cannot be achieved through efforts applied largely after it comes into being. Accordingly, it is essential that anticipated outcomes during, as well as after, system utilization be considered during the early stages of design and development. Responsibility for <emphasis>life-cycle engineering,</emphasis> largely neglected in the past, must become the central engineering focus.

</para><section id="ch02lev2sec5" label="2.2.1"><title id="ch02lev2sec5.title"><inst></inst>The Product and System Life Cycles
</title>
<para>Fundamental to the application of systems engineering is an understanding of the life-cycle process, illustrated for the system in . The product life cycle begins with the identification of a need and extends through conceptual and preliminary design, detail design and development, production or construction, distribution, utilization, support, phase-out, and disposal. The life-cycle phases are classified as <emphasis>acquisition</emphasis> and <emphasis>utilization</emphasis> to recognize producer and customer activities.

\subsection{The Systems Engineering Process}\index{The Systems Engineering Process}

Although there is general agreement regarding the principles and objectives of systems engineering, its actual implementation will vary from one system and engineering team to the next. The process approach and steps used will depend on the nature of the system application and the backgrounds and experiences of the individuals on the team. To establish a common frame of reference for improving communication and understanding, it is important that a “baseline” be defined that describes the systems engineering process, along with the essential life-cycle phases and steps within that process. Augmenting this common frame of reference are top-down and bottom-up approaches. And, there are other process models that have attracted various degrees of attention. Each of these topics is presented in this section.
</para>
Model Based Systems Engineering

Model based systems engineering has process model (not decision model) characteristics. In the INCOSE vision 2020 document, v Model Based Systems Engineering (MBSE) is the formalized application of modeling to support the SEP. It brings together, normally on a computer-based platform, the information to support functional analysis and allocation, requirements management and decomposition, synthesis, analysis, verification and validation. Figures 6.4 and 6.5. provide the essential framework for the MBSE model.

Figure 6.4 HERE
Figure 6.5 HERE
	
Specifically, revisit the information flows in Figure 6.5 in light of the abstract regarding the “On line linking of Individual Designers Decisions . . . FIND IT

%------------------------------------------------

\section{Schematic Models for the Systems Engineering Process}\index{Schematic Models for the Systems Engineering Process}

System Life-Cycle Models
<para>Experience over many decades indicates that a properly functioning system that is effective and economically competitive cannot be achieved through efforts applied largely after it comes into being. Accordingly, it is essential that anticipated outcomes during, as well as after, system utilization be considered during the early stages of design and development. Responsibility for <emphasis>life-cycle engineering,</emphasis> largely neglected in the past, must become the central engineering focus.</para>
<section id="ch02lev2sec5" label="2.2.1"><title id="ch02lev2sec5.title"><inst></inst>The Product and System Life Cycles</title>
<para>Fundamental to the application of systems engineering is an understanding of the life-cycle process, illustrated for the system in . The product life cycle begins with the identification of a need and extends through conceptual and preliminary design, detail design and development, production or construction, distribution, utilization, support, phase-out, and disposal. The life cycle phases are classified as <emphasis>acquisition</emphasis> and <emphasis>utilization</emphasis> to recognize producer and customer activities.

Figure 6.1 Here
<para>System life cycle engineering goes beyond the product life cycle viewed in isolation. It must simultaneously embrace the life cycle of the production or construction subsystem, the life cycle of the maintenance and support subsystem, and the life cycle for retirement, phase-out, reuse, and disposal as another subsystem. The overall system is made up of four concurrent life cycles progressing in parallel, as is illustrated in . This conceptualization is the basis g. for <emphasis>concurrent engineering</emphasis>.

<para>The need for the product comes into focus first. This recognition initiates conceptual design to meet the need. Then, during conceptual design of the product, consideration should simultaneously be given to its production. This gives rise to a parallel life cycle for a production and/or construction capability. Many producer-related activities are needed to prepare for the production of a product, whether the production capability is a manufacturing plant, construction contractors, or a service activity.</para>
<para>Also shown in is another life cycle of considerable importance that is often neglected until product and production design is completed. This is the life cycle for support activities, including the maintenance, logistic support, and technical skills needed to service the product during use, to support the production capability during its duty cycle, and to maintain the viability of the entire system. Logistic, maintenance, technical support, and regeneration requirements planning should begin during product conceptual design in a coordinated manner.</para>
<para>As each of the life cycles is considered, design features should be integrated to facilitate phase out, regeneration, or retirement having minimal impact on interrelated systems. For example, attention to end-of-life recyclability, reusability, and disposability will contribute to environmental sustainability. Also, the system should be made ready for regeneration by anticipating and addressing changes in requirements, such as increases in complexity, incorporation of planned new technology, likely new regulations, market expansion, and others.</para>
<para>In addition, the interactions between the product and system and any related systems should begin receiving compatibility attention during conceptual design to minimize the need for product and system redesign. Whether the interrelated system is a companion product sold by the same company, an environmental system that may be degraded, or a computer system on which a software product runs, the relationship with the product and system under development must be engineered concurrently.</para></section>
Figure 6.2 sheet 1 and sheet 2 HERE</para></section></section>

<section id="ch02lev2sec6" label="2.2.2"><title id="ch02lev2sec6.title"><inst></inst>Designing for the Life Cycle</title>
<para>Design within the system life-cycle context differs from design in the ordinary sense. Life-cycle-guided design is simultaneously responsive to customer needs (i.e., to requirements expressed in functional terms) and to life-cycle outcomes. Design should not only transform a need into a system configuration but should also ensure the design’s compatibility with related physical and functional requirements. Further, it should consider operational outcomes expressed as producibility, reliability, maintainability, usability, supportability, serviceability, disposability, sustainability, and others, in addition to performance, effectiveness, and affordability.</para>
<para>A detailed presentation of the elaborate technological activities and interactions that must be integrated over the system life-cycle process is given in . The progression is iterative from left to right and not serial in nature, as might be inferred.</para>
<para>Although the level of activity and detail may vary, the life-cycle functions described and illustrated are generic. They are applicable whenever a new need or changed requirement is identified, with the process being common to large as well as small-scale systems. It is essential that this process be implemented completely—at an appropriate level of detail—not only in the engineering of new systems but also in the re-engineering of existing or legacy systems.</para>
<para>Major technical activities performed during the design, production or construction, utilization, support, and phase-out phases of the life cycle are highlighted in. These are initiated when a new need is identified. A planning function is followed by conceptual, preliminary, and detail design activities. Producing and/or constructing the system are the function that completes the acquisition phase. System operation and support functions occur during the utilization phase of the life cycle. Phase-out and disposal are important final functions of utilization to be considered as part of design for the life cycle.</para>
<para>The numbered blocks in “map” and elaborate on the phases of the life cycles depicted in <as follows:</para>
<para>The communication and coordination needed to design and develop the product, the production capability, the system support capability, and the relationships with interrelated systems—so that they traverse the life cycle together seamlessly—is not easy to accomplish. Progress in this area is facilitated by technologies that make more timely acquisition and use of design information possible. Computer-Aided Design (CAD) technology with internet/intranet connectivity enables a geographically dispersed multidiscipline team to collaborate effectively on complex physical designs.</para>
<para>For certain products, the addition of Computer-Aided Manufacturing (CAM) software can automatically translate approved three-dimensional CAD drawings into manufacturing instructions for numerically controlled equipment. Generic or custom parametric CAD software can facilitate exploration of alternative design solutions. Once a design has been created in CAD/CAM, iterative improvements to the design are relatively easy to make. The CAD drawings also facilitate maintenance, technical support, regeneration (re-engineering), and disposal. A broad range of other electronic communication and collaboration tools can help integrate relevant geographically dispersed design and development activities over the life cycle of the system.</para>
<para>Concern for the entire life cycle is particularly strong within the U.S. Department of Defense (DOD) and its non-U.S. counterparts. This may be attributed to the fact that acquired defense systems are owned, operated, and maintained by the DOD. This is unlike the situation most often encountered in the private sector, where the consumer or user is usually not the producer. Those private firms serving as defense contractors are obliged to design and develop in accordance with DOD directives, specifications, and standards. Because the DOD is the customer and also the user of the resulting system, considerable DOD intervention occurs during the acquisition phase.<footnoteref preference="1" label="5" role="generated" linkend="ch02fn05"/>3</para>
<para>Many firms that produce for private-sector markets have chosen to design with the life cycle in mind. For example, design for energy efficiency is now common in appliances such as water heaters and air conditioners. Fuel efficiency is a required design characteristic for automobiles. Some truck manufacturers promise that life-cycle maintenance costs will be within stated limits. These developments are commendable, but they do not go far enough. When the producer is not the consumer, it is less likely that potential operational problems will be addressed during development. Undesirable outcomes too often end up as problems for the user of the product instead of the producer.

%------------------------------------------------

\section{The Systems Engineering Process}\index{The Systems Engineering Process}

Although there is general agreement regarding the principles and objectives of systems engineering, its actual implementation will vary from one system and engineering team to the next. The process approach and steps used will depend on the nature of the system application and the backgrounds and experiences of the individuals on the team. To establish a common frame of reference for improving communication and understanding, it is important that a “baseline” be defined that describes the systems engineering process, along with the essential life-cycle phases and steps within that process. Augmenting this common frame of reference are top-down and bottom-up approaches. And, there are other process models that have attracted various degrees of attention. Each of these topics is presented in this section.

\subsection{Development of Engineering Models}\index{Development of Engineering Models}

As the system design and development effort progresses, the basic process evolves from a description of the design in the form of drawings, documentation, and databases to the construction of a physical model or mock-up, to the construction of an engineering or laboratory model, to the construction of a prototype, and ultimately to the production of a final product. The purpose in proceeding through these steps is to provide a solid basis for design evaluation and/or validation. The earlier in the design process that one can accomplish this purpose, the better, since the incorporation of any necessary changes for corrective action will be more costly and disruptive later when the design progresses toward the production/construction phase.</para>
<para>The first two steps in this development sequence are discussed in the previous sections: <link linkend="ch05lev1sec4" preference="0" type="backward">Section <xref linkend="ch05lev1sec4" label="5.4"><inst>8.3.4</inst></xref></link> discusses the development of a mock-up and <link linkend="ch05lev1sec5" preference="0" type="backward">Section <xref linkend="ch05lev1sec5" label="5.5"><inst>8.3.5</inst></xref></link> discusses the development of data and documentation requirements. At this stage, the results have not produced an actual “working” model of the system.</para>
<para>At some point in the detail design and development phase, it may be appropriate to produce an <emphasis>engineering model</emphasis> (or a laboratory model). The objective is to demonstrate some (if not all) of the functions that the system is to ultimately perform for the customer by constructing an “operating” model and utilizing it in a research-oriented environment in an engineering shop, or equivalent. This model may be constructed using nonstandard and unqualified parts and will not necessarily reflect the design configuration that will ultimately be produced for the customer. The intent is to verify certain performance characteristics and to gain confidence that “all is well” at the time.

\subsection{Analytical Models and Modeling}\index{Analytical Models and Modeling}

<para>The design evaluation process may be further facilitated through the use of various analytical models, methods, and tools in support of the Macro-CAD objective. A model, in this context, is a simplified representation of the real world that abstracts features of the situation relative to the problem being analyzed. It is a tool employed by an analyst to assess the likely consequences of various alternative courses of action being examined. The model must be adapted to the problem at hand and the output must be oriented to the selected evaluation criteria. The model, in itself, is not the decision maker but is a tool that provides the necessary data in a timely manner in support of the decision-making process.</para>
<para>The extensiveness of the <emphasis>model</emphasis> will depend on the nature of the problem, the number of variables, input parameter relationships, number of alternatives being evaluated, and the complexity of operation. The ultimate objective in the selection and development of a model is simplicity and usefulness. The model used should incorporate the following features:</para>
<orderedlist numeration="arabic" spacing="normal" inheritnum="ignore" continuation="restarts"><listitem><inst>	1.	</inst><para>The model should represent the dynamics of the system configuration being evaluated in a way that is simple enough to understand and manipulate, and yet close enough to the operating reality to yield successful results.</para></listitem>
<listitem><inst>	2.	</inst><para>The model should highlight those factors that are most relevant to the problem at hand and suppress (with discretion) those that are not as important.</para></listitem>
<listitem><inst>	8.1.	</inst><para>The model should be comprehensive, by including <emphasis>all</emphasis> relevant factors, and be reliable in terms of repeatability of results.</para></listitem>
<listitem><inst>	8.2.	</inst><para>Model design should be simple enough to allow for timely implementation in problem solving. Unless the tool can be utilized in a timely and efficient manner by the analyst (or the manager), it is of little value. If the model is large and highly complex, it may be appropriate to develop a series of models where the output of one can be tied to the input of another. Also, it may be desirable to evaluate a specific element of the system independent of other elements.</para></listitem>
<listitem><inst>	5.	</inst><para>Model design should incorporate provisions for ease of modification or expansion to permit the evaluation of additional factors as required. Successful model development often includes a series of trials before the overall objective is met. Initial attempts may suggest information gaps, which are not immediately apparent and consequently may suggest beneficial changes.</para></listitem></orderedlist>
<para>The use of mathematical models offers significant benefits. In terms of system application, several considerations exist—operational considerations, design considerations, product/construction considerations, testing considerations, logistic support considerations, and recycling and disposal considerations. There are many interrelated elements that must be integrated as a system and not treated on an individual basis. The mathematical model makes it possible to deal with the problem as an entity and allows consideration of all major variables of the problem on a simultaneous basis. More specifically:</para>
    1. <orderedlist numeration="arabic" spacing="normal" inheritnum="ignore" continuation="restarts"><listitem><inst>	1.	</inst><para>The mathematical model will uncover relations between the various aspects of a problem that are not apparent in the verbal description.</para></listitem>
    2. <listitem><inst>	2.	</inst><para>The mathematical model enables a comparison of <emphasis>many</emphasis> possible solutions and aids in selecting the best among them rapidly and efficiently.</para></listitem>
    3. <listitem><inst>	8.1.	</inst><para>The mathematical model often explains situations that have been left unexplained in the past by indicating cause-and-effect relationships.</para></listitem>
    4. <listitem><inst>	8.2.	</inst><para>The mathematical model readily indicates the type of data that should be collected to deal with the problem in a quantitative manner.</para></listitem>
    5. <listitem><inst>	5.	</inst><para>The mathematical model facilitates the prediction of future events, such as effectiveness factors, reliability and maintainability parameters, logistics requirements, and so on.</para></listitem>
    6. <listitem><inst>	6.	</inst><para>The mathematical model aids in identifying areas of risk and uncertainty.</para></listitem></orderedlist>
    7. <para>When analyzing a problem in terms of selecting a mathematical model for evaluation purposes, it is desirable to first investigate the tools that are currently available. If a model already exists and is proven, then it may be feasible to adopt that model. However, extreme care must be exercised to relate the right technique with the problem being addressed and to apply it to the depth necessary to provide the sensitivity required in arriving at a solution. Improper application may not provide the results desired, and the consequence may be costly.</para>
<para>Conversely, it might be necessary to construct a new model. In accomplishing this task, one should generate a comprehensive list of system parameters that will describe the situation being simulated. Next, it is necessary to develop a matrix showing parameter relationships, each parameter being analyzed with respect to every other parameter to determine the magnitude of relationship. Model input–output factors and parameter feedback relationships must be established. The model is constructed by combining the various factors and then testing it for validity. Testing is difficult to do because the problems addressed primarily deal with actions in the future that are impossible to verify. However, it may be possible to select a known system or equipment item that has been in existence for several years and exercise the model using established parameters. Data and relationships are known and can be compared with historical experience. In any event, the analyst might attempt to answer the following questions: <emphasis>Can the model describe known facts and situations sufficiently well? When major input parameters are varied, do the results remain consistent and are they realistic? Relative to system application, is the model sensitive to changes in operational requirements, design, production/construction, and logistics and maintenance support? Can cause-and-effect relationships be established?</emphasis></para>
<para>Models and appropriate systems analysis methods are presented further in <link olinkend="part03" preference="0">Part <xref olinkend="part03" label="III"><inst>III</inst></xref></link>. Trade-off and system optimization relies on these methods and their application. A fundamental knowledge of probability and statistics, economic analysis methods, modeling and optimization, simulation, queuing theory, control techniques, and the other analytical techniques is essential to accomplishing effective systems analysis.

\subsection{Design, Data, Information, and Integration}\index{Design, Data, Information, and Integration}

As a result of advances in information systems technology, the methods for documenting design are changing rapidly. These advances have promoted the use of vast electronic databases for the purposes of information processing, storage, and retrieval. Through the use of CAD techniques, information can be stored in the form of three-dimensional representations, in regular two-dimensional line drawings, in digital format, or in combinations of these. By using computer graphics, word processing capability, e-mail communications, video-disc technology, and the like, design can be presented in more detail, in an easily modified format, and displayed faster.</para>
<para>Although many advances have been made in the application of computerized methods to data acquisition, storage, and retrieval, the need for some of the more conventional methods of design documentation remains. These include a combination of the following:</para>
<orderedlist numeration="arabic" spacing="normal" inheritnum="ignore" continuation="restarts"><listitem><inst>	1.	</inst><para><emphasis>Design drawings</emphasis>—assembly drawings, control drawings, logic diagrams, structural layouts, installation drawings, schematics, and so on.</para></listitem>
<listitem><inst>	2.	</inst><para><emphasis>Material and part lists</emphasis>—part lists, material lists, long-lead-item lists, bulk-item lists, provisioning lists, and so on.</para></listitem>
<listitem><inst>	3.	</inst><para><emphasis>Analyses and reports</emphasis>—trade-off study reports supporting design decisions, reliability and maintainability analyses and predictions, human factors analyses, safety reports, supportability analyses, configuration identification reports, computer documentation, installation and assembly procedures, and so on.</para></listitem></orderedlist>
<para>Design drawings, constituting a primary source of definition, may vary in form and function depending on the design objective; that is, the type of equipment being developed, the extent of development required, whether the design is to be subcontracted, and so on. Some typical types of drawings from the past are illustrated in 
<para>During the process of detail design, engineering documentation is rather preliminary and then gradually progresses to the depth and extent of definition necessary to enable product manufacture. The responsible designer, using appropriate design aids, produces a functional diagram of the overall system. The system functions are analyzed and initial packaging concepts are assumed. With the aid of specialists representing various disciplines (e.g., civil, electrical, mechanical, structural, components, reliability, maintainability, and sustainability) and supplier data, detail design layouts are prepared for subsystems, units, assemblies, and subassemblies. The results are analyzed and evaluated in terms of functional capability, reliability, maintainability, human factors, safety, producibility, and other design parameters to assure compliance with the allocated requirements and the initially established design criteria. This review and evaluation occurs at each stage in the basic design sequence and generally follows the steps presented in 
<para>Engineering data are reviewed against design standards and checklist criteria. Throughout the industrial and governmental sectors are design standards manuals and handbooks developed to cover preferred component parts and supplier data, preferred design and manufacturing practices, designated levels of quality for specified products, requirements for safety, and the like. These standards (ANSI, EIA, ISO, etc.), as applicable, may serve as a basis for design review and evaluation.</para>

Patterns Leading to Models
</para></section></section>
3. An Introduction to TRIZ (SE-110406, Blackburn, 15 April 2011, Page 6 of 25)

TRIZ began to appear in the West in the 1980s, and is evolving to be an inventive problem
solving approach for any system. (Stratton and Mann 2003) TRIZ (often pronounced trees) is a
Russian acronym for the theory of inventive problem solving, an approach to solving problems
systematically, creating innovation by identifying and eliminating system conflicts or
contradictions. (Stratton and Mann 2003; Harvard 2009) The Russian title for TRIZ
transliterated to English is Teoriya Resheniya Izobretatelskikh Zadatch. (Low, Lamvik et al.
2002) TRIZ uses generalized principles derived from patent data analysis to offer solutions for
system problems. (Stratton and Mann 2003)
TRIZ follows a scheme of abstracting a problem to identify an abstracted solution (ReVelle
2002) and offers abstracted inventive principles to solve problems based on a prior solution.
Savransky estimates that “about 95% of the inventive problems in any particular field have
already been solved in another field.” (Savransky 2000) Mann makes the case that TRIZ’
 “generic problem solving framework . . . allows engineers and scientists working in any one field to access the good practices of everyone working in not just their own, but every other field of science.” (Mann 2002) Savransky describes this as accumulating and condensing “all respective
human knowledge and then” applying “it to solve inventive problems.” (Savransky 2000)
As noted previously, TRIZ began when Genrich Altshuller studied trends in patents while
serving in the Russian Navy in 1946. He discovered that innovation is not a random process, but
is governed by learnable principles. (Savransky 2000; Silverstein, DeCarlo et al. 2007) He
identified patterns in patents, and later developed an approach that describes conflicting features
with mapped solutions. Specifically, his study identified 39 engineering or desired features (or
parameters) and 40 inventive solutions to apply when one or more parameters conflict – see
Appendices A-C. (ReVelle 2002; Stratton and Mann 2003) TRIZ is based on three premises as
follows. (ReVelle 2002; Stratton and Mann 2003)
1. Ideality or Ideal Final Result (IFR): The ultimate goal is to design a system with no
harmful functions. IFR is useful to describe what the desired final system state should be,
enabling the identification of system contradictions that will prevent success.
2. Contradictions (or conflicts): It is necessary to eliminate (wholly or partially) a system
contradiction to achieve higher success. A physical contradiction occurs when an aspect of a
system needs to be in opposite states (e.g. hot versus cold).
3. Resources: Maximize the utility of current resources prior to introducing any additional
complexity or resources to the system.
IFR can be useful to mitigate the impact of PI (Psychological Inertia), where it might be
difficult for the systems engineer to innovate beyond his or her normal experience. Altshuller
viewed PI as a barrier to innovation. (Low, Lamvik et al. 2002) PI relates to the likelihood that
one will limit the solution space inside a known or comfortable paradigm, or define a solution
path defined by his or her paradigm. (ReVelle 2002) PI has been described as “the sum of one’s
intellectual, emotional, academic, experiential, and other biases” (Silverstein, DeCarlo et al.
2007) and “the effort made by a system to preserve the current (meta-)stable state or to resist
change in that state.” (Savransky 2000)
SE-110406, Blackburn, 15 April 2011, Page 7 of 25
Starting at the ideal state and working backwards can stimulate new and innovative thinking.
The idea of eliminating contradictions is useful in that it enables the system to have breakthrough
performance. Last, focusing on maximizing current resources avoids adding additional cost and
complexity to the system, which can in turn trigger undesired emergent behaviors.
Using the findings from patents, TRIZ offers abstracted inventive principles to solve
problems. By aligning abstracted engineering or desired features with already successful
principles as observed in patents, system contradictions can be eliminated. While there are other
advanced TRIZ tools and methods, this paper will focus on the application of the classic TRIZ
Contradiction Matrix. The following presents a TRIZ Trade Study framework that uses the
TRIZ Contradiction Matrix.

RAW:
Conceptually sound system design derives from focusing on what the system is intended to do before determining what the system is, with form following function. This focus is most effective when based on essential design dependent parameters, recognizing the concurrent life-cycle factors of production, support, maintenance, phase-out, and disposal. It invokes integrating and iterating synthesis, analysis, and evaluation. These considerations are germane to system and product design when embedded within the systems engineering process. The purpose of this presentation is to provide an overview of the embedded relationship of design dependent parameters as key controllables in the effective, efficient, and orderly process of bringing cost-effective systems, products, structures, and services (the human-made world) into being.

\subsection{Patterns as Models for Systems Engineering}\index{Patterns as Models for Systems Engineering}

This group is probably quite advanced in the reflections on patterns and their role in systems science / systems thinking / systems engineering. I have been in contact with some members such as Len Troncale, Joseph Simpson a while ago in a discussion Len involved me in but it seems I lost the thread, and Peter Tuddenham, with whom I have been reflecting on the role of pattern literacy in support of systems literacy. In addition to whatwe presented at ISSS 2017on the topic we presented a paper at the pattern languages of programs conference, with pattern literacy in support of systems literacy seen from a pattern language perspective. An approach to patterns quite different than the systems science approach.
The issue you raise Aleksandar is key. Patterns can be everything or nothing, and can we really find a bottom line theory that is not a reduction from one or another perspective (sorry if I 'reduce' your point that much here, I am looking forward to get deeper in Dennett's understanding of patterns)? I am very interested in Len's work, because I think that what Len has done over the years can provide some responses although I do not know his work well enough to know exactly which piece. The isomorphies Len has been pursuing are patterns, and it seems to me that there would be ways of interconnecting patterns of different types on different types of criteria with some sort of 'semantic' if not semiotic relationship, to see clusters of 'probable' isomorphy emerge, around which conversations can take place. This is something we have discussed in Vienna, and that is outlined around slides 16 to 22 of the presentation linked above.

What I would very much like to do, with, and beyond this survey, is to gather research that has already been done on different approaches to patterns. When reading the paperDefining “System”: a Comprehensive Approach, [proceedings of?] 27th Annual INCOSE International Symposium (IS 2017) in Adelaide, with some of you I recognize as authors, I was wondering whether something like this type of research had been done on patterns within a systems context, and thought maybe some in this group had started looking at patterns similarly.

The survey is an attempt to scratch the surface. The input of your group will be very valuable, especially if key questions are asked there, and I offer to share here a compilation of responses I will have collected. I look forward to further discussions.

RAW:
Better for ST definition. From a cognitive perspective, systems thinking integrates analysis and synthesis. Natural science has been primarily reductionistic, studying the components of systems and using quantitative empirical verification. Human science, as a response to the use of positivistic methods for studying human phenomena, has embraced more holistic approaches, studying social phenomena through qualitative means to create meaning. Systems thinking bridges these two approaches by using both analysis and synthesis to create knowledge and understanding and integrating an ethical perspective. Analysis answers the ‘what’ and ‘how’ questions while synthesis answers the ‘why’ and ‘what for’ questions. Better for ST definition. From a cognitive perspective, systems thinking integrates analysis and synthesis. Natural science has been primarily reductionistic, studying the components of systems and using quantitative empirical verification. Human science, as a response to the use of positivistic methods for studying human phenomena, has embraced more holistic approaches, studying social phenomena through qualitative means to create meaning. Systems thinking bridges these two approaches by using both analysis and synthesis to create knowledge and understanding and integrating an ethical perspective. Analysis answers the ‘what’ and ‘how’ questions while synthesis answers the ‘why’ and ‘what for’ questions.

%------------------------------------------------

\section{Modeling Quality Function Deployment}\index{Modeling Quality Function Deployment}

A useful tool that can be applied to aid in the establishment and prioritization of TPMs is the <emphasis>quality function deployment</emphasis> (QFD) model. QFD constitutes a team approach to help ensure that the “voice of the customer” is reflected in the ultimate design. The purpose is to establish the necessary requirements and to translate those requirements into technical solutions. Customer requirements are defined and classified as <emphasis>attributes</emphasis>, which are then weighted based on the degree of importance. The QFD method provides the design team an understanding of customer desires, forces the customer to prioritize those desires, and enables a comparison of one design approach against another. Each customer attribute is then satisfied by a technical solution.1<footnoteref preference="1" label="1" role="generated" linkend="ch03fn01"/></para>
<para>The QFD process involves constructing one or more matrices, the first of which is often referred to as the <emphasis>House of Quality</emphasis> (HOQ). A modified version of the HOQ is presented in . Starting on the left side of the structure is the identification of customer needs and the ranking of those needs in terms of priority, the levels of importance being specified quantitatively. This side reflects the <emphasis>whats</emphasis> that must be addressed. The top part of the HOQ identifies the designer’s <emphasis>technical</emphasis> response relative to the attributes that must be incorporated into the design in order to respond to the needs (i.e., the voice of the customer). This part constitutes the <emphasis>hows</emphasis>, and there should be at least one technical solution for each identified customer need. The interrelationships among attributes (or technical correlations) are identified, as well as possible areas of conflict. The center part of the HOQ conveys the strength or impact of the proposed technical response on the identified requirement. The bottom part allows a comparison between possible alternatives, and the right side of the HOQ is used for planning purposes.</para>
<para>The QFD method is used to facilitate the translation of a prioritized set of subjective customer requirements into a set of system-level requirements during conceptual design. A similar approach may be used to subsequently translate system-level requirements into a more detailed set of requirements at each stage in the design and development process. In, the <emphasis>hows</emphasis> from one house become the <emphasis>whats</emphasis> for a succeeding house. Requirements may be developed for the system, subsystem, component, manufacturing process, support infrastructure, and so on. The objective is to ensure the required justification and traceability of requirements from the top down. Further, requirements should be stated in <emphasis>functional</emphasis> terms.</para>
<para>Although the QFD method may not be the only approach used in helping define the specific requirements for system design, it does constitute an excellent tool for creating the necessary visibility from the beginning. One of the largest contributors to <emphasis>risk</emphasis> is the lack of a good set of requirements and an adequate system specification. Inherent within the system specification should be the identification and prioritization of TPMs. The TPM, its associated metric, its relative importance, and benchmark objective in terms of what is currently available will provide designers with the necessary guidance for accomplishing their task. This is essential for establishing the appropriate levels of design emphasis, for defining the criteria as an input to the design, and for identifying the levels of possible risk should the requirements not be met. Again, care must be taken to ensure that these system-level requirements are not adversely impacted by external factors from other systems within the same overall SOS structure.

%------------------------------------------------

\section{Model-Based Systems Engineering}\index{Model-Based Systems Engineering}

Model based systems engineering has process model (not decision model) characteristics. In the INCOSE vision 2020 document, v Model Based Systems Engineering (MBSE) is the formalized application of modeling to support the SEP. It brings together, normally on a computer-based platform, the information to support functional analysis and allocation, requirements management and decomposition, synthesis, analysis, verification and validation. Figures 6.4 and 6.5. provide the essential framework for the MBSE model.

Figure 6.4 HERE
Figure 6.5 HERE
	
Specifically, revisit the information flows in Figure 6.5 in light of the abstract regarding the “On line linking of Individual Designers Decisions . . . FIND IT

One of the challenges to MBSE is a reliance on network effects, the idea that the benefit to the team grows nonlinearly with the adoption by other users. Furthermore, there is little data about adoption of modeling by engineering-driven practitioners or firms. Obtaining and leveraging data in this area is important to assess whether engineers apply effective modeling techniques in practice.

Finally, I am enclosing the Exploratory Research Proposal in reoriented form now entitled:

EXPLORATORY RESEARCH ON THE ON-LINE LINKING OF
INDIVIDUAL DESIGNER’S DECISIONS TO THE SYSTEM LEVEL AND
TO SYSTEM LIFE -CYCLE VALUE IN THE FACE OF MULTIPLE CRITERIA

Our SDEM/SEP (System Design Evaluation Methodology Integrated With The Systems Engineering Process) capability provides the enabling foundation and visible interface with firms comprising the Software Productivity Consortium (SPC). New funding effective January 1 for SDEM/SEP brings the total to \$120,000, provided jointly by SPC and the Virginia Center for Innovative Technology. It is the SPC member firms to whom we will go to support a comprehensive proposal based on findings from the requested Small Grant for Exploratory Research.
Emerging academic and educational experiences, including massive open online courses (MOOCs) and similar online offerings, are a new opportunity to learn about industry practices. In this paper, we draw on data from the “Architecture and Systems Engineering: Models and Methods to Manage Complex Systems” online program from the Massachusetts Institute of Technology (MIT), which has enrolled 4200 participants, to understand how models are used in practice. We use aggregated data from enrollment surveys and poll questions to investigate model use by participants and their organizations.
The results in this paper demonstrate that while modeling is widely recognized among course participants for its potential, the actual deployment of models against problems of interest is substantially lower than the participant-stated potential. We also find that in the practice of engineering, many participants reported low use of basic modeling best practices, such as documentation and specialized programs. Furthermore, we find that participants overstate the deployment of modeling practices within their organizations.
We believe that this research is a fresh use of descriptive analysis to report on actual modeling practices. Future research should target more detail about modeling practices in engineering-driven firms, and the reasons why modeling adoption is low or overstated.
The “Architecture and Systems Engineering: Models and Methods to Manage Complex Systems” certificate program is a series of four online courses offered by the Massachusetts Institute of Technology (MIT) for professional audiences covering models in engineering and model-based system engineering. For more information, visithttps://sysengonline.mit.edu/.

Modern educational experiences include environments beyond the traditional classroom. Massive open online courses (MOOCs) assemble learners from around the world from different backgrounds. For example, edX.org lists over 1900 available courses in Introductory, Intermediate, and Advanced levels at the time of this writing. These programs extend the educational reach across geographies, industries, firms, and knowledge silos through interactive videos, surveys, and discussion boards. Beyond exposure to course content, MOOC participants engage in both individually-paced and community learning environments.
MOOCs provide access to large groups of participants simultaneously, orders of magnitude larger than is otherwise practical within traditional constraints. In a prior study of participant “subpopulations” in MOOCs, Kizilcec et al. (2013) concludes that effectively leveraging of participants’ “prior knowledge” enhances MOOC content effectiveness because it “mediates encounters with new information.” We propose that the overall MOOC environment and interactive mechanisms facilitate a practice-sharing role, in addition to the traditional content delivery style of online and video education.
The “Architecture and Systems Engineering: Models and Methods to Manage Complex Systems” certificate program offered by the Massachusetts Institute of Technology (MIT) is a series of four online courses with enrollment of 4200 participants with common interest in model-based systems engineering over the past two years. As career opportunity becomes increasingly chaotic and professionals endure a “life-long” state of “transition” (Vuolo et al., 2012), the participants who enroll in these courses for career training and growth represent a potential population sample to learn about engineering practices. While it is debatable whether this fee-for-service offering with fewer than 10,000 participants fits the definition of a MOOC (Jordan 2015), participants with experience in industry engage through both public and corporate-supported program enrollments.
During the “Architecture and Systems Engineering” program at MIT, participants complete an enrollment survey and respond to interactive polls about modeling in engineering.Responses reported by participants through these mechanisms is an opportunity to learn about the modeling use and sophistication in engineering-driven firms. This data is sufficiently anonymized to support useful learning about models in engineering without exposing participants or their organizations to privacy risk. In this paper, our research objective is to analyze survey and polls data from these courses to measure model use and user sophistication in engineering-driven firms. Through a descriptive analysis of participants and the firms they represent, this paper seeks to fill a perceived gap in the currently available research in this area.
This paper begins with a research outline and analysis of methods used on survey and polls data from the “Architecture and Systems Engineering” program at MIT. The results are then presented, followed by discussions about the implications about the use of models in industry, the future of MBSE, the use of MOOC data for research, and approaches for studies in engineering.
INTRODUCTION: Model-based Systems Engineering

This is Model-based Systems Engineering

Systems engineering is concerned with the design, building, and use of concrete entities such as engines, machines, and structures. It is equally concerned with business systems which are composed of processes. In order to engage in systems engineering we need an organized means of thinking about those systems in their operational contexts. 

The discipline of systems engineering uses the knowledge and techniques from various branches of engineering and science in the planning and development of new solutions for answering a need. In fact, systems engineering is a multi-disciplinary approach to satisfying the needs of stakeholders through the creation or improvement of a system. The focus of the systems engineer is on the whole system and the system’s external interfaces. 

Model-based systems engineering quite literally means the use of a model (or models) to gain insight into engineering the solution for a project. Each project model uses a restricted language framework that spans the problem and solution spaces. This primer discusses the use of such a model- its advantages and the various aspects of its use.

Why use a model-based approach?
There are major advantages arising from using models as the basis of systems engineering. Models offer a complete consideration of the entire engineering problem, the use of a consistent language to describe the problem and the solution, the production of a coherently designed solution and the comprehensive and verifiable answering of all the system requirements posed by the problem. All of these offer significant advantages in seeking a solution to the systems design problem at hand.

Basic Concepts

Although the term ``system'' is defined in a variety of ways in the systems engineering community, most definitions are similar to the one used in the DoDAF framework - ``any organized assembly of resources and procedures united and regulated by interaction or interdependence to accomplish a set of specific functions.'' However, this primer takes an intentionally broad view of the systems definition because it allows for the use of model-based systems engineering to undertake and provide solutions for a much wider range of problems than would be possible if the operating definition of ``system'' were restricted to a narrower definition.

The INCOSE definition of system is ``a construct or collection of different entities that together produce results not obtainable by the entities alone.''  Some examples of systems include:

\begin{itemize}
\item A set of things working together as parts of a mechanism or an interconnecting network
\item A set of organs in the body with a common structure or function
\item A group of related hardware units or software programs or both, especially when dedicated to a single application
\item A major range of strata that corresponds to a period in time, subdivided into series
\item A group of celestial objects connected by their mutual attractive forces, especially moving in orbits about a center
The entities of a human constructed system can include people, hardware, software, facilities, policies, and documents; that is, all the things required to produce system-level results. These results include system-level qualities, properties, characteristics, functions, behavior and performance. The value added by the system as a whole, beyond that contributed independently by the parts, is primarily created by the relationships among the parts. In other words, the ``value-add'' of the system emerges in the synergy created when the parts come together.
\end{itemize}

Systems Composition
A system begins with an idea that must be translated into reality. The model of a system serves to link the engineered system “reality” to the theoretical idea and vice versa (bi-directionally). Using a model clearly shows when and how the theory explains reality and how reality confirms the theory. 

Properties – Within the boundary of a system there are three kinds of properties:
Entities – these are the parts (things or substances) that make up a system. These parts may be atoms or molecules, larger bodies of matter like sand grains, raindrops, plants, animals, or even components like motors, planes, missiles, etc.
Attributes – attributes are characteristics of the entities that may be perceived and measured. For example: quantity, size, color, volume, temperature, reliability, maintainability and mass.
Relationships – relationships are the associations that occur between entities and attributes. These associations are based on cause and effect. 

Language – When combined these entities, attributes and relationships form a system “language.” This language is fundamental to being able to describe and communicate the system among the engineering team as well as to other stakeholders.
Layers – The system is represented hierarchically, allowing it to be understood as decomposable into meaningful layers of subunits. These subunits are conventionally named: 
    • a system is composed of subsystems; 
    • subsystems in turn are composed of assemblies; 
    • assemblies are composed of subassemblies, and 
    • subassemblies are composed of parts. 
It is important to note that what may be considered a ``part'' in the context of a particular system may be a complete ``system'' in its own right.
Often the terms used in this hierarchy are not well specified; some engineers use the term sub-subsystem, others use the terms component and subcomponent in the hierarchy; contributing to confusion. In order to avoid such usage confusion, the term ``component'' is used here as an abstract term representing the physical or logical entity that performs a specific function or functions.

Systems engineering begins with the identification of the needs of the users and the stakeholders while assuring that the right problem is addressed. The systems engineer crafts those needs into a definition of the system, identifies the functions that meet those needs, allocates those functions to the system entities (components) and finally confirms that the system performs as designed and satisfies the needs of the user.

Systems engineering is both a technical and a management process. The technical process addresses the analytical, design, and implementation efforts necessary to transform the operational need into a system of the proper size and configuration and produces the documentation necessary to implement, operate and maintain the system. The management process involves planning, assessing risks, integrating the various engineering specialties and design groups, maintaining configuration control, and continuously auditing the effort to ensure that cost, schedule, and technical performance objectives are satisfied.

To be effective in all these areas systems engineering must, therefore, be an organized, repeatable, iterative and convergent approach to the development of complex systems. The approach must be ``organized'' because without an organized approach the details of the system under development will be overlooked, confused, and misunderstood. The approach must be “repeatable” so that it will apply to other system development efforts in such a way that creates reasonable assurances of success. Such an approach should be both iterative and convergent as well, which means the engineering processes repeat at each level of system design and insure process convergence to a solution.

The Process. The first task of the systems engineer is to develop a clear statement of the problem setting out what issue or issues are being addressed by the proposed system. This involves working with others (especially system stakeholders and subject matter experts) to identify the stated requirements that govern what would characterize an acceptable solution. The systems engineer must provide design focus and facilitate proper and effective communication between the various subject matter experts and the stakeholders. The systems engineer must have a broad knowledge base to understand the various disciplines involved in the development of the system, and to participate in and evaluate system level design decisions and to resolve system issues. Often some system requirements conflict with each other and the systems engineer must resolve these conflicts in a way that does not lose sight of the system’s purpose. The goal of the engineer is to develop a system that maximizes the strengths and benefits of the system while minimizing its flaws and weaknesses. 

For illustrative purposes, we will present the Model-based systems engineering effort from a top-down perspective. Figure 1- Intra-layer Systems Engineering Process, and Figure 2- Onion Layers, illustrate the process of working within a layer and between layers respectively.
 
Figure 1- Intra-layer Systems Engineering Process

Once the problem is clearly defined, the process steps follow the flow in Figure 1. The originating requirements (which identify what the system will provide) are extracted from the source documentation, market studies or other expressions of the system definition and analyzed. This analysis identifies numerous aspects of the desired system. Analyzing the requirements allows the systems engineer to define the system boundaries and identify what is inside and outside those boundaries. The definition of these boundaries- an often-overlooked step- is critical to the implementation of the system. Any change in the system boundaries will impact the complexity and character of the way in which the system interacts and interfaces with the environment.

Specifying the functional requirements and the interactions of the system with the external entities leads directly to developing a clear picture of the system. It is critical that the functional requirements (the “what”) are understood before attempting to define the implementation (the “how”) of the system. Therefore, this analysis is repeated throughout the system design process to test the rigor and integrity of the system.

In parallel with the functional behavior definition, what must be built to perform the needed behavior is derived through the decomposition of the system into components. The systems engineer then analyzes the constraints and allocates system behavior to the physical components. This leads to the specification of each system component. Because of this allocation, the identification and definition of all interfaces between the physical parts of the system, including hardware, software, and people can take place. 

The engineer uses a formal specification language to characterize the various design entities (requirements, functions, components, etc.) in a repository. Using this language and a repository allows the engineer to construct a system ``model.'' By capturing all the system information in the repository and correcting all errors, the engineer builds the repository to contain the system model from which the design team will produce the system, segment, and interface specifications.

It should be emphasized that these activities are performed concurrently or in parallel but not independently. The activities in one area are influenced by and have influence on the other activities. The layer-by-layer approach of MBSE assures that these interrelationships are considered in context. Approaches that involve ``deep dives'' into one area (e.g., requirements) run the substantial risk of obscuring the systemic risks incurred when the complex relationships are not fully considered. The central power of the MBSE approach lies in its careful and complete consideration of the system design in an orderly and systematic fashion. This can only happen through the orderly process and excellent communication which must be the hallmarks of an effective systems engineering process. 

This look at the systems engineering process uses a description that best fits top down systems design. With suitable approach variations the systems engineer can address reverse or bottom-up and middle-out systems engineering perspectives as well. For example, in conducting reverse engineering, the engineer would begin with the existing physical description and work from the physical representation and interfaces to ultimately derive the original system’s requirements. Once these are obtained, the engineering process would proceed to incorporate the desired changes and enhancements as in a top-down design.

Figure 2 above shows the layers with the domains represented in each. The movement of the system design process from a conceptual to a detailed description happens by moving downward with increasing detail or granularity from one layer to the next. Each of the iterations analyzes a layer within the system design process. Beginning with the basic, high-level user requirements serves to define broad system characteristics and objectives. Clarity is brought to these high-level expressions until the process has defined the system to a point of sufficient granularity to allow the system’s physical implementation to begin.

The ``Onion'' Model. As the threads are developed in increasing detail or ``granularity'' a layered structure takes shape. The engineering process follows these layers drilling deeper and deeper into the system design. Every iteration of the systems engineering process increases the level of specificity, removes ambiguity and resolves unknowns. The domains (Requirements, Functional Behavior, Architecture and Validation and Verification) are all addressed in context as each successive layer is peeled back. 

This approach to solving the problem in layers is the heart of MBSE. With this approach (addressing all domains in each layer), there comes an assurance that all aspects of the engineering problem at hand are addressed completely and consistently. The layers and their interrelationships also provide a solution that can be easily verified and validated against the needs that created the problem. By ``peeling the onion,'' we can be assured that we have indeed addressed the problem in a meaningful and productive way converging on a solution that responds to the entire problem in all its aspects.

The application of an iterative, convergent (layer-by-layer) systems engineering process leads to reducing ambiguity by resolving open and uncovered issues and mitigating risks. Systems engineers and stakeholders collaborate with other team members to make decisions that advance the design to completion. When the design is complete, validation and verification can take the form of ``walking through the design,'' verifying that all the requirements are valid and can be verified, all the functions are present that are necessary to meet the requirements, all appropriate analytical and simulation activities have been performed and all the components needed to perform the functions at this level are defined. In other words, the engineers and stakeholders can verify that the layer-by-layer process has converged on an engineering solution that satisfies all the requirements for the system being designed.

%------------------------------------------------

\section{Summary and Extensions}\index{Summary and Extensions}

There is not unanimity around the definition of Model-based systems engineering in the market place and most uses of the term Model-based systems engineering are not as broad as what is addressed here. The same iterative approach that is used to define and design a new system (the ``top-down'' engineering approach) can be used to analyze and improve an existing system (``middle-out'' or ``reverse'' engineering). By intentionally adopting a ``broad'' definition of a system the approach can be used across the widest possible spectrum of systems to be analyzed or constructed. Clearly, from the definitions and discussion, the aim of a system model is to be able to analyze and gain insight into real systems whether human-made or not. With human-made or ``engineered'' systems, the aim is to find a realizable solution to a stated need using effective engineering and management processes.

The chapters that follow will not go into exhaustive detail but will strive to provide a high-level view of the Model-based systems engineering process. Various aspects of the iterative systems engineering process will be addressed to develop a clear understanding of ``what is model-based systems engineering'' and ``what are its advantages?'' Because of its very inter-relatedness it is difficult to determine the order in which discuss MBSE’s concepts. The discussion here will begin with language, the foundational blocks of the systems engineering process. From there it will turn to the model, the process and the functional and physical topics. Along the way, most of the major concepts of MBSE will be introduced and explained. It is the hope of the creative team who write this Primer that it will prove a useful introduction to the world of problem solving with MBSE.

%------------------------------------------------

%% SEA CHAPTER 7 - ALTERNATIVES AND MODELS IN DECISION MAKING
% SEA Question Location in \label{sea-Chapter#-Problem#}
\begin{exercises}
    \begin{exercise}
    \label{sea-7-1}
        A complete and all-inclusive alternative rarely emerges in its final state. Explain.
    \end{exercise}
    \begin{solution}
    \end{solution}
    
    \begin{exercise}
    \label{sea-7-2}
        Should decision making be classified as an art or as a science?
    \end{exercise}
    \begin{solution}
    \end{solution}
    
    \begin{exercise}
    \label{sea-7-3}
        Contrast limiting and strategic factors.
    \end{exercise}
    \begin{solution}
    \end{solution}
    
    \begin{exercise}
    \label{sea-7-4}
        Discuss the various meanings of the word model.
    \end{exercise}
    \begin{solution}
    \end{solution}
    
    \begin{exercise}
    \label{sea-7-5}
        Describe briefly physical models, schematic models, and mathematical models.
    \end{exercise}
    \begin{solution}
    \end{solution}
    
    \begin{exercise}
    \label{sea-7-6}
        How do mathematical models directed to decision situations differ from those traditionally used in the physical sciences?
    \end{exercise}
    \begin{solution}
    \end{solution}
    
    \begin{exercise}
    \label{sea-7-7}
        Contrast direct and indirect experimentation.
    \end{exercise}
    \begin{solution}
    \end{solution}
    
    \begin{exercise}
    \label{sea-7-8}
        Write the general form of the evaluation function for money flow modeling and define its symbols.
    \end{exercise}
    \begin{solution}
    \end{solution}
    
    \begin{exercise}
    \label{sea-7-9}
        Identify a decision situation and indicate the variables under the control of the decision maker and those not directly under his or her control.
    \end{exercise}
    \begin{solution}
    \end{solution}
    
    \begin{exercise}
    \label{sea-7-10}
        Contrast the similarities and differences in the economic optimization functions given by Equations 7.2 and 7.3 (see Figure 7.2).
    \end{exercise}
    \begin{solution}
    \end{solution}
    
    \begin{exercise}
    \label{sea-7-11}
        Why is it not possible to formulate a model that accurately represents reality?
    \end{exercise}
    \begin{solution}
    \end{solution}
    
    \begin{exercise}
    \label{sea-7-12}
        Under what conditions may a properly formulated model become useless as an aid in decision making?
    \end{exercise}
    \begin{solution}
    \end{solution}
    
    \begin{exercise}
    \label{sea-7-13}
        Explain the nature of the cost components that should be considered in deciding how frequently to review a dynamic environment.
    \end{exercise}
    \begin{solution}
    \end{solution}
    
    \begin{exercise}
    \label{sea-7-14}
        What caution must be exercised in the use of models?
    \end{exercise}
    \begin{solution}
    \end{solution}
    
    \begin{exercise}
    \label{sea-7-15}
        Discuss several specific reasons why models are of value in decision making.
    \end{exercise}
    \begin{solution}
    \end{solution}
    
    \begin{exercise}
    \label{sea-7-16}
        Identify a multiple-criteria decision situation with which you have experience. Select the three to five most important criteria.
    \end{exercise}
    \begin{solution}
    \end{solution}
    
    \begin{exercise}
    \label{sea-7-17}
        Discuss the degree to which you think the criteria you selected in Question 16 are truly independent. Weight each criterion, check for consistency, and then normalize the weight so that the total sums to 100.Use the method of paired comparisons to rank the criteria in order of decreasing importance.
    \end{exercise}
    \begin{solution}
    \end{solution}
    
    \begin{exercise}
    \label{sea-7-18}
        Extend Question 16 to include two or three alternatives. Evaluate how well each alternative ranks against each criterion on a scale of 1 to 10. Then compute the product of the ratings and the criterion weights and sum them for each criterion to determine the weighted evaluation for each alternative.
    \end{exercise}
    \begin{solution}
    \end{solution}
    
    \begin{exercise}
    \label{sea-7-19}
        The values for three alternatives considered against four criteria are given (with higher values being better). What can you conclude using the following systematic elimination methods?
        \begin{enumerate}[label=\alph*)]
            \item Comparing the alternatives against each other (dominance).
            \item Comparing the alternatives against a standard (Rule 1 and Rule 2).
            \item Comparing criteria across alternatives (criteria ranked  ).
        \end{enumerate}
        \begin{table}[h]
        \centering
        \begin{tabular}{r r r r r r}
        \toprule
         & \multicolumn{3}{c}{\textbf{Alternative}} & \multicolumn{1}{c}{\textbf{Minimum}} \\
        \cmidrule{2-4}
        \multicolumn{1}{c}{\textbf{Criterion}} & A & B & C & \multicolumn{1}{c}{\textbf{Ideal}} & \multicolumn{1}{c}{\textbf{Standard}} \\
        \midrule
        1. & 6  & 5  & 8  & 10  & 7  \\
        2. & 90 & 80 & 75 & 100 & 70 \\
        3. & 40 & 35 & 50 & 50  & 30 \\
        4. & G  & P  & VG & E   & F  \\
        \bottomrule
        \end{tabular}
        %\caption{Table caption}
        \label{tab:sea-7-19} % Unique label used for referencing the table in-text
        %\addcontentsline{toc}{table}{Table \ref{tab:example}} % Uncomment to add the table to the table of contents
        \end{table}
    \end{exercise}
    \begin{solution}
    \end{solution}
    
    \begin{exercise}
    \label{sea-7-20}
        A small software firm is planning to offer one of four new software products and wishes to maximize profit, minimize risk, and increase market share. A weight of 65\% is assigned to annual profit potential, 20\% to profitability risk, and 15\% to market share. Use the tabular additive method for this situation and identify the product that would be best for the firm to introduce.
        \begin{table}[h]
        \centering
        \begin{tabular}{l l l l}
        \toprule
        New Product & Profit Potential & Profit Risk & Market Share \\
        \midrule
        SW \RomanNumeralCaps{1} & \$100K & \$40K & HIGH \\
        SW \RomanNumeralCaps{2} & \$140K & \$35K & MEDIUM \\
        SW \RomanNumeralCaps{3} & \$150K & \$50K & LOW \\
        SW \RomanNumeralCaps{4} & \$130K & \$45K & MEDIUM \\
        \bottomrule
        \end{tabular}
        %\caption{Table caption}
        \label{tab:sea-7-20} % Unique label used for referencing the table in-text
        %\addcontentsline{toc}{table}{Table \ref{tab:example}} % Uncomment to add the table to the table of contents
        \end{table}
    \end{exercise}
    \begin{solution}
    \end{solution}
    
    \begin{exercise}
    \label{sea-7-21}
        Convert the tabular additive results from Problem 20 into a stacked bar chart.
    \end{exercise}
    \begin{solution}
    \end{solution}
    
    \begin{exercise}
    \label{sea-7-22}
        Rework the example in Section 7.5 if the importance ratings for Better, Cheaper, Faster in Table 7.5 change from 7, 9, 4 to 9, 7, 4.
    \end{exercise}
    \begin{solution}
    \end{solution}
    
    \begin{exercise}
    \label{sea-7-23}
        Sketch a decision evaluation display that would apply to making a choice from among three automobile makes in the face of the three top criteria of your selection.
    \end{exercise}
    \begin{solution}
    \end{solution}
    
    \begin{exercise}
    \label{sea-7-24}
        Superimpose another source alternative (remanufacture) on the decision evaluation display in Section 7.5 to illustrate its expandability. Now discuss the criteria values that this source alternative would require to make it the preferred alternative.
    \end{exercise}
    \begin{solution}
    \end{solution}
    
    \begin{exercise}
    \label{sea-7-25}
        Formulate an evaluation matrix for a hypothetical decision situation of your choice.
    \end{exercise}
    \begin{solution}
    \end{solution}
    
    \begin{exercise}
    \label{sea-7-26}
        Formulate an evaluation vector for a hypothetical decision situation under assumed certainty.
    \end{exercise}
    \begin{solution}
    \end{solution}
    
    \begin{exercise}
    \label{sea-7-27}
        Develop an example to illustrate the application of paired outcomes in decision making among a number of nonquantifiable alternatives.
    \end{exercise}
    \begin{solution}
    \end{solution}
    
    \begin{exercise}
    \label{sea-7-28}
        What approaches may be used to assign probabilities to future outcomes?
    \end{exercise}
    \begin{solution}
    \end{solution}
    
    \begin{exercise}
    \label{sea-7-29}
        What is the role of dominance in decision making among alternatives?
    \end{exercise}
    \begin{solution}
    \end{solution}
    
    \begin{exercise}
    \label{sea-7-30}
        Give an example of an aspiration level in decision making.
    \end{exercise}
    \begin{solution}
    \end{solution}
    
    \begin{exercise}
    \label{sea-7-31}
        When would one follow the most probable future criterion in decision making?
    \end{exercise}
    \begin{solution}
    \end{solution}
    
    \begin{exercise}
    \label{sea-7-32}
        What drawback exists in using the most probable future criterion?
    \end{exercise}
    \begin{solution}
    \end{solution}
    
    \begin{exercise}
    \label{sea-7-33}
        How does the Laplace criterion for decision making under uncertainty actually convert the situation to decision making under risk?
    \end{exercise}
    \begin{solution}
    \end{solution}
    
    \begin{exercise}
    \label{sea-7-34}
        Discuss the maximin and the maximax rules as special cases of the Hurwicz rule.
    \end{exercise}
    \begin{solution}
    \end{solution}
    
    \begin{exercise}
    \label{sea-7-35}
        The cost of developing an internal training program for office automation is unknown but described by the following probability distribution:
        \begin{table}[h]
        \centering
        \begin{tabular}{r c}
        \toprule
        \textbf{Cost} & \textbf{Probability of Occurence} \\
        \midrule
        \$80,000  & 0.20 \\
        \$95,000  & 0.30 \\
        \$105,000 & 0.25 \\
        \115,000  & 0.20 \\
        \$130,000 & 0.05 \\
        \bottomrule
        \end{tabular}
        %\caption{Table caption}
        \label{tab:sea-7-35} % Unique label used for referencing the table in-text
        %\addcontentsline{toc}{table}{Table \ref{tab:example}} % Uncomment to add the table to the table of contents
        \end{table}
        What is the expected cost of the course? What is the most probable cost? What is the maximum cost that will occur with a 95\% assurance?
    \end{exercise}
    \begin{solution}
    \end{solution}
    
    \begin{exercise}
    \label{sea-7-36}
        Net profit has been calculated for five investment opportunities under three possible futures. Which alternative should be selected under the most probable future criterion? Which under the expected value criterion?
        \begin{table}[h]
        \centering
        \begin{tabular}{r r r r}
        \toprule
         & \textbf{$F_1$(0.3)} & \textbf{$F_2$(0.2)} & \textbf{$F_3$(0.5)} \\
        \midrule
        $A_1$ &  \$100,000 & \$100,000 & \$380,000 \\
        $A_2$ & -\$200,000 & \$160,000 & \$590,000 \\
        $A_3$ &  \$0       & \$180,000 & \$500,000 \\
        $A_4$ &  \$110,000 & \$280,000 & \$200,000 \\
        $A_5$ &  \$400,000 & \$90,000  & \$180,000 \\
        \bottomrule
        \end{tabular}
        %\caption{Table caption}
        \label{tab:sea-7-36} % Unique label used for referencing the table in-text
        %\addcontentsline{toc}{table}{Table \ref{tab:example}} % Uncomment to add the table to the table of contents
        \end{table}
    \end{exercise}
    \begin{solution}
    \end{solution}
    
    \begin{exercise}
    \label{sea-7-37}
        Daily positive and negative payoffs are given for five alternatives and five futures in the following matrix. Which alternative should be chosen to maximize the probability of receiving a payoff of at least 9? What choice would be made by using the most probable future criterion?
        \begin{table}[h]
        \centering
        \begin{tabular}{r r r r r r}
        \toprule
         & \textbf{$F_1$(0.15)} & \textbf{$F_2$(0.20)} & \textbf{$F_3$(0.30)} & \textbf{$F_4$(0.20)} & \textbf{$F_5$(0.15)} \\
        \midrule
        $A_1$ & 12 &  8 & -4 &  0 &  9 \\
        $A_2$ & 10 &  0 &  5 & 10 & 16 \\
        $A_3$ &  6 &  5 & 10 & 15 & -4 \\
        $A_4$ &  4 & 14 & 20 &  6 & 12 \\
        $A_5$ & -8 & 22 & 12 &  4 &  9 \\
        \bottomrule
        \end{tabular}
        %\caption{Table caption}
        \label{tab:sea-7-37} % Unique label used for referencing the table in-text
        %\addcontentsline{toc}{table}{Table \ref{tab:example}} % Uncomment to add the table to the table of contents
        \end{table}
    \end{exercise}
    \begin{solution}
    \end{solution}
    
    \begin{exercise}
    \label{sea-7-38}
        The following matrix gives the payoffs in utiles (a measure of utility) for three alternatives and three possible states of nature:
        \begin{table}[h]
        \centering
        \begin{tabular}{r r r r}
        \toprule
         & \multicolumn{3}{c}{\textbf{State of Nature}} \\
         & \textbf{$S_1$} & \textbf{$S_2$} & \textbf{$S_3$} \\
        \midrule
        $A_1$ & 50 & 80 & 80 \\
        $A_2$ & 60 & 70 & 20 \\
        $A_3$ & 90 & 30 & 60 \\
        \bottomrule
        \end{tabular}
        %\caption{Table caption}
        \label{tab:sea-7-38} % Unique label used for referencing the table in-text
        %\addcontentsline{toc}{table}{Table \ref{tab:example}} % Uncomment to add the table to the table of contents
        \end{table}
        Which alternative would be chosen under the Laplace principle? The maximin rule? The maximax rule? The Hurwicz rule with  $\alpha=0.75$?
    \end{exercise}
    \begin{solution}
    \end{solution}
    
    \begin{exercise}
    \label{sea-7-39}
        The following payoff matrix indicates the costs associated with three decision options and four states of nature:
        \begin{table}[h]
        \centering
        \begin{tabular}{r r r r}
        \toprule
         & \multicolumn{4}{c}{\textbf{State of Nature}} \\
        \textbf{Option} & \textbf{$S_1$} & \textbf{$S_2$} & \textbf{$S_3$} & \textbf{$S_4$} \\
        \midrule
        $T_1$ & 20 & 25 & 30 & 35 \\
        $T_2$ & 40 & 30 & 40 & 20 \\
        $T_3$ & 10 & 60 & 30 & 25 \\
        \bottomrule
        \end{tabular}
        %\caption{Table caption}
        \label{tab:sea-7-38} % Unique label used for referencing the table in-text
        %\addcontentsline{toc}{table}{Table \ref{tab:example}} % Uncomment to add the table to the table of contents
        \end{table}
        Select the decision option that should be selected for the maximin rule; the maximax rule; the Laplace rule; the minimax regret rule; and the Hurwicz rule with   How do the rules applied to the cost matrix differ from those that are applied to a payoff matrix of profits?
    \end{exercise}
    \begin{solution}
    \end{solution}
    
    \begin{exercise}
    \label{sea-7-40}
        A cargo aircraft is being designed to operate in different parts of the world where external navigation aids vary in quality.  The manufacturer has identified five areas where the aircraft is likely to fly and the likelihood that the aircraft will be in each area. The aircraft will be in Areas 1 through 5 with the frequency given in the table below. The table also gives the 5 navigation systems under consideration, N1, N2, N3, N4, and N5. The manufacturer has estimated the typical navigation errors to be expected in each Area and these are shown in the table. Navigation errors are in nautical miles (nm). Small navigation errors are good; large navigation errors are bad.
        \begin{table}[h]
        \centering
        \begin{tabular}{r r r r}
        \toprule
        \textbf{Navigation Systems} & \textbf{Area1 (0.15)} & \textbf{Area2 (0.20)} & \textbf{Area3 (0.30)} & \textbf{Area4 (0.20)} & \textbf{Area5 (0.15)} \\
        \midrule
        $N_1$ & 0.09 & 0.15 & 0.20 & 0.30 & 0.10 \\
        $N_1$ & 0.09 & 0.30 & 0.25 & 0.09 & 0.06 \\
        $N_1$ & 0.15 & 0.14 & 0.09 & 0.06 & 0.25 \\
        $N_1$ & 0.21 & 0.07 & 0.05 & 0.12 & 0.06 \\
        $N_1$ & 0.28 & 0.03 & 0.08 & 0.19 & 0.10 \\
        \bottomrule
        \end{tabular}
        %\caption{Table caption}
        \label{tab:sea-7-40} % Unique label used for referencing the table in-text
        %\addcontentsline{toc}{table}{Table \ref{tab:example}} % Uncomment to add the table to the table of contents
        \end{table}
        \begin{enumerate}[label=\alph*)]
            \item Which system(s) will maximize the probability of achieving a navigation error of 0.10 nm or less?
            \item What is the probability of achieving this navigation error?
            \item Which choice would be made by using the most probable future criterion?
        \end{enumerate}
    \end{exercise}
    \begin{solution}
    \end{solution}
    
    \begin{exercise}
    \label{sea-7-41}
        The following matrix gives the expected profit in thousands of dollars for five marketing strategies and five potential levels of sales:
        \begin{table}[h]
        \centering
        \begin{tabular}{r r r r}
        \toprule
         & \multicolumn{5}{c}{\textbf{Level of Sales}} \\
        \textbf{Strategy} & \textbf{$L_1$} & \textbf{$L_2$} & \textbf{$L_3$} & \textbf{$L_4$} & \textbf{$L_5$} \\
        \midrule
        $M_!$ & 10 & 20 & 30 & 40 & 50 \\
        $M_2$ & 20 & 25 & 25 & 30 & 35 \\
        $M_3$ & 50 & 40 &  5 & 15 & 20 \\
        $M_4$ & 40 & 35 & 30 & 25 & 25 \\
        $M_5$ & 10 & 20 & 25 & 30 & 20 \\
        \bottomrule
        \end{tabular}
        %\caption{Table caption}
        \label{tab:sea-7-41} % Unique label used for referencing the table in-text
        %\addcontentsline{toc}{table}{Table \ref{tab:example}} % Uncomment to add the table to the table of contents
        \end{table}
        Which marketing strategy would be chosen under the maximin rule? The maximax rule? The Hurwicz rule with $\alpha=0.4$?
    \end{exercise}
    \begin{solution}
    \end{solution}
    
    \begin{exercise}
    \label{sea-7-42}
        Graph the Hurwicz rule for all values of   using the payoff matrix of Problem 40.
    \end{exercise}
    \begin{solution}
    \end{solution}
    
    \begin{exercise}
    \label{sea-7-43}
        The following decision evaluation matrix gives the expected savings in maintenance costs (in thousands of dollars) for three policies of preventive maintenance and three levels of operation of equipment. Given the probabilities of each level of operation,   and   determine the best policy based on the most probable future criterion.
        \begin{table}[h]
        \centering
        \begin{tabular}{r r r r}
        \toprule
         & \multicolumn{3}{c}{\textbf{Level of Operation}} \\
         \textbf{Policy} & \textbf{$L_1$} & \textbf{$L_2$} & \textbf{$L_3$} \\
        \midrule
        $M_!$ & 10 & 20 & 30 \\
        $M_2$ & 22 & 26 & 26 \\
        $M_3$ & 40 & 30 & 15 \\
        \bottomrule
        \end{tabular}
        %\caption{Table caption}
        \label{tab:sea-7-43} % Unique label used for referencing the table in-text
        %\addcontentsline{toc}{table}{Table \ref{tab:example}} % Uncomment to add the table to the table of contents
        \end{table}
    \end{exercise}
    \begin{solution}
    \end{solution}
    
    \begin{exercise}
    \label{sea-7-44}
        Use the decision evaluation display of Figure 7.4 to make visible a design decision situation of your choice from Part II of this text.
    \end{exercise}
    \begin{solution}
    \end{solution}
    
    \begin{exercise}
    \label{sea-7-45}
        What should be done with those facets of a decision situation that cannot be explained by the model?
    \end{exercise}
    \begin{solution}
    \end{solution}
\end{exercises}
% SKIPPED