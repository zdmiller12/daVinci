\chapter{BRINGING ENGINEERED SYSTEMS INTO BEING}\label{chap:6}

Systems engineering and analysis reveals unexpected ways of using technology to bring new and improved systems and products into being that will be more competitive in the global economy. New and emerging technologies are expanding physically realizable design options and enhancing capabilities for developing more cost-effective systems. And, unprecedented improvement possibilities arise from the proper application of the concepts and principles of systems engineering to legacy systems.

This chapter introduces a technologically based interdisciplinary process encompassing an extension of engineering through all phases of the system life cycle; that is, design and development, production or construction, utilization and support, and phase-out and disposal. The process is derived from the concepts and general systems thinking presented in Part I of this textbook.

The final section of this chapter provides a summary and extension of the key concepts and ideas pertaining to the process of bringing systems into being. It is augmented with some annotated references and website addresses identifying opportunities for further inquiry. General references and website addresses supporting the ideas in this chapter are given in 

\section{What the System is Intended to Do}\index{What the System is Intended to Do}

The tangible outcome of systems engineering is an engineered or technical system, whether human-made or human-modified. Systems concepts, general systems theory, and systems thinking presented in provide a scientific foundation for engineering the pervasive domain of the human-made world. This section and the material that follows pertain to the organized technological activities for bringing engineered systems into being. To begin on solid ground, it is necessary to define the engineered system in terms of its characteristics.

\subsection{Characteristics of the Engineered System}\index{Characteristics of the Engineered System}

A technical or engineered system is a human-made or human-modified system designed to meet functional purposes or objectives. Systems can be engineered well or poorly. The phrase ``engineered system'' in this book implies a well-engineered system. A well-engineered system has the following characteristics:

\begin{enumerate}
\item Engineered systems have functional purposes in response to identified needs and have the ability to achieve stated operational objectives
\item Engineered systems are brought into being and operate over a life cycle, beginning with identification of needs and ending with phase-out and disposal
\item Engineered systems have design momentum that steadily increases throughout design, production, and deployment, and then decreases throughout phase-out, retirement, and disposal
\item Engineered systems are composed of a harmonized <emphasis>combination of resources, such as facilities, equipment, materials, people, information, software, and money
\item Engineered systems are composed of subsystems and related components that interact with each other to produce a desired system response or behavior
\item Engineered systems are part of a hiearchy and are influenced by external factors from larger systems of which they are a part and from sibling systems from which they are composed
\item Engineered systems are embedded into the natural world and interact with it in desirable as well as undesirable ways
\end{enumerate}

Systems engineering is defined in several ways. Basically, systems engineering is a functionally-oriented, technologically-based interdisciplinary process for bringing systems and products (human-made entities) into being as well as for improving existing systems. The outcome of systems engineering is the engineered system as previously described. Its overarching purpose is to make the world better, primarily for people. Accordingly, human-made entities should be designed to satisfy human needs and/or objectives effectively while minimizing system life-cycle cost, as well as the intangible costs of societal and ecological impacts.

Organization, humankind’s most important innovation, is the time-tested means for bringing human-made entities into being. While the focus is nominally on the entities themselves, systems engineering embraces a better strategy. Systems engineering concentrates on what the entities are intended to do before determining what the entities are composed of. As simply stated within the profession of architecture, form follows function. Thus, instead of offering systems or system elements and products per se, the organizational focus should shift to designing, delivering, and sustaining functionality, a capability, or a solution.

\subsection{System Life Cycle Engineering}\index{System Life Cycle Engineering}

Design within the system life-cycle context as in Figure 5.1 differs from design in the general sense. Life-cycle-guided design is simultaneously responsive to customer needs (i.e., to requirements expressed in functional terms) and to life-cycle outcomes. Design should not only transform a need into a system configuration but should also ensure the design’s compatibility with related physical and functional requirements. Further, it should consider operational outcomes expressed as producibility, reliability, maintainability, usability, supportability, serviceability, disposability, sustainability, and others, in addition to performance, effectiveness, and affordability.</para>

Concurrent Engineering

A detailed presentation of the elaborate technological activities and interactions that must be integrated over the system life-cycle process is given in Figure 6.2. The progression is iterative from left to right and not serial in nature, as might be inferred.

Figure 6.1 Here (was 2.3)

Although the level of activity and detail may vary, the life-cycle functions described and illustrated are generic. They are applicable whenever a new need or changed requirement is identified, with the process being common to large as well as small-scale systems. It is essential that this process be implemented completely at an appropriate level of detail not only in the engineering of new systems but also in the re-engineering of existing or legacy systems.

Major technical activities performed during the design, production or construction, utilization, support, and phase-out phases of the life cycle are highlighted in Figure 6.1. These are initiated when a new need is identified. A planning function is followed by conceptual, preliminary, and detail design activities. Producing and/or constructing the system are the function that completes the acquisition phase. System operation and support functions occur during the utilization phase of the life cycle. Phase-out and disposal are important final functions of utilization to be considered as part of design for the life cycle.

The numbered blocks in ``map'' and elaborate on the phases of the life cycles depicted in 

The acquisition phase 
The utilization phase
The design phase
The startup phase
The operation phase
The retirement phase

The communication and coordination needed to design and develop the product, the production capability, the system support capability, and the relationships with interrelated systems so that they traverse the life cycle together seamlessly is not easy to accomplish. Progress in this area is facilitated by technologies that make more timely acquisition and use of design information possible. Computer-Aided Design (CAD) technology with internet/intranet connectivity enables a geographically dispersed multidiscipline team to collaborate effectively on complex physical designs.

For certain products, the addition of Computer-Aided Manufacturing (CAM) software can automatically translate approved three-dimensional CAD drawings into manufacturing instructions for numerically controlled equipment. Generic or custom parametric CAD software can facilitate exploration of alternative design solutions. Once a design has been created in CAD/CAM, iterative improvements to the design are relatively easy to make. The CAD drawings also facilitate maintenance, technical support, regeneration (re-engineering), and disposal. A broad range of other electronic communication and collaboration tools can help integrate relevant geographically dispersed design and development activities over the life cycle of the system.

Concern for the entire life cycle is particularly strong within the U.S. Department of Defense (DOD) and its non-U.S. counterparts. This may be attributed to the fact that acquired defense systems are owned, operated, and maintained by the DOD. This is unlike the situation most often encountered in the private sector, where the consumer or user is usually not the producer. Those private firms serving as defense contractors are obliged to design and develop in accordance with DOD directives, specifications, and standards. Because the DOD is the customer and the user of the resulting system, considerable DOD intervention occurs during the acquisition phase.

Many firms that produce for private-sector markets have chosen to design with the life cycle in mind. For example, design for energy efficiency is now common in appliances such as water heaters and air conditioners. Fuel efficiency is a required design characteristic for automobiles. Some truck manufacturers promise that life cycle maintenance costs will be within stated limits. These developments are commendable, but they do not go far enough. When the producer is not the consumer, it is less likely that potential operational problems will be addressed during development. Undesirable outcomes too often end up as problems for the user of the product instead of the producer.

%------------------------------------------------

\section{Engineering the System and Product}\index{Engineering the System and Product}

Engineering has always been concerned with the economical use of limited resources to achieve objectives. The purpose of the engineering activities of design and analysis is to determine how physical and conceptual factors may be altered to create the most utility for the least cost, in terms of product cost, product service cost, social cost, and environmental cost. Viewed in this context, engineering should be practiced in an expanded way, with engineering of the system placed ahead of concern for product components thereof.

Classical engineering focuses on physical factors such as the selection and design of physical components and their behaviors and interfaces. Achieving the best overall results requires focusing initially on conceptual factors, such as needs, requirements, and functions. The ultimate system, however, is manifested physically where the main objective is usually considered to be product performance, rather than the design and development of the overall system of which the product is a part. A product cannot come into being and be sustained without a production or construction capability, without support and maintenance, and so on. Engineering the system and product usually requires an interdisciplinary approach embracing both the product and associated capabilities for production or construction, product and production system maintenance, support and regeneration, logistics, connected system relationships, and phase-out and disposal.

A product may also be known as a service such as health care, learning modules, entertainment packets, financial services and controls, and orderly traffic flow. In these service examples, the engineered system is a health care system, an educational system, an entertainment system, a financial system, and a traffic control system.

\subsection{Systems are Known by Their Products}\index{Systems are Known by Their Products}

Systems and their associated products are designed, developed, deployed, renewed, and phased out in accordance with processes that are not as well understood as they might be. The cost-effectiveness of the resulting technical entities can be enhanced by placing emphasis on the following:

\begin{enumerate}
\item Improving methods for determining the scope of needs to be met by the system. All aspects of a systems engineering project are profoundly affected by the scope of needs, so this determination should be accomplished first. Initial consideration of a broad set of needs often yields a consolidated solution that addresses multiple needs in a more cost-effective manner than a separate solution for each need
\item Improving methods for defining product and system requirements as they relate to verified customer needs and external mandates. This should be done early in the design phase, along with a determination of performance, effectiveness, and specification of essential system characteristics
\item Addressing the total system with all its elements from a life-cycle perspective, and from the product to its elements of support and renewal. This means defining the system in functional terms before identifying hardware, software, people, facilities, information, or combinations thereof
\item Considering interactions in the overall system hierarchy. This includes relationships between pairs of system components, between higher and lower levels within the system hierarchy, and between sibling systems or subsystems
\item Organizing and integrating the necessary engineering and related disciplines into a top-down system-engineering effort in a concurrent and timely manner
\item Establishing a disciplined approach with appropriate review, evaluation, and feedback provisions to ensure orderly and efficient progress from the initial identification of needs through phase-out and disposal
\end{enumerate}

Any useful system must respond to identified <emphasis>functional needs</emphasis>. Accordingly, the elements of a system must not only include those items that relate directly to the accomplishment of a given operational scenario or mission profile but must also include those elements of logistics and maintenance support for use should failure of a prime element(s) occur. To ensure the successful completion of a mission, all necessary supporting elements must be available, in place, and ready to respond. System sustainability can help insure that the system continues meeting the functional needs in a competitive manner as the needs and the competition evolves. And, system sustainability contributes to overall sustainability of the environment.

\subsection{Product and System Categories}\index{Product and System Categories}

It is interesting and useful to note that systems are often known by their products. They are identified in terms of the products they propose, produce, deliver, or in other ways bring into being. Examples are manufacturing systems that produce products, construction systems that erect structures, transportation systems that move people or goods, traffic control systems that manage vehicle or aircraft flow, maintenance systems that repair or restore, and service systems that meet the need of a consumer or patient. What the system does is manifested through the product it provides. The product and its companion system are inexorably linked.

As frameworks for study, or baselines, two generic product/system categories are presented and characterized in this section. Consideration of these categories is intended to serve two purposes. First, it will help explain and clarify the topics and steps in the process of bringing engineered or technical systems into being. Then, in subsequent chapters, these categories and examples will provide opportunities for look-back reference to generic situations. Although there are other less generic examples in those chapters, greater understanding of them may be imparted by reference to the categories established in this section.

Single-Entity Product Systems. A single-entity product system, for example, may manifest itself as a bridge, a custom-designed home entertainment center, a custom software system, or a unique consulting session. The product may be a consumable (a nonstandard banquet or a counseling session) or a repairable (a highway or a supercomputer). Another useful classification is to distinguish consumer goods from producer goods, the latter being employed to produce the former. A product, as considered in this textbook, is not an engineered system no matter how complex it might be.

The product standing alone is not an engineered system. Consider a bridge constructed to meet the need for crossing an obstacle (a river, a water body, or another roadway). The engineered system is composed of the bridge structure plus a construction subsystem, a maintenance subsystem, an operating and support subsystem, and a phase-out and demolition process. Likewise, an item of equipment for producer or consumer use is not a system within the definition and description of an engineered system given in 

Manufacturing plants that produce repairable or consumable products, warehouses or shopping centers that distribute products, hospital facilities that provide health care services, and air traffic control systems that produce orderly traffic flow are also single-entity systems when the plant, shopping center, or hospital is the product being brought into being. These entities, in combination with appended and companion subsystems, may rightfully be considered technical systems.

The preceding recognizes the engineered system as more than just the consumable or repairable product, be it a single entity, a population of homogenous entities, or a flow of entities. The product must be treated as part of a system to be engineered, deployed, and operated. Although the product subsystem (including structure or service) directly meets the customer’s need, this need must be functionally decomposed and allocated to the subsystems and components comprising the overall system.

Multiple-Entity Population Systems. Multiple-entity populations, often homogenous in nature, are quite common. Thinking of these populations as being aggregated generic products permits them to be studied probabilistically. However, the engineered system is more than just a single entity in the population, or even the entities as a population. It is composed of the population together with the subsystems of production, maintenance and support, regeneration, and phase-out and disposal.

A set of needs provides justification for bringing the population into being. This set of needs drives the life-cycle phases of acquisition and utilization, made up of design, construction or production, maintenance and support, renovation, and eventually ending with phase-out and demolition/disposal. As with single-entity product systems, the product is the subsystem that directly meets the customer’s need.

Examples of repairable-entity populations include the following: The airlines and the military acquire, operate, and maintain aircraft with population characteristics. In ground transit, vehicles (such as taxicabs, rental automobiles, and commercial trucks) constitute repairable equipment populations. Production equipment types (such as machine tools, weaving looms, and autoclaves) are populations of equipment classified as producer goods.

Also consider repairable (renovatable) populations of structures, often homogenous in nature. In multi-family housing, a population of structures is composed of individual dwelling units constructed to meet the need for shelter at a certain location. In multi-tenant office buildings, the population of individual offices constitutes a population of renovatable entities. And in urban or suburban areas, public clinics constitute a distributed population of structures to provide health care.

The simplest multi-entity populations are called inventories. These inventories may be made up of consumables or repairable. Examples of consumables are small appliances, batteries, foodstuffs, toiletries, publications, and many other entities that are a part of everyday life. Repairable-entity inventories are often subsystems or components for prime equipment. For example, aircraft hydraulic pumps, automobile starters and alternators, and automation controllers are repairable entities that are components of higher-level systems.

Homogenous populations lend themselves to designs that are targeted to the end product or prime equipment, as well as to the population as a whole. Economies of scale, production and maintenance learning, mortality considerations, operational analyses based on probability and statistics, and so on, all apply to the repairable-entity population to a greater or lesser degree. But the system to be brought into being must be larger in scope than the population itself, if the end result is to be satisfactory to the producer and customer.

\subsection{Engineering for Product Competitiveness}\index{Engineering for Product Competitiveness}

Product competitiveness is desired by both commercial and public-sector producers worldwide. Thus, the systems engineering challenge is to bring products and systems into being that meet customer expectations cost-effectively.

Because of intensifying international competition, producers are seeking ways to gain a sustainable competitive advantage in the marketplace. Acquisitions, mergers, and advertising campaigns seem unable to create the intrinsic wealth and goodwill necessary for the long-term health of the organization. Economic competitiveness is essential. Engineering with an emphasis on economic competitiveness must become coequal with concerns for advertising, production distribution, finance, and the like.

Available human and physical resources are dwindling. The industrial base is expanding worldwide, and international competition is increasing rapidly. Many organizations are downsizing, seeking to improve their operations, and considering international partners. Competition has reduced the number of suppliers and subcontractors able to respond. This is occurring at a time when the number of qualified team members required for complex system development is increasing. Consequently, needed new systems are being deferred in favor of extending the life of existing systems.

All other factors being equal, people will meet their needs by purchasing products and services that offer the highest value–cost ratio, subjectively evaluated. This ratio can be increased by giving more attention to the resource-constrained world within which engineering is practiced. To ensure economic competitiveness of the product and enabling system, engineering must become more closely associated with economics and economic feasibility. This is best accomplished through a system life-cycle approach to engineering.

Limiting and Strategic Factors. </title><para>Those factors that stand in the way of attaining objectives are known as <emphasis>limiting factors</emphasis>. An important element of the systems engineering process is the identification of the limiting factors restricting accomplishment of a desired objective. Once the limiting factors have been identified, they are examined to locate strategic factors, those factors that can be altered to make progress possible.

The identification of strategic factors is important, for it allows the decision maker to concentrate effort on those areas in which success is obtainable. This may require inventive ability, or the ability to put known things together in new combination and is distinctly creative in character. The means achieving the desired objective may consist of a procedure, a technical process, or an organizational or managerial change. Strategic factors limiting success may be circumvented by operating on engineering, human, and economic factors individually and jointly.

An important element of the process of defining alternatives is the identification of the limiting factors restricting the accomplishment of a desired objective. Once the limiting factors have been identified, they are examined to locate those strategic factors that can be altered in a cost-efficient way so that a selection from among the alternatives may be made.

Seeking Desirable Emergent Properties. Those factors that stand in the way of attaining objectives are known as limiting factors. An important element of the systems engineering process is the identification of the limiting factors restricting accomplishment of a desired objective. Once the limiting factors have been identified, they are examined to locate <emphasis>strategic factors</emphasis>, those factors that can be altered to make progress possible.

The identification of strategic factors is important, for it allows the decision maker to concentrate effort on those areas in which success is obtainable. This may require inventive ability, or the ability to put known things together in new combinations and is distinctly creative in character. The means achieving the desired objective may consist of a procedure, a technical process, or an organizational or managerial change. Strategic factors limiting success may be circumvented by operating on engineering, human, and economic factors individually and jointly.

An important element of the process of defining alternatives is the identification of the limiting factors restricting the accomplishment of a desired objective. Once the limiting factors have been identified, they are examined to locate those strategic factors that can be altered in a cost-efficient way so that a selection from among the alternatives may be made.

People often acquire diverse products to meet specific needs without companion contributing systems to ensure the best overall results, and without adequately considering the effects of the products on the natural world, on humans, and on other human-made systems. Proper application of systems engineering ensures timely and balanced evaluation of all issues to harmonize overall results from human investments, minimizing problems and maximizing satisfaction.

Maximizing Satisfaction Through Emergent Properties. Maximizing satisfaction requires a focus on emergent properties with linkage to requirements derived from the customer.

An emergent property cannot be a property determined from analysis of constituent part interactions when assembled. No single part provides flight. It only occurs when the parts are correctly assembled, and engines engaged. That situation conforms to the simple description given and all would agree that “dominance in transatlantic business class performance” is also an emergent property but the system assembled to attain that distinction is a whole for which the Dreamliner 787 is but one part.

In the general case, however, it may prove rather difficult to undertake ``total systems design'' of some complex system with particular emergent properties in mind. It has been an idealist target, however, since - if it can be achieved - there is the prospect of "something for nothing," or the whole being greater than the sum of its parts per Derek Hitchins.

The concept of emergence is very useful but may not help as much as expected in some cases. Take a passenger jet airplane as an example. The airplane itself has the emergent property of flight, whereas an airplane engine, on its own, cannot fly. If an airplane engine is considered to be the total system, then the emergent property is thrust. The jet fan does not produce thrust, neither does the jet compressor. However, all together, the jet engine produces thrust.

Thus, I am not sure all would agree about your notion of emergence. A simple description of emergence is: those properties of the whole which may not be ascribedexclusivelyto any of the parts. The property of flight is clearly not emergent: it comes as a linear balance between the forces of drag and thrust in the forward axis and lift and weight on the vertical axis. Newton's First Law applies.

And there is no ``whole is greater than the sum of its parts,'' either, as Aristotle required. So, flight is not emergent. If one were to seek emergent properties from an airliner, then we would be looking at the airliner in operation in some future, competitive environment, and a hoped for emergent property might be "dominance in transatlantic business class performance."

Can one undertake systems design with that emergent property in mind? Yes, one can. Ask the Dreamliner 787 folks.

%------------------------------------------------

\section{System Design Considerations}\index{System Design Considerations}

The systems engineering process is suggested as a preferred approach for bringing systems and their products into being that will be cost-effective and globally competitive. An essential technical activity within the process is that of system design evaluation. Evaluation must be inherent within the systems engineering process and be invoked regularly as the system design activity progresses. However, system evaluation should not proceed without either accepting guidance from customer requirements and applicable system design criteria, or direct involvement of the customer. When conducted with full recognition of customer requirements and design criteria, evaluation enhances assurance of continuous design improvement.

\subsection{Development of Design Criteria}\index{Development of Design Criteria}

As was depicted in 6.1, the definition of needs at the system level is the starting point for determining customer requirements and developing design criteria. The requirements for the system as an entity are established by describing the functions that must be performed. The operational and support functions (i.e., those required to accomplish a specified mission scenario, or series of missions, and those required to ensure that the system is able to perform the needed functions when required) must be described at the top level. Also, the general concepts and nonnegotiable requirements for production, systems integration, and retirement must be described.
<para>In design evaluation, an early step that fully recognizes design criteria is to establish a <emphasis>baseline</emphasis> against which a given alternative or design configuration may be evaluated (refer to ). This baseline is determined through the iterative process of requirements analysis (i.e., identification of needs, analysis of feasibility, definition of system operational and support requirements, and selection of concepts for production, systems integration, and retirement). A more specific baseline is developed for each <emphasis>system level</emphasis> in. The functions that the system must perform to satisfy a specific scope of customer needs should be described, along with expectations for cycle time, frequency, speed, cost, effectiveness, and other relevant factors. Functional requirements must be met by incorporating design characteristics within the system at the appropriate level.
<As part of this process, it is necessary to establish some system “metrics” related to performance, effectiveness, cost, and similar quantitative factors as required to meet customer expectations. For instance, what functions must the system perform, where are these to be accomplished, at what frequency, with what degree of reliability, and at what cost? Some of these factors may be more important than others by the customer which will, in turn, influence the design process by placing different levels of emphasis on meeting criteria. Candidate systems result from design synthesis and become the appropriate targets for design analysis and evaluation.</para>
<para>Evaluation is invoked to determine the degree to which each candidate system satisfies design criteria. Applicable criteria regarding the system should be expressed in terms of <emphasis>technical performance measures</emphasis> (TPMs) and exhibited at the system level. TPMs are measures for characteristics that are, or derive from, attributes inherent in the design itself. Attributes that depend directly on design characteristics are called <emphasis>design-dependent parameters</emphasis> (DDPs), with specific measures thereof being the TPMs. In contrast, relevant factors external to the design are called <emphasis>design-independent parameters</emphasis> (DIPs).</para>
<para>It is essential that the development of <emphasis>design criteria</emphasis> be based on an appropriate set of <emphasis>design considerations</emphasis>, considerations that lead to the identification of both <emphasis>design-dependent</emphasis> and <emphasis>design-independent parameters</emphasis> and that support the derivation of <emphasis>technical performance measures</emphasis>. More precise definitions for these terms are as follows:</para>
1.	Design considerations the full range of attributes and characteristics that could be exhibited by an engineered system, product, or service. These are of interest to both the producer and the customer.
2.	Design-dependent parameters (DDPs) attributes and/or characteristics inherent in the design for which predicted or estimated measures are required or desired (e.g., design life, weight, reliability, producibility, maintainability, pollutability, and others).
3.	Design-independent parameters (DIPs) factors external to the design that must be estimated and/or forecasted for use during design evaluation (e.g., fuel cost per pound, labor rates, material cost per pound, interest rates, and others). These depend upon the production and operating environment for the system.
4.	Technical performance measures (TPMs) predicted and/or estimated values for DDPs. They also include values for higher level (derived) performance considerations (e.g., availability, cost, flexibility, and supportability).
5.	Design criteria customer specified or negotiated target values for technical performance measures. Also, desired values for TPMs as specified by the customer as requirements.</para></listitem></orderedlist>

<para>The issue and impact of multiple criteria will be presented in the paragraphs that follow. Then, the next section will direct attention to design criteria as an important part of a morphology for synthesis, analysis, and evaluation. In so doing, the terms defined above will be better related to each other and to the system realization process.

\subsection{Considering Multiple Criteria}\index{Considering Multiple Criteria}

In, the prioritized TPMs at the top level reflect the overall performance characteristics of the system as it accomplishes its mission objectives in response to customer needs. There may be numerous factors, such as system size and weight, range and accuracy, speed of performance, capacity, operational availability, reliability and maintainability, supportability, cost, and so on. These <emphasis>measures of effectiveness</emphasis> (MOEs) must be specified in terms of a level of importance, as determined by the customer based on the criticality of the functions to be performed.</para>
<para>For example, there may be certain mission scenarios where system availability is critical, with reliability being less important if there are maintainability considerations built into the system that facilitate ease of repair. Conversely, for missions where the accomplishment of maintenance is not feasible, reliability becomes more important. Thus, the nature and criticality of the mission(s) to be accomplished will lead to the identification of specific requirements and the relative levels of importance of the applicable TPMs.</para>
<para>Given the requirements at the top level, it may be appropriate to develop a “design-objectives” tree similar to that presented in. First , second , and third order (and lower-level) considerations are noted. Based on the established MOEs for the system, a top-down breakout of requirements will lead to the identification of characteristics that should be included and made inherent within the design; for example, a first-order consideration may be <emphasis>system value</emphasis>, which, in turn, may be subdivided into <emphasis>economic</emphasis> factors and <emphasis>technical</emphasis> factors.</para>
<para>Technical factors may be expressed in terms of <emphasis>system effectiveness</emphasis>, which is a function of performance, operational availability, dependability, and so on. This leads to the consideration of such features as speed of performance, reliability and maintainability, size and weight, and flexibility. If maintainability represents a high priority in design, then such features as packaging, accessibility, diagnostics, mounting, and interchangeability should be stressed in the design. Thus, the <emphasis>criteria</emphasis> for design and the associated DDPs should be established early, during conceptual design, and then carried through the entire design cycle. The DDPs establish the exent and scope of the design space within which <emphasis>trade-off</emphasis> decisions may be made. During the process of making these trade-offs, requirements must be related to the appropriate hierarchical level in the system structure (i.e., system, subsystem, and configuration item) as in 

\subsection{System(s) of Systems and System(s) of Systems Engineering}\index{System(s) of Systems and System(s) of Systems Engineering}

System of Systems:Modern systems that comprise system of systems problems are not monolithic; rather they have five common characteristics: operational independence of the individual systems, managerial independence of the systems, geographical distribution, emergent behavior and evolutionary development:
Source:Sage, A.P., and C.D. Cuppan. "On the Systems Engineering and Management of Systems of Systems and Federations of Systems," Information, Knowledge, Systems Management, Vol. 2, No. 4, 2001, pp. 325-345.

System of Systems:System of systems problems are a collection of trans-domain networks of heterogeneous systems that are likely to exhibit operational and managerial independence, geographical distribution, and emergent and evolutionary behaviors that would not be apparent if the systems and their interactions are modeled separately.
Source:DeLaurentis, D. "Understanding Transportation as a System of Systems Design Problem," 43rd AIAA Aerospace Sciences Meeting, Reno, Nevada, January 10-13, 2005. AIAA-2005-0123.

Robert's Comments:I like the above definitions of system of systems a lot, because they make a useful distinction between systems in general, and systems, the engineering of which gives rise to particular challenges and emphases, because of the distinguishing features stated in the definitions. Of course, in a purely literal sense, almost any system is a system of systems (things constructed of two or more interacting parts).

System-of-Systems Engineering (SoSE):System-of-Systems Engineering (SoSE) is a set of developing processes, tools, and methods for designing, re-designing and deploying solutions to System-of-Systems challenges.
Robert's Comments: The above definition of System-of-Systems Engineering (SoSE) sits perfectly with the two definitions of system of systems.

System of Systems Engineering (SoSE):System of Systems (SoS) Engineering is an emerging interdisciplinary approach focusing on the effort required to transform capabilities into SoS solutions and shape the requirements for systems. SoS Engineering ensures that:
1. Individually developed, managed, and operated systems function as autonomous constituents of one or more SoS and provide appropriate functional capabilities to each of those SoS
2. Ppolitical, financial, legal, technical, social, operational, and organizational factors, including the stakeholders' perspectives and relationships, are considered in SoS development, management, and operations
3. A SoS can accommodate changes to its conceptual, functional, physical, and temporal boundaries without negative impacts on its management and operations
a SoS collective behavior, and its dynamic interactions with its environment to adapt and respond, enables the SoS to meet or exceed the required capability.
Source:Systems of Systems Engineering Center of Excellence

Robert's Comments:The first bullet point above seems to capture the essence of System-of-Systems Engineering (SoSE). Regarding the second bullet point, one would hope and expect that applicable political, financial, legal, technical, social, operational, and organizational factors, if any, including the stakeholders' perspectives and relationships, are considered in system development without exception, regardless of the system. To do otherwise is bad systems engineering. The third bullet point seems obtuse. The last bullet point would seem to apply to every engineered system, no exceptions.
System-of-Systems Engineering (SoSE):System of Systems (SoS) Engineering deals with planning, analyzing, organizing, and integrating the capabilities of a mix of existing and new systems into a SoS capability greater than the sum of the capabilities of the constituent parts.
SoSs should be treated and managed as a system in their own right, and should therefore be subject to the same systems engineering processes and best practices as applied to individual systems.
Differs from the engineering of a single system. The considerations should include the following factors or attributes:
Larger scope and greater complexity of integration efforts;
Collaborative and dynamic engineering;
Engineering under the condition of uncertainty;
Emphasis on design optimization;
Continuing architectural reconfiguration;
Simultaneous modeling and simulation of emergent System of Systems behavior; and
Rigorous interface design and management.
Source:Defense Acquisition Guidebook (DAG) - 2006 Definition of SoSE

Robert's Comments:The above definition of System of Systems (SoS) Engineering would seem to apply to the engineering of all systems. Of the list of so-called differences, the first is arguable; continuing architectural reconfiguration may or may not apply to any system depending on circumstances, and the rest of the alleged differences would seem to be factors in engineering of any system.

I am not convinced that a differentiation of "system of systems engineering" and "systems engineering" is helpful, at least with respect to choice of words. However, recognition of the specific additional challenges of engineering systems from subsystems that are subject to operational and managerial independence, may be geographically distributed, and may well exhibit evolutionary behaviors not under the control of the developers of the parent (SoS) system – that is extremely valuable. Research in this area from participants such as Purdue University's College of Engineering (USA), National Centers for System of Systems Engineering (Old Dominion University – USA), and others will contribute to mankind's ability to successfully engineer complex socio-technical systems.

%------------------------------------------------

\section{System Life-Cycle Engineering}\index{System Life-Cycle Engineering}

%------------------------------------------------

\section{Synthesis, Analysis, and Evaluation}\index{Synthesis, Analysis, and Evaluation}

System design is the prime mover of systems engineering, with system design evaluation being its compass. System design requires both integration and iteration, invoking a process that coordinates synthesis, analysis, and evaluation, as is shown conceptually in 6.6 It is essential that the technological activities of synthesis, analysis, and evaluation be integrated and applied iteratively and continuously over the system life cycle. The benefits of continuous improvement in system design are thereby more likely to be obtained.

\subsection{A Morphology for Synthesis, Analysis, and Evaluation}\index{A Morphology for Synthesis, Analysis, and Evaluation}

Figure presents a high-level schematic of the systems engineering process from a product realization perspective. It is a morphology for linking applied research and technologies (Block 0) to customer needs (Block 1). It also provides a structure for visualizing the technological activities of synthesis, analysis, and evaluation. Each of these activities is summarized in the paragraphs that follow, with reference to relevant blocks within the morphology.

Figure 6.6 Here

Synthesis.</title><para><inst></inst>To design is to synthesize, project, and propose what might be for a specific set of customer needs and requirements, normally expressed in functional terms (Block 2). Synthesis is the creative process of putting known things together into new and more useful combinations. Meeting a need in compliance with customer requirements is the objective of design synthesis.</para>
<para>The primary elements enabling design synthesis are the design team (Block 3), supported by traditional and computer-based tools for design synthesis (Block 4). Design synthesis is best accomplished by combining top-down and bottom-up activities (Block 5). Existing and newly developed components, parts, and subsystems are then integrated to generate candidate system designs in a form ready for analysis and evaluation.</para></section>
<section id="ch02lev3sec4"><title id="ch02lev3sec4.title">Analysis.</title><para><inst></inst>Analysis of candidate system and product designs is a necessary but not sufficient ingredient in system design evaluation. It involves the functions of estimation and prediction of DDP values (TPMs) (Block 6) and the determining or forecasting of DIP values from information found in physical and economic databases (Block 7).</para>
<para>Systems analysis and operations research provides a step on the way to system design evaluation, but adaptation of those models and methods to the domain of design is necessary. The adaptation explicitly recognizes DDPs and is developed in <link olinkend="part03" preference="0">Part <xref olinkend="part03" label="III"><inst>III</inst></xref></link> of this book.</para></section>
<section id="ch02lev3sec5"><title id="ch02lev3sec5.title">Evaluation.</title><para><inst></inst>Each candidate design (or design alternative) should be evaluated against other candidates and checked for compliance with customer requirements. Evaluation of each candidate (Block 8) is accomplished after receiving DDP values for the candidate from Block 6. It is the specific values for DDPs (the TPMs) that differentiate (or instance) candidate designs.</para>
<para>DIP values determined in Block 7 are externalities. They apply to and across all candidate designs being presented for evaluation. Each candidate is optimized in Block 8 before being presented for design decision (Block 9). It is in Block 9 that the best candidate is sought. Since the preferred choice is subjective, it should ultimately be made by the customer.

\subsection{Discussion of the 10 Block Morphology}\index{Discussion of the 10 Block Morphology}

This section presents and discusses the functions accomplished by each block in the system design morphology that is exhibited in . The discussion will be at a greater level of detail than the general description of synthesis, analysis, and evaluation given above.
Technologies, exhibited in Block 0, are the product of applied research. They evolve from the activities of engineering research and development and are available to be considered for incorporation into candidate system designs. As a driving force for innovation, technologies are the most potent ingredient for advancing the capabilities of human-made entities.</para>
<para>It is the responsibility of the designer/producer to propose and help the customer understand what might be for each technological choice. Those producers able to articulate and deliver better technological solutions, on time and within budget, will attain and retain a competitive edge in the global marketplace.</para></section>
<section id="ch02lev3sec7"><title id="ch02lev3sec7.title">The Customer (Block 1).</title><para><inst></inst>The purpose of system design is to satisfy customer (and stakeholder) needs and expectations. This must be with the full realization that the perceived success of a design is ultimately determined by the customer, identified in Block 1; the customary being Number 1.</para>
<para>During the design process, all functions to be provided and all requirements to be satisfied should be determined from the perspective of the customer or the customer’s representative. Stakeholder and any other special interests should also be included in the “voice of the customer” in a way that reflects all needs and concerns. Included among these must be ecological and human impacts. Arrow A represents the elicitation of customer needs, desired functionality, and requirements.</para></section>
<section id="ch02lev3sec8"><title id="ch02lev3sec8.title">Need, Functions, and Requirements (Block 2).</title><para><inst></inst>The purpose of this block is to identify and specify the desired behavior of the system or product in functional terms. A market study identifies a need, an opportunity, or a deficiency. From the need comes a definition of the basic requirements, often stated in functional terms. Requirements are the input for design and operational criteria, and criteria are the basis for the evaluation of candidate system configurations.</para>
<para>At this point, the system and its product should be defined by its function, not its form. Arrow A indicates customer inputs that define need, functionality, and operational requirements. Arrows B and C depict the translation and transfer of this information to the design process.</para></section>
<section id="ch02lev3sec9"><title id="ch02lev3sec9.title">The Design Team (Block 3).</title><para><inst></inst>The design team should be organized to incorporate in-depth technical expertise, as well as a broader systems view. Included must be expertise in each of the product life-cycle phases and elements contained within the set of system requirements.</para>
<para>Balanced consideration should be present for each phase of the design. Included should be the satisfaction of intended purpose, followed by producibility, reliability, maintainability, disposability, sustainability, and others. Arrow B depicts requirements and design criteria being made available to the design team and Arrow D indicates the team’s contributed synthesis effort wherein need, functions, and requirements are the overarching consideration (Arrow C).</para></section>
<section id="ch02lev3sec10"><title id="ch02lev3sec10.title">Design Synthesis (Block 4).</title><para><inst></inst>To design is to project and propose what might be. Design synthesis is a creative activity that relies on the knowledge of experts about the state of the art as well as the state of technology. From this knowledge, a few feasible design alternatives are fashioned and presented for analysis. Depending upon the phase of the product life cycle, the synthesis can be in conceptual, preliminary, or detailed form.</para>
<para>The candidate design is driven by both a top-down functional decomposition from Block 2 and a bottom-up combinatorial approach utilizing available system elements from Block 6. Arrow E represents a blending of these approaches. Adequate definition of each design alternative must be obtained to allow for life-cycle analysis in view of the requirements. Arrow F highlights this definition process as it pertains to the passing of candidate design alternatives to design analysis in Block 6.</para></section>
<section id="ch02lev3sec11"><title id="ch02lev3sec11.title">Top-Down and Bottom-Up (Block 5).</title><para><inst></inst>Traditional engineering design methodology is based largely on a bottom-up approach. Starting with a set of defined elements, designers synthesize the system/product by finding the most appropriate combination of elements. The bottom-up process is iterative with the number of iterations determined by the creativity and skill of the design team, as well as by the complexity of the system design.</para>
<para>A top-down approach to design is inherent within systems engineering. Starting with requirements for the external behavior of any component of the system (in terms of the function provided by that component), that behavior is then decomposed. These decomposed functional behaviors are then described in more detail and made specific through an analysis process. Then, the appropriateness of the choice of functional components is verified by synthesizing the original entity. Most systems and products are realized through an intelligent combination of the top-down and bottom-up approaches, with the best mix being largely a matter of judgment and experience.</para></section>
<section id="ch02lev3sec12"><title id="ch02lev3sec12.title">Design Analysis (Block 6).</title><para><inst></inst>Design analysis is focused largely on determining values for cost and effectiveness measures generated during estimation and prediction activities. Models, database information, and simulation are employed to obtain DDP values (or TPMs) for each synthesized design alternative from Block 4. Output Arrow G passes the analysis results to design evaluation (Block 8).</para>
<para>The TPM values provide the basis for comparing system designs against input criteria to determine the relative merit of each candidate. Arrow H represents input from the available databases and from relevant studies.</para></section>
<section id="ch02lev3sec13"><title id="ch02lev3sec13.title">Physical and Economic Databases (Block 7).</title><para><inst></inst>Block 7 provides a resource for the design process, rather than being an actual step in the process flow. There exists a body of knowledge and information that engineers, technologists, economists, and others rely on to perform the tasks of analysis and evaluation. This knowledge consists of physical laws, empirical data, price information, economic forecasts, and numerous other studies and models.</para>
<para>Block 7 also includes descriptions of existing system components, parts, and subsystems, often “commercial off-the-shelf.” It is important to use existing databases in doing analysis and synthesis to avoid duplication of effort. This body of knowledge and experience can be utilized both formally and informally in performing needed studies, as well as in supporting the decisions to follow.</para>
<para>At this point, and as represented by Arrow I, DIP values are estimated or forecasted and provided to the activity of design evaluation in Block 8.</para></section>
<section id="ch02lev3sec14"><title id="ch02lev3sec14.title">Design Evaluation (Block 8).</title><para><inst></inst>Design evaluation is an essential activity within system and product design and the systems engineering process. It should be embedded appropriately within the process and then pursued continuously as design and development progresses.</para>
<para>Life-cycle cost is one basis for comparing alternative designs that otherwise meet minimum requirements under performance criteria. The life-cycle cost of each alternative is determined based on the activity of estimation and prediction just completed. Arrow J indicates the passing of the evaluated candidates to the decision process. The selection of preferred alternative(s) can only be made after the life-cycle cost analysis is completed and after effectiveness measures are defined and applied.</para></section>
<section id="ch02lev3sec15"><title id="ch02lev3sec15.title">Design Decision (Block 9).</title><para><inst></inst>Given the variety of customer needs and perceptions as collected in Block 2, choosing a preferred alternative is not just the simple task of picking the least expensive design. Input criteria, derived from customer and product requirements, are represented by Arrow K and by the DDP values (TPMs) and life-cycle costs indicated by Arrow J. The customer or decision maker must now trade off life-cycle cost against effectiveness criteria subjectively. The result is the identification of one or more preferred alternatives that can be used to take the design process to the next level of detail.</para>
<para>Alternatives must ultimately be judged by the customer. Accordingly, arrow L depicts the passing of evaluated candidate designs to the customer for review and decision. Alternatives that are found to be unacceptable in performance can be either discarded or reworked and new alternatives created. Alternatives that meet all, or the most important, performance criteria can then be evaluated based on estimations and predictions of TPM values, along with an assessment of risk.

%------------------------------------------------

\section{Standards and Systems Engineering Maturity Measures}\index{Standards and Systems Engineering Maturity Measures}

%------------------------------------------------

\section{Implementing Systems Engineering}\index{Implementing Systems Engineering}

Within the context of synthesis, analysis, and evaluation is the opportunity to implement systems engineering over the system life cycle in measured ways that can help ensure its effectiveness. These measured implementations are necessary because the complexity of technical systems continues to increase, and many systems in use today are not meeting the needs of the customer in terms of performance, effectiveness, and overall cost. New technologies are being introduced on a continuing basis, while the life cycles for many systems are being extended. The length of time that it takes to develop and acquire a new system needs to be reduced, the costs of modifying existing systems are increasing, and available resources are dwindling. At the same time, there is more international cooperation, and competition is increasing worldwide.</para>
	<para>There are many categories of human-made systems, and there are several application domains where the concepts and principles of systems engineering can be effectively implemented. Every time that there is a newly identified need to accomplish some function, a new <emphasis>system</emphasis> requirement is established. In each instance, there is a new design and development effort that must be accomplished at the <emphasis>system</emphasis> level. This, in turn, may lead to a variety of approaches at the subsystem level and below (i.e., the design and development of new equipment and software, the selection and integration of new commercial off-the-shelf items, the modification of existing items already in use, or combinations thereof).</para>
<para>Accordingly, for every new customer requirement, there is a needed design effort for the system overall, to which the steps described in Section are applicable. Although the extent and depth of effort will vary, the concepts and principles for bringing a system into being are basically the same. Some specific application areas are highlighted in, and application domains include the following:
1.	</inst><para>Large-scale systems with many components, such as a space-based system, an urban transportation system, a hydroelectric power-generating system, or a health-care delivery system.</para></listitem>
<listitem><inst>	2.	</inst><para>Small-scale systems with relatively few components, such as a local area communications network, a computer system, a hydraulic system, or a mechanical braking system, or a cash receipt system.</para></listitem>
<listitem><inst>3. 	</inst><para>Manufacturing or production systems where there is input–output relationships, processes, processors, control software, facilities, and people.</para></listitem>
<listitem><inst>	4.	</inst><para>Systems where a great deal of new design and development effort is required (e.g., in the introduction of advanced technologies).</para></listitem>
<listitem><inst>	5.	</inst><para>Systems where the design is based largely on the use of existing equipment, commercial software, or existing facilities.</para></listitem>
<listitem><inst>	6.	</inst><para>Systems that are highly equipment, software, facilities, or data intensive.</para></listitem>
<listitem><inst>	7.	</inst><para>Systems where there are several suppliers involved in the design and development process at the local, and possibly international, level.</para></listitem>
<listitem><inst>	8.	</inst><para>Systems being designed and developed for use in the defense, civilian, commercial, or private sectors separately or jointly.</para></listitem>
<listitem><inst>	9.	</inst><para>Human-modified systems wherein a natural system is altered or augmented to make it serve human needs more completely, while being retained/sustained largely in its natural state.

\subsection{Recognizing and Managing Life Cycle Impacts}\index{Recognizing and Managing Life Cycle Impacts}

In evaluating past experiences regarding the development of technical systems, it is discovered that most of the problems experienced have been the direct result of not applying a <emphasis>disciplined</emphasis> top-down “systems approach.” The overall requirements for the system were not defined well from the beginning; the perspective in terms of meeting a need has been relatively “short term” in nature; and, in many instances, the approach followed has been to “deliver it now and fix it later,” using a bottom-up approach to design. The systems design and development process has suffered from the lack of good early planning and the subsequent definition and allocation of requirements in a complete and methodical manner. Yet, it is at this early stage in the life cycle when decisions are made that have a large impact on the overall effectiveness and cost of the system. This is illustrated conceptually in 
<para>Referring to, experience indicates that there can be a large commitment in terms of technology applications, the establishment of a system configuration and its performance characteristics, the obligation of resources, and potential life-cycle cost at the early stages of a program. It is at this point when system-specific knowledge is limited, but when major decisions are made pertaining to the selection of technologies, the selection of materials and potential sources of supply, equipment packaging schemes and levels of diagnostics, the selection of a manufacturing process, the establishment of a maintenance approach, and so on. It is estimated that from 50\% to 75\% of the projected life-cycle cost for a given system can be committed (i.e., “locked in”) based on engineering design and management decisions made during the early stages of conceptual and preliminary design. Thus, it is at this stage where the implementation of systems engineering concepts and principles is critical. It is essential that one start off with a good understanding of the customer need and a definition of system requirements.</para>
<para>The systems engineering process is applicable over all phases of the life cycle, with the greatest benefit being derived from its emphasis on the early stages, as illustrated in . The objective is to influence design early, in an effective and efficient manner, through a comprehensive needs analysis, requirements definition, functional analysis and allocation, and then to address the follow-on activities in a logical and progressive manner with the provision of appropriate feedback. As conveyed in , the overall objective is to influence design in the early phases of system acquisition, leading to the identification of individual discipline-based design needs. These should be applied in a timely manner as one evolves from system level requirements to the design of various subsystems and components thereof.

\subsection{Systems Thinking for Systems Being}\index{Systems Thinking for Systems Being}

Systems thinking is a gateway to seeing interconnections. Once we see a new reality, we cannot go back and ignore it. More importantly, that “seen” has an emotional connection, beautifully captured in the statement by Rusty Schweickart after his experience of seeing his home planet from space. OWH, a man’s mind one expanded cannot go back to …
This is the leading edge of the sustainability movement: the realization that no matter how many solar panels we install, how many green products we consume, how much CO2 we remove from the atmosphere, we will not be living better lives if we do not transform ourselves, our lifestyles, choices and priorities. Sustainability is an inside job, a learning journey to live lightly, joyfully, peacefully, meaningfully.
Systems being involves embodying a new consciousness, an expanded sense of self, a recognition that we cannot survive alone, that a future that works for humanity needs also to work for other species and the planet. It involves empathy and love for the greater human family and for all our relationships – plants and animals, earth and sky, ancestors and descendents, and the many peoples and beings that inhabit our Earth. This is the wisdom of many indigenous cultures around the world, this is part of the heritage that we have forgotten, and we are in the process of recovering.
Systems being and systems living brings it all together: linking head, heart and hands. The expression of systems being is an integration of our full human capacities. It involves rationality with reverence to the mystery of life, listening beyond words, sensing with our whole being, and expressing our authentic self in every moment of our life. The journey from systems thinking to systems being is a transformative learning process of expansion of consciousness—from awareness to embodiment. Kathia Laszlo, Ph.D., directs Saybrook University's program in Leadership of Sustainable Systems.

%------------------------------------------------

\section{Summary and Extensions}\index{Summary and Extensions}

The overarching goal of systems engineering is embodied in the title of this chapter. This goal is to bring successful systems and their products into being. Each and every definition of systems
engineering presented in <link olinkend="ch01lev1sec6" preference="0">Section <xref olinkend="ch01lev1sec6" label="1.6"><inst>1.6</inst></xref></link> supports this desired goal. Accordingly, it is appropriate to devote this chapter to a high-level presentation of the essentials involved in the engineering of systems. Subsequent chapters are devoted to specific topics needed to achieve the goal.</para>


4 This excerpt is from the plenary presentation “Beyond Systems Thinking: The role of beauty and love in the transformation of our world” by Dr. Kathia Laszlo at the 55th Meeting of the International Society for the Systems Sciences at the University of Hull, U.K., on July 21, 2011.


<para>This chapter introduces a technologically based interdisciplinary process encompassing an extension of engineering through all phases of the system life cycle; that is, design and development, production or construction, utilization and support, and phase-out and disposal. The process is derived from the systems concepts and general systems thinking presented in <link olinkend="ch01" preference="0">Chapter <xref olinkend="ch01" label="1"><inst>1</inst></xref></link>. Upon completion, <link preference="0" linkend="ch02">Chapter 6</link> should provide the reader with an in-depth understanding of the following:</para>
    1. <itemizedlist mark="bull" spacing="normal"><listitem><inst>	</inst><para>A more complete definition and description for the category of systems that are human-made, in contrast with the definition of general systems given in <link olinkend="ch01" preference="0">Chapter <xref olinkend="ch01" label="1"><inst>1</inst></xref></link>;</para></listitem>
    2. <listitem><inst>	</inst><para>The product as part of the engineered system, yet distinguishable from it, with emphasis on the system as the overarching entity to be brought into being;</para></listitem>
    3. <listitem><inst>	</inst><para>Product and system categories with life-cycle engineering and design as a generic paradigm for the realization of competitive products and systems;</para></listitem>
    4. <listitem><inst>	</inst><para>Engineering the relationships among systems to achieve sustainability of the product and the environment, synergy among human-made systems, and continuous improvement of human existence;</para></listitem>
    5. <listitem><inst>	</inst><para>System design evaluation and the multiple-criteria domain within which it is best pursued;</para></listitem>
    6. <listitem><inst>	</inst><para>Integration and iteration in system design, invoking the major activities of synthesis, analysis, and evaluation;</para></listitem>
    7. <listitem><inst>	</inst><para>A morphology for synthesis, analysis, and evaluation and its effective utilization within the systems engineering process;</para></listitem>
    8. <listitem><inst>	</inst><para>The importance of investing in systems thinking and engineering early in the life cycle and the importance thereto of systems engineering management; and</para></listitem>
    9. <listitem><inst>	</inst><para>Potential benefits to be obtained from the proper and timely implementation of systems engineering and analysis.</para></listitem></itemizedlist>
<para>The engineered or technical system is to be brought into being; it is a system destined to become part of the human-made world. Therefore, the definition and description of the engineered system is given early in this chapter. It narrows the conceptualization of systems set forth in <link olinkend="ch01lev1sec1" preference="0">Sections <xref olinkend="ch01lev1sec1" label="1.1"><inst>1.1</inst></xref></link> and <link olinkend="ch01lev1sec2" preference="0"><xref olinkend="ch01lev1sec2" label="1.2"><inst>1.2</inst></xref></link>. In most cases, there is a product coexistent with or within the system, and in others the system is the product. But in either case, there must exist a human need to be met.</para>
<para>Since systems are often known by their products, product and system categories are identified as frameworks for study in this and subsequent chapters. Major categories are single-entity product systems and multiple-entity population systems. Availability of these example categories is intended to help underpin and clarify the topics and steps in the process of bringing engineered or technical systems into being.</para>
<para>The product and system life cycle is the <emphasis>enduring paradigm</emphasis> used throughout this book. It is argued that the defense origin of this life-cycle paradigm has profitable applications in the private sector. The life cycle is first introduced in <link linkend="ch02lev1sec2" preference="0" type="backward">Section <xref linkend="ch02lev1sec2" label="2.2"><inst>6.2</inst></xref></link> with two simple diagrams; the first provides the product and the second gives an expanded concurrent life-cycle view. Then, designing for the life cycle is addressed with the aid of more elaborate life-cycle diagrams, showing many more activities and interactions. Other systems engineering process models are then exhibited to conclude an overview of the popular process structures for bringing systems and products into being.</para>
<para>Since design is the fundamental technical activity for both the product and the system, it is important to proceed with full knowledge of all system design considerations. The identification of DDPs and their counterparts, DIPs, follows. Emanating from DDPs are technical performance measures to be predicted and/or estimated. The deviation or difference between predicted TPMs and customer-specified criteria provides the basis for design improvement through iteration, with the expectation of convergence to a preferred design. During this design activity, criteria or requirements must be given center stage. Accordingly, the largest section of this chapter is devoted to an explanation of design evaluation based on customer-specified criteria. The explanation is enhanced by the development and presentation of a 10-block morphology for synthesis, analysis, and evaluation.</para>
<para>This chapter closes with some challenges and opportunities that will surely arise during the implementation of systems engineering. The available application domains are numerous. A general notion is that systems engineering is an engineering interdiscipline in its own right, with important engineering domain manifestations. It is hereby conjectured that the systems engineering body of systematic knowledge will not advance significantly without engineering domain opportunities for application. However, it is clear that significant improvements in domain-specific projects do occur when resources are allocated to systems engineering activities early in the life cycle. Two views of this observed benefit are illustrated in this chapter.</para>
<para>It is recognized that some readers may need and desire to probe beyond the content of this textbook. If so, we would recommend two edited works: The <emphasis>Handbook of Systems Engineering and Management</emphasis>, A. P. Sage and W. B. Rouse (Eds), John Wiley \& Sons, Inc., 2009, augments the technical and managerial topics encompassed by systems engineering. <emphasis>Design and Systems</emphasis>, A. Collen and W. W. Gasparski (Eds), Transaction Publishers, 1995, makes visible the pervasive nature of design in the many arenas of human endeavor from a philosophical and praxiological perspective.
Regarding the body of systems engineering knowledge, there is a timely project being pursued within the INCOSE the SEBoK (Systems Engineering Body of Knowledge) activity involving hundreds of members. Interested individuals may review the current state of development and/or make contributions to it by visiting . An earlier effort along this line was to engage the intellectual leaders of INCOSE (including the authors) in the writing of 16 seminal articles. These were published in the inaugural issue of <emphasis>Systems Engineering</emphasis>, Vol. 1, No. 1, July September 1994. Copies of this special issue and subsequent issues of the journal may be obtained through the INCOSE website.

%------------------------------------------------

%% QUESTIONS, PROBLEMS, AND EXERCISES
% SEA Question Location in \label{sea-Chapter#-Problem#}
\begin{exercises}
    \begin{exercise}
    \label{sea-6-1}
    
    \end{exercise}
    \begin{solution}
    \end{solution}

\end{exercises}
% SKIPPED