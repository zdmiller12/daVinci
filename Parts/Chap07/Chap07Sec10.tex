\section{Schematic Models for the Systems Engineering Process}\index{Schematic Models for the Systems Engineering Process}

System Life-Cycle Models
<para>Experience over many decades indicates that a properly functioning system that is effective and economically competitive cannot be achieved through efforts applied largely after it comes into being. Accordingly, it is essential that anticipated outcomes during, as well as after, system utilization be considered during the early stages of design and development. Responsibility for <emphasis>life-cycle engineering,</emphasis> largely neglected in the past, must become the central engineering focus.</para>
<section id="ch02lev2sec5" label="2.2.1"><title id="ch02lev2sec5.title"><inst></inst>The Product and System Life Cycles</title>
<para>Fundamental to the application of systems engineering is an understanding of the life-cycle process, illustrated for the system in . The product life cycle begins with the identification of a need and extends through conceptual and preliminary design, detail design and development, production or construction, distribution, utilization, support, phase-out, and disposal. The life cycle phases are classified as <emphasis>acquisition</emphasis> and <emphasis>utilization</emphasis> to recognize producer and customer activities.

Figure 6.1 Here
<para>System life cycle engineering goes beyond the product life cycle viewed in isolation. It must simultaneously embrace the life cycle of the production or construction subsystem, the life cycle of the maintenance and support subsystem, and the life cycle for retirement, phase-out, reuse, and disposal as another subsystem. The overall system is made up of four concurrent life cycles progressing in parallel, as is illustrated in . This conceptualization is the basis g. for <emphasis>concurrent engineering</emphasis>.

<para>The need for the product comes into focus first. This recognition initiates conceptual design to meet the need. Then, during conceptual design of the product, consideration should simultaneously be given to its production. This gives rise to a parallel life cycle for a production and/or construction capability. Many producer-related activities are needed to prepare for the production of a product, whether the production capability is a manufacturing plant, construction contractors, or a service activity.</para>
<para>Also shown in is another life cycle of considerable importance that is often neglected until product and production design is completed. This is the life cycle for support activities, including the maintenance, logistic support, and technical skills needed to service the product during use, to support the production capability during its duty cycle, and to maintain the viability of the entire system. Logistic, maintenance, technical support, and regeneration requirements planning should begin during product conceptual design in a coordinated manner.</para>
<para>As each of the life cycles is considered, design features should be integrated to facilitate phase out, regeneration, or retirement having minimal impact on interrelated systems. For example, attention to end-of-life recyclability, reusability, and disposability will contribute to environmental sustainability. Also, the system should be made ready for regeneration by anticipating and addressing changes in requirements, such as increases in complexity, incorporation of planned new technology, likely new regulations, market expansion, and others.</para>
<para>In addition, the interactions between the product and system and any related systems should begin receiving compatibility attention during conceptual design to minimize the need for product and system redesign. Whether the interrelated system is a companion product sold by the same company, an environmental system that may be degraded, or a computer system on which a software product runs, the relationship with the product and system under development must be engineered concurrently.</para></section>
Figure 6.2 sheet 1 and sheet 2 HERE</para></section></section>

<section id="ch02lev2sec6" label="2.2.2"><title id="ch02lev2sec6.title"><inst></inst>Designing for the Life Cycle</title>
<para>Design within the system life-cycle context differs from design in the ordinary sense. Life-cycle-guided design is simultaneously responsive to customer needs (i.e., to requirements expressed in functional terms) and to life-cycle outcomes. Design should not only transform a need into a system configuration but should also ensure the design’s compatibility with related physical and functional requirements. Further, it should consider operational outcomes expressed as producibility, reliability, maintainability, usability, supportability, serviceability, disposability, sustainability, and others, in addition to performance, effectiveness, and affordability.</para>
<para>A detailed presentation of the elaborate technological activities and interactions that must be integrated over the system life-cycle process is given in . The progression is iterative from left to right and not serial in nature, as might be inferred.</para>
<para>Although the level of activity and detail may vary, the life-cycle functions described and illustrated are generic. They are applicable whenever a new need or changed requirement is identified, with the process being common to large as well as small-scale systems. It is essential that this process be implemented completely—at an appropriate level of detail—not only in the engineering of new systems but also in the re-engineering of existing or legacy systems.</para>
<para>Major technical activities performed during the design, production or construction, utilization, support, and phase-out phases of the life cycle are highlighted in. These are initiated when a new need is identified. A planning function is followed by conceptual, preliminary, and detail design activities. Producing and/or constructing the system are the function that completes the acquisition phase. System operation and support functions occur during the utilization phase of the life cycle. Phase-out and disposal are important final functions of utilization to be considered as part of design for the life cycle.</para>
<para>The numbered blocks in “map” and elaborate on the phases of the life cycles depicted in <as follows:</para>
<para>The communication and coordination needed to design and develop the product, the production capability, the system support capability, and the relationships with interrelated systems—so that they traverse the life cycle together seamlessly—is not easy to accomplish. Progress in this area is facilitated by technologies that make more timely acquisition and use of design information possible. Computer-Aided Design (CAD) technology with internet/intranet connectivity enables a geographically dispersed multidiscipline team to collaborate effectively on complex physical designs.</para>
<para>For certain products, the addition of Computer-Aided Manufacturing (CAM) software can automatically translate approved three-dimensional CAD drawings into manufacturing instructions for numerically controlled equipment. Generic or custom parametric CAD software can facilitate exploration of alternative design solutions. Once a design has been created in CAD/CAM, iterative improvements to the design are relatively easy to make. The CAD drawings also facilitate maintenance, technical support, regeneration (re-engineering), and disposal. A broad range of other electronic communication and collaboration tools can help integrate relevant geographically dispersed design and development activities over the life cycle of the system.</para>
<para>Concern for the entire life cycle is particularly strong within the U.S. Department of Defense (DOD) and its non-U.S. counterparts. This may be attributed to the fact that acquired defense systems are owned, operated, and maintained by the DOD. This is unlike the situation most often encountered in the private sector, where the consumer or user is usually not the producer. Those private firms serving as defense contractors are obliged to design and develop in accordance with DOD directives, specifications, and standards. Because the DOD is the customer and also the user of the resulting system, considerable DOD intervention occurs during the acquisition phase.<footnoteref preference="1" label="5" role="generated" linkend="ch02fn05"/>3</para>
<para>Many firms that produce for private-sector markets have chosen to design with the life cycle in mind. For example, design for energy efficiency is now common in appliances such as water heaters and air conditioners. Fuel efficiency is a required design characteristic for automobiles. Some truck manufacturers promise that life-cycle maintenance costs will be within stated limits. These developments are commendable, but they do not go far enough. When the producer is not the consumer, it is less likely that potential operational problems will be addressed during development. Undesirable outcomes too often end up as problems for the user of the product instead of the producer.
