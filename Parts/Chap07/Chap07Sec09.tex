\section{Systems Engineering Process Models}\index{Systems Engineering Process Models}

System Life-Cycle Models

<para>Experience over many decades indicates that a properly functioning system that is effective and economically competitive cannot be achieved through efforts applied largely after it comes into being. Accordingly, it is essential that anticipated outcomes during, as well as after, system utilization be considered during the early stages of design and development. Responsibility for <emphasis>life-cycle engineering,</emphasis> largely neglected in the past, must become the central engineering focus.

</para><section id="ch02lev2sec5" label="2.2.1"><title id="ch02lev2sec5.title"><inst></inst>The Product and System Life Cycles
</title>
<para>Fundamental to the application of systems engineering is an understanding of the life-cycle process, illustrated for the system in . The product life cycle begins with the identification of a need and extends through conceptual and preliminary design, detail design and development, production or construction, distribution, utilization, support, phase-out, and disposal. The life-cycle phases are classified as <emphasis>acquisition</emphasis> and <emphasis>utilization</emphasis> to recognize producer and customer activities.

\subsection{The Systems Engineering Process}\index{The Systems Engineering Process}

Although there is general agreement regarding the principles and objectives of systems engineering, its actual implementation will vary from one system and engineering team to the next. The process approach and steps used will depend on the nature of the system application and the backgrounds and experiences of the individuals on the team. To establish a common frame of reference for improving communication and understanding, it is important that a “baseline” be defined that describes the systems engineering process, along with the essential life-cycle phases and steps within that process. Augmenting this common frame of reference are top-down and bottom-up approaches. And, there are other process models that have attracted various degrees of attention. Each of these topics is presented in this section.
</para>
Model Based Systems Engineering

Model based systems engineering has process model (not decision model) characteristics. In the INCOSE vision 2020 document, v Model Based Systems Engineering (MBSE) is the formalized application of modeling to support the SEP. It brings together, normally on a computer-based platform, the information to support functional analysis and allocation, requirements management and decomposition, synthesis, analysis, verification and validation. Figures 6.4 and 6.5. provide the essential framework for the MBSE model.

Figure 6.4 HERE
Figure 6.5 HERE
	
Specifically, revisit the information flows in Figure 6.5 in light of the abstract regarding the “On line linking of Individual Designers Decisions . . . FIND IT