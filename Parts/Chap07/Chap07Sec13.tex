\section{Model-Based Systems Engineering}\index{Model-Based Systems Engineering}

Model based systems engineering has process model (not decision model) characteristics. In the INCOSE vision 2020 document, v Model Based Systems Engineering (MBSE) is the formalized application of modeling to support the SEP. It brings together, normally on a computer-based platform, the information to support functional analysis and allocation, requirements management and decomposition, synthesis, analysis, verification and validation. Figures 6.4 and 6.5. provide the essential framework for the MBSE model.

Figure 6.4 HERE
Figure 6.5 HERE
	
Specifically, revisit the information flows in Figure 6.5 in light of the abstract regarding the “On line linking of Individual Designers Decisions . . . FIND IT

One of the challenges to MBSE is a reliance on network effects, the idea that the benefit to the team grows nonlinearly with the adoption by other users. Furthermore, there is little data about adoption of modeling by engineering-driven practitioners or firms. Obtaining and leveraging data in this area is important to assess whether engineers apply effective modeling techniques in practice.

Finally, I am enclosing the Exploratory Research Proposal in reoriented form now entitled:

EXPLORATORY RESEARCH ON THE ON-LINE LINKING OF
INDIVIDUAL DESIGNER’S DECISIONS TO THE SYSTEM LEVEL AND
TO SYSTEM LIFE -CYCLE VALUE IN THE FACE OF MULTIPLE CRITERIA

Our SDEM/SEP (System Design Evaluation Methodology Integrated With The Systems Engineering Process) capability provides the enabling foundation and visible interface with firms comprising the Software Productivity Consortium (SPC). New funding effective January 1 for SDEM/SEP brings the total to \$120,000, provided jointly by SPC and the Virginia Center for Innovative Technology. It is the SPC member firms to whom we will go to support a comprehensive proposal based on findings from the requested Small Grant for Exploratory Research.
Emerging academic and educational experiences, including massive open online courses (MOOCs) and similar online offerings, are a new opportunity to learn about industry practices. In this paper, we draw on data from the “Architecture and Systems Engineering: Models and Methods to Manage Complex Systems” online program from the Massachusetts Institute of Technology (MIT), which has enrolled 4200 participants, to understand how models are used in practice. We use aggregated data from enrollment surveys and poll questions to investigate model use by participants and their organizations.
The results in this paper demonstrate that while modeling is widely recognized among course participants for its potential, the actual deployment of models against problems of interest is substantially lower than the participant-stated potential. We also find that in the practice of engineering, many participants reported low use of basic modeling best practices, such as documentation and specialized programs. Furthermore, we find that participants overstate the deployment of modeling practices within their organizations.
We believe that this research is a fresh use of descriptive analysis to report on actual modeling practices. Future research should target more detail about modeling practices in engineering-driven firms, and the reasons why modeling adoption is low or overstated.
The “Architecture and Systems Engineering: Models and Methods to Manage Complex Systems” certificate program is a series of four online courses offered by the Massachusetts Institute of Technology (MIT) for professional audiences covering models in engineering and model-based system engineering. For more information, visithttps://sysengonline.mit.edu/.

Modern educational experiences include environments beyond the traditional classroom. Massive open online courses (MOOCs) assemble learners from around the world from different backgrounds. For example, edX.org lists over 1900 available courses in Introductory, Intermediate, and Advanced levels at the time of this writing. These programs extend the educational reach across geographies, industries, firms, and knowledge silos through interactive videos, surveys, and discussion boards. Beyond exposure to course content, MOOC participants engage in both individually-paced and community learning environments.
MOOCs provide access to large groups of participants simultaneously, orders of magnitude larger than is otherwise practical within traditional constraints. In a prior study of participant “subpopulations” in MOOCs, Kizilcec et al. (2013) concludes that effectively leveraging of participants’ “prior knowledge” enhances MOOC content effectiveness because it “mediates encounters with new information.” We propose that the overall MOOC environment and interactive mechanisms facilitate a practice-sharing role, in addition to the traditional content delivery style of online and video education.
The “Architecture and Systems Engineering: Models and Methods to Manage Complex Systems” certificate program offered by the Massachusetts Institute of Technology (MIT) is a series of four online courses with enrollment of 4200 participants with common interest in model-based systems engineering over the past two years. As career opportunity becomes increasingly chaotic and professionals endure a “life-long” state of “transition” (Vuolo et al., 2012), the participants who enroll in these courses for career training and growth represent a potential population sample to learn about engineering practices. While it is debatable whether this fee-for-service offering with fewer than 10,000 participants fits the definition of a MOOC (Jordan 2015), participants with experience in industry engage through both public and corporate-supported program enrollments.
During the “Architecture and Systems Engineering” program at MIT, participants complete an enrollment survey and respond to interactive polls about modeling in engineering.Responses reported by participants through these mechanisms is an opportunity to learn about the modeling use and sophistication in engineering-driven firms. This data is sufficiently anonymized to support useful learning about models in engineering without exposing participants or their organizations to privacy risk. In this paper, our research objective is to analyze survey and polls data from these courses to measure model use and user sophistication in engineering-driven firms. Through a descriptive analysis of participants and the firms they represent, this paper seeks to fill a perceived gap in the currently available research in this area.
This paper begins with a research outline and analysis of methods used on survey and polls data from the “Architecture and Systems Engineering” program at MIT. The results are then presented, followed by discussions about the implications about the use of models in industry, the future of MBSE, the use of MOOC data for research, and approaches for studies in engineering.
INTRODUCTION: Model-based Systems Engineering

This is Model-based Systems Engineering

Systems engineering is concerned with the design, building, and use of concrete entities such as engines, machines, and structures. It is equally concerned with business systems which are composed of processes. In order to engage in systems engineering we need an organized means of thinking about those systems in their operational contexts. 

The discipline of systems engineering uses the knowledge and techniques from various branches of engineering and science in the planning and development of new solutions for answering a need. In fact, systems engineering is a multi-disciplinary approach to satisfying the needs of stakeholders through the creation or improvement of a system. The focus of the systems engineer is on the whole system and the system’s external interfaces. 

Model-based systems engineering quite literally means the use of a model (or models) to gain insight into engineering the solution for a project. Each project model uses a restricted language framework that spans the problem and solution spaces. This primer discusses the use of such a model- its advantages and the various aspects of its use.

Why use a model-based approach?
There are major advantages arising from using models as the basis of systems engineering. Models offer a complete consideration of the entire engineering problem, the use of a consistent language to describe the problem and the solution, the production of a coherently designed solution and the comprehensive and verifiable answering of all the system requirements posed by the problem. All of these offer significant advantages in seeking a solution to the systems design problem at hand.

Basic Concepts

Although the term ``system'' is defined in a variety of ways in the systems engineering community, most definitions are similar to the one used in the DoDAF framework - ``any organized assembly of resources and procedures united and regulated by interaction or interdependence to accomplish a set of specific functions.'' However, this primer takes an intentionally broad view of the systems definition because it allows for the use of model-based systems engineering to undertake and provide solutions for a much wider range of problems than would be possible if the operating definition of ``system'' were restricted to a narrower definition.

The INCOSE definition of system is ``a construct or collection of different entities that together produce results not obtainable by the entities alone.''  Some examples of systems include:

\begin{itemize}
\item A set of things working together as parts of a mechanism or an interconnecting network
\item A set of organs in the body with a common structure or function
\item A group of related hardware units or software programs or both, especially when dedicated to a single application
\item A major range of strata that corresponds to a period in time, subdivided into series
\item A group of celestial objects connected by their mutual attractive forces, especially moving in orbits about a center
The entities of a human constructed system can include people, hardware, software, facilities, policies, and documents; that is, all the things required to produce system-level results. These results include system-level qualities, properties, characteristics, functions, behavior and performance. The value added by the system as a whole, beyond that contributed independently by the parts, is primarily created by the relationships among the parts. In other words, the ``value-add'' of the system emerges in the synergy created when the parts come together.
\end{itemize}

Systems Composition
A system begins with an idea that must be translated into reality. The model of a system serves to link the engineered system “reality” to the theoretical idea and vice versa (bi-directionally). Using a model clearly shows when and how the theory explains reality and how reality confirms the theory. 

Properties – Within the boundary of a system there are three kinds of properties:
Entities – these are the parts (things or substances) that make up a system. These parts may be atoms or molecules, larger bodies of matter like sand grains, raindrops, plants, animals, or even components like motors, planes, missiles, etc.
Attributes – attributes are characteristics of the entities that may be perceived and measured. For example: quantity, size, color, volume, temperature, reliability, maintainability and mass.
Relationships – relationships are the associations that occur between entities and attributes. These associations are based on cause and effect. 

Language – When combined these entities, attributes and relationships form a system “language.” This language is fundamental to being able to describe and communicate the system among the engineering team as well as to other stakeholders.
Layers – The system is represented hierarchically, allowing it to be understood as decomposable into meaningful layers of subunits. These subunits are conventionally named: 
    • a system is composed of subsystems; 
    • subsystems in turn are composed of assemblies; 
    • assemblies are composed of subassemblies, and 
    • subassemblies are composed of parts. 
It is important to note that what may be considered a ``part'' in the context of a particular system may be a complete ``system'' in its own right.
Often the terms used in this hierarchy are not well specified; some engineers use the term sub-subsystem, others use the terms component and subcomponent in the hierarchy; contributing to confusion. In order to avoid such usage confusion, the term ``component'' is used here as an abstract term representing the physical or logical entity that performs a specific function or functions.

Systems engineering begins with the identification of the needs of the users and the stakeholders while assuring that the right problem is addressed. The systems engineer crafts those needs into a definition of the system, identifies the functions that meet those needs, allocates those functions to the system entities (components) and finally confirms that the system performs as designed and satisfies the needs of the user.

Systems engineering is both a technical and a management process. The technical process addresses the analytical, design, and implementation efforts necessary to transform the operational need into a system of the proper size and configuration and produces the documentation necessary to implement, operate and maintain the system. The management process involves planning, assessing risks, integrating the various engineering specialties and design groups, maintaining configuration control, and continuously auditing the effort to ensure that cost, schedule, and technical performance objectives are satisfied.

To be effective in all these areas systems engineering must, therefore, be an organized, repeatable, iterative and convergent approach to the development of complex systems. The approach must be ``organized'' because without an organized approach the details of the system under development will be overlooked, confused, and misunderstood. The approach must be “repeatable” so that it will apply to other system development efforts in such a way that creates reasonable assurances of success. Such an approach should be both iterative and convergent as well, which means the engineering processes repeat at each level of system design and insure process convergence to a solution.

The Process. The first task of the systems engineer is to develop a clear statement of the problem setting out what issue or issues are being addressed by the proposed system. This involves working with others (especially system stakeholders and subject matter experts) to identify the stated requirements that govern what would characterize an acceptable solution. The systems engineer must provide design focus and facilitate proper and effective communication between the various subject matter experts and the stakeholders. The systems engineer must have a broad knowledge base to understand the various disciplines involved in the development of the system, and to participate in and evaluate system level design decisions and to resolve system issues. Often some system requirements conflict with each other and the systems engineer must resolve these conflicts in a way that does not lose sight of the system’s purpose. The goal of the engineer is to develop a system that maximizes the strengths and benefits of the system while minimizing its flaws and weaknesses. 

For illustrative purposes, we will present the Model-based systems engineering effort from a top-down perspective. Figure 1- Intra-layer Systems Engineering Process, and Figure 2- Onion Layers, illustrate the process of working within a layer and between layers respectively.
 
Figure 1- Intra-layer Systems Engineering Process

Once the problem is clearly defined, the process steps follow the flow in Figure 1. The originating requirements (which identify what the system will provide) are extracted from the source documentation, market studies or other expressions of the system definition and analyzed. This analysis identifies numerous aspects of the desired system. Analyzing the requirements allows the systems engineer to define the system boundaries and identify what is inside and outside those boundaries. The definition of these boundaries- an often-overlooked step- is critical to the implementation of the system. Any change in the system boundaries will impact the complexity and character of the way in which the system interacts and interfaces with the environment.

Specifying the functional requirements and the interactions of the system with the external entities leads directly to developing a clear picture of the system. It is critical that the functional requirements (the “what”) are understood before attempting to define the implementation (the “how”) of the system. Therefore, this analysis is repeated throughout the system design process to test the rigor and integrity of the system.

In parallel with the functional behavior definition, what must be built to perform the needed behavior is derived through the decomposition of the system into components. The systems engineer then analyzes the constraints and allocates system behavior to the physical components. This leads to the specification of each system component. Because of this allocation, the identification and definition of all interfaces between the physical parts of the system, including hardware, software, and people can take place. 

The engineer uses a formal specification language to characterize the various design entities (requirements, functions, components, etc.) in a repository. Using this language and a repository allows the engineer to construct a system ``model.'' By capturing all the system information in the repository and correcting all errors, the engineer builds the repository to contain the system model from which the design team will produce the system, segment, and interface specifications.

It should be emphasized that these activities are performed concurrently or in parallel but not independently. The activities in one area are influenced by and have influence on the other activities. The layer-by-layer approach of MBSE assures that these interrelationships are considered in context. Approaches that involve ``deep dives'' into one area (e.g., requirements) run the substantial risk of obscuring the systemic risks incurred when the complex relationships are not fully considered. The central power of the MBSE approach lies in its careful and complete consideration of the system design in an orderly and systematic fashion. This can only happen through the orderly process and excellent communication which must be the hallmarks of an effective systems engineering process. 

This look at the systems engineering process uses a description that best fits top down systems design. With suitable approach variations the systems engineer can address reverse or bottom-up and middle-out systems engineering perspectives as well. For example, in conducting reverse engineering, the engineer would begin with the existing physical description and work from the physical representation and interfaces to ultimately derive the original system’s requirements. Once these are obtained, the engineering process would proceed to incorporate the desired changes and enhancements as in a top-down design.

Figure 2 above shows the layers with the domains represented in each. The movement of the system design process from a conceptual to a detailed description happens by moving downward with increasing detail or granularity from one layer to the next. Each of the iterations analyzes a layer within the system design process. Beginning with the basic, high-level user requirements serves to define broad system characteristics and objectives. Clarity is brought to these high-level expressions until the process has defined the system to a point of sufficient granularity to allow the system’s physical implementation to begin.

The ``Onion'' Model. As the threads are developed in increasing detail or ``granularity'' a layered structure takes shape. The engineering process follows these layers drilling deeper and deeper into the system design. Every iteration of the systems engineering process increases the level of specificity, removes ambiguity and resolves unknowns. The domains (Requirements, Functional Behavior, Architecture and Validation and Verification) are all addressed in context as each successive layer is peeled back. 

This approach to solving the problem in layers is the heart of MBSE. With this approach (addressing all domains in each layer), there comes an assurance that all aspects of the engineering problem at hand are addressed completely and consistently. The layers and their interrelationships also provide a solution that can be easily verified and validated against the needs that created the problem. By ``peeling the onion,'' we can be assured that we have indeed addressed the problem in a meaningful and productive way converging on a solution that responds to the entire problem in all its aspects.

The application of an iterative, convergent (layer-by-layer) systems engineering process leads to reducing ambiguity by resolving open and uncovered issues and mitigating risks. Systems engineers and stakeholders collaborate with other team members to make decisions that advance the design to completion. When the design is complete, validation and verification can take the form of ``walking through the design,'' verifying that all the requirements are valid and can be verified, all the functions are present that are necessary to meet the requirements, all appropriate analytical and simulation activities have been performed and all the components needed to perform the functions at this level are defined. In other words, the engineers and stakeholders can verify that the layer-by-layer process has converged on an engineering solution that satisfies all the requirements for the system being designed.