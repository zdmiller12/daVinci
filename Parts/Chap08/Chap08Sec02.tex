\section{Conceptual System Design}\index{Conceptual System Design}

Conceptually sound system design focuses on what the system is intended to do before determining what the system is, with form following function. This focus is most effective when based on essential design dependent parameters, recognizing the concurrent life-cycle factors of production, support, maintenance, phase-out, recycle, and disposal. It invokes iterating synthesis, analysis, and evaluation as in Chapter 6. These considerations are germane to system and product design as embedded within the systems engineering process and its inherent life cycle.

\subsection{Problem Definition and Need Identification}\index{Problem Definition and Need Identification}

The systems engineering process usually begins with the identification of a ``want'' or ``desire'' for something based on some ``real'' or ``perceived'' deficiency. For instance, suppose that a current system capability is not adequate in terms of meeting certain required performance goals, is not available when needed, cannot be properly supported, or is too costly to operate. Or, there is a lack of capability to communicate between point A and point B, at a desired rate X, with reliability of Y, and within a specified cost of Z. Or, a regional transportation authority is faced with the problem of providing for increased two-way traffic flow across a river that divides a growing municipality (to illustrate the overall process, this particular example is developed further in, and under preliminary design in).

It is important to commence by first defining the ``problem'' and then defining the need for a specific system capability that (hopefully) is responsive. It is not uncommon to first identify some ``perceived'' need which, in the end, doesn’t really solve the problem at hand. In other words, why is this particular system capability needed? Given the problem definition, a new system requirement is defined along with the priority for introduction, the date when the new system capability is required for customer use, and an estimate of the resources necessary for its acquisition. To ensure a good start, a comprehensive statement of the problem should be presented in specific qualitative and quantitative terms and in enough detail to justify progressing to the next step. It is essential that the process begin by defining a ``real'' problem and its importance.

The necessity for identifying the need may seem to be basic or self-evident; however, a design effort is often initiated as a result of a personal interest or a political whim, without the requirements first having been adequately defined. In the software and information technology area, in particular, there is a tendency to accomplish considerable coding and software development at the detailed level before adequately identifying the real need. In addition, there are instances when engineers sincerely believe that they know what the customer needs, without having involved the customer in the ``discovery'' process. The ``design-it-now-fix-it-later'' philosophy often prevails, which, in turn, leads to unnecessary cost and delivery delay.

Defining the problem is often the most difficult part of the process, particularly if there is a rush to get underway. The number of false starts and the resulting cost commitment can be significant unless a good foundation is laid from the beginning, as illustrated by . A complete description of the need, expressed in quantitatively related criteria whenever possible, is essential. It is important that the problem definition reflects true customer requirements, particularly in an environment of limited resources.

Having defined the problem completely and thoroughly, a needs analysis should be performed with the objective of translating a broadly defined ``want'' into a more specific system-level requirement. The questions are as follows: What is required of the system in ``functional'' terms? What functions must the system perform? What are the ``primary'' functions? What are the ``secondary'' functions? What must be accomplished to alleviate the stated deficiency? When must this be accomplished? Where is it to be accomplished? How many times or at what frequency must this be accomplished?

There are many basic questions of this nature, making it important to describe the customer requirements in a functional manner to avoid a premature commitment to a specific design concept or configuration. Unless form follows function, there is likely to be an unnecessary expenditure of valuable resources. The ultimate objective is to define the whats first, deferring the hows until later.

Identifying the problem and accomplishing a needs analysis in a satisfactory manner can best be realized through a team approach involving the customer, the ultimate consumer or user (if different from the customer), the prime contractor or producer, and major suppliers, as appropriate. The objective is to ensure that proper and effective communications exist between all parties involved in the process. Above all, the ``voice of the customer'' must be heard, providing the system developer(s) an opportunity to respond in a timely and appropriate manner.

\subsection{Advanced System Planning and Architecting}\index{Advanced System Planning and Architecting}

Given an identified ``need'' for a new or improved system, the advanced stages of system planning and architecting can be initiated. Planning and architecting are essential and coequal activities for bringing a new or improved capability into being. The overall ``program requirements'' for bringing the capability into being initiate an advanced system planning activity and the development of a <emphasis>program management plan</emphasis> (PMP), shown as the second block in . While the specific nomenclature for this top-level plan may differ with each program, the objective is to prepare a ``management-related'' plan providing the necessary guidance for all subsequent managerial and technical activities.

Referring to, the PMP guides the development of requirements for implementation of a systems engineering program and the preparation of a systems engineering management plan (SEMP), or system engineering plan (SEP). The ``technical requirements'' for the system are simultaneously determined. This involves development of a system-level architecture (functional first and physical later) to include development of system operational requirements, determination of a functional architecture, proposing alternative technical concepts, performing feasibility analysis of proposed concepts, selecting a maintenance and support approach, and so on, as is illustrated by. The results lead to the preparation of the system specification (Type A). The preparation of the SEMP and the system specification should be accomplished concurrently in a coordinated manner. The two documents must ``talk'' to each other and be mutually supportive.

It can be observed from and  that the identified requirements are directly aligned and supportive of the activities and milestones shown in . The system specification (Type A) contains the highest-level architecture and forms the basis for the preparation of all lower-level specifications in a top-down manner. These lower-level specifications include development (Type B),product (Type C), process (Type D), and material (Type E) specifications, and are described further in . The systems engineering management plan is addressed in detail in . The systems engineering process and the steps illustrated in the lower part of are described in the remaining sections of this chapter.

Conceptual design is the first and most important phase of the system design and development process. It is an early and high-level life-cycle activity with the potential to establish, commit, and otherwise predetermine the function, form, cost, and development schedule of the desired system and its product(s). The identification of a problem and an associated definition of need provide a valid and appropriate starting point for conceptual system design.	

Selection of a path forward for the design and development of a preferred system architecture that will ultimately be responsive to the identified customer need is a major purpose of conceptual design. Establishing this foundation early, as well as initiating the early planning and evaluation of alternative technological approaches, is a critical initial step in the implementation of the systems engineering process. Systems engineering, from an organizational perspective, should take the lead in the solicitation of system 

\subsection{System Design and Feasibility Analysis}\index{System Design and Feasibility Analysis}

Having justified the need for a new system, it is necessary to (1) identify various system-level design approaches or alternatives that could be pursued in response to the need; (2) evaluate the feasible approaches to find the most desirable in terms of performance, effectiveness, maintenance and sustaining support, and life-cycle economic criteria; and (3) recommend a preferred course of action. There may be many possible alternatives; however, the number of these must be narrowed down to those that are physically feasible and realizable within schedule requirements and available resources.

In considering alternative system design approaches, different technology applications are investigated. For instance, in response to the river crossing problem (identified in ), alternative design concepts may include a tunnel under the river, a bridge spanning the river, an airlift capability over the river, the use of barges and ferries on the river, or possibly re-routing the river itself. Then a feasibility study would be accomplished to determine a preferred approach. In performing such a study, one must address limiting factors such as geological and geotechnical, atmospheric and weather, hydrology and water flow, as well as the projected capability of each alternative to meet life-cycle cost objectives. In this case, the feasibility results might tentatively indicate that some type of bridge structure spanning the river appears to be best.

At a more detailed level, in the design of a communications system, is a fiber optics technology or the conventional twisted-wire approach preferred? In aircraft design, to what extent should the use of composite materials be considered? In automobile design, should high-speed electronic circuitry in a certain control application be incorporated or should an electromechanical approach be utilized? In the design of a data transmission capability, should a digital or an analogue format be used? In the design of a process, to what extent should embedded computer capabilities be incorporated? Included in the evaluation process are considerations pertaining to the type and maturity of the technology, its stability and growth potential, the anticipated lifetime of the technology, the number of supplier sources, and so on.

It is at this early stage of the system life cycle that major decisions are made relative to adopting a specific design approach and related technology application. Accordingly, it is at this stage that the results of such design decisions can have a great impact on the ultimate behavioral characteristics and life-cycle cost of a system (refer again to ). Technology applications are evaluated and, in some instances where there is not enough information available (or a good solution is not readily evident), research may be initiated with the objective of developing new knowledge to enable other approaches. Finally, it must be agreed that the ``need'' should dictate and drive the ``technology,'' and not vice versa.

The identification of alternatives and feasibility considerations will significantly impact the operational characteristics of the system and its design for constructability, producibility, supportability, sustainability, disposability, and other design characteristics. The selection and application of a given technology has reliability and maintainability implications, may impact human performance, may affect construction or manufacturing and assembly operations in terms of the processes required, and may significantly impact the need for system maintenance and support. Each will certainly affect life-cycle cost differently. Thus, it is essential that life-cycle considerations be an inherent part of the process of determining the feasible set of system design alternatives.

\subsection{System Operational Requirements}\index{System Operational Requirements}

Once the need and technical approach have been defined, it is necessary to translate this into some form of an ``operational scenario,'' or a set of operational requirements. At this point, the following questions may be asked: What are the anticipated types and quantities of equipment, software, personnel, facilities, information, and so on, required, and where are they to be located? How is the system to be utilized, and for how long? What is the anticipated environment at each operational site (user location)? What are the expected interoperability requirements (i.e., interfaces with other ``operating'' systems in the area)? How is the system to be supported, by whom, and for how long? The answer to these and comparable questions leads to the definition of system operational requirements, the follow-on maintenance and support concept, and the identification of specific design-to criteria, and related guidelines.</para>

Defining Operational requirements. System operational requirements should be identified and defined early, carefully, and as completely as possible, based on an established need and selected technical approach. The operational concept and scenario as defined herein is identified in and . It should include the following:

\begin{enumerate}
\item Identification of the prime and alternate or secondary missions of the system. What is the system to accomplish? How will the system accomplish its objectives? The mission may be defined through one or a set of scenarios or operational profiles. It is important that the <emphasis>dynamics</emphasis> of system operating conditions be identified to the extent possible.
\item Performance and physical parameters: Definition of the operating characteristics or functions of the system (e.g., size, weight, speed, range, accuracy, flow rate, capacity, transmit, receive, throughput, etc.). What are the critical system performance parameters? How are they related to the mission scenario(s)?
\item Operational deployment or distribution: Identification of the quantity of equipment, software, personnel, facilities, and so on and the expected geographical location to include transportation and mobility requirements. How much equipment and associated software is to be distributed, and where is it to be located and for how long? When does the system become fully operational?
\item Operational life cycle (horizon): Anticipated time that the system will be in operational use (expected period of sustainment). What is the total inventory profile throughout the system life cycle? Who will be operating the system and for what period of time?
\item Utilization requirements: Anticipated usage of the system and its elements (e.g., hours of operation per day, percentage of total capacity, operational cycles per month, facility loading). How is the system to be used by the customer, operator, or operating authority in the field?
\item Effectiveness factors: System requirements specified as figures-of-merit (FOMs) such as cost/system effectiveness, operational availability (Ao), readiness rate, dependability, logistic support effectiveness, mean time between maintenance (MTBM), failure rate (), maintenance downtime (MDT), facility utilization (in percent), operator skill levels and task accomplishment requirements, and personnel efficiency. Given that the system will perform, how effective or efficient is it? How are these factors related to the mission scenario(s)?
\item Environmental factors: Definition of the environment in which the system is expected to operate (e.g., temperature, humidity, arctic or tropics, mountainous or flat terrain, airborne, ground, or shipboard). This should include a range of values as applicable and should cover all transportation, handling, and storage modes. How will the system be handled in transit? To what will the system be subjected during its operational use, and for how long? A complete environmental profile should be developed.
\end{enumerate}

In addition to defining operational requirements that are system specific, the system being developed may be imbedded within an overall higher-level structure making it necessary to give consideration to interoperability requirements. For example, an aircraft system may be contained within a higher-level airline transportation system, which is part of a regional transportation capability, and so on. There may be both ground and marine transportation systems within the same overall structure, where major interface requirements must be addressed when system operational requirements are being defined.

In some instances, there may be both vertical and horizontal impacts when addressing the system in question, within the context of some larger overall configuration. There are two important questions to be addressed: <emphasis>What is the potential impact of this new system on the other systems in the same SOS configuration? What are the external impacts from the other systems within the same SOS structure on this new system?

Illustrating System Operational Requirements. Further consideration of system operational requirements (as presented in is provided through five sample illustrations, each covering different degrees or levels of detail. The first illustration is an extension of the river crossing problem introduced in . The second illustration, covering operational requirements in more depth, is an aircraft system with worldwide deployment. The third illustration is a communication system with ground and airborne applications. The fourth illustration deals with commercial airline capability for a metropolitan area. The fifth illustration considers a hospital as part of a community healthcare system. Finally, it is noted that there may exist many other applications and situations to which the illustrated methodology applies.<

The intent of these examples is to encourage consideration of operational requirements at a greater depth than before and to do so early in the system life cycle when the specification of such requirements will have the greatest impact on design, as emphasized in. While it may be easier to delay such considerations until later in the system design process, the consequences of such are likely to be very costly in the long term. The objective in systems engineering is to ``force'' considerations of operational requirements as early as practicable in the design process.

\subsection{System Maintenance and Support}\index{System Maintenance and Support}

In addressing system requirements, the normal tendency is to deal primarily with those elements of the system that relate directly to the ``performance of the mission,'' that is, prime equipment, operator personnel, operational software, operating facilities, and associated operational data and information. At the same time, too little attention is given to system maintenance and support and the sustainment of the system throughout its planned life cycle. In general, the emphasis has been directed toward only part of the system and not the entire system as an entity, which has led to costly results in the past.

To realize the overall benefits of systems engineering, it is essential that <emphasis>all</emphasis> elements of the system be considered on an integrated basis from the beginning. This includes not only the prime mission-related elements of the system but the maintenance and support infrastructure as well. The prime system elements must be designed in such a way that they can be effectively and efficiently supported through the entire system life cycle, and the maintenance and support infrastructure must be responsive to this requirement. This, in turn, means that one should also address the design characteristics as they pertain to transportation and handling equipment, test and support equipment, maintenance facilities, the supply chain process, and other applicable elements of logistic support.

The maintenance and support concept developed during the conceptual design phase evolves from the definition of system operational requirements described in . It constitutes a before-the-fact series of illustrations and statements leading to the definition of reliability, maintainability, human factors and safety, constructability and producibility, supportability, sustainability, disposability, and related requirements for design. It constitutes an ``input'' to the design process, whereas the maintenance plan (developed later) defines the follow-on requirements for system support based on a known design configuration and the results of the supportability analysis presented in 

The maintenance and support concept is reflected by the network and the activities and their interrelationships, illustrated in . The network exists whenever there are requirements for corrective and/or preventive maintenance at any time and throughout the system life cycle. By reviewing these requirements, one should address such issues as the levels of maintenance, functions to be performed at each level, responsibilities for the accomplishment of these functions, design criteria pertaining to the various elements of support (e.g., type of spares and levels of inventory, reliability of the test equipment, personnel quantities and skill levels), and the effectiveness factors and ``design-to'' requirements for the overall maintenance and support infrastructure. Although the design of the prime elements of the system may appear to be adequate, the overall ability of the system to perform its intended mission objective highly depends on the effectiveness of the support infrastructure as well.

While there may be some variations that arise, depending on the type and nature of the system, the maintenance and support concept generally includes the following items

Levels of maintenance: Corrective and preventive maintenance may be performed on the system itself (or an element thereof) at the site where the system is operating and used by the customer, in an intermediate shop near the customer’s operational site, and/or at a depot or manufacturer’s facility. Maintenance level pertains to the division of functions and tasks for each area where maintenance is performed. Anticipated frequency of maintenance, task complexity, personnel skill-level requirements, special facility needs, supply chain requirements, and so on, dictate to a great extent the specific functions to be accomplished at each level. Depending on the nature and mission of the system, there may be two, three, or four levels of maintenance; however, for the purposes of further discussion, maintenance may be classified as organizational, intermediate, and manufacturer/depot/supplier. 
describes the basic criteria and differences between these levels.

Repair policies: Within the constraints illustrated in there are a number of possible policies specifying the extent to which the repair of an element or component of a system should be accomplished (if at all). A repair policy may dictate that an item should be designed such that, in the event of failure, it should be nonrepairable, partially repairable, or fully repairable. Stemming from the operational requirements described in (refer to the five system illustrations), an initial ``repair policy'' for the system being developed should be established with the objective of providing some early guidelines for the design of the different components that make up the system. Referring to the example of the repair policy, illustrated in, it can be seen that there are numerous quantitative factors, which were initially derived from the definition of system operational requirement, that provide ``design-to'' guidelines as an input to the overall design process; for example, the system shall be designed such that the MTBM shall be 175 hours or greater, the MDT shall be 2 hours or less, the MLH/OH shall not exceed 0.1, and so on. A repair policy should be initially developed and established during the conceptual design phase, and subsequently updated as the design progresses and the results of the level-of-repair and supportability analyses become available.

Organizational responsibilities: The accomplishment of maintenance may be the responsibility of the customer, the producer (or supplier), a third party, or a combination thereof. In addition, the responsibilities may vary, not only with different components of the system but also as one progresses in time through the system operational use and sustaining support phase. Decisions pertaining to organizational responsibilities may affect system design from a diagnostic and packaging standpoint, as well as dictate repair policies, product warranty provisions, and the like. Although conditions may change, some initial assumptions are required at this point in time.

Maintenance support elements: As part of the initial maintenance concept, criteria must be established relating to the various elements of maintenance support. These elements include supply support (spares and repair parts, associated inventories, and provisioning data), test and support equipment, personnel and training, transportation and handling equipment, facilities, data, and computer resources. Such criteria, as an input to design, may cover self-test provisions, built-in versus external test requirements, packaging and standardization factors, personnel quantities and skill levels, transportation and handling factors, constraints, and so on. The maintenance concept provides some initial system design criteria pertaining to the activities illustrated in, and the final determination of specific logistic and maintenance support requirements will occur through the completion of a supportability analysis as design progresses.

Effectiveness requirements: These constitute the effectiveness factors associated with the support capability. In the supply support area, they may include a spare-part demand rate, the probability that a spare part will be available when required, the probability of mission success given a designated quantity of spares in the inventory, and the economic order quantity as related to inventory procurement. For test equipment, the length of the queue while waiting for test, the test station process time, and the test equipment reliability are key factors. In transportation, transportation rates, transportation times, the reliability of transportation, and transportation costs are of significance. For personnel and training, one should be interested in personnel quantities and skill levels, human error rates, training rates, training times, and training equipment reliability. In software, the number of errors per mission segment, per module of software, or per line of code may be important measures. For the supply chain overall, reliability of service, item processing time, and cost per item processed may be appropriate metrics to consider. These factors, as related to a specific system-level requirement, must be addressed. It is meaningless to specify a tight quantitative requirement applicable to the repair of a prime element of the system when it takes 6 months to acquire a needed spare part (for example). The effectiveness requirements applicable to the supply chain and support capability must complement the requirements for the system overall.

Environment: Definition of environmental requirements as they pertain to the maintenance and support infrastructure is equally important. This includes the impact of external factors such as temperature, shock and vibration, humidity, noise, arctic versus tropical applications, operating in mountainous versus flat terrain country, shipboard versus ground conditions, and so on, on the design of the maintenance and support infrastructure. In addition, it is also necessary to address possible ``outward'' environmental impact(s) of the maintenance and support infrastructure on other systems and on the environment in general (with the ``design for sustainability'' as being a major objective).

The maintenance concept provides the foundation that leads to the design and development of the maintenance and support infrastructure and defines the specific design-to requirements for the various elements of support (e.g., the supply support capability, transportation and handling equipment, test and support equipment, and facilities). These requirements, as they apply to system life-cycle support, can have a significant ``feedback'' effect (impact) on the prime elements of the system as well. Thus, the definition of system operational requirements and the development of the maintenance concept must be accomplished concurrently and early during the conceptual design phase. The combined result forms the basis for development of much of the material throughout the subsequent sections of this text, particularly with regard to the subject areas included in , that is, design for reliability, design for maintainability, design for usability (human factors), design for supportability, and others.

In summary, when defining the maintenance concept, it is important that consideration be given to the interfaces that may exist between the support requirements and infrastructure for the new system being developed and those comparable requirements for other systems that may be contained within the same overall system-of-systems configuration (refer to ). The requirements for this new system must first be defined, the impacts on (and from) the other systems evaluated, major conflicting areas noted, and finally modifications be incorporated as required. Care must be taken to ensure that the requirements for this new system are not compromised in any way. Finally, the selection of the ultimate maintenance and support infrastructure configuration must be justified on the basis of the LCC. What may seem to be a least-cost approach in providing maintenance support at the local level may not be such when considering the costs associated with all of the supply chain activities for the system in question.

\subsection{The System Architecture}\index{The System Architecture}

With the definition of system operational requirements, the maintenance and support concept, and the identification and prioritization of the TPMs, the basic system architecture has been established. The architecture deals with a top-level system structure (configuration), its operational interfaces, anticipated utilization profiles (mission scenarios), and the environment within which it is to operate.

Architecture describes how various requirements for the system interact. This, in turn, leads into a description of the functional architecture, which evolves from a functional analysis and is a description of the system in functional terms. From this analysis, and through the requirements allocation process and the definition of the various resource requirements necessary for the system to accomplish its mission, the physical architecture is defined. Through application of this process, one is able to evolve from the whats to the hows.

Many different trade-offs are possible as the system design progresses. Decisions must be made regarding the evaluation and selection of appropriate technologies, the evaluation and selection of commercial off-the-shelf (COTS) components, subsystem and component packaging schemes, possible degrees of automation, alternative test and diagnostic routines, various maintenance and support policies, and so on. Later in the design cycle, there may be alternative manufacturing processes, alternative factory maintenance plans, alternative logistic support structures, and alternative methods of material phase-out, recycling, and/or disposal.

One must first define the problem and then identify the design criteria or measures against which the various alternatives will be evaluated (i.e., the applicable TPMs), select the appropriate evaluation techniques, select or develop a model to facilitate the evaluation process, acquire the necessary input data, evaluate each of the candidates under consideration, perform a sensitivity analysis to identify potential areas of risk, and finally recommend a preferred approach. This process is illustrated in , and can be tailored and applied at any point in the life cycle as illustrated in and. Only the depth of the analysis and evaluation effort will vary, depending on the nature of the problem.

Referring to , trade-off analysis involves synthesis. Synthesis refers to the combining and structuring of components to create a feasible system configuration. Synthesis is design. Initially, synthesis is used in the development of preliminary concepts and to establish relationships among various components of the system. Later, when sufficient functional definition and decomposition have occurred, synthesis is used to further define the hows at a lower level. Synthesis involves the creation of a configuration that could be representative of the form that the system will ultimately take, although a final configuration should not be assumed at this early point in the design process.

Given a synthesized configuration, its characteristics need to be evaluated in terms of the system requirements initially specified. Changes are incorporated as required, leading to a preferred design configuration. This iterative process of synthesis, analysis, evaluation, and design refinement leads initially to the establishment of the functional baseline, then the allocated baseline, and finally the product baseline (refer to ). A good description of these configuration baselines, combined with a disciplined approach to baseline management, is essential for the successful implementation of the systems engineering process.

Throughout the conceptual system design phase (commencing with the needs analysis), one of the major objectives is to develop and define the specific ``design-to'' requirements for the system as an entity. The results from the activities described in are combined, integrated, and included in a system specification (Type A). This specification constitutes the top ``technical-requirements'' document that provides overall guidance for system design from the beginning. Referring to , this specification is usually prepared at the conclusion of conceptual design. Further, this top-level specification provides the baseline for the development of all lower-level specifications to include development (Type B), product (Type C), process (Type D), and material (Type E) specifications. While there may be a variety of formats used in the preparation of the system specification, an example of one approach is presented in .

\subsection{Conceptual Design Review}\index{Conceptual Design Review}

Design progresses from an abstract notion to something that has form and function, is firm, and can ultimately be reproduced in designated quantities to satisfy a need. Initially, a need is identified. From this point, design evolves through a series of stages (i.e., conceptual design, preliminary system design, and detail design and development). In each major stage of the design process, an evaluative function is accomplished to ensure that the design is correct at that point before proceeding with the next stage. The evaluative function includes both the informal day-to-day project coordination and data/documentation review and the formal design review. Design information is released and reviewed for compliance with the basic system-equipment requirements (i.e., performance, reliability, maintainability, usability, supportability, sustainability, etc., as defined by the system specification). If the requirements are satisfied, the design is approved as is. If not, recommendations for corrective action are initiated and discussed as part of the formal design review.

The formal design review constitutes a coordinated activity (including a meeting or series of meetings) directed to satisfy the interests of the design engineer and the technical discipline support areas (reliability, maintainability, human factors, logistics, manufacturing engineering, quality assurance, and program management). The purpose of the design review is to formally and logically cover the proposed design from the total system standpoint in the most effective and economical manner through a combined integrated review effort. The formal design review serves a number of purposes.

\begin{enumerate}
\item It provides a formalized check (audit) of the proposed system/subsystem design with respect to specification requirements. Major problem areas are discussed and corrective action is taken as required
\item It provides a common baseline for all project personnel. The design engineer is provided the opportunity to explain and justify his or her design approach, and representatives from the various supporting organizations (e.g., maintainability, logistic support, and marketing) are provided the opportunity to learn of the design engineer’s problems. This serves as an excellent communication medium and creates a better understanding among design and support personnel
\item It provides a means for solving interface problems and promotes the assurance that all system elements will be compatible, internally and externally
\item It provides a formalized record of what design decisions were made and the reasons for making them. Analyses, predictions, and trade-off study reports are noted and are available to support design decisions. Compromises to performance, reliability, maintainability, human factors, cost, and/or logistic support are documented and included in the trade-off study reports
\item It promotes a higher probability of mature design, as well as the incorporation of the latest techniques (where appropriate). Group review may identify new ideas, possibly resulting in simplified processes and ultimate cost savings
\end{enumerate}

The formal design review, when appropriately scheduled and conducted in an effective manner, leads to reduction in the producer’s risk relative to meeting specification requirements and results in improvement of the producer’s methods of operation. Also, the customer often benefits from the receipt of a better product.

Design reviews are generally scheduled before each major evolutionary step in the design process, as illustrated in . In some instances, this may entail a single review toward the end of each stage (i.e., conceptual, preliminary system design, and detail design and development). For the other projects, where a large system is involved and the amount of new design is extensive, a series of formal reviews may be conducted on designated elements of the system. This may be desirable to allow for the early processing of some items while concentrating on the more complex, high-risk items.

Although the number and type of design reviews scheduled may vary from program to program, four basic types are readily identifiable and are common to most programs. They include the conceptual design review (i.e., system requirements review), the system design review, the equipment/software design review, and the critical design review. Of particular interest relative to the activities discussed in this section is the conceptual design review, which is dedicated to the review and validation of system operational requirements, maintenance and support concept, specified TPMs, and the functional analysis and allocation of requirements at the system level. Referring to , this review is usually conducted at the end of the conceptual design phase and prior to the accomplishment of preliminary design.

\subsection{Conceptual Design for the River Crossing Problem}\index{Conceptual Design for the River Crossing Problem}

Write paras about CD for the RCP
