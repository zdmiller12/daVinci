\section{Methods for Modeling Requirements}\index{Methods for Modeling Requirements}


The Requirements Framework

Transforming the need or opportunity into a system that satisfies it is achieved by achieved by applying the systems engineering framework. Nevertheless, there is no single model. Several models have been proposed, each one with its own pros and cons. The systems engineer needs to know the different roads or methods available, their advantages and drawbacks, and select in each case the most appropriate one. Each of the systems engineering models or frameworks can potentially deliver the desired results (namely, the design of the required system); in any case, what is important is not to lose sight of the goal sought, and to rely on timely feedback in order to alter the course of action when and as needed. The systems process has been described by a number of authors [Blanchard, 1991; Sage, 2000; Kossiakoff and Sweet, 2003; Sage and Rouse, 2009].

ABSTRACT: Systems and software qualities (SQs) are also known as non-functional requirements (NFRs). Where functional requirements (FRs) specify what a system should do, the NFRs specify how well the system should do them. Many of them, such as Reliability, Availability, Maintainability, Usability, Affordability, Interoperability, and Adaptability, are often called “ilities,” but not to the exclusion of other SQs such as Safety, Security, Resilience, Robustness, Accuracy, and Speed.
In 2012, the US Department of Defense (DoD) identified seven Critical Technology Areas needing emphasis in its technology investments. One of them was called Engineered Resilient Systems (ERS). The SERC sponsor, the DoD Undersecretary for Systems Engineering, and the lead ERS research organization, the Army Engineering Research Center (ERC), held two workshops to explore what research was being addressed, and how the SERC could complement it. It turned out that the existing ERS research underway was primarily directed at field testing, supercomputer modeling, and resilient design of physical systems, and that the SERC could best complement this research by addressing the resilient design of cyber-physical-human (CPH) systems, Some of the SERC universities were performing such research, such as AFIT, Georgia Tech, MIT, NPS, Penn State, USC, U. Virginia, and Wayne State. These universities have been addressing aspects of this research area as a team since 2013.
Initially, the team found a veritable quagmire of SQ definitions and relationships. For example, looking up “resilience” in Wikipedia, the team found over 20 different definitions of “resilience,” with over 10 different definitions of a system’s post-resilient state. The leading standard in the area, ISC/IEC 25010, had weak and inconsistent definitions of the qualities. For example, it defined Reliability with respect to the satisfaction of a system’s functional requirements, but not its quality requirements. Some of the SERC universities had developed partial ontologies of the SQs, and exploration of alternative ontology structures identified found one that addressed not only the inter-quality relationships, but also their sources of value variation.
BIO:Dr. Barry Boehm received his B.A. degree from Harvard in 1957, and his M.S. and Ph.D. degrees from UCLA in 1961 and 1964, all in Mathematics. He has also received honorary Sc.D. in Computer Science from the U. of Massachusetts in 2000 and in Software Engineering from the Chinese Academy of Sciences in 2011.

\subsection{Functional and Non\-Functional Requirements}\index{Functional and Non\-Functional Requirements}

A second way for classifying requirements is by dividing them into functional and non-functional ones. Functional requirements are those that express what the system is to be able to do or perform, whereas non-functional are those that define overall qualities or attributes of the system of or a relevant part of it (like a subsystem). Functional requirements describe what the system is to do, while the non-functional requirements set conditions or constraints on how the functional ones are to be implemented. For example, a functional requirement for a cellular phone may be the hours of autonomy of its battery; a non-functional requirement may be one stating its expected useful life (a reliability-related requirement). One thing is what the system can perform (the functional requirements). Typical non-functional requirements are related to safety, security, availability, reliability, maintainability, supportability, interfaces, system acceptance, and the like. Non-functional requirements can be divided into the following three groups:
    a. Product requirements. These specify characteristics or properties that a system, or a part of it, has to have. For example, if the need is to have a means for transporting a certain cargo and the solution is going to be a special truck, a couple of non-functional requirements could be for the truck to have a certain operational availability and for its power pack (engine and transmission) to have a certain mission reliability.
    b. Process requirements. These relate to the processes associated with the design, development, operation and/or support of the system. Examples could be meeting a certain standard, using a given tool, or following an indicated method or procedure. For example, in the Eurofighter program the Air Forces of the four nations involved (Germany, Italy, Spain, and the United Kingdom) decided to exchange information electronically and consequently new logistics information systems had to be developed. It was decided that those systems would be developed in compliance with standard AECMA 2000M International Specification for Material Management, which regulates the exchange of material support data in electronic form.
    c. External requirements. These are derived from the environment (in the broadest sense of the term) in which the system will be deployed. For example, the need to interface and work together with other systems is an external requirement, as it is also the need for a certain fault tolerance capability if the environment in which the system is intended to operate will make access for maintenance very difficult and/or expensive.
    
\subsection{Characteristics of Requirements}\index{Characteristics of Requirements}

There are a number of essential characteristics or abilities that the requirements are to have in order for them to truly constitute the needed solid foundation on which to design and develop the system desired. Each individual requirement has to have certain abilities that will be somewhat different depending on whether they are stakeholder or system requirements. Furthermore, the entire set of requirements needs to meet some characteristics, too. In a nutshell, requirements have their own requirements. Too many projects derail or end up in serious trouble because requirements have not been well defined and fail to translate in an accurate manner the perceived need or opportunity. Such distortion in the translation contaminates all subsequent steps in the process. The likelihood of having the requirements well written and formulated from the beginning increases if the systems engineer has a clear understanding of the needed characteristics of the requirements, of each one individually and of the entire set of them.

Characteristics of Individual Requirements. Each individual stakeholder or system requirement is to be:
Needed. Seems trivial, but it is not. Every requirement poses constraints and demands on the system and consequently has potential impacts on aspects, such as performance or cost. It is therefore important to ensure that every requirement is really needed and to avoid the mistake of requesting things that are not needed. Customers and other stakeholders are generally little aware of the broad and deep implications of their requirements and, if they had better visibility of those consequences, they would probably waive a number of them. Unneeded requirements will take a toll on the project, be it in terms of performance, cost and/or schedule. For the set of requirements, more is not necessarily better. Actually, the effect of unnecessary requirements can be as bad, or even worse, than that of missed requirements. The need or opportunity is to be accurately translated into requirements can be as bad or even worse, than that of missed requirements, and that is what really matters. A requirement that is not needed reduces the solutions space, being it then more difficult to optimize the solution for the customer and the stakeholders.
Atomic. Only one needed functionality or property per requirement; that is, requirements are not to be compounded. If several aspects need to be covered, then it is best to split them and address each one in a separate requirement. This implies that the use of conjunctions (and, or, and the like) is to be avoided. It is in general more difficult to verify the fulfillment of non-atomic requirements and problems arise easily when there is evidence that part of a requirement has been met, while others have not. This complicates the verification effort, which is highly facilitated when requirements are atomic.
Unique. Each needed system characteristic or functionality is to be stated only once, in a single requirement. When things are repeated in several requirements nothing positive is added and there is a large risk of conflict if the different wordings do not really convey the same message. Even if perfectly consistent, duplications or multiple repetitions make it more complicated for the designer to grasp the entire picture. Later, if a change to a requirement is proposed and accepted, it would be necessary to amend all others expressing the same need so that their wordings remain aligned. Consistency problems will easily surface if requirements are not unique.
Positive. The requirement has to state what functionality is needed and not what is not to be present. Negative statements are more likely to be misinterpreted, especially when there are several negative statements embedded in the requirement. Positive statements are always easier to understand.
Objective. A requirement is not to be open to interpretation. Fuzziness is to be avoided; subjectivity is not desirable as it only increases the chances for reworks and disputes. Furthermore, the likelihood of different interpretations by different parties may lead to a design that eventually will not satisfy the user; that is, subjective requirements increase the risk of eventual problems in the validation of the solution and should thus be avoided.
    a. Understandable. Goes without saying, but the purpose of a requirement is to translate the perception of a need or opportunity in such a way that it can be communicated to, and understood by, the recipient party. In some instances the recipient party may be the stakeholders themselves, who are sent the requirements seeking confirmation or amendment as necessary. In some other cases, the recipient party may be the designers, who will take the requirements to perform the functional analysis. If a requirement is not understood, it does not serve any purpose.
    b. Correct. Apparently another obvious thing, but frequently overlooked. Requirements are to depict needed functionalities or properties of the system. Should a requirement be incorrect and demand something that is not needed, it will imply either unnecessary costs to the end user, or pose a conflict with other truly-needed requirements, or complicate unnecessarily the design by introducing gratuitous or superfluous constraints. A requirement that is not correct displaces the solution space, increasing thus the risk of not fulfilling the needs of expectations of customer and stakeholders.
    c. Concise. Requirements should state simply and directly what is needed, avoiding superfluous details and unnecessary information that would make it more difficult to be understood.
    d. Traceable. It is essential to know who is behind each requirement for a number of reasons. It may be necessary to seek confirmation to the wording of a requirement of to its interpretation, or it could be the case that a change is needed and the originator and owner of the requirement is to be approached, or that evidence on the verification of the fulfillment of the requirement is to be furnished. Knowing which stakeholder originated which requirement is thus a sine qua non condition. Traceability is to be done forward and backward, from stakeholder to actually physical characteristic in the system and back.
    e. Prioritizable. Not all requirements need to have the same relevance or importance to the end customer and to the rest of stakeholders. In principle there should not be any conflicts among the requirements, but even if that is the case there is always the possibility that some requirements will only be completely fulfilled at the expense of others not being so. That means that requirements should be prioritized; that is, that they should be classified or arranged as per their importance in order to facilitate the decisions to be taken whenever unsolvable conflicts show up. It is typical to classify requirements in three categories: ‘must’, ‘desirable’, and ‘nice to have’. The first group includes those that are absolutely mandatory. The second are the requirements that should definitely be met unless they pose a conflict with one of the ‘must’ group. Finally the third group integrates those other requirements that are desirable but only as long as they do not jeopardize the fulfillment of any other requirement in the two other groups. In any case the requirements priority taxonomy or classification, in whatever number of levels, has to be decided for each particular project and agreed upon with the stakeholders.
    f. Solution-independent. The requirements are to be stated without any prejudice or bias as to what the solution might be. The important thing is that they reflect what functionalities and characteristics are required, not how they will be implemented and rendered by the system. A common mistake is to state requirements for a certain type of solution simply because the customer is familiar with it. Doing this will constitute an unnecessary constraint that will reduce the freedom of the systems engineer to find the best solution to the problem at stake.
In addition to the above, each system requirement is to be:
    a. Concept-dependent. Once the list of stakeholder requirements is considered to the complete and is validated in terms of reflecting the perceived need or opportunity, the next step in the systems engineering framework is to identify all (or as many as possible) ways for solving it. Each potential solution is a design concept and, out of the many that are identified, the best one will be selected. All stakeholder requirements will then be translated to the chosen design concept in a one-to-many relationship as portrayed in figure 5.3.
    b. Verifiable. There is no point in stating a requirement if its fulfillment cannot be objectively confirmed, and yet many projects are plagued with requirements for which an objective verification is not specified. Whenever a functionality is required by using a standard concept or metric there is no further need to state how its fulfillment will be verified. During pre-production some models or prototypes may enable physical measurements. During serial production other more accurate measurements will be feasible, and during the operational life the true characteristics of the real system, once fielded and in operation by the user, may be measured. The relationship of system requirements and verification methods is thus, in general, one-to-many (refer again to figure 5.2).
    c. Feasible. Upfront, it may be difficult to know if a particular requirement is attainable or not. To further complicate things it could be that some requirements can only be achieved if others are not, at least to their full extent. Such case would mean a certain lack of coherence or consistency, a property that the entire set of requirements is to have, as explained in detail in the next subsection. Yet, it is highly desirable that all accepted requirements be considered, in principle, achievable; otherwise, they should be reconsidered. Given the subjectivity of this characteristic, it is essential to conduct a thorough risk management program that considers the effect in the end system of the potential lack of achievement of certain requirements (for example, due to a certain achievement of certain technology not reaching a needed level of maturity, resulting in a certain requirement being partially unattainable).
    d. Modifiable. Although system requirements are to be feasible, it may be that the design effort eventually proves the contrary. The appropriate thing is to go back to the stakeholder that generated the stakeholder requirement from which the system requirement was derived, in order to modify it as appropriate. This means that desirably all stakeholders formulate their requirements with some latitude for amendments or modification, should it prove necessary.
Figure 5.5 summarizes the characteristics that the individual requirements need to have, whether at the stakeholder level or at the system level. Failure of the requirements to meet these characteristics will disrupt the subsequent design and development effort. The needs perceived by the user are always legitimate, but too frequently they are wrongly stated in form of stakeholder requirements. As important as detecting a need or opportunity is to translate it into the right set of well-written stakeholder requirements.
(Figure 5.5. Characteristics of the individual requirements.)
	Ill-defined stakeholder and/or system requirements almost inevitably spell out the lack of customer satisfaction with the design and developed system.

Characteristics of the Set of Requirements. 	If each requirement is to meet a number of characteristics of abilities, as described in the previous section, the entire set is to be:
    a. Complete. The identified need or the perceived opportunity is to be fully explained through the set of requirements. If any required functionalities are not reflected, then the system will most likely not truly fulfill its purpose. This has three implications. First, it means that all stakeholders are to be considered because if any one is not considered and involved, it will then follow that all requirements coming from it will be missed. Second, it is essential that each stakeholder addresses its specific needs or demands regarding the system and expresses it in a complete set of requirements. The requirements coming from the different stakeholders will have to be merged; redundancies will have to be eliminated and potential conflicts will have to be sorted out. Third and final, every stakeholder requirement will have to be completely translated into the appropriate array of system requirements for the selected design concept. Failure to do so will mean that the input received by the designers will not truly reflect what the end users and the rest of the stakeholders need.
    b. Coherent. As stated in the previous characteristic, no conflicts or contradictions are to be present in the set of requirements at any level (stakeholder or system), which is the same as to say that the set has to be consistent.
    c. Structured. A well-organized set of requirements will be easier to understand than a poorly-structured one. As there will be requirements covering many areas or aspects, it is mandatory that they are organized in a logical manner that facilitates their understanding and the grasping of the global picture.
    d. Non-redundant. As it was also mentioned when addressing the need for the set of requirements to be complete, it is clear that it should not contain any repetitions. Each aspect is to be covered just once.
