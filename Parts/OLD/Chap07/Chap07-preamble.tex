The utilization of models has become essential in almost all disciplines, especially the applied sciences and engineering disciplines of. Early examples of models include diagrams of structures and schematics to plan battlefield movements, while modern examples cover a variety of applications for indirect experimentation ranging from mathematical manipulation through complex computer simulations.
Beyond classical models, recent software, analytical, and technology advancements have positioned modeling in engineering with promises to reduce cost, improve performance, and make collaboration more effective. While modeling has a rich history, wide application in engineering, and profound potential benefits, the realization from advanced and innovative modeling techniques depends entirely on awareness, application, and adoption by technology-based enterprises.
But, there are fundamental differences between models used in science and engineering. Science is concerned with the natural world, whereas engineering is concerned primarily with the human-made world. Science uses models to gain an understanding of the way things are in the natural world. Engineering uses models of the human-made world in an attempt to achieve what humans want. The validated models of science are used in engineering to establish bounds for engineered systems and to improve the products of such systems.