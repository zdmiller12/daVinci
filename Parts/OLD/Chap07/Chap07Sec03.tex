\section{Modeling System Functions}\index{Modeling System Functions}

Once the operational analysis has been concluded, the next step is to perform the functional analysis, which transforms the system requirements into the architecture of the system. In an interative process, the requirements are analyzed with the focus on the whats, and not on the hows. The analysis enables the identification of the functionalities that the system is to render in order to fulfill the requirements, and this allows the identification of the elements needed in the system to deliver those identified functionalities (Soles).

\subsection{Functional Analysis and Allocation}\index{Functional Analysis and Allocation}

A critical step in implementing the systems engineering process is the accomplishment of the functional analysis and the definition of the system in “functional” terms. Functions are initially identified as part of defining the need and the basic requirements for the system (, Block 0.1). System operational requirements and the maintenance concept are defined, and the functional analysis is expanded to establish a functional baseline, from which the resource requirements for the system are identified; that is, equipment, software, people, facilities, data, the various elements of maintenance and support, and so on. The functional analysis is initiated during the conceptual design phase and is described in <link olinkend="ch03lev1sec7" preference="0">Section <xref olinkend="ch03lev1sec7" label="3.7"><inst>3.7</inst></xref></link>. As design and development continues, the functional analysis is accomplished to a greater depth, to the subsystem level and below, during the preliminary system design phase, as described in <link olinkend="ch04lev1sec3" preference="0">Section <xref olinkend="ch04lev1sec3" label="4.3"><inst>4.3</inst></xref></link>. This appendix provides guidance as to the detailed steps involved in accomplishing a functional analysis and in the development of functional flow block diagrams (FFBDs).</para>
<para>Functional analysis includes the process of translating top-level system requirements into specific qualitative and quantitative design-to requirements. Given an identified need for a system, supported by the definition of system operational requirements and the maintenance concept, it is necessary to translate this information into meaningful design criteria. This translation task constitutes an iterative process of breaking down system-level requirements into successive levels of detail; a convenient mechanism for communicating this information is through the various levels of FFBDs.</para>

</title>
<para>An essential activity in early conceptual and preliminary design is the development of a <emphasis>functional</emphasis> description of the system to serve as a basis for identification of the resources necessary for the system to accomplish its mission. A <emphasis>function</emphasis> refers to a specific or discrete action (or series of actions) that is necessary to achieve a given objective; that is, an operation that the system must perform, or a maintenance action that is necessary to restore a faulty system to operational use. Such actions may ultimately be accomplished through the use of equipment, software, people, facilities, data, or various combinations thereof. However, at this point in the life cycle, the objective is to specify the <emphasis>whats</emphasis> and not the <emphasis>hows</emphasis>; that is, <emphasis>what</emphasis> needs to be accomplished versus <emphasis>how</emphasis> it is to be done.</para>
<para>The <emphasis>functional analysis</emphasis> is an iterative process of translating system requirements into detailed design criteria and the subsequent identification of the resources required for system operation and support. It includes breaking requirements at the system level down to the subsystem, and as far down the hierarchical structure as necessary to identify input design criteria and/or constraints for the various elements of the system. The purpose is to develop the top-level <emphasis>system architecture</emphasis>, which deals with both “requirements” and “structure.”</para>
<para>Referring to the iterative process in , the functional analysis actually commences (in a broad context) in conceptual design as part of the problem definition and needs analysis task (refer to ). Subsequently, “operational” and “maintenance” functions are identified, leading to the development of top-level systems requirements, as described in . The purpose of the “functional analysis” is to present an overall integrated and composite description of the system’s <emphasis>functional architecture</emphasis>, to establish a functional baseline for all subsequent design and support activities, and to provide a foundation from which all physical resource requirements are identified and justified; that is, the system’s <emphasis>physical architecture</emphasis>. A continuation of the functional analysis at the subsystem level and below is presented in , and the specific mechanics for the development of functional flow block diagrams (FFBDs) are discussed further in <link olinkend="app01" preference="0">Appendix <xref olinkend="app01" label="A"><inst>A</inst></xref></link>.</para>

Functional Flow Block Diagram. </title><para>Accomplishment of the functional analysis is facilitated through the use of <emphasis>functional flow block diagrams</emphasis> (FFBDs). The preparation of these diagrams may be accomplished through the application of any one of a number of graphical methods, including the Integrated DEFinition (IDEF) modeling method, the Behavioral Diagram method, the N-Squared Charting method, and so on. Although the graphical presentations are different, the ultimate objectives are similar. The approach assumed here is illustrated in 
, a simplified flow diagram with some decomposition is shown. Top-level functions are broken down into second-level functions, second-level functions into third-level functions, and so on, down to the level necessary to adequately describe the system and its various elements in functional terms to show the various applicable functional interface relationships and to identify the resources needed for functional implementation. Block numbers are used to show sequential and parallel relationships, initially for the purpose of providing top-down “traceability” of requirements, and later as a bottom-up “traceability” and justification of the physical resources necessary to accomplish these functions.</para>
 shows an expansion of a functional flow block diagram (FFBD), identifying a partial top level of activity, a breakdown of Function 4.0 into a top “operational” flow, and a breakdown of Function 4.2 into a top “maintenance” flow in the event that this function does not “perform” as required. Note that the words in each block are “action-oriented.” Each block represents some operational or maintenance support function that must be performed for the system to accomplish its designated mission, and there are performance measures (i.e., metrics) associated with each block that are allocated from the top. In addition, each block can be expanded (through further downward iteration) and then evaluated in terms of inputs, outputs, controls and/or constraints, and enabling mechanisms. Basically, the “mechanisms” lead to the identification of the physical resources necessary to accomplish the function, evolving from the <emphasis>whats</emphasis> to the <emphasis>hows</emphasis>. The identification of the appropriate resources in terms of equipment needs, software, people, facilities, information, data, and so on, is a result of one or more trade-off studies leading to a preferred approach as to <emphasis>how best</emphasis> to accomplish a given function.</para>
 shows an expansion of one of the functions identified in the overall basic community health care infrastructure illustrated in . The objective here is to show traceability from the definition of operational requirements in <link linkend="ch03lev1sec4" preference="0" type="backward">Section <xref linkend="ch03lev1sec4" label="3.4"><inst>8.1.4</inst></xref></link> by selecting a specific functional block from one of the example illustrations; that is, the “Community Hospital” in Illustration 5.</para>
<para>The functional analysis evolves through a series of steps illustrated in. Initially, there is a need to accomplish one or more functions . Through the definition and system operational requirements () and the maintenance and support concept (<link linkend="ch03lev1sec5" preference="0" type="backward">Section <xref linkend="ch03lev1sec5" label="3.5"><inst>8.1.5</inst></xref></link>), the required functions are further delineated, with the functional analysis providing an overall description of system requirements; that is, <emphasis>functional architecture</emphasis>. FFBDs are developed for the primary purpose of structuring these requirements by illustrating organizational and functional interfaces.</para>
<para>As one progresses through the functional analysis, and particularly when developing a new system that is within a higher-level SOS structure, there may be functions identified as “common” and shared with a different system. For example, a “transmitter” or “receiver” function may be common for two different and separate communication systems; a “power supply” function may provide the necessary power for more than one system; an “imaging center” may provide the necessary medical diagnostic services for more than one hospital and/or doctor’s office complex (refer to ), and so on. illustrates the application of several “common functions” as part of the functional breakdown for three different systems; that is, Systems <emphasis>A</emphasis>, <emphasis>B</emphasis>, and <emphasis>C.</emphasis></para>
<para>The formal functional analysis is fully initiated during the latter stages of conceptual design, and is intended to enable the completion of the system design and development process in a comprehensive and logical manner. More specifically, the functional approach helps to ensure the following:</para>
<orderedlist numeration="arabic" spacing="normal" inheritnum="ignore" continuation="restarts"><listitem><inst>	1.	</inst><para>All facets of system design and development, production, operation, support, and phase-out are considered (i.e., all significant activities within the system life cycle).</para></listitem>
<listitem><inst>	2.	</inst><para>All elements of the system are fully recognized and defined (i.e., prime equipment, spare/repair parts, test and support equipment, facilities, personnel, data, and software).</para></listitem>
<listitem><inst>	3.	</inst><para>A means is provided for relating system packaging concepts and support requirements to specific system functions (i.e., satisfying the requirements of good functional design).</para></listitem>
<listitem><inst>	4.	</inst><para>The proper sequences of activity and design relationships are established along with critical design interfaces.</para></listitem></orderedlist>
<para>Finally, it should be emphasized that the functional analysis provides the baseline from which reliability requirements (refer to and the reliability “model” in), maintainability requirements, human factors requirements, and supportability requirements are determined. Referring to, the <emphasis>functional baseline</emphasis> leads to the <emphasis>allocated baseline</emphasis>, which leads to the <emphasis>product baseline</emphasis>.</para></section>

Functional Allocation. </title><para>Given a top-level description of the system through the functional analysis, the next step is to break the system down into elements (or components) by <emphasis>partitioning</emphasis>. The challenge is to identify and group closely related functions into packages employing a common resource (e.g., an equipment item or a software package) and to accomplish multiple functions to the extent possible. Although it may be relatively easy to identify individual functional requirements and associated resources on an independent basis, this process may turn out to be rather costly when it comes to packaging system components, weight, size, and so on. The questions are as follows: <emphasis>What hardware or software (or other) can be selected that will perform multiple functions reliably, effectively, and efficiently? How can new functional requirements be added without adding new physical elements to the system structure?</emphasis></para>
<para>The partitioning of the system into elements is evolutionary in nature. Common functions may be grouped or combined to provide a system packaging scheme, with the following objectives in mind:</para>
<orderedlist numeration="arabic" spacing="normal" inheritnum="ignore" continuation="restarts"><listitem><inst>	1.	</inst><para>System elements may be grouped by geographical location, a common environment, or by similar types of items (e.g., equipment, software, data packages, etc.) that have similar functions.</para></listitem>
<listitem><inst>	2.	</inst><para>Individual system packages should be as independent as possible with a minimum of “interaction effects” with other packages. A design objective is to enable the removal and replacement of a given package without having to remove and replace other packages in the process, or requiring an extensive amount of alignment and adjustment as a result.</para></listitem>
<listitem><inst>	3.	</inst><para>In breaking a system down into subsystems, select a configuration in which the “communications” between the various different subsystems is minimized. In other words, whereas the subsystem’s <emphasis>internal</emphasis> complexity may be high, the <emphasis>external</emphasis> complexity should be low. Breaking the system down into packages requiring high rates of information exchange between these packages should be avoided.</para></listitem>
<listitem><inst>	4.	</inst><para>An objective is to pursue an <emphasis>open-architecture</emphasis> approach in system design. This includes the application of common and standard modules with well-defined standard interfaces, grouped in such a way as to allow for system upgrade modifications without destroying the overall functionality of the system.</para></listitem></orderedlist>

<para role="continued">An overall design objective is to break the system down into elements such that only a few critical events can influence or change the inner workings of the various packages that make up the system architecture.</para>
<para>Through the process of partitioning and functional packaging, trade-off studies are conducted in evaluating the different design approaches that can be followed in responding to a given functional requirement. It may be appropriate to perform a designated function through the use of equipment, software, people, facilities, data, and/or various combinations thereof. The proper mix is established, and the result may take the form of a system structure similar to the example presented in. With this structure representing the proposed system “make-up,” the next step is to determine the design-to requirements for each of the system elements; that is, system operator, Equipment 123, Unit <emphasis>B</emphasis>, computer resources, and facilities.</para>
<para>Referring to (and ), specific quantitative design-to requirements have been established at the top level for the <emphasis>system</emphasis>. These TPMs, which evolved from the definition of operational requirements and the maintenance support concept, must be <emphasis>allocated</emphasis> or <emphasis>apportioned</emphasis> down to the appropriate subsystems or the elements that make up the system. For instance, given an operational availability Ao requirement of 0.985 for the system, <emphasis>what design requirements should be specified for the equipment, software, facility, and operator such that, when combined, they will meet the availability requirements for the overall system?</emphasis> In other words, to guarantee an ultimate system design configuration that will meet all customer (user) requirements, there must be a top-down allocation of design criteria from the beginning; that is, during the latter stages of conceptual design. The allocation process starts at this point and continues with the appropriate subsystems and below as required. presents a case-study example of allocation from the system to next level below, and the process is expanded for the allocation of reliability requirements in , maintainability requirements in and so on.

\subsection{Functional Flow Block Diagrams}\index{Functional Flow Block Diagrams}

Functional flow block diagrams (FFBDs) are models developed to describe the system and its elements in functional terms. These diagrams reflect both operational and support activities as they occur throughout the system life cycle, and they are structured in a manner that illustrates the hierarchical aspects of the system (see . Some of the key features of the overall functional flow process are noted as follows:</para>
<orderedlist numeration="arabic" spacing="normal" inheritnum="ignore" continuation="restarts"><listitem><inst>1.	</inst><para>The functional block diagram approach should include coverage of all activities throughout the system life cycle, and the method of presentation should reflect proper activity sequences and interface interrelationships.</para></listitem>
<listitem><inst>2.	</inst><para>The information included within the functional blocks should be concerned with <emphasis>what</emphasis> is required before looking at <emphasis>how</emphasis> it should be accomplished.</para></listitem>
<listitem><inst>3.	</inst><para>The process should be flexible to allow for expansion if additional definition is required or reduction if too much detail is presented. The objective is to progressively and systematically work down to the level where resources can be identified with how a task should be accomplished (refer to 
<para role="continued">In the development of functional flow diagrams, some degree of standardization is necessary (for communication) in defining the system. Thus, certain basic practices and symbols should be used, whenever possible, in the physical layout of functional diagrams. The paragraphs below provide some guidance in this direction.</para>
<orderedlist numeration="arabic" spacing="normal" inheritnum="ignore" continuation="restarts"><listitem><inst>1.	</inst><para><emphasis>Function block.</emphasis> Each separate function in a functional diagram should be presented in a single box enclosed by a solid line. Blocks used for reference to other flows should be indicated as partially enclosed boxes labeled “Ref.” Each function may be as gross or detailed as required by the level of functional diagram on which it appears, but it should stand for a definite, finite, discrete action to be accomplished by equipment, personnel, facilities, software, or any combination thereof. Questionable or tentative functions should be enclosed in dotted blocks.</para></listitem>
<listitem><inst>2.	</inst><para><emphasis>Function numbering.</emphasis> Functions identified in the functional flow diagrams at each level should be numbered in a manner which preserves the continuity of functions and provides information with respect to function origin throughout the system. Functions on the top-level functional diagram should be numbered 1.0, 2.0, 3.0, and so on. Functions which further indenture these top functions should contain the same parent identifier and should be coded at the next decimal level for each indenture. For example, the first indenture of function 3.0 would be 3.1, the second 3.1.1, the third 3.1.1.1, and so on. For expansion of a higher-level function within a particular level of indenture, a numerical sequence should be used to preserve the continuity of the function. For example, if more than one function is required to amplify function 3.0 at the first level of indenture, the sequence should be 3.1, 3.2, 3.3, ..., 3.<emphasis>n</emphasis>. For expansion of function 3.3 at the second level, the numbering shall be 3.3.1, 3.3.2, ..., 3.3.<emphasis>n</emphasis>. Where several levels of indentures appear in a single functional diagram, the same pattern should be maintained. While the basic ground rule should be to maintain a minimum level of indentures in any one particular flow, it may become necessary to include several levels to preserve the continuity of functions and to minimize the number of flows required to functionally depict the system.</para></listitem>
<listitem><inst>3.	</inst><para><emphasis>Functional reference.</emphasis> Each functional diagram should contain a reference to its next higher functional diagram through the use of a reference block. For example, function 4.3 should be shown as a reference block in the case where the functions 4.3.1, 4.3.2, . . . , 4.3.<emphasis>n</emphasis>, and so on, are being used to expand function 4.3. Reference blocks shall also be used to indicate interfacing functions as appropriate.</para></listitem>
<listitem><inst>4.	</inst><para><emphasis>Flow connection.</emphasis> Lines connecting functions should indicate only the functional flow and should not represent either a lapse in time or any intermediate activity. Vertical and horizontal lines between blocks should indicate that all functions so interrelated must be performed in either a parallel or a series sequence. Diagonal lines may be used to indicate alternative sequences (cases where alternative paths lead to the next function in the sequence).</para></listitem>
<listitem><inst>	5.	</inst><para><emphasis>Flow directions.</emphasis> Functional diagrams should be laid out so that the functional flow is generally from left to right and the reverse flow, in the case of a feedback functional loop, from right to left. Primary input lines should enter the function block from the left side; the primary output, or go line, should exit from the right, and the no-go line should exit from the bottom of the box.</para></listitem>
<listitem><inst>	6.	</inst><para><emphasis>Summing gates.</emphasis> A circle should be used to depict a summing gate. As in the case of functional blocks, lines should enter or exit the summing gate as appropriate. The summing gate is used to indicate the convergence, divergence, parallel, or alternative functional paths and is annotated with the term AND or OR. The term AND is used to indicate that parallel functions leading into the gate must be accomplished before proceeding to the next function, or that paths emerging from the AND gate must be accomplished after the preceding functions. The term OR is used to indicate that any of the several alternative paths (alternative functions) converge to, or diverge from, the OR gate. The OR gate thus indicates that alternative paths may lead or follow a particular function.</para></listitem>
<listitem><inst>	7.	</inst><para><emphasis>Go and no-go paths</emphasis>. The symbols G and <inlineequation id="app01ie01"><inlinemediaobject><textobject role="xpressmath"></textobject></inlinemediaobject></inlineequation> are used to indicate go and no-go paths, respectively. The symbols are entered adjacent to the lines leaving a particular function to indicate alternative functional paths.</para></listitem>
<listitem><inst>	8.	</inst><para><emphasis>Numbering procedure for changes</emphasis>. Additions of functions to existing data should be accomplished by locating the new function in its correct position without regard to sequence of numbering. The new function should be numbered using the first unused number at the level of indenture appropriate for the new function.

\subsection{Some Example Application}\index{Some Example Application}

With the objective of illustrating how some of these general guidelines are employed, are included to present a few simple applications.</para>

<para role="continued">Although these sample block diagrams do not cover the selected systems entirely, it is hoped that the material is presented in enough detail to provide an appropriate level of guidance for the development of functional block diagrams.