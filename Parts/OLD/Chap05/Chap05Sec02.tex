From a cognitive perspective, systems thinking integrates analysis and synthesis. Natural science has been primarily reductionistic, studying the components of systems and using quantitative empirical verification. Human science, as a response to the use of positivistic methods for studying human phenomena, has embraced more holistic approaches, studying social phenomena through qualitative means to create meaning. Systems thinking bridges these two approaches by using both analysis and synthesis to create knowledge and understanding and integrating an ethical perspective. Analysis answers the `what’ and `how’ questions while synthesis answers the rich inquiring platform for approaches such as social systems design, developed by Bela H. Banathy, and evolutionary systems design, as Alexander Laszlo and myself have developed to include a deeper understanding of a system in its larger context as well as a vision of the future for co-creating ethical innovations for sustainability.

Accordingly, the overall purpose of the chapters in Part II is to impart an in-depth understanding of system design as a process; a process that is greater than the sum of its artificial categories as identified by chapter titles. To achieve this process purpose, it is necessary to introduce some terminology and notation somewhat prematurely; that is, before the more complete treatment provided in subsequent chapters.

Systems Engineering and the Scientific Method – The Scientific Method is a fundamental basis for scientific development. Like SE it is not a one-size-fits-all cookbook method and requires intelligent tailoring of fundamental principles to achieve desired outcomes. The Scientific Method is learned early in STEM education and applied across multiple courses that follow. Systems Engineering, however, is just as fundamental to engineering of even the most fundamental concepts. The proposed research task would develop a simple and easy way to teach the concept of SE based on the scientific method model that all STEM students can also learn and internalize with the same ubiquity as the Scientific Method. This task will require revising current internet sites that tout an ``Engineering Method'' that bears little resemblance to how engineering is practiced except on very simple projects. FROM: Chaput, et al.

\subsection{Systems Engineering: Understanding the Differences}\index{Systems Engineering: Understanding the Differences}

Fundamental to the application of systems engineering is an understanding of life-cycle thinking, illustrated for the product in. The product life cycle begins with the identification of a need and extends through conceptual and preliminary design, detail design and development, production or construction, distribution, utilization, support, phase-out, and disposal. The life-cycle phases are classified as acquisition and utilization to recognize producer and customer activities.

Figure 5.1 Here

There is a need to properly define the ``intellectual foundations of systems engineering;'' we need to look beyond systems engineering to do this. This paper presents a new framework for understanding and integrating the distinct and complementary contributions of systems science, systems thinking, and systems engineering (in both domain-independent and domain-specific forms) to an ``integrated systems approach.'' The key step is to properly separate out and show the relationship between the triad of: systems science as an objective ``science of systems;'' systems engineering as ``creating, adjusting, and configuring systems for a purpose.'' None of these is a subset of another; all must be considered as distinct through interdependent subjects. They can be so considered if they are correctly defined and bounded. The key conclusion of the analysis in that the ``correct'' choice of system boundary for a purpose depends on the property of interest. This choice seems to belong in the domain of ``systems thinking,'' which thus provides a key input to ``systems engineering.'' In practice in many systems businesses, the role of “systems architect” or ``systems engineer'' integrates the skills of systems science, systems thinking and systems engineering which are therefore essential competencies for the role.

This starts with the premise that there is a ``systems science,'' or at least a ``theory of systems,'' that defines systems and system properties. These properties of a system - which can be either natural or human-made, and which can exhibit a greater or lesser degree of adaptive behavior - are independent of how and why it was created. The nature of such systems is independent of human intentionality, and we see similar or identical system properties and patterns recurring across different domains of application.

We consider systems engineering as those activities relating to the purposeful creation or adaptation, adjustment, and operation, of systems. From this perspective, the core function of systems engineering is ``making choices about how to create and adjust a new system or modify an existing on the better to achieve a purpose.''

So, if systems science is independent of human intentionality, and systems engineering makes choices about creating or modifying systems ``for a purpose,'' where do we establish purpose?  If we consider systems thinking to be about ``understanding systems in a human context,'' we can then view it as the activity that looks at systems through a lens of human intentionality and establishes the ``purpose'' and ``value'' that drive systems engineering. This triad approach, treating systems science, systems thinking and systems engineering as equal partners in an integrated systems approach, allows us to resolve paradoxes that have troubled practitioners for some time.

A systems approach in a domain will apply these general principles within the context of existing knowledge about the domain. This may include an understanding of domain problems, constraints, risks and opportunities; and the best to tackle issues as we approach is better efficiency based on risk-aware replication of known practices and proven design rules. Potential disadvantages are blindness to cross-domain opportunities and issues, and a risk of the ``wrong-problem syndrome,'' solving the problems that interest domain experts rather than what is needed to resolve the problem situation. Draft Thales Copyright 2011 12 Jan 2011

Background. There has long been a divide between systems practitioners concerned with ``hard'' systems - often involving software and complex technologies - and ``soft systems,'' concerned with social systems and human understanding of systems and human response to complex situations. Both sets of practice seek an underpinning theory or science of systems. However, the relationship of systems science to systems thinking and systems engineering is a process-oriented approach that developed out of US DoD and NASA practices in the 1960s and onwards. Peter Checkland developed ``soft systems methodology'' as a way of introducing systems thinking and systemic understanding into complex organizational problem situations. Derek Hitchins in a number of publications between 1990 and 2010 has presented a set of definitions that allow us to characterize ``systems'' and systemic behavior. There is increasing interest in ``complex adaptive systems.'' Recently Bristol University has sought to unify the ``Systems engineering'' and ``systems thinking'' communities using the principle ``hard systems exist within soft systems.'' This view, while useful for integrating the domains of ``systems engineers'' and ``systems thinkers,'' does not account the fact that many systems - all natural ones and many man-made ones as well - exist and evolve independent of human intentionality. Jack Ring’s ``System value cycle'' describes an integrated system approach that addresses a “community situation” focusing in turn on ``value,'' ``purpose,'' and ``system.'' But Lawson’s ``system coupling model'' shows systems as being configured from available assets in response to a ``problem situation.'' There is a renewed interest within INCOSE - often regarded as a purely ``hard systems'' and process-oriented community in systems science as a theoretical underpinning for systems engineering and in the use of ontologies and model based systems engineering methods to improve the rigor and predictability of systems engineering endeavor.

Theory of systems
Applying systems approach to a domain
“whole systems thinking”, “understanding systems in a human context”
Establish human interest and intentionality wrt systems
Making choices about how to create and adjust a new system or modify an existing one the better to achieve a purpose.
Purpose and value
Stakeholder alignment
Properties of interest;
Appropriate boundary
Effect of proposed changes

Success in systems engineering derives from the realization that the design activity requires a <emphasis>team</emphasis> approach. As one proceeds from conceptual design to the subsequent phases of the life cycle, the actual team ``make-up'' will vary in terms of the specific expertise required and the number of project personnel assigned. Early in the conceptual and preliminary design phases, there is a need for a few highly qualified individuals with broad technical knowledge who understand the customer’s requirements and the user’s operational environment, the major functional elements of the system and their interface relationships, and the general process for bringing a system into being. These key individuals are those who understand and believe in the systems approach and know when to call on the appropriate disciplinary expertise for assistance. The objective is to ensure the early consideration of all of the ``design-for'' requirements described in 

As the design progresses, the need for representation from the various design disciplines increases. Referring to (which is an extension of the concept in ), a systems engineering implementation goal is to ensure the proper integration of the design disciplines as appropriate to the need. Depending on the project, there may be relatively few individuals assigned, or there may be hundreds involved. Further, some of the expertise desired may be located within the same physical facility, whereas other members of the project team may be remotely located in supplier organizations (both locally and internationally). In a system-of-systems (SOS) context, the project team may need to include individuals representing major “interfacing” systems.

A major goal in the implementation of the system engineering process is first to understand system requirements and the expectations of the customer and then to provide the technical guidance necessary to ensure that the ultimate system configuration will meet the need. Realization of this goal depends on providing the right personnel and material resources at the right location and in a timely manner. Such resources may include a combination of the following:	

\begin{enumerate}
\item Engineering technical expertise (e.g., aeronautical engineers, civil engineers, electrical engineers, mechanical engineers, software engineers, reliability engineers, logistics engineers, and environmental engineers)
\item Engineering technical support (e.g., technicians, component part specialists, computer programmers, model builders, drafting personnel, test technicians, and data analysts)
\item Nontechnical support (e.g., marketing, purchasing and procurement, contracts, budgeting and accounting, industrial relations, manufacturing personnel, and logistics supply chain specialists). 
\end{enumerate}

Whatever the case, the objective of systems engineering is to promote the “team” approach and to create the proper working environment for the necessary ongoing communications and exchange of information on a continuing day-to-day basis. illustrates this essential integration function, and the “system engineer” must be knowledgeable of the various disciplines, their respective objectives, and when to integrate these requirements into the overall design process. The organization for systems engineering is discussed further in
