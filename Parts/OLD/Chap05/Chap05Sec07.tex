An understanding of the interrelationships among the factors identified in Figures 5.2 and 5.3 is essential if the full benefit of systems engineering is to be realized. There is a need to ensure that the applicable engineering disciplines responsible for the design of individual system elements are properly integrated. This need extends to the proper implementation of concurrent engineering to address the life cycles for the product and for the supporting capabilities of production, support, and phase-out, as is illustrated in Figure 6.2. A good communication network, with local-area and wide-area capability, must also be in place and available to all critical project personnel. This is a particular challenge when essential project personnel are located remotely, often worldwide.

Successful implementation of the systems engineering process depends not only on the availability and application required to accomplish the overall objective. Although the steps described in Section 6.3 may be specified for a given program, successful implementation (and the benefits to be derived) will not be realized unless the proper organizational environment is established that will encourage it to happen. There have been numerous instances where a project organization included a “systems engineering” function but where the impact on design has been almost nonexistent, resulting in objectives not being met.

Although some of the benefits associate with application of the concepts and principle of systems engineering have been provided through this chapter, it may be helpful to provide a compact summary for reference. Accordingly, application of the systems engineering process can lead to the following benefits:

Reduction in the cost of system design and development, production and/or construction, system operation and support, system retirement and material disposal; hence, a reduction in life-cycle cost should occur. Often it is perceived that the implantation of systems engineering will increase the cost of system acquisition. Although there may be a few more steps to perform during the early (conceptual and preliminary) system design phases, this investment could significantly reduce the requirements in the integration, test, and evaluation efforts accomplished late in the detail design and development phase. The bottom-up approach involved in making the system work can be simplified if a holistic engineering effort is initiated from the beginning. In addition, experience indicates that the early emphasis on systems engineering can result in a cost savings later in the production, operations and support, and retirement phases of the life cycle.

Reduction in system acquisition time (or the time from the initial identification of a customer need to the delivery of a system to the customer). Evaluation of all feasible alternative approaches to design early in the life cycle (with the support of available design aids such as the use of CAD technology) should help to promote greater design maturity earlier. Changes can be incorporated at an early stage before the design is “fixed” and costlier to modify. Further, the results should enable a reduction in the time that it takes for final system integration, tests, and evaluation.

More visibility and a reduction in the risks associated with the design decision-making process. Increased visibility is provided through viewing the system from a long-term and life-cycle perspective. The long-term impacts because of early design decisions and “cause-and-effect” relationships can be assessed at an early stage. This should cause a reduction in potential risks, resulting in greater customer satisfaction.

Without the proper organizational emphasis from top management, the establishment of an environment that will allow for creativity and innovation, a leadership style that will promote a “team” approach to design, and so on, implementation of the concepts and the methodologies described herein may not occur. Thus, systems engineering must be implemented in terms of both technology and management. This joint implementation of systems engineering is the responsibility of systems engineering management. Part V of this textbook, consisting of two chapters, is devoted to this important activity.

The NAE Grand Challenges FROM ASEE PAPER.
