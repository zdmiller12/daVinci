Engineering education has been subjected to in-depth study every decade or so, beginning with the Mann Report in 1918. The most recent and authoritative study was conducted by the National Academy of Engineering (NAE) and published in 2005 under the title Educating the Engineer of 2020.
Although acknowledging that certain basics of engineering will not change, this NAE report concluded that the explosion of knowledge, the global economy, and the way engineers will work will reflect on ongoing evolution that began to gain momentum at the end of the twentieth century. The report gives three overarching trends to be reckoned with the engineering educators, interacting with engineering leaders in government and industry:

\begin{enumerate}
\item The economy in which we work will be strongly influenced by the global marketplace for engineering services, evidenced by the outsourcing of engineering jobs, a growing need for interdisciplinary and system-based approaches, demands for new paradigms of customization, and an increasingly international talent pool.
\item The steady integration of technology in our public infrastructures and lives will call for more involvement by engineers in the setting of public policy and for participation in the civic arena.
\item The external forces in society, the economy, and the professional environment will all challenge the stability of the engineering workforce and affect our ability to attract the most talented individuals to an engineering career.
\end{enumerate}

Continuing technological advances have created in increasing demand for engineers in most fields. But certain engineering and technical specialties will be merged or become obsolete with time. There will always be a demand for engineers who can synthesize and adapt to changes. The astute engineer should be able to detect trends and plan for satisfactory transitions by acquiring knowledge to broaden his or her capability. To help obtain the knowledge needed for adaptation is one aim of this systems engineering textbook.

\subsection{Systems Engineering}\index{Systems Engineering}

From its modest beginnings more than a half-century ago, systems engineering (SE) is emerging as an effective technologically based interdisciplinary process for bringing systems and services into being. While the primary focus is nominally on the entities themselves, systems engineering is inherently oriented to considering “the end before the beginning”. It concentrates on what the entities are intended to do before determining what the entities are, with form following function.

Systems engineering is concerned with the engineering of human-made systems and with systems analysis. In the first case, emphasis is one the process of bringing systems into being, beginning with the identification of a need or deficiency and extending through requirements determination, functional analysis and allocation, design synthesis and evaluation, design validation, deployment and distribution, operation and support, sustainment, and phase-out and disposal. In the second case, focus is on the improvement of systems already in being. By utilizing the iterative process of analysis, evaluation, modification, and feedback, most systems now in existence can be improved in their operational effectiveness, delivered service quality, user affordability, product and environmental sustainability, and stakeholder satisfaction. The systems approach is increasingly being considered by forward-looking private and public organizations and enterprises. It is applicable to most types of systems, encompassing the human activity domains of communication, defense, education, healthcare, manufacturing, transportation, and other named in the National Academy of Engineering compilation of Grand Challenges.

\subsection{Operations Research and Management Science}\index{Operations Research and Management Science}

Operations research (OR) and the management sciences (MS), provide a body of systematic knowledge embracing models, modeling, and simulation approaching for performing systems analysis (SA). Applicable to operations and management as the name implies, OR/MS has been found to be necessary but not sufficient for Engineering Economy to be rigorously linked to operations. Accordingly, both OR and MS will be recognized in the sections that follow as primary enablers for EE@SL to provide more opportunities to “think about the end before the beginning”.

Systems and Systems Science

Sillito, H. G. (2012). Integrating systems science, systems thinking, and systems engineering: Understanding the differences and exploiting the synergies. Proceedings of INCOSE International Symposium, Rome, July 2012.

One score and five years ago, our beloved Council (on SE, now international) was founded by visionary Charter Members to serve society through Systems Engineering, and to do so in collaboration with the disciplines (domains) of engineering. Our INCOSE of today has embraced and subsumed that vision, held also by then President Brian Mar. Brian recognized and promulgated domain centric systems engineering without the acronym DCSE, believing that SE is best developed within the domains of engineering. I supported that founding position.

It was not until INCOSE began seeking inroads into the venerable engineering organizations of ABET and ASEE that we felt compelled to explain our intent. I credit colleague Dennis Buede and my co-author Elizabeth McCrae for the inspiration leading to the acronym pair that now includes systems centric systems engineering (SCSE), ad developed in our seminal 2005 INCOSE paper. The non-threatening and peacekeeping benefit of DCSE/SCSE partitioning has proven its value over time in our relations with the engineering profession.

Not all SE degree programs are administered through the classical departmental structure of the host institution. Although most undergraduate programs are classically organized, the following variants will be found:

\begin{enumerate}
\item There are instances where an academic administrative unit will be the home for more than one-degree program or major; e.g., Systems Engineering and Industrial Engineering. The department name may or may not subsume the names of all degree programs.
\item There are instances where the institution will offer both a SE Centric (SEC) and a Domain Centric SE (DCSE) program; e.g., Systems Engineering and Manufacturing Systems Engineering. The DCSE program may be administered in an interdepartmental mode, whereas the SEC program will usually be administered within a department.
\item In those instances where an institution offers a SEC program at the basic and advanced levels, all are usually administered within a department. This is also true for DCSE programs, except that the SE component may not exist at all degree levels.
\end{enumerate}

The above variants are mentioned to emphasize that one must be aware of the administrative and organizational home for a degree program of interest.

The focus in this paper is always on the degree program itself. In discussing the basic and advanced level programs in the SEC and SCSE categories, this program focus will be strengthened by recognizing that Systems Engineering is broad in nature. It cannot be viewed in the same context as the traditional engineering disciplines. This notwithstanding, many domains of engineering are seeking a better technological balance by adopting systems thinking. This is the primary reason for the rapid growth in the number of engineering domains adding a systems component to their programs.
