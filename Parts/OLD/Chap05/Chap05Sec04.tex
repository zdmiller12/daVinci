The content of this chapter is anchored by the domains of systems science and systems engineering, beginning with the former and ending with the latter. Accordingly, it is important to recognize that at least one professional organization exists for each domain. For systems science, there is the International Society for the System Sciences (ISSS), originally named the ``Society for General Systems Research.'' ISSS was established at the 1956 meeting of the American Association for the Advancement of Science under the leadership of biologist Ludwig von Bertalanffy, economist Kenneth Boulding, mathematician-biologist Anatol Rapoport, neurophysiologist Ralph Gerard, psychologist James Miller, and anthropologist Margaret Mead.

The founders of the International Society for the System Sciences felt strongly that the systematic (holistic) aspect of reality was being overlooked or downgraded by the conventional disciplines, which emphasize specialization and reductionist approaches to science. They stressed the need for more general principles and theories and sought to create a professional organization that would transcend the tendency toward fragmentation in the scientific enterprise. The reader interested in exploring the field of systems science and learning more about the work of the International Society for System Sciences, may visit the ISSS website at http://www.isss.org.

Technology, human-made, and human-modified systems comprise the core of this chapter, with science and systems science as the foundation. It is the engineered system that is the justification for almost every concept, process, method, and model presented in this textbook. Accordingly, and in-depth understanding of the engineered system (through a focused definition and description) is not part of this chapter. It is deferred for the opening of Chapter 6, where the paradigm for bringing systems into being is treated at a high level. The purpose is to clarify the distinction between systems that are engineered and systems that exist naturally.

The most prominent professional organization for systems engineering is the International Council on Systems Engineering (INCOSE). Originally named the ``National Council on Systems Engineering,'' NCOSE was chartered in 1991 in the United States; it has now expanded worldwide to become the leading society to develop, nurture, and enhance the interdisciplinary approach and means to enable the realization of successful systems. INCOSE has strong and enduring ties with industry, academia, and government to achieve the following goals:

\begin{enumerate}
\item Provide a focal point for the dissemination of systems engineering knowledge
\item Promote collaboration in systems engineering education and research
\item Assure the establishment of professional standards for integrity in the practice of systems engineering
\item Improve the professional status of persons engaged in the practice of systems engineering
\item Ecourage governmental and industrial support for research and educational programs that will improve the systems engineering process and its practice. 
\end{enumerate}

An expanded view of systems engineering, as promulgated through the International Council of Systems Engineering, as well as a window into a wealth of information about this relatively new engineering interdisciplines may be obtained at http://www.incose.org.

The ISSS was established in 1956, originally the Society for General Systems Research, an affiliate of the American Association for the Advancement of Science. It is often associated with the work of Ludwig von Bertalanffy and the efforts to develop a General System Theory, but its members and leaders have spanned a wide range of professional backgrounds and practice areas. The international Federation for Systems Research (IFSR: www.ifsr.org ), founded in 1981, is a federation of systems organizations around the world (including the ISSS and since 2011 even INCOSE). Those member organizations cover a spectrum of geographic regions and of specific systems orientations (e.g. cybernetics). INCOSE, founded in 1991, has a membership of nearly 8000 systems engineers, most prominently from aerospace and defense industries in the US, but also spanning many other countries and other domains.

In 2010, INCOSE began an internal working group focused on a systems science, whose charter is to “promote the advancement and understanding of Systems Science and its application to SE” (http://www.incose.org/practice/techactivities/wg/syssciwg/ ). The stated objectives were to: “1) encourage advancement of systems science principles and concepts as they apply to systems engineering, 2) promote awareness of systems science as a foundation for systems engineering, and 3) highlight linkages between systems science theories and empirical practices of systems engineering.”

In concert with this same intent, INCOSE and ISSS began formally inviting representatives as guests to each other’s meetings, in an exchange of ideas and interests. In 2011, ISSS and INCOSE formally signed a memorandum of understanding, based on the following principles:

\begin{enumerate}
\item The ISSS and INCOSE agree to a relationship for mutual benefit, to be reconfirmed every three years. The purpose of the relationship is to further the practices and knowledge jointly in systems sciences and systems engineering.
\item INCOSE members are interested in gaining foundational knowledge in systems science concepts, methods and tools that may be applied in the practice of systems engineering.
\item ISSS members are interested in seeing systems theories applied in practice, and further developing approaches on practical problems in systems engineering, based on the rich legacy of research in the systems sciences and on the feedback from systems engineering experience in applying the theories.
\end{enumerate}

Most scientific and professional societies in the United States interact and collaborate with cognizant but independent honor societies. The cognizant honor society for systems engineering is the Omega Alpha Association (OAA), emerging under the motto “Think About the End Before the Beginning.”  Chartered in 2006 as an international honor association, OAA has the overarching objective of advancing the systems engineering process and its professional practice in service to humankind. Among subordinate objectives are opportunities to (1) inculate a greater appreciation within the engineering profession that every human being decision shapes the human-made world and determines its impact upon the natural world and upon people; (2) advance system design and development morphology through a better comprehension and adaptation of the da Vinci philosophy of thinking about the end before the beginning; that is, determining what designed entities are intended to do before specifying what the entities are and concentrating on the provision of functionality, capability, or a solution before designing the entities per se; and (3) encourage excellence in systems engineering education and research through collaboration with academic institutions and professional societies to evolve robust policies and procedures for recognizing superb academic programs and students. The OAA website, http://www.omegalpha.org, provides information about OAA goals and objectives, as well as the OAA vision for recognizing and advancing excellence in systems engineering, particularly in academia.