Engineering is not science, but an application of science. It is an art composed of skill and ingenuity in adapting knowledge to the uses of humanity. The Accreditation Board for Engineering and Technology has adopted the definition:

Engineering is the profession in which a knowledge of the mathematical and natural sciences gained by study, experience, and practice is applied with judgment to develop ways to utilize, economically, the materials and forces of nature for the benefit of mankind.

This, like most other accepted definitions, emphasizes the applied nature of engineering.

The role of the scientist is to add to mankind’s accumulated body of systematic knowledge and to discover universal laws of behavior. The role of the engineer is to apply this knowledge to situations to produce products and services. To the engineer, knowledge is not an end in itself but is the raw material from which he or she fashions structures, systems, products, and services. Thus, engineering involves the determination of the combination of materials, forces, information, and human factors that will yield a desired result. Engineering activities are rarely carried out for the satisfaction that may be derived from them directly. With few exceptions, their use is confined to satisfying human wants.

Modern civilization depends to a large degree upon engineering. Most products and services used to facilitate work, communication, transportation, and national defense and to furnish sustenance, shelter, and health are directly or indirectly a result of engineering activity. Engineering has also been instrumental in providing leisure time for pursuing and enjoying culture. Through the development of instant communication and rapid transportation, engineering has provided the means for both cultural and economic improvement of humanity.

Science is the foundation upon which the engineer builds toward the advancement of mankind. With the continued development of science and the worldwide application of engineering, the general standard of living may be expected to improve and further increase the demand for those things that contribute to people’s love for the comfortable and beautiful. The fact that these human wants may be expected increasingly to engage the attention of engineers is, in part, the basis for the incorporation of humanistic and social considerations in engineering curricula. An understanding of these fields is essential as engineers seek solutions to the complex socioecological problems of today.

5.1.1 Historical Perspectives of Science and Engineering

Historically, science and engineering were two separate cultures that did not interact. Only in the past two or three centuries has a scientific foundation been explicitly built and used for engineering. A description of how the relationship between science and engineering develops as part of the maturation process is given for engineering disciplines such as mechanical and chemical engineering, by Shaw (1990) and Finch (1951). The summary of this process provided here is from Hybertson (2009).

The first two parts of this section address broader perspectives on science and engineering in general. The last three parts discuss perspectives on SE and SS.

On 7 Nov 2013, at 1846, Hillary Sillitto hsillitto@googlemail.com wrote:

Second, someone said ``SE is not E.''  An interesting philosophical discussion, but so politically fraught as not to be useful, indeed actively damaging to our cause: the claim makes many enemies and no friends. SE is ``a kind of'' E and goes beyond the boundaries of the technology-focused kind of “E” into the realms of the deliciously ambiguous other meaning of ``E'', ``to cunningly arrange...'' and ‘twas I who said it. Moreover, I believe this inappropriate notion that SE is E, to be the main issue restraining the advancement of SE. Nor is the clear distinction between SE and E a political bogyman – on the contrary, it may be a political trump card. And the hackneyed idea of calling upon the ambiguous employment of “E” is surely unworthy. We all know what engineering is - even the French.

Consider a very direct analogy between SE and E, on the one hand, with Architectural Design and Civil Engineering on the other. The Norman Fosters of this world are not engineers, but instead are creative, innovative designers of sophisticated buildings and complexes – often in competition. They employ firms of civil design and construction engineers to undertake building work, frequently supervised by the architects.

It would be naïve to imagine that such international architects are of the same discipline as civil engineers; they are clearly not. And architects are concerned with the conception and design of whole buildings, whole complexes, and in their future context and environment, into which their putative solution must fit in full harmony. Architects must surely know enough about civil engineering not to design something that cannot be built - but they are not civil engineers per se. They may not even make their own fly-thu simulations and models of future buildings/complexes. OTOH, civil engineers would never have designed the Sydney Opera House. So, should it be with SEs and Es. Sytems investigation/conception/CONOPS/architecting/etc., are different from engineering: the more complex the problem and its solution, the more different - SE must be free to be creative and innovative, so long as it stays within the bounds of that which can be engineered.

So, may I reiterate: SE is not E – they are different, in that they are different disciplines, with different tenets, principles, methods, practices and, yes, different sciences. Yet, they are connected – as Architects are connected with Civil Engineers – in that SAs and SEs conceive, design and architect the whole, while Es make the products and artefacts. Cheers, Derek H.

The initial phase is ad hoc practice, in which problems are solved by talented amateurs and virtuosos. The solutions of these craftsmen then gradually move into the folklore, become routine, and begin limited production. This gradually leads to commercial-scale production. The problems of large-scale production then stimulate the development of a supporting science that defines the principles that explain the behavior that practitioners have already observed and exploited. As the science matures, it finally gets out ahead of current practice in the sense that it formulates principles that can be used to improve current practice: more predictability, fewer errors, more precise safety margins that allow less over-design and therefore lower cost. At this point, people can be educated in both the theory and practice of the field, and the field is now a mature discipline.

As both science and engineering mature, they contribute to each other. Engineering builds thing, the properties of which science needs to analyze and understand to enable more efficient and economical engineering. It should be noted that not all supporting science is stimulated by or results from problems in an engineering discipline. Often a science already exists and is found to be helpful to an engineering discipline.

How are (or can or should) the systems sciences and systems engineering (be) related?  For the web conference for the INCOSE (International Council of Systems Engineers) Complex Systems Working Group on November 22, 2010, I decided to present a personal perspective on linkages. The ideas were essentially in two parts with:

\begin{enumerate}
\item The systems movement as a system of ideas including
\item The systems science community as some individuals, some organizations, and some publications; and
\item Ten frames to guide thinking and discussion about changes in society, economics and technology in the 21st century (based on Ing (2011)); and
\item John N. Warfield’s ``A Challenge for Systems Engineers: To Evolve towards Systems Science,'' published in INCOSE Insight (2007).
\end{enumerate}

The first point reflects my view of the breadth and diversity of the system sciences. The second point reviewed some challenges presented by John N. Warfield, who was both a pioneer in the systems engineering community and a luminary in the systems sciences community. As a guide for the web conference, I provided a context map.

\subsection{Science, Engineering, and Systems Engineering}\index{Science, Engineering, and Systems Engineering}

Excerpts from Systems Engineering, Alberto Sols, Pp21-23

Several recognized scholars and historians have written extensively on science and engineering, from its meaning and origin to its current status [Leonard, 1989; van Doren, 1991; Checkland, 1993; McClellan and Dorn, 1999; Gribbin, 2002; Bryson, 2004; Priestley, 2014]. It is not the purpose of this chapter to present a thorough review of science and engineering, but just to set a basic framework for the better understanding of the advent of systems engineering as a discipline in the second half of the 20th century.

        1.1. Science

Long gone are the days when our ancestors would look terrified at a thunderstorm, at an erupting volcano or at so many other natural phenomena. In their inability to come up with an explanation they resorted to gods and mythology to explain what in one way or another caught their attention or shocked them.

Science is one of the major activities of the human mind. It can be defined as the knowledge gained from the physical and material world through systematic study, observation and experimentation. The word ``science'' comes from the Latin word scientia, which means knowledge. Science as we know it originated about 2500 years ago in the ancient Greece. Thales was a pre-Socratic Greek philosopher who lived in Miletus (Asia Minor) circa 624 to 546 B.C.E. He is known as the first true scientist as, according to available records, he was the first person to try to explain natural phenomena without reference to gods and mythology. Thales tried to develop testable and verifiable explanations about the universe and the behavior of things. The Greeks succeeded first by discovering nature, willing as they were to consider the world as a natural system, governed by natural laws, thus leaving the gods out. The Greeks were interested primarily in knowledge and understanding, and only secondarily in practical usefulness. This focus on knowledge per se is what we would call today basic research.

Science involves structured and unbiased observations, from which understanding of the observed phenomena is gained. The conducted observations enable the formulation of general laws or rules; further observations will result in those laws either being re-confirmed, rejected, or in need of amendments. That is the process of the so-called scientific method, which has been described by a number of authors [Carey, 2010; Gauch, 2012]. The scientific method is the body of techniques for investigating phenomena and gaining new knowledge, as well as for validating or amending what we already know. Knowledge is gained through the recording and analysis of observable, measurable, empirical and reproducible evidence, subject to the laws of reasoning. The scientific method is thus the procedure by which knowledge is generated in empirical studies. The main advantage of the scientific method is that it is unprejudiced as theories are accepted based on the result of tests and experiments that anyone can reproduce. The scientific method is divided into six steps, as depicted in figure 1.1. The first step is conducted somehow informally and noc necessarily in a structured manner. Second, a hypothesis is constructed and it will allow the prediction of future observations. Third, an experiment is designed to test the validity of the hypothesis. Fourth, the experiment is run and data are collected. Fifth, based on the gathered evidence the hypothesis is approved or rejected in total or in part. If the hypothesis is not approved then it is necessary to go back to the second step, and to reformulate the hypothesis and/or redesign the experiment, as needed. In any case, the sixth and final step is to report the results. This last step is essential because it enables the needed repeatability of the experiments, which is the cornerstone of the scientific method.

Research enables the advancement of science. In the words of Hungarian-born and Nobel Prize winner Albert Szent-Gyorgyi, to do research is to see what everybody has seen and to thing what nobody has thought. Research is frequently divided into two categories: basic research and applied research; the former is performed without thought of practical ends, seeking to widen the understanding of the phenomena of a certain field, whereas the latter refers to scientific studies and activities that seek to solve practical problems or to achieve specific goals. Basic research is the true pacemaker of technological progress. If basic research aims at extending the boundaries of fundamental understanding, applied research is always directed toward some identified societal problem or need.

Systems Engineering from? Systems engineering (SE) is an interdisciplinary field of engineering, that focuses on the development and organization of complex systems. It is the “art and science of creating whole solutions to complex problems,” for example: signal processing systems, control systems and communication system, or other forms of high-level modelling and design in specific fields of engineering.

Systems Methodologies. There are several types of Systems Methodologies, that is, disciplines for analysis of systems. For example:

\begin{itemize}
\item Soft systems methodology (SSM): in the field of organizational studies is an approach to organizational process modelling, and it can be used both for general problem solving and in the management of change. It was developed in England by academics at the University of Lancaster Systems Department through a ten-year Action Research programme.
\item System development methodology (SDM) in the field of IT development is a general term applied to a variety of structured, organized processes for developing information technology and embedded software systems.
\end{itemize}
    
\subsection{Synergies Between Systems Science and Engineering}\index{Synergies Between Systems Science and Engineering}

The purpose of this section is to describe some of the commonalities, complementary roles, and potential synergies between systems science and systems engineering. Two general themes are prominent:

\begin{enumerate}
\item The essence of the relationships between SS and SE, the stable relation and potential synergies between them.
\item The changes needed in SE and SS to address emerging challenges and opportunities in the systems arena.
\end{enumerate}

Theme 1: As human endeavors, both fields will continue to evolve, as will their potential connections to each other. Nevertheless, the essence of the relationship between these fields is expected to be relatively stable. The intent is that this paper will mature over time in its understanding of the essence of the relationship and will evolve to reflect the evolution of the fields and to indicate future directions for both fields.

Theme 2: Both SE and SS face emerging challenges and opportunities, and the authors believe that both need to change to adequately address the challenges and exploit the opportunities. This paper will provide a rationale for those beliefs and outline a possible path forward.

There have been movements recently to more closely align systems science and systems engineering organizations, both to understand and leverage the synergies and to mutually foster the needed changes.

Following this introduction, Section 2 of the paper describes specific organizations which represent significant members and practitioners of each area: the International Society for the Systems Sciences (ISSS: www.isss.org), the International Federation for Systems Research (IFSR: www.ifsr.org ), and the International Council on Systems Engineering (INCOSE: www.incose.org ). Section 3 addresses some of the histories of systems science and systems engineering. Section 4discusses emerging changes and challenges for system engineering, and Section 5 discusses emerging changes and challenges for systems science. Section 6 explores some of the interdependencies and synergies between systems science and systems engineering, including ways in which science and engineering (i.e., understanding and acting) more generally might be brought closer together. Section 7 offers some very preliminary conclusions.