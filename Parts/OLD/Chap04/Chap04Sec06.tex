Society is confronted with two important interconnected environments, the physical and the economic. Their success of humans in altering the physical environment to produce products and services depends upon a knowledge of physical laws. However, the worth of these products and services lies in their utility measured in economic items. There are numerous examples of structures, machines, processes, and systems that exhibit excellent physical design but have little economic merit.

\subsection{The Physical and Economic Interface}\index{The Physical and Economic Interface}

Want satisfaction in the economic environment and engineering proposals in the physical environment are linked by the production or the construction process. Figure 4.1 illustrates the relationship between engineering proposals, production or construction, and want satisfaction.

Figure 4.1 Relating Physical and Economic Factors

In dealing with the physical environment engineers have a body of physical laws upon which to base their reasoning. Such laws as Boyle’s law, Ohm’s law, and Newton’s laws of motion were developed primarily by collecting and comparing numerous similar instances and by the use of an inductive process. These laws may then be applied by deduction to specific instances. They are supplemented by many formulas and known facts, all of which enable the engineer to come to conclusions that match the facts of the physical environment within narrow limits. Much is known with certainty about the physical environment.

Much less, particularly of a quantitative nature, is known about the economic environment. Since economics is involved with the actions of people, it is apparent that economic laws must be based upon their behavior. Economic laws can be no more exact than the description of the behavior of people acting singly and collectively.

The usual function of engineering is to manipulate the elements of one environment, the physical, to create value in a second environment, the economic. However, engineers sometimes have a tendency to disregard economic feasibility and are often applied in practice by the necessity for meeting situations in which action must be based on estimates and judgment. Yet today’s engineering graduates are increasingly finding themselves in positions in which their responsibility is extended to include economic considerations.

Engineers can readily extend their inherent ability of analysis to become proficient in the analysis of the economic aspects of engineering application. Furthermore, the engineer who aspires to a creative position in engineering will find proficiency in economic analysis helpful. The large percentage of engineers who will eventually be engaged in managerial activities will find such proficiency a necessity.

Initiative for the use of engineering rests, for the most part, upon those who will concern themselves with social and economic consequences. To maintain the initiative, engineers must operate successfully in both the physical and economic sectors of the total environment. It is the objective of engineering economy to prepare engineers to cope effectively with the bi-environmental nature of engineering application.

\subsection{Physical and Economic Efficiency}\index{Physical and Economic Efficiency}

Both individuals and organizations possess limited resources. This makes it necessary to produce the greatest output for a given input – that is, to operate at high efficiency. Thus, the search is not merely for a good opportunity for the employment of limited resources, but for the best opportunity.

People are continually seeking to satisfy their wants. They give up certain utilities to gain others that they value more. This is essentially an economic process, in which the objective is the maximization of economic efficiency.

Engineering is primarily a producer activity that comes into being to satisfy human wants. Its objective is to get the greatest end result per unit of resource expenditure. This is essentially a physical process in which the objective is the maximization of physical efficiency, which may be stated as

				(1.1)

If interpreted broadly enough, physical efficiency is a measure of the success of engineering activity in the physical environment. However, the engineer must be concerned with two levels of efficiency. On the first level is physical efficiency expressed as outputs divided by inputs of such physical units as Btu’s, kilowatts, and foot-pounds. When such physical units are involved, efficiency will always be less than unity, or less than 100\%

On the second level are economic efficiencies. These are expressed in terms of economic units of output divided by economic units of input, each expressed in terms of a medium of exchange such as money. Economic efficiency may be stated as

				(1.2)

It is well known that physical efficiencies over 100\% are not possible. However, economic efficiencies can exceed 100\% and must do so for economic ventures to be successful.

Physical efficiency is related to economic efficiency. For example, a power plant may be profitable in economic terms even through its physical efficiency in converting units of energy in coal to electrical energy may be relatively low. As an example, in the conversion of energy in a certain plant, assume that the physical efficiency is only 36\%. Assuming that output Btu’s in the form of electrical energy have an economic worth of \$14.65 per million and that input Btu’s in the form of coal have an economic cost of \$1.80 per million, then

Since physical processes are of necessity carried out at efficiencies less than 100\% and economic ventures are feasible only if they attain efficiencies greater than 100\%, it is clear that in feasible economic ventures the economic worth per unit of physical output must always be greater than the economic cost per unit of physical input. Consequently, economic efficiency must depend more upon the worth and cost per unit of physical outputs and inputs than upon physical efficiency. Physical efficiency is always significant, but only to the extent that it contributes to economic efficiency.
In the final evaluation of most ventures, even those in which engineering plays a leading role, economic efficiencies must take precedence over physical efficiencies. This is because the function of engineering is to create utility in the economic environment by altering elements of the physical environment.