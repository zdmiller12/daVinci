Economics is not simply a topic on which to express opinions or vent emotions. It is a systematic study of cause and effect, showing what happens when you do specific things in specific ways. In economics analysis, the methods used by a Marxist economist like Oskar Lange did not differ in any fundamental way from the methods used by a conservative economist like Milton Friedman. It is these basic economic principles that this book is about.

One of the ways of understanding the consequences of economic decisions is to look at them in terms of the incentives they create, rather than simply the goals they pursue. This means that consequences matter more than intentions – and not just the immediate consequences, but also the longer run repercussions 

When economists analyze prices, wages, profits, or the international balance of trade, for example, it is from the standpoint of how decisions in various parts of the economy affect the allocation of scarce resources in a way that raises or lowers the material standard of living of the people as a whole.

\subsection{History and the Role of Economics}\index{History and the Role of Economics}

While there are controversies in economics, as there are in science, this does not mean that the basic principles of chemistry or physics are just a matter of opinion. Einstein’s analysis of physics, for example, was not just Einstein’s opinion, as the world discovered at Hiroshima and Nagasaki. Economic reactions may not be as spectacular or as tragic, as of a given day, but the worldwide depression of the 1930’s plunged millions of people into poverty, even in the richest countries, producing malnutrition in countries with surplus food, probably causing more deaths around the world than those at Hiroshima and Nagasaki.

Conversely, when India and China – historically, two of the poorest nations on earth – began in the late twentieth century to make fundamental changes in their economic policies, the economies began growing dramatically. It has been estimated that 20 million people in India rose out of destitution in a decade. In China, the number of people living on a dollar a day or less fell from 374 million – one third of the country’s population in 1990 – to 128 million by 2004, now just 10 percent of a growing population. In other words, nearly a quarter of a billion Chinese were now better off as a result of a change in economic policy.

Things like this are what make the study of economics important – and not just a matter of opinions or emotions. Economics is a tool of cause and effect analysis, a body of tested knowledge – and principles derived from that knowledge.

Most of us hate to even think of having to make such choices. Indeed, as we have already seen, some middle-class Americans are distressed at having to make much milder choices and trade-offs. But life does not ask us what we want. It presents us with options. Economics is one of the way of trying to make the most of those options.

The History of Economics. People have been talking about economic issues, and some writing about them, for thousands of years, so it is not possible to put a specific date on when the study of economics began as a separate field. Modern economics is often dated from 1776, when Adam Smith wrote his classic, The Wealth of Nations, but there were substantial books devoted to economics at least a century earlier, and there was a contemporary school of French economists called the Physiocrats, some of whose members Smith met while traveling in France, years before he wrote his own treatise on economics. What was different about The Wealth of Nations was that it became the foundation for a whole school of economists who continued and developed its ideas over the next two generations, including such leading figures as David Ricardo (1772-1823) and John Stuart Mill (1806-1873), and the influence of Adam Smith has to some extent persisted on to the present day. No such claim could be made for any previous economist, despite many people who had written knowledgeably and insightfully on the subject in earlier times.

More than two thousand years ago, Xenophon, a student of Socrates, analyzed economics policies in ancient Athens. In the Middle Ages, religious conceptions of a ``fair'' or ``just'' price, and a ban on usury, let Thomas Aquinas to analyze the economic implications of those doctrines and the exceptions that might therefore be morally acceptable. For example, Aquinas argued that selling something for more than was paid for it could be done ``lawfully'' when the seller has ``improved the thing in some way,'' or as compensation for risk, or because of having incurred costs of transportation. Another way of saying the same thing is that much that looks like sheer taking advantage of other people is often in fact compensation for various costs and risks incurred in the process of bringing goods to consumers or lending money to those who seek to borrow.

However far economists have moved beyond the medieval notion of a fair and just price, that concept still lingers in the background of much present-day thinking among people who speak of things being sold for more or less their ``real'' value and individuals being paid more or less than they are ``really'' worth, as well as in such emotionally powerful but empirically undefined notions as price ``gouging.''

From isolated individual writing about economics there evolved, over time, coherent schools of thought, people writing within a common framework of assumptions - the medieval scholastics, of whom Thomas Aquinas was a prominent example, the mercantilists, the classical economists, the Keynesians, the ``Chicago School,'' and others. Individuals coalesced into various schools of thought even before economists became a profession in the nineteenth century.

Classical Economics. Within a decade after Sir James Steuart’s multi-volume mercantilist treatise, Adam Smith’s The Wealth of Nations was published and dealt a historic blow against mercantilist theories and the whole mercantilist conception of the world. Smith conceived of the nation as all the people living in it. Thus, you could not enrich a nation by keeping wages down in order to export. “No society can surely be flourishing and happy, of which the far greater part of the members is poor and miserable,” Smith said. He also rejected the notion of economic activity as a zero-sum process, in which one nation loses what another nation gains. To him, all nations could advance at the same time in terms of the prosperity of their respective peoples, even though military power – a major concern of the mercantilists – was of course relative and a zero-sum competition.

In short, the mercantilists were preoccupied with the transfer of wealth, whether by export surpluses, imperialism, or slavery – all of which benefit some at the expense of others. Adam Smith was concerned with the creation of wealth, which is not a zero-sum process. Smith rejected government intervention in the economy to help merchants – the source of the name ``mercantilism'' – and instead advocated free markets along the lines of the French economists, the Physiocrats, who had coined the term laissez faire. Smith repeatedly excoriated special-interest legislation to help ``merchants and manufacturers,'' whom he characterized as people whose political activities were designed to deceive and oppress the public. In the context of the times, laissez faire was a doctrine that opposed government favors to business.

The most fundamental difference between Adam Smith and the mercantilists was that Smith did not regard gold as being wealth. The very title of his book - The Wealth of Nations - raised the fundamental question of what wealth consisted of. Smith argued that wealth consisted of the goods and services which determined the standard of living of the people - the whole people, who to Smith constituted the nation. Smith rejected both imperialism and slavery – on economic grounds as well as moral grounds, saying that the ``great fleets and armies'' necessary for imperialism ``acquire nothing which can compensate the expense of maintaining them.'' The Wealth of Nations closed by urging Britain to give up dreams of empire. As for slavery, Smith considered it economically inefficient, as well as morally repugnant, and dismissed with contempt the idea that enslaved Africans were inferior to people of European ancestry.

Although Adam Smith is today often regarded as a “conservative” figure, he in fact attacked many of the dominant ideas and interests of his own times. Moreover, the idea of a spontaneously self-equilibrating system – the market economy – first developed by the Physiocrats and later made part of the tradition of classical economics by Adam Smith, represented a radically new departure, not only in analysis of social causation but also in seeing a reduced role for political, intellectual, or other elites as guides or controllers of the masses.

For centuries, landmark intellectual figures from Plato onward had discussed what policies wise leaders might impose for the benefit of society in various ways. But, in the economy, Smith argued that governments were giving ``a most unnecessary attention'' to things that would work out better if left alone to be sorted out by individuals interacting with one another and making their own mutual accommodations. Government intervention in the economy, which mercantilist Sir James Steuart saw as the role of a wise ``statesman,'' Smith saw as the notions and actions of ``crafty'' politicians, who created more problems than they solved.

While The Wealth of Nations was not the first systematic treatise on economics, it became the foundation of a tradition known as classical economics, which built upon Smith’s work over the next century. Not all earlier treatises were mercantilist by any means. Books by Richard Cantillon in the 1730s and by Ferdinando Galiani in 1751, for example, presented sophisticated economic analyses, and Fancois Quesnay’s Tableau Economique in 1758, contained insights that inspired the transient but significant school of economists called the Physiocrats. But, as already noted, these earlier pioneers created no enduring school of leading economists in later generations who based themselves on their work, as Adam Smith did.

\subsection{Entreprenuership and Profit}\index{Entreprenuership and Profit}

In the most fundamental humans are all, with each of our actions, always and invariably profit-seeking entrepreneurs. Whenever we act, we employ some physical means (things valued as goods) — at a minimum our body and its standing room, but in most cases also various other, ``external'' things — so as to divert the ``natural'' course of events (the course of events we expect to happen if we were to act differently) to reach some more highly valued anticipated future state of affairs instead. With every action we aim at substituting a more favorable future for a less favorable one that would result if we were to act differently. In this sense, with every action we seek to increase our satisfaction and attain a psychic profit.

But every action is threatened also with the possibility of loss. For every action refers to the future and the future is uncertain or at best only partially known. Every actor, in deciding on a course of action, compares the value of two anticipated states of affairs: the state he wants to effect through his action but that has not yet been realized, and another state that would result if he were to act differently but cannot come into existence, because he acts the way he does. This makes every action a risky enterprise. An actor can always fail and suffer a loss. He may not be able to effect the anticipated future state of affairs - that is, the actor’s technical knowledge, his ``know how'' may be deficient or it may be temporarily ``superseded,'' due to some unforeseen external contingencies. Or else, even if he has successfully produced the desired state of physical affairs, he may still consider his action a failure and suffer a loss, if this state of affairs provides him with less satisfaction than what he could have attained had he chosen otherwise (some earlier-on rejected alternative course of action) - that is, the actor’s speculative knowledge - his knowledge of the temporal change and fluctuation of values and valuations - may be deficient.

Since all our actions display entrepreneurship and are aimed at being successful and yielding the actor a profit, there can be nothing wrong with entrepreneurship and profit. Wrong, in any meaningful sense of the term, are only failure and loss, and accordingly, in all our actions, we always try to avoid them.

The question of justice, i.e., whether a specific action and the profit or loss resulting from it is ethically right or wrong, arises only in connection with conflicts.

Since every action requires the employment of specific physical means — a body, standing room, external objects — a conflict between different actors must arise, whenever two actors try to use the same physical means for the attainment of different purposes. The source of conflict is always and invariably the same: the scarcity of physical means. Two actors cannot at the same time use the same physical means – the same bodies, spaces and objects - for alternative purposes. If they try to do so, they must clash. Therefore, in order to avoid conflict or resolve it if it occurs, an action-able principle and criterion of justice is required, i.e., a principle regulating the just or ``proper'' vs. the unjust or ``improper'' use and control (ownership) of scarce physical means.

Logically, what is required to avoid all conflict is clear: It is only necessary that every good be always and at all times owned privately, i.e., controlled exclusively by some specified individual (or individual partnership or association), and that it be always recognizable which good is owned and by whom, and which is not. The plans and purposes of various profit-seeking actor-entrepreneurs may then be as different as can be, and yet no conflict will arise so long as their respective actions involve only and exclusively the use of their own, private property.

Yet how can this state of affairs: the complete and unambiguously clear privatization of all goods, be practically accomplished? How can physical things become private property in the first place; and how can conflict be avoided from the beginning of mankind on?

A single — praxeo-logical — solution to this problem exists and has been essentially known to mankind since its beginnings — even if it has only been slowly and gradually elaborated and logically re-constructed. To avoid conflict from the start, it is necessary that private property be founded through acts of original appropriation. Property must be established through acts (instead of mere words or declarations), because only through actions, taking place in time and space, can an objective — inter-subjectively ascertainable — link be established between a particular person and a particular thing. And only the first appropriator of a previously un-appropriated thing can acquire this thing as his property without conflict. For, by definition, as the first appropriator he cannot have run into conflict with anyone in appropriating the good in question, as everyone else appeared on the scene only later.

This importantly implies that while every person is the exclusive owner of its own physical body as his primary means of action, no person can ever be the owner of any other person’s body. For we can use another person’s body only indirectly, i.e., in using our directly appropriated and controlled own body first. Thus, direct appropriation temporally and logically precedes indirect appropriation; and accordingly, any non-consensual use of another person’s body is an unjust misappropriation of something already directly appropriated by someone else.

All just property, then, goes back directly or indirectly, through a chain of mutually beneficial — and thus likewise conflict-free — property-title transfers, to original appropriators and acts of original appropriation. Mutatis mutandis, all claims to and uses made of things by a person who had neither appropriated or produced these things, nor acquired them through a conflict-free exchange from some previous owner, are unjust. And by implication: All profits gained or losses suffered by an actor-entrepreneur with justly acquired means are just profits (or losses); and all profits and losses accruing to him through the use of unjustly acquired means are unjust.

This analysis applies in full also to the case of the entrepreneur in the term’s narrower definition, as a capitalist-entrepreneur.

The capitalist entrepreneur acts with a specific goal in mind: to attain a monetary profit. He saves or borrows saved money, he hires labor, and he buys or rents raw materials, capital goods and land. He then proceeds to produce his product or service, whatever it may be, and he hopes to sell this product for a monetary profit. For the capitalist, ``profit appears as a surplus of money received over money expended and loss as a surplus of money expended over money received. Profit and loss can be expressed in definite amounts of money.'' (Mises 1966, p. 289)

As all action, a capitalist enterprise is risky. The cost of production - the money expended - does not determine the revenue received. In fact, if the cost of production determined price and revenue, no capitalist would ever fail. Rather, it is anticipated prices and revenues that determine what production costs the capitalist can possibly afford.

Yet, the capitalist does not know what future prices will be paid or what quantity of his product will be bought at such prices. This depends exclusively on the buyers of his product, and the capitalist has no control over them. The capitalist must speculate what the future demand will be. If he is correct and the expected future prices do correspond to the later fixed market prices, he will earn a profit. On the other hand, while no capitalist aims at making losses - because losses imply that he must ultimately give up his function as a capitalist and become either a hired employee of another capitalist or a self-sufficient producer-consumer - every capitalist can err with his speculation and the actually realized prices fall below his expectations and his accordingly assumed production cost, in which case he does not earn a profit but incurs a loss.

While it is possible to determine exactly how much money a capitalist has gained or lost in the course of time, his money profit or loss do not imply much if anything about the capitalist’s state of happiness, i.e., about his psychic profit or loss. For the capitalist, money is rarely if ever the ultimate goal (safe, may be, for Scrooge McDuck, and only under a gold standard). In practically all cases, money is a means to further action, motivated by still more distant and ultimate goals. The capitalist may want to use it to continue or expand his role as a profit-seeking capitalist. He may use it as cash to be held for not yet determined future employments. He may want to spend it on consumer goods and personal consumption. Or he may wish to use it for philanthropic or charitable causes, etc.

What can be unambiguously stated about a capitalist’s profit or loss is this: His profit or loss are the quantitative expression of the size of his contribution to the well-being of his fellow men, i.e., the buyers and consumers of his product, who have surrendered their money in exchange for his (by the buyers) more highly valued product. The capitalist’s profit indicates that he has successfully transformed socially less highly valued and appraised means of action into socially more highly valued and appraised ones and thus increased and enhanced social welfare. Mutatis mutandis, the capitalist’s loss indicates that he has used some more valuable inputs for the production of a less valuable output and so wasted scarce physical means and impoverished society.

Money profits are not just good for the capitalist, then, they are also good for his fellow men. The higher a capitalist’s profit, the greater has been his contribution to social welfare. Likewise, money losses are bad not only for the capitalist, but they are bad also for his fellow men, whose welfare has been impaired by his error.

The question of justice: of the ethically ``right'' or ``wrong'' of the actions of a capitalist-entrepreneur, arises, as in the case of all actions, again only in connection with conflicts, i.e., with rivalrous ownership claims and disputes regarding specific physical means of action. And the answer for the capitalist here is the same as for everyone, in any one of his actions.

The capitalist’s actions and profits are just, if he has originally appropriated or produced his production factors or has acquired them - either bought or rented them - in a mutually beneficial exchange from a previous owner, if all his employees are hired freely at mutually agreeable terms, and if he does not physically damage the property of others in the production process. Otherwise, if some or all of the capitalist’s production factors are neither appropriated or produced by him, nor bought or rented by him from a previous owner (but derived instead from the ex-propriation of another person’s previous property), if he employs non-consensual, ``forced'' labor in his production, or if he causes physical damage to others’ property during production, his actions and resulting profits are unjust.

In that case, the unjustly harmed person, the slave, or any person in possession of proof of his own un-relinquished older title to some or all of the capitalist’s means of production, has a just claim against him and can insist on restitution — exactly as the matter would be judged and handled outside the business world, in all civil affairs.