The application of knowledge gained from the scientific method concerns prediction of the future. If X happens then Y will happen. That is the utility. Y may include probability functions.

\subsection{Classic Greek Philosophy}\index{Classic Greek Philosophy}

Philosophy in classical Greece is the ultimate origin of the western conception of the nature of a thing. According to Aristotle, the philosophical study of human nature itself originated with Socrates, who turned philosophy from study of the heavens to study of the human things.

Socrates is said to have studied the question of how a person should best live, but he left no written works. It is clear from the works of his students Plato and Xenophon, and also by what was said about him by Aristotle (Plato’s student), that Socrates was a rationalist and believed that the best life and the life most suited to human nature involved reasoning. The Socratic school was the dominant surviving influence in philosophical discussion in the Middle Ages, amongst Islamic, Christian, and Jewish philosophers.

The human soul in the works of Plato and Aristotle has a divided nature, divided in a specifically human way. One part is specifically human and rational, and divided into a part which is rational on its own, and a spirited part which can understand reason. Other parts of the soul are home to desires or passions similar to those found in animals. In both Aristotle and Plato, spiritedness (thumos) is distinguished from the other passions (epithumiai). The proper function of the ``rational'' was to rule the other parts of the soul, helped by spiritedness. By this account, using one’s reason is the best way to live, and philosophers are the highest types of human.

Aristotle – Plato’s most famous student – made some of the most famous and influential statements about human nature. In his works, apart from using a similar scheme of a divided human soul, some clear statements about human nature are made:
\begin{enumerate}
\item Man is a conjugal animal, meaning an animal which is born to couple when an adult, thus building a household (oikos) and, in more successful cases, a clan or small village still run upon patriarchal lines.
\item Man is a political animal, meaning an animal with an innate propensity to develop more complex communities the size of a city or town, with a division of labor and law-making. This type of community is different in a kind from a large family and requires the special use of human reason.
\item Man is mimetic animal. Man loves to use his imagination (and not only to make laws and run town councils). He says ``we enjoy looking at accurate likenesses of things which are themselves painful to see, obscene beasts, for instance, and corpses.''  And the ``reason why we enjoy seeing likenesses is that, as we look, we learn and infer what each is, for instance, ‘that is so and so.’''
\end{enumerate}
For Aristotle, reason is not only what is most special about humanity compared to other animals, but it is also what we were meant to achieve at our best. Much of Aristotle’s description of human nature is still influential today. However, the particular teleological idea that humans are ``meant'' or intended to be something has become much less popular in modern times.

For the Socratics, human nature, and all natures, are metaphysical concepts. Aristotle developed the standard presentation of this approach with his theory of four causes. Every living thing exhibits four aspects or ``causes'': matter, form, effect, and end. For example, an oak tree is made of plant cells (matter), grew from an acorn (effect), exhibits the nature of oak trees (form), and grows into a fully mature oak tree (end). Human nature is an example of a formal cause, according to Aristotle. Likewise, to become a fully actualized human being (including fully actualizing the mind) is our end. Aristotle (Nicomachean Ethics, Book X) suggests that the human intellect is ``smallest in bulk'' but the most significant part of the human psyche, and should be cultivated above all else. The cultivation of learning and intellectual growth of the philosopher, which is thereby also the happiest and least painful life.

Although this new realism applied to the study of human life from the beginning – for example, in Machiavelli’s works – the definitive argument for the final rejection of Aristotle was associated especially with Francis Bacon. Bacon sometimes wrote as if he accepted the traditional four causes (``It is a correct position that ``true knowledge is knowledge by causes.'' And causes again are not improperly distributed into kinds: the material, the formal, the efficient, and the final.'') but he adapted these terms and rejected one of the three:

``But of these the final cause rather corrupts than advances the sciences, except such as have to do with human action. The discovery of the formal is despaired of. The efficient and the material (as they are investigated and received, that is, as remote causes, without reference to the latent process leading to the form) are but slight and superficial, and contribute little, if anything, to true and active science.''

This line of thinking continued with Rene Descartes, whose new approach returned philosophy or science to its pre-Socratic focus upon non-human things. Thomas Hobbes, then Grammatist Vico, and David Hume all claimed to be the first to properly use a modern Baconian scientific approach to human things.

Hobbes famously followed Descartes in describing humanity as matter in motion just like machines. He also very influentially described man’s natural state (without science and artifice) as one where life would be “solitary, poor, nasty, brutish, and short.”  Following him, John Locke’s philosophy of empiricism also saw human nature as a tabula rasa. In this view, the mind is at birth a “blank slate” without rules, so data are added, and rules for processing them are formed solely by our sensory experiences.

Jean-Jacques Rousseau pushed the approach of Hobbes to an extreme and criticized it at the same time. He was a contemporary and acquaintance of Hume, writing before the French Revolution and long before Darwin and Freud. He shocked Western civilization with his Second Discourse by proposing that humans had once been solitary animals, without reason or language or communities, and had developed these things due to accidents of pre-history. (This proposal was also less famously made by Giambattista Vico.)  In other words, Rousseau argued that human nature was not only not fixed, but not even approximately fixed compared to what had been assumed before him. Humans are political, and rational, and have language now, but originally, they had none of these things. This in turn implied that living under the management of human reason might not be a happy way to live at all, and perhaps there is no ideal way to live. Rousseau is also unusual in the extent to which he took the approach of Hobbes, asserting that primitive humans were not even naturally social. A civilized human is therefore not only imbalanced and unhappy because of the mismatch between civilized life and human nature, but unlike Hobbes, Rousseau also became well known for the suggestion that primitive humans had been happier, “noble savages.”

Rousseau’s conception of human nature has been seen as the origin of many intellectual and political developments of the 19th and 20th centuries. He was an important influence upon Kant, Hegel, and Marx, and the development of German idealism, historicism, and romanticism.

What human nature did entail, according to Rousseau and the other modernists of the 17th and 18th centuries, were animal-like passions that led humanity to develop language and reasoning, and more complex communities (or communities of any kind, according to Rousseau).

In contrast to Rousseau, David Hume was a critic of the oversimplifying and systematic approach of Hobbes, Rousseau, and some others whereby, for example, all human nature is assumed to be driven by variations of selfishness. Influenced by Hutcheson and Shaftesbury, he argued against oversimplification. On the one hand, he accepted that, for many political and economic subjects, people could be assumed to be driven by such simple selfishness, and he also wrote of some of the more social aspects of ``human nature'' as something which could be destroyed, for example if people did not associate in just societies. On the other hand, he rejected what he called the ``paradox of the sceptics'', saying that no politician could have invented words like \textit{honorable}, \textit{shameful}, \textit{lovely}, \textit{odious}, \textit{noble}, and \textit{despicable}, unless there was not some natural ``original constitution of the mind.''

Hume – like Rousseau – was controversial in his own time for his modernist approach, following the example of Bacon and Hobbes, of avoiding consideration of metaphysical explanations for any type of cause and effect. He was accused to being an atheist. He wrote:

``We needn’t push our researches so far as to ask `Why do we have humanity, i.e. a fellow-feeling with others?'  It’s enough that we experience this as a force in human nature. Our examination of causes must stop somewhere.''

After Rousseau and Hume, the nature of philosophy and science changed, branching into different disciplines and approaches, and the study of human nature changed accordingly. Rousseau’s proposal that human nature is malleable became a major influence upon international revolutionary movements of various kinds, while Hume’s approach has been more typical in Anglo-Saxon countries, including the United States.

\subsection{What Man Has Built}\index{What Man Has Built}

Humanity (???) Twentieth Century, more than our ancestors, must attempt to understand the varied peoples with whom he shares an increasingly small planet. To reach this understanding he needs to know the cultures which molded other people’s outlook, the history that carried them to this point.

How to select the civilizations that must be examined in a limited series of books on the history of the world’s cultures?  That is the subject of Jaques Barzun’s introduction to the Time-Life series entitled The Great Ages of Man. Mr. Barzun, Dean of Faculties and Provost of Columbia University, is one of the pre-eminent cultural historians of this generation. He describes how the “revolution … in our conception of humanity” wrought by the emergence of “dozens of new peoples, new states, and new pasts” has made essential the realization that “nothing human is alien.”

In explaining the criteria for the selection of historic cultures examined in this series, he also suggests the path that present-day cultures may follow in the future.

At the end of this introductory booklet is a comprehensive chronological chart. This shows the meaningful relationships of great cultures the world over – in time, in place, and in the interpenetrations discussed by Dean Barzum. A number of these cultures provide the subject matter for volumes in this series. This overall chart will be found useful in connection with each book; in addition, each book includes an appropriate segment from the chart. THE EDITORS OF TIME-LIFE BOOKS (1965 Time Inc.)