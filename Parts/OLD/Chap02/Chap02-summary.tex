Science and systems thinking is increasingly being used to tackle a wide variety of subjects in fields such as computing, engineering, epidemiology, information science, health, manufacture, management, and the environment.

\begin{itemize}
\item Systemology and synthesis. The science of systems or their formation is called systemology. Problems and problem complexities faced by humankind do not organize themselves along disciplinary lines. New arrangements of scientific and professional efforts based on the common attributes and characteristics of needs and problems should contribute to the progress. More attention should be paid to human action and praxeology applications at the macro-level to help understand both the economic and non-economic dimensions of the world in which we live.
\item The formation of interdisciplines began in the middle of the last century and that has brought about an evolutionary synthesis of knowledge. This has occurred not only within science, but between science and technology and between science and humanities. The forward progress of systemology in the study of large-scale complex systems requires a synthesis of science and the humanities in addition to a synthesis of science and technology.
\item When synthesizing human-made systems, unintended effects can be minimized and the natural system can sometimes be improved by engineering the larger human-modified system instead of engineering only the human-made. If system evaluation is applied beyond the human-made, then the boundary of the target system (meant to include both natural and human-made systems) should be adopted as the boundary of the human-modified domain.
\item Systems are as pervasive as the universe in which they exist. They are as grand as the universe itself or as infinitesimal as the atom. Systems appeared first in natural forms, but with the advent of human beings, a variety of human-made systems have come into existence. In recent decades, we have begun to understand structure and characteristics of natural and human-made systems in a scientific way.
\end{itemize}

Upon completion of , the reader will have obtained essential insight into systems and systems thinking, with an orientation toward systems engineering and analysis. The system definitions, classifications, and concepts presented in this chapter are intended to impart a general understanding about the following:

\begin{enumerate}
\item System classifications, similarities, and dissimilarities
\item The fundamental distinction between natural and human-made systems
\item The elements of a system and the position of the system in the hierarchy of systems
\item The domain of systems science, with consideration of cybernetics, general systems theory, and systemology
\item Technology as the progenitor for the creation of technical systems, recognizing its impact on the natural world
\item The transition from the machine or industrial age to the Systems Age, with recognition of its impact upon people and society
\item System complexity and scope and the demands these factors make on engineering in the Systems Age
\item The range of contemporary definitions of systems engineering used within the profession.
\end{enumerate}