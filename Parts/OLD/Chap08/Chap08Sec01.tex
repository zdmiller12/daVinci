\section{System Life-Cycle Phases}\index{System Life-Cycle Phases}

A newly identified human need, or an evolving need, reveals a new set of system requirements. If a decision is made to seek a solution for the need, then a decision is needed whether to consider other needs in designing the solution. Based on an initial determination regarding the scope of needs, the basic phases of conceptual design and onward through system retirement and phase-out are then applicable, as described in the paragraphs that follow. The scope of needs may contract or expand, but the scope should be stabilized as early as possible during conceptual design, preferably based on an evaluation of value and cost by the customer.

\subsection{Chronology of Life-Cycle Phases}\index{Chronology of Life-Cycle Phases}

Figure 8.1 8.18 illustrates the major life-cycle process phases and selected milestones for a generic system. This is the model that will serve as a frame of reference for material presented in subsequent sections. Included are the basic steps in the systems engineering process (i.e., requirements analysis, functional analysis and allocation, synthesis, trade-off studies, design evaluation, etc.

Figure 8.1 Here (From SEA 2.4)

Program phases described in are not intended to convey specific tasks, or time periods, or levels of funding, or numbers of iterations. Individual program requirements will vary from one application to the next. The figure exhibits an overall process that needs to be followed during system acquisition and deployment. Regardless of the type, size, and complexity of the system, there is a conceptual design requirement (i.e., to include requirements analysis), a preliminary design requirement, and so on. Also, to ensure maximum effectiveness, the concepts presented in must be properly ``tailored'' to the particular system application being addressed.

shows the basic steps in the systems engineering process to be iterative in nature, providing a top-down definition of the system, and then proceeding down to the subsystem level (and below as necessary). Focused on the needs, and beginning with conceptual design, the completion of Block 0.2 defines the system in functional terms (having identified the ``what'' from a requirements perspective). These ``whats'' are translated into an applicable set of ``hows'' through the iterative process of functional partitioning and requirements allocation, together with conceptual design synthesis, analysis, and evaluation. This conceptual design phase is where the initial configuration of the system (or system architecture) is defined.

During preliminary design, completion of Block 1.1 defines the system in refined functional terms providing a top-down definition of subsystems with preparation for moving down to the component level. Here the ``whats'' are extracted from (provided by) the conceptual design phase. These ``whats'' are translated into an applicable set of ``hows'' through the iterative process of functional partitioning and requirements allocation, together with preliminary design synthesis, analysis, and evaluation. This preliminary design phase is where the initial configuration of subsystems (or subsystem architecture) is defined.

Blocks 1.1–1.7 are an evolution from Blocks 0.1–0.8, Blocks 2.1–2.5 are an evolution from Blocks 1.1–1.7, and Blocks 3.1–3.6 are an evolution from Blocks 2.1–2.5. The overall process reflected in the figure constitutes an evolutionary design and development process. With appropriate feedback and design refinement provisions incorporated, the process should eventually converge to a successful design. The functional definition of the system, its subsystems, and its components serves as the baseline for the identification of resource requirements for production and then operational use (i.e., hardware, software, people, facilities, data, elements of support, or a combination thereof).

\subsection{Checklist of Life\-Cycle Steps}\index{Checklist of Life\-Cycle Steps}

This section addresses certain steps in the systems engineering process and, in doing so, provides basic insight and knowledge about the following:

\begin{enumerate}
\item Identifying and translating a problem or deficiency into a definition of need for a system that will provide a preferred solution
\item Accomplishing advanced system planning and architecting in response to the identified need
\item Developing system operational requirements describing the functions that the system must perform to accomplish its intended purpose(s) or mission(s);
\item Conducting exploratory studies leading to the definition of a technical approach for system design
\item Proposing a maintenance concept for the sustaining support of the system throughout its planned life cycle
\item Identifying and prioritizing technical performance measures (TPMs) and related criteria for design
\item Accomplishing a system-level functional analysis and allocating requirements to various subsystems and components
\item Performing systems analysis and producing trade-off studies
\item Developing a system specification
\item Conducting a conceptual design review
\end{enumerate}

The completion of the steps above constitutes the system definition process at the conceptual level. Although the depth, effort, and cost of accomplishing these steps may vary, the process is applicable to any type or category of system, complex or simple, large or small. It is important that these steps, which encompass the front end of the systems engineering process, be thoroughly understood. Collectively, they serve as a learning objective with the goal being to provide a comprehensive step-by-step approach for addressing this critical early phase of the systems engineering process.
    
\subsection{A Hypothetical Example}\index{A Hypothetical Example}

System life-cycle phases and their associated steps will be more easily explained and visualized if presented in connection with a hypothetical but realistic example. Accordingly, this is the time to offer and describe an example that will lend itself as a basis for discussing the material in this chapter. The chosen example is that of engineering a solution to the problem possessed by a municipality that exists due to the division of the community into east and west by the presence of a river. This will be referred to as a river crossing problem.

Assume that a regional transportation authority is faced with the problem of providing for increased two-way traffic flow across a river that divides a growing municipality (to illustrate the overall process, this particular example is developed further in, and under preliminary design in 

In considering alternative system design approaches, different technology applications are investigated. For instance, in response to the river crossing problem (identified in ), alternative design concepts may include a tunnel under the river, a bridge spanning the river, an airlift capability over the river, the use of barges and ferries on the river, or possibly re-routing the river itself. Then a feasibility study would be accomplished to determine a preferred approach. In performing such a study, one must address limiting factors such as geological and geotechnical, atmospheric and weather, hydrology and water flow, as well as the projected capability of each alternative to meet life-cycle cost objectives. In this case, the feasibility results might tentatively indicate that some type of bridge structure spanning the river appears to be best.

Returning to the regional public transportation authority facing the problem of providing for capability that will allow for a significant increase in the two-way traffic flow across a river dividing a growing municipality (a <emphasis>what</emphasis>). Further study of the problem (i.e., the current deficiency) revealed requirements for the two-way flow of private vehicles, taxicabs, buses, rail and rapid transit cars, commercial vehicles, large trucks, people on motor cycles and bicycles, and pedestrians across the river. Through advanced system planning and consideration of possible architectures, various river crossing concepts were proposed and evaluated for physical and economic feasibility. These included going under the river, on the river, spanning the river, over the river, or possibly re-routing the river itself. Feasibility considerations determined that the river is not a good candidate for rerouting, both physically and due to its role in providing navigable traffic flow upstream and downstream.

Results from the study indicated that the most attractive approach is the construction of some type of a bridge structure spanning the river (a how). From this point on, it is necessary to delve further into the operational requirements leading to the selection and evaluation of a bridge type (suspension, pier and superstructure, causeway, etc.) by considering some detailed ``design-to'' factors as below

While some of the specific design-to qualitative and quantitative factors introduced in this example may vary from one project to the next, this example is presented with the intention of illustrating those considerations that must be addressed early in conceptual design as it pertains to the river crossing problem.
    
\subsection{Other Hypothetical Examples}\index{Other Hypothetical Examples}


These additional illustrations. The five illustrations presented, derived from the results of feasibility analyses conducted earlier, are representative of typical ``needs'' for which system operational requirements must be defined at the inception of a program, and must serve as the basis for all subsequent program activities. These requirements must not only be implemented within the bounds of the specific system configuration in question but must also consider all possible external interfaces that may exist.

For example, as illustrated in, one may be dealing with a number of different systems, all of which are closely related and may have a direct impact on each other. Also, there may be some ``sharing'' of capabilities across the board; for example, the air and ground transportation systems utilizing some of the same components which operate as part of the communication systems. Thus, the design of any new system must consider all possible impacts that it will have on other systems within the same <emphasis>SOS</emphasis> configuration (as shown in the figure), as well as those possible impacts that the other systems may have on the new system.

The methodology employed is basically the same for any system, whether the subject is a relatively small item as part of the river crossing bridge, installed in an aircraft or on a ship, a factory, or a large one-of-a-kind project such as the community hospital involving design and construction. In any case, the system must be defined in terms of its projected mission, performance, operational deployment, life cycle, utilization, effectiveness factors, and the anticipated environment.