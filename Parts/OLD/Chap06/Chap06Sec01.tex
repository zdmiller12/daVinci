The tangible outcome of systems engineering is an engineered or technical system, whether human-made or human-modified. Systems concepts, general systems theory, and systems thinking presented in provide a scientific foundation for engineering the pervasive domain of the human-made world. This section and the material that follows pertain to the organized technological activities for bringing engineered systems into being. To begin on solid ground, it is necessary to define the engineered system in terms of its characteristics.

\subsection{Characteristics of the Engineered System}\index{Characteristics of the Engineered System}

A technical or engineered system is a human-made or human-modified system designed to meet functional purposes or objectives. Systems can be engineered well or poorly. The phrase ``engineered system'' in this book implies a well-engineered system. A well-engineered system has the following characteristics:

\begin{enumerate}
\item Engineered systems have functional purposes in response to identified needs and have the ability to achieve stated operational objectives
\item Engineered systems are brought into being and operate over a life cycle, beginning with identification of needs and ending with phase-out and disposal
\item Engineered systems have design momentum that steadily increases throughout design, production, and deployment, and then decreases throughout phase-out, retirement, and disposal
\item Engineered systems are composed of a harmonized <emphasis>combination of resources, such as facilities, equipment, materials, people, information, software, and money
\item Engineered systems are composed of subsystems and related components that interact with each other to produce a desired system response or behavior
\item Engineered systems are part of a hiearchy and are influenced by external factors from larger systems of which they are a part and from sibling systems from which they are composed
\item Engineered systems are embedded into the natural world and interact with it in desirable as well as undesirable ways
\end{enumerate}

Systems engineering is defined in several ways. Basically, systems engineering is a functionally-oriented, technologically-based interdisciplinary process for bringing systems and products (human-made entities) into being as well as for improving existing systems. The outcome of systems engineering is the engineered system as previously described. Its overarching purpose is to make the world better, primarily for people. Accordingly, human-made entities should be designed to satisfy human needs and/or objectives effectively while minimizing system life-cycle cost, as well as the intangible costs of societal and ecological impacts.

Organization, humankind’s most important innovation, is the time-tested means for bringing human-made entities into being. While the focus is nominally on the entities themselves, systems engineering embraces a better strategy. Systems engineering concentrates on what the entities are intended to do before determining what the entities are composed of. As simply stated within the profession of architecture, form follows function. Thus, instead of offering systems or system elements and products per se, the organizational focus should shift to designing, delivering, and sustaining functionality, a capability, or a solution.

\subsection{System Life Cycle Engineering}\index{System Life Cycle Engineering}

Design within the system life-cycle context as in Figure 5.1 differs from design in the general sense. Life-cycle-guided design is simultaneously responsive to customer needs (i.e., to requirements expressed in functional terms) and to life-cycle outcomes. Design should not only transform a need into a system configuration but should also ensure the design’s compatibility with related physical and functional requirements. Further, it should consider operational outcomes expressed as producibility, reliability, maintainability, usability, supportability, serviceability, disposability, sustainability, and others, in addition to performance, effectiveness, and affordability.</para>

Concurrent Engineering

A detailed presentation of the elaborate technological activities and interactions that must be integrated over the system life-cycle process is given in Figure 6.2. The progression is iterative from left to right and not serial in nature, as might be inferred.

Figure 6.1 Here (was 2.3)

Although the level of activity and detail may vary, the life-cycle functions described and illustrated are generic. They are applicable whenever a new need or changed requirement is identified, with the process being common to large as well as small-scale systems. It is essential that this process be implemented completely at an appropriate level of detail not only in the engineering of new systems but also in the re-engineering of existing or legacy systems.

Major technical activities performed during the design, production or construction, utilization, support, and phase-out phases of the life cycle are highlighted in Figure 6.1. These are initiated when a new need is identified. A planning function is followed by conceptual, preliminary, and detail design activities. Producing and/or constructing the system are the function that completes the acquisition phase. System operation and support functions occur during the utilization phase of the life cycle. Phase-out and disposal are important final functions of utilization to be considered as part of design for the life cycle.

The numbered blocks in ``map'' and elaborate on the phases of the life cycles depicted in 

The acquisition phase 
The utilization phase
The design phase
The startup phase
The operation phase
The retirement phase

The communication and coordination needed to design and develop the product, the production capability, the system support capability, and the relationships with interrelated systems so that they traverse the life cycle together seamlessly is not easy to accomplish. Progress in this area is facilitated by technologies that make more timely acquisition and use of design information possible. Computer-Aided Design (CAD) technology with internet/intranet connectivity enables a geographically dispersed multidiscipline team to collaborate effectively on complex physical designs.

For certain products, the addition of Computer-Aided Manufacturing (CAM) software can automatically translate approved three-dimensional CAD drawings into manufacturing instructions for numerically controlled equipment. Generic or custom parametric CAD software can facilitate exploration of alternative design solutions. Once a design has been created in CAD/CAM, iterative improvements to the design are relatively easy to make. The CAD drawings also facilitate maintenance, technical support, regeneration (re-engineering), and disposal. A broad range of other electronic communication and collaboration tools can help integrate relevant geographically dispersed design and development activities over the life cycle of the system.

Concern for the entire life cycle is particularly strong within the U.S. Department of Defense (DOD) and its non-U.S. counterparts. This may be attributed to the fact that acquired defense systems are owned, operated, and maintained by the DOD. This is unlike the situation most often encountered in the private sector, where the consumer or user is usually not the producer. Those private firms serving as defense contractors are obliged to design and develop in accordance with DOD directives, specifications, and standards. Because the DOD is the customer and the user of the resulting system, considerable DOD intervention occurs during the acquisition phase.

Many firms that produce for private-sector markets have chosen to design with the life cycle in mind. For example, design for energy efficiency is now common in appliances such as water heaters and air conditioners. Fuel efficiency is a required design characteristic for automobiles. Some truck manufacturers promise that life cycle maintenance costs will be within stated limits. These developments are commendable, but they do not go far enough. When the producer is not the consumer, it is less likely that potential operational problems will be addressed during development. Undesirable outcomes too often end up as problems for the user of the product instead of the producer.