System design is the prime mover of systems engineering, with system design evaluation being its compass. System design requires both integration and iteration, invoking a process that coordinates synthesis, analysis, and evaluation, as is shown conceptually in 6.6 It is essential that the technological activities of synthesis, analysis, and evaluation be integrated and applied iteratively and continuously over the system life cycle. The benefits of continuous improvement in system design are thereby more likely to be obtained.

\subsection{A Morphology for Synthesis, Analysis, and Evaluation}\index{A Morphology for Synthesis, Analysis, and Evaluation}

Figure presents a high-level schematic of the systems engineering process from a product realization perspective. It is a morphology for linking applied research and technologies (Block 0) to customer needs (Block 1). It also provides a structure for visualizing the technological activities of synthesis, analysis, and evaluation. Each of these activities is summarized in the paragraphs that follow, with reference to relevant blocks within the morphology.

Figure 6.6 Here

Synthesis.</title><para><inst></inst>To design is to synthesize, project, and propose what might be for a specific set of customer needs and requirements, normally expressed in functional terms (Block 2). Synthesis is the creative process of putting known things together into new and more useful combinations. Meeting a need in compliance with customer requirements is the objective of design synthesis.</para>
<para>The primary elements enabling design synthesis are the design team (Block 3), supported by traditional and computer-based tools for design synthesis (Block 4). Design synthesis is best accomplished by combining top-down and bottom-up activities (Block 5). Existing and newly developed components, parts, and subsystems are then integrated to generate candidate system designs in a form ready for analysis and evaluation.</para></section>
<section id="ch02lev3sec4"><title id="ch02lev3sec4.title">Analysis.</title><para><inst></inst>Analysis of candidate system and product designs is a necessary but not sufficient ingredient in system design evaluation. It involves the functions of estimation and prediction of DDP values (TPMs) (Block 6) and the determining or forecasting of DIP values from information found in physical and economic databases (Block 7).</para>
<para>Systems analysis and operations research provides a step on the way to system design evaluation, but adaptation of those models and methods to the domain of design is necessary. The adaptation explicitly recognizes DDPs and is developed in <link olinkend="part03" preference="0">Part <xref olinkend="part03" label="III"><inst>III</inst></xref></link> of this book.</para></section>
<section id="ch02lev3sec5"><title id="ch02lev3sec5.title">Evaluation.</title><para><inst></inst>Each candidate design (or design alternative) should be evaluated against other candidates and checked for compliance with customer requirements. Evaluation of each candidate (Block 8) is accomplished after receiving DDP values for the candidate from Block 6. It is the specific values for DDPs (the TPMs) that differentiate (or instance) candidate designs.</para>
<para>DIP values determined in Block 7 are externalities. They apply to and across all candidate designs being presented for evaluation. Each candidate is optimized in Block 8 before being presented for design decision (Block 9). It is in Block 9 that the best candidate is sought. Since the preferred choice is subjective, it should ultimately be made by the customer.

\subsection{Discussion of the 10 Block Morphology}\index{Discussion of the 10 Block Morphology}

This section presents and discusses the functions accomplished by each block in the system design morphology that is exhibited in . The discussion will be at a greater level of detail than the general description of synthesis, analysis, and evaluation given above.
Technologies, exhibited in Block 0, are the product of applied research. They evolve from the activities of engineering research and development and are available to be considered for incorporation into candidate system designs. As a driving force for innovation, technologies are the most potent ingredient for advancing the capabilities of human-made entities.</para>
<para>It is the responsibility of the designer/producer to propose and help the customer understand what might be for each technological choice. Those producers able to articulate and deliver better technological solutions, on time and within budget, will attain and retain a competitive edge in the global marketplace.</para></section>
<section id="ch02lev3sec7"><title id="ch02lev3sec7.title">The Customer (Block 1).</title><para><inst></inst>The purpose of system design is to satisfy customer (and stakeholder) needs and expectations. This must be with the full realization that the perceived success of a design is ultimately determined by the customer, identified in Block 1; the customary being Number 1.</para>
<para>During the design process, all functions to be provided and all requirements to be satisfied should be determined from the perspective of the customer or the customer’s representative. Stakeholder and any other special interests should also be included in the “voice of the customer” in a way that reflects all needs and concerns. Included among these must be ecological and human impacts. Arrow A represents the elicitation of customer needs, desired functionality, and requirements.</para></section>
<section id="ch02lev3sec8"><title id="ch02lev3sec8.title">Need, Functions, and Requirements (Block 2).</title><para><inst></inst>The purpose of this block is to identify and specify the desired behavior of the system or product in functional terms. A market study identifies a need, an opportunity, or a deficiency. From the need comes a definition of the basic requirements, often stated in functional terms. Requirements are the input for design and operational criteria, and criteria are the basis for the evaluation of candidate system configurations.</para>
<para>At this point, the system and its product should be defined by its function, not its form. Arrow A indicates customer inputs that define need, functionality, and operational requirements. Arrows B and C depict the translation and transfer of this information to the design process.</para></section>
<section id="ch02lev3sec9"><title id="ch02lev3sec9.title">The Design Team (Block 3).</title><para><inst></inst>The design team should be organized to incorporate in-depth technical expertise, as well as a broader systems view. Included must be expertise in each of the product life-cycle phases and elements contained within the set of system requirements.</para>
<para>Balanced consideration should be present for each phase of the design. Included should be the satisfaction of intended purpose, followed by producibility, reliability, maintainability, disposability, sustainability, and others. Arrow B depicts requirements and design criteria being made available to the design team and Arrow D indicates the team’s contributed synthesis effort wherein need, functions, and requirements are the overarching consideration (Arrow C).</para></section>
<section id="ch02lev3sec10"><title id="ch02lev3sec10.title">Design Synthesis (Block 4).</title><para><inst></inst>To design is to project and propose what might be. Design synthesis is a creative activity that relies on the knowledge of experts about the state of the art as well as the state of technology. From this knowledge, a few feasible design alternatives are fashioned and presented for analysis. Depending upon the phase of the product life cycle, the synthesis can be in conceptual, preliminary, or detailed form.</para>
<para>The candidate design is driven by both a top-down functional decomposition from Block 2 and a bottom-up combinatorial approach utilizing available system elements from Block 6. Arrow E represents a blending of these approaches. Adequate definition of each design alternative must be obtained to allow for life-cycle analysis in view of the requirements. Arrow F highlights this definition process as it pertains to the passing of candidate design alternatives to design analysis in Block 6.</para></section>
<section id="ch02lev3sec11"><title id="ch02lev3sec11.title">Top-Down and Bottom-Up (Block 5).</title><para><inst></inst>Traditional engineering design methodology is based largely on a bottom-up approach. Starting with a set of defined elements, designers synthesize the system/product by finding the most appropriate combination of elements. The bottom-up process is iterative with the number of iterations determined by the creativity and skill of the design team, as well as by the complexity of the system design.</para>
<para>A top-down approach to design is inherent within systems engineering. Starting with requirements for the external behavior of any component of the system (in terms of the function provided by that component), that behavior is then decomposed. These decomposed functional behaviors are then described in more detail and made specific through an analysis process. Then, the appropriateness of the choice of functional components is verified by synthesizing the original entity. Most systems and products are realized through an intelligent combination of the top-down and bottom-up approaches, with the best mix being largely a matter of judgment and experience.</para></section>
<section id="ch02lev3sec12"><title id="ch02lev3sec12.title">Design Analysis (Block 6).</title><para><inst></inst>Design analysis is focused largely on determining values for cost and effectiveness measures generated during estimation and prediction activities. Models, database information, and simulation are employed to obtain DDP values (or TPMs) for each synthesized design alternative from Block 4. Output Arrow G passes the analysis results to design evaluation (Block 8).</para>
<para>The TPM values provide the basis for comparing system designs against input criteria to determine the relative merit of each candidate. Arrow H represents input from the available databases and from relevant studies.</para></section>
<section id="ch02lev3sec13"><title id="ch02lev3sec13.title">Physical and Economic Databases (Block 7).</title><para><inst></inst>Block 7 provides a resource for the design process, rather than being an actual step in the process flow. There exists a body of knowledge and information that engineers, technologists, economists, and others rely on to perform the tasks of analysis and evaluation. This knowledge consists of physical laws, empirical data, price information, economic forecasts, and numerous other studies and models.</para>
<para>Block 7 also includes descriptions of existing system components, parts, and subsystems, often “commercial off-the-shelf.” It is important to use existing databases in doing analysis and synthesis to avoid duplication of effort. This body of knowledge and experience can be utilized both formally and informally in performing needed studies, as well as in supporting the decisions to follow.</para>
<para>At this point, and as represented by Arrow I, DIP values are estimated or forecasted and provided to the activity of design evaluation in Block 8.</para></section>
<section id="ch02lev3sec14"><title id="ch02lev3sec14.title">Design Evaluation (Block 8).</title><para><inst></inst>Design evaluation is an essential activity within system and product design and the systems engineering process. It should be embedded appropriately within the process and then pursued continuously as design and development progresses.</para>
<para>Life-cycle cost is one basis for comparing alternative designs that otherwise meet minimum requirements under performance criteria. The life-cycle cost of each alternative is determined based on the activity of estimation and prediction just completed. Arrow J indicates the passing of the evaluated candidates to the decision process. The selection of preferred alternative(s) can only be made after the life-cycle cost analysis is completed and after effectiveness measures are defined and applied.</para></section>
<section id="ch02lev3sec15"><title id="ch02lev3sec15.title">Design Decision (Block 9).</title><para><inst></inst>Given the variety of customer needs and perceptions as collected in Block 2, choosing a preferred alternative is not just the simple task of picking the least expensive design. Input criteria, derived from customer and product requirements, are represented by Arrow K and by the DDP values (TPMs) and life-cycle costs indicated by Arrow J. The customer or decision maker must now trade off life-cycle cost against effectiveness criteria subjectively. The result is the identification of one or more preferred alternatives that can be used to take the design process to the next level of detail.</para>
<para>Alternatives must ultimately be judged by the customer. Accordingly, arrow L depicts the passing of evaluated candidate designs to the customer for review and decision. Alternatives that are found to be unacceptable in performance can be either discarded or reworked and new alternatives created. Alternatives that meet all, or the most important, performance criteria can then be evaluated based on estimations and predictions of TPM values, along with an assessment of risk.