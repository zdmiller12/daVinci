Central to the physics of  STEA are the physical aspects of organization emphasized by 

Chester Barnard. Chester Barnard was best known as the author of The Functions of the Executive, perhaps the 20thcentury’s most influential book on management and leadership. 

Barnard offers a systems approach to the study of organization, which contains a psychological theory of motivation and behavior, a sociological theory of cooperation and complex inter dependencies, and an ideology based on a meritocracy. Barnard’s teachings drew on personal insights as a senior executive of ATT in the 1920s and 1930s, and he emphasized the role of the manager as both a professional and as a steward of the corporation. For leadership to be effective, it had to be perceived as legitimate, Barnard maintained. Barnard sensed that the central challenge of management was balancing both the technological and human dimensions of organization.

Chester Barnard was best known as the author of The Functions of the Executive, perhaps the 20th century’s most influential book on management and leadership. Barnard offers a systems approach to the study of organization, which contains a psychological theory of motivation and behavior, a sociological theory of cooperation and complex inter dependencies, and an ideology based on a meritocracy. Barnard’s teachings drew on personal insights as a senior executive of ATT in the 1920s and 1930s, and he emphasized the role of the manager as both a professional and as a steward of the corporation. For leadership to be effective, it had to be perceived as legitimate, Barnard maintained. Barnard sensed that the central challenge of management was balancing both the technological and human dimensions of organization.

The challenge for the executive was to communicate organizational goals and to win the cooperation of both the formal and the informal organization; but he cautioned against relying exclusively on incentive schemes to win that cooperation. Responsibility in terms of the honor and faithfulness with which managers carry out their responsibilities is the most important function of the executive. Published: 2010

Chester Barnard and the Systems Approach to Nurturing Organizations Andrea Gabor Associate Professor, and Michael R. Bloomberg Professor of Business Journalism, Department of Journalism Baruch College City University of New York One Bernard Baruch Way 55 Lexington at 24th St. New York, NY 10010 (646) 312-3970 AAGabor@aol.com, andrea.gabor@baruch.cuny.edu 1999- Joseph T. Mahoney Investors in Business Education Professor of Strategy, \& Director of Graduate Studies, Department of Business Administration College of Business University of Illinois at Urbana-Champaign 350 Wohlers Hall 1206 South Sixth Street Champaign, IL 61820 (217) 244-8257 josephm@illinois.edu Chester Barnard (1886-1961) was best known as the author of The Functions of the Executive, perhaps the 20th century’s most influential book on management and leadership.1 The book emphasizes competence, moral integrity, rational stewardship, professionalism, and a systems approach, and was written for posterity. For generations, The Functions of the Executive proved to be an inspiration to the leading thinkers in a host of disciplines. Perrow writes that: ``This ... remarkable book contains within it the seeds of three distinct trends of organizational theory that were to dominate the field for the next three decades.''

One was the institutional theory as represented by Philip Selznick [1957]; another was the decision-making school as represented by Herbert Simon [1947]; the third was the human relations school [Mayo, 1933; Roethlisberger \& Dickson, 1939]" (1986: 63). 2 Barnard’s work also influenced sociology’s Parsons and Gouldner and informed the institutional economics of Williamson (1975, 2005). Indeed, Andrews states that: “The Functions of the Executive remains today, as it has been since its publication, the most thought-provoking book on organization and management ever written by a practicing executive” (1968: xxi). Barnard combined the capacity for abstract thought 1 As of July 25th, 2010, Barnard’s (1938) Functions of the Executive had been cited over 8,000 times (Google Scholar). See Bedeian and Wren (2001) for their ranking of the top 25 most influential management books of the 20th century with Taylor (1911) and Barnard (1938) occupying the top two positions. 2 Classic works influenced by Barnard’s The Function of the Executive include: Boulding (1956), Coser (1956), Cyert \& March (1963), Dalton (1959), Downs (1967), Gouldner (1954), Homans (1950), Katz \& Kahn (1966), Likert (1961), March \& Simon (1958), Mayo (1945), McGregor (1960), Merton (1949), Mintzberg (1973), Selznick (1957), Simon (1947), Thompson (1967), and Williamson (1975). Of particular note is the Barnard-Simon connection (Simon, 1991, 1994; Wolf, 1995a). It is worth nothing, however, that although Barnard knew both Roethlisberger and Mayo, he later claimed to have known little about the Hawthorne studies, which were completed before he wrote Functions of the Executive. Barnard did serve as a major influence on Likert (1961) and McGregor (1960). 1 with the ability to apply reason to professional experiences toward developing a “science of organization” (1938: 290).3

The challenge for the executive was to communicate organizational goals and to win the cooperation of both the formal and the informal organization; but he cautioned against relying exclusively on incentive schemes to win that cooperation. Responsibility in terms of the honor and faithfulness with which managers carry out their responsibilities is the most important function of the executive. Published: 2010 URL: 

One was the institutional theory as represented by Philip Selznick [1957]; another was the decision-making school as represented by Herbert Simon [1947]; the third was the human relations school [Mayo, 1933; Roethlisberger \& Dickson, 1939]" (1986: 63). 2 Barnard’s work also influenced sociology’s Parsons and Gouldner and informed the institutional economics of Williamson (1975, 2005). Indeed, Andrews states that: ``The Functions of the Executive remains today, as it has been since its publication, the most thought-provoking book on organization and management ever written by a practicing executive'' (1968: xxi). Barnard combined the capacity for abstract thought 1 As of July 25th, 2010, Barnard’s (1938) Functions of the Executive had been cited over 8,000 times (Google Scholar). See Bedeian and Wren (2001) for their ranking of the top 25 most influential management books of the 20th century with Taylor (1911) and Barnard (1938) occupying the top two positions. 2 Classic works influenced by Barnard’s The Function of the Executive include: Boulding (1956), Coser (1956), Cyert \& March (1963), Dalton (1959), Downs (1967), Gouldner (1954), Homans (1950), Katz \& Kahn (1966), Likert (1961), March \& Simon (1958), Mayo (1945), McGregor (1960), Merton (1949), Mintzberg (1973), Selznick (1957), Simon (1947), Thompson (1967), and Williamson (1975). Of note is the Barnard-Simon connection (Simon, 1991, 1994; Wolf, 1995a). It is worth nothing, however, that although Barnard knew both Roethlisberger and Mayo, he later claimed to have known little about the Hawthorne studies, which were completed before he wrote Functions of the Executive. Barnard did serve as a major influence on Likert (1961) and McGregor (1960). 1 with the ability to apply reason to professional experiences toward developing a ``science of organization'' (1938: 290).3

The Systems Approach to Nurturing Organizations Andrea Gabor Joseph T. Mahoney Baruch College, City University of New York University of Illinois at Urbana Champaign, College of Business.
    
\subsection{Organization Theory per Thusen}\index{Organization Theory per Thusen}

\subsection{Organization Theory per Torgesen}\index{Organization Theory per Torgesen}

Organization of the efforts of individuals is an invention of man. Organizations are often thought of as consisting of or being made up of people. Thought persons are always associated with organizations, it is not persons that are organized but the actions or in fact more clearly the muscular forces of persons. These muscular forces are of course physical.

This concept of organizations is embodied in Barnard’s definition which states, “An organization is the consciously coordinated forces or actions of two or more persons.”  This discussion will be based on this concept which is thoroughly analyzed in Barnard’s The Functions of the Executive. It is the physical aspects of organizations that are to be examined. The author is not unaware of and has often felt what is described as the spirit or esprit de corps of organizations. Such subjective reactions to participants are undoubtedly very important in organizations, but they are subjective and therefore difficult to analyze and evaluate; it is particularly difficult to determine the causes or origins of favorable and unfavorable subjective reactions.

The physical aspects of organizations which are observable and important aspects of organized action can be quite objectively delineated. For some strange reason, few observers have devoted their energies to this task. Rather the emphasis has been placed upon subjective aspects and the viewpoints held about these aspects are as varied as the number of observers and so perhaps of limited use.

It is believed that if attention were first directed to the objective aspects and if these became well understood, it is likely that the subjective aspects of organization could be more profitably dealt with.
There is much to be gained from examinations of the physical manifestations of organizations.

Since this discussion is based on Barnard’s definition of organization which considers the basic ingredient of organization to be the ``efforts or activities of persons,'' it is desirable to consider the physical attributes of persons and personal efforts or activities.

Physical characteristics of persons. A person has certain physical aspects. He is a discrete entity that has weight or mass, temperature, light reflective capacity, color, etc., as do inanimate masses of matter. He further has the capacity to ingest or absorb food, water, air, heat energy, light energy, etc., and the ability to expel waste products, air, heat and light energy. By this process he maintains an equilibrium within the physical environment as long as he lives. The individual also converts physical intake into physical energy. It is this physical energy produced by his muscles that lead to the ``efforts and activities'' which are the crux of organized human effort.

Much has been said about the spirit of an organization. It should be clear that spirit in an observer is essentially a subjective evaluation generated in him by a leader or associates being perceived by virtue of physical energy flows to his sense organs.

The Activities of Persons

The observable external manifestation of a person’s activity, meaning a series of acts, is movement of his body parts. To move a person must produce forces by contracting muscles against his skeletal frame. In other words, he accelerates and displaces some or all of his body parts with physical forces generated within his muscles.

The external activities of persons consist of movement of body parts. They are objective and readily observable.

A person is also capable of mental activity. Mental activity is not directly observable except perhaps by means of an encephalograph. As important as mental activities are in relation to the behavior of persons, the fact remains that for organizational or managerial purposes they are not directly and objectively observable.

Then mental activity leads to action of body parts, an observer may attempt inference of the mental activity has taken place. Inference of the mental activities, that is the inference of thoughts, which have led to a given muscular action is fraught with great uncertainty. Therein lays the difficulty of conveying the thoughts of one person to another. The first person must always code his thoughts in terms that are possible of being decoded by a second person.

There is no way to transfer thoughts from one person to another except by movement of body parts that can be observed by other persons through physical media that can be detected by one or more of the five sense organs.

If a person’s mental activities lead to no physical action, they are in fact of no consequence in so far as other persons are concerned. It should be clear that a manager, a poet, a teacher, a researcher, or a parent influences people by his physical actions only, and not by his mental activities no matter how wise or profound the latter may be. If the manager’s plans are to be put into effect, he must code them in physically manifested verbal or written directions which can be observed by his followers and decoded in their minds into meaningful calls for action. Similarly a researcher must make known through some physical actions understandable to others what truths he has discovered or his efforts go for naught.

Cooperative effort is an essential characteristic of organized effort. When two or more persons coordinate their actions, i.e. their muscular force for a common purpose, the resulting fused actions of forces are said to be organized.

The mental activities of two or more persons may also be joined by proper coding of thoughts. For example, assume that an executive does not know the address of a client but that his secretary does. The executive may find the desired address with considerable expenditure of time by looking through notes in his desk. But he may gain the desired information with much less total expenditure by organized efforts by asking, ``What is the address of the Smith Company which I called upon last week,'' and immediately receive the reply, ``It is located at 6857 North Sylvester Avenue,'' from his secretary who has the address stored in her mind, i.e. memorized.

In this exchange the executive coded his message and transmitted it as sound waves by physically expelling air over his vocal cords. These waves were received by the ears of the secretary, alerted her attention, were decoded for meaning, produced a search of memory for the desired information and a coded message expressed in terms of sound waves which in turn were received by the executive as physical sound waves and decoded by him.

There is little question that communication requires expenditure of energy on the part of the ``sender.'' Also messages are usually transmitted between persons through intervening physical mediums such as air, light waves, electric waves, and physical bodies where senses of touch, taste and smell are involved.

But is there expenditure of muscular energy on the part of the receiver?  Some observers in the fields of psychology and physiology have found that thinking is accompanied by a series of incipient muscular conscious thinking will be absent.

This view is adhered to in this section for it simplifies consideration of cooperative action involving communication. By accepting this view cooperative communication, i.e. communication between two or more persons, requires coordination of muscular forces of sender and receivers to a degree.

\subsection{The Environment of Organizations}\index{The Environment of Organizations}

The total environment may be considered to consist of the universe. Though the universe is inseperable, it is convenient to regard it to consist of three segments, namely the physical, biological, and societal segments. By the physical segment we mean the parts of the environment that have mass such as steel bars, water, air etc., and which embody physical energy such as heat, light power. The biological segment refers to those elements that are characterized by life, that is the capacity to maintain an equilibrium with the balance of the environment by internal self-directed processes, the capacity to base action on experience and by capacity to reproduce its kind. The social segment refers to the interaction of humans. None of the segments can exist alone. For example, there appears to be no example of living beings without physical characteristics. Nor can there be social reactions without life.

Since the total environment is all pervasive, an organization functions in and accomplishes its objectives if at all within the environment.

The question thus arises, how do organizations accomplish their ends. At this point it is well to recall the definitions of organization being used. At this point it is well to recall the definitions of organization being used. Namely, an organization is the consciously coordinated forces of two or more persons.

To understand this definition in toto, it will be helpful to define its elements. The forces of two or more persons are muscular forces external to the persons. These forces are manifested as force applied through distance and are measurable in such units as foot pounds and gram centimeters. By their very nature the forces involved in organization are readily objectively observable.

By the word conscious in the present connection is meant awareness. That is, a person who contributes a force senses or is aware that he is so doing. Not only is a contributor aware of the forces he contributes to an organization but is also usually cognizant in some degree why he is doing so.

The term coordination embodies a spectrum of meanings. But in connection with the definition of organization coordination means that the force or forces supplied by one person is in some relationship with the force or forces contributed to an organization by other persons. The organization in its fundamental aspects consists essentially of the resultant of the several forces supplied at any given instant by the several persons.

This alteration of the physical environment may have utility. And this example is illustrative of the basic method by which utilities are created by organizations. That is, utilities are created in organizations would be sharpened and easier to understand if the underlined word were added to cause the definition to read: An organization is the consciously, continuously coordinated forces or actions of two or more persons. Strictly speaking, this is the definition of a unit organization.

The function of organization is to create utility. Utility is created by altering the physical environment in such a manner that is increases the satisfaction of people.

A violinist draws his bow across the strings setting in motion air waves that may create a satisfaction in the mind of a person whose ears perceive the waves. A delivery service created utility by moving a package from sender to receiver. A machinist creates utility by cutting away part of the metal of a bar with the aid of a machine. In typical factory operations utility is created by a sequence of acts expended upon material.

These steps are typical of the production of useful goods and services. It is apparent any product or service can be produced by similar incremental steps. During some steps continuously coordinated action of two or more persons is used as are the steps in 1 and 3 above. At other time only the forces of individuals are needed either to be expended on the basic material directly or to control machine elements or mechanical power sources of one kind or another.

In the interests of economy the usual practice is to reduce the need for unit organization. This can often be done by supplying outside power sources. For example, in Step 1 a fork truck operated by one man could be employed to move the bar from platform to crane.

\subsection{The Physical Aspects of Control}\index{The Physical Aspects of Control}

A considerable amount of physical energy expended in organizations is directed not to produce the product directly but to direct the activities of persons so that they may produce a desired product. Spontaneous coordination of human efforts rarely produces a useful result and even more rarely a useful pre-determined result.

Consider by the way of analogy an internal combustion engine. It takes in fuel and converts some of the energy in it to physical force energy. Some of this energy is delivered through the crank shaft for external application, some is wasted in internal friction and in heat losses and some of it is utilized in driving certain elements such as the cam shaft and ignition system which regulate the functioning of valves and the ignition of the explosive mixtures in a proper relationship with various other elements of the engine as necessary for the engine to function. It should be noted that the energy consumed in regulation does not contribute directly to the energy delivered to the crank shaft for external use.

The energy expended in organizations to achieve coordinated effort of two or more persons is expended in communication, which is to transfer information from one person to one or more persons.

How is information transferred from one person to another?  The method for transferring information is for the first person to code the information, the ideas in his mind, in a system of symbols that can be transmitted through some physical medium, which can be discerned, decoded, and understood, by a second person.

In the transmission of information as necessary to coordinate human activities and thus in the maintenance of organized action the most usual receptor means are the sight and hearing senses. The receptors involving the senses of touch, tasting, and smelling appear to be used infrequently but the general steps discussed above with respect to sight and hearing.

If, as has been assumed above that communication require expenditure of energy in the form of muscular forces by the transmitter and the receiver, the question arises, do the forces of the transmitter and the receiver produce a resultant force or forces.

Consider voice transmission through air. Most of the energy expended to produce air waves is damped out by air friction and by being absorbed by the surrounding physical environment such as walls, clothing, etc. Only a small portion of the total energy enters the ear of the receiver and this small amount is probably damped out in the process of stimulating the auditory sense receptors. Whatever muscular tension the receiver must exert to ``listen'' seems not to produce forces that join with forces expended by the sender to produce a resultant force as is the case when two persons join forces as in a unit organization to alter the external environment.

Why no resultant forces are produced by the communications efforts of two or more persons may be because they are not usually simultaneous.

The roll of communication in organization is not, so to speak, to do the work of the organization but to regulate or coordinate the physical efforts of contributors who supply the physical forces through which organization objectives will be achieved. In this respect communication plays a roll analogous to that of cam shafts and ignition systems of an engine which regulates the power producing elements of the engine, at the expense of and not contribution to the engine external power output.

\subsection{Other Energy Expenditures in Organization}\index{Other Energy Expenditures in Organization}

Above, it has been indicated that a chief essential ingredient of organizations as defined by Barnard is muscular force of persons.

Also it has been indicated that the objectives of organizations are achieved by altering the physical environment. This is true regardless of whether the ultimate purpose of the organization is to alter principally the physical, biological, or social segment of the environment.

It has been pointed out that the limitations of persons can be overcome by coordinating their forces. For example, two men may be able to lift a greater weight than one. Two persons can look in two directions at once, etc. These are representative of the energies that are employed to achieve the objectives of organizations and require coordination of the forces of persons.

Coordination of forces of persons rarely occurs spontaneously and perhaps never without the expenditure of energy to achieve this purpose.

Energy expended to achieve coordination is largely energy used to transfer information in communication. This energy is in the form of physically encoded ideas transmitted by the muscular forces of the sender through a physical media to a receiver. To be cognizant of the incoming coded message the receiver, it seems, must be under at least some muscular tension and then once having received the encoded message, must be able to decode it to gain an understanding of the idea transmitted. There is no question that there is some loss in information in the process. But in any event the energy expended by the two participants is dissipated in the process of altering the mind of the recipient and apparently has no other tangible effect that might be useful in achieving the objective of the organization. That is to say that the energy is expended in the setting up and maintenance.

To this point nothing has been mentioned about the energy, both physical and mental, that is expended by a contributor in preparation to communicate. This is essentially an individual activity. Consider for example, a manager who prepares a communication for a heads of departments meeting to set forth the production objectives to be reached during the ensuing fiscal year. He may make many calculations, notes, graphs, etc. in arriving at the illustrated presentations he finally makes. This preparation may possibly be done exclusively through the manager’s individual efforts and thus not require coordination of efforts. Certainly preparation of parts of any communication will have to be the result of individual efforts and thus not require coordination of efforts. Certainly preparation of parts of any communication will have to be the result of individual effort. As a second example, consider the common method of communicating technical directions via mechanical drawings. It is apparent that preparations of drawing require mental and physical effort at least some of which is individually done.

Closely associated with but probably not actually a part of formal organization is the intertwined informal organization that seems to always accompany formal organizations because of the human needs of contributors. The activity of informal organizations centers about communication. The communication in question requires expenditures of energy and is carried on essentially as communication necessary to coordinate forces in formal organization and differs only in objective.