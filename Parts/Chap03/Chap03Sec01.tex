\section{Humans and Human Nature}\index{Humans and Human Nature}

\subsection{The Nature of the Individual}\index{The Nature of the Individual}

The human being can be treated as a distinct individual. The study of individual differences may be pursued across many dimension1. The most obvious distinctions are physical. Some individuals may be said to be attractive while others may be considered plain or even unattractive. This latter evaluation, however, is a subjective one and influenced by the cultural and social mores of the society.

Individuals are different one from another across dimensions that are more subtle than mere appearances or capabilities. They may differ in intelligence, in aptitudes, in varying attributes of personality and in interests and attitudes. Some students are said to be smarter than others. They receive better grades and they may be more successful after school and accomplish more in life. This success may be partly due to intelligence. The trait of intelligence is apparently more difficult to define than it is to measure. Intelligence has been variously described as adaptability to new circumstances or the ability to deal in abstractions and with complexity. Regardless of the definition, measures of intelligence have been developed that do have predictive validity, and people are known to differ in this important trait.

People also differ in aptitudes. Individuals may possess greater or lesser degrees of mechanical aptitude, which in turn might include such specifics as motor skills and manual dexterity. Other aptitudes may include spatial and perceptual abilities as well as clerical capabilities. People differ in aptitudes and these differences have become recognized in recent years as important in individual job placement. Individuals are also different regarding personality. Some people are emotionally stable while others are not. Some people are nervous while others are calm. Some are dominant while others are quiet and more passive. Personality is a very profound attribute and people differ along this dimension as they do along other dimensions.

Interests and attitudes are other traits in which people differ. Some people are interested in social activity while others are not. Some are oriented more toward economics and practical matters while other people might prefer to deal in theoretical concepts or abstractions. Some people are more interested than others in mystical experiences. In attitudes also, people are different. In classifying attitudes, one might speak of radical as opposed to conservative beliefs, as degrees of support in the established social order. Regarding almost any contemporary controversial issue, we have our opinions, and other individuals with divergent opinions have their prejudices. These are attitudes.

Part of the differences among people may be attributed to the influence of environment and the other part may be due to heredity. The relative effect of each is not as important herein as the combined resultant of their effect. The uniqueness of individuals will be particularly evident and important in the relationship of these individuals to cooperative systems. An objective of a cooperative system may be important to one individual and trivial to another. An activity may appear correct to one person and improper to another. An incentive may be valuable to one person and insignificant to a second. All people are assumed to be different about their subjective evaluation of the value or utility associated with any object, stat, or event. It will subsequently be demonstrated that there are influences acting within cooperative systems to stabilize the values of individuals. A church may attempt to inculcate a code of morality upon its membership. An industrial organization might be concerned and act to improve its image as an employer in a community. In addition to these more overt attempts to structure values, it will be demonstrated that there are also subtle mechanisms acting to stabilize values within cooperative systems. These will be discussed in subsequent chapters. For the present, the uniqueness of individuals is assumed. The attempts to manipulate and stabilize values will be explored, but it will be assumed that the composite value structure of any individual is at least slightly different from those of all other individuals.

Differences within an individual also develop over time. In a physiological sense, body cells die and are replaced. So also may an individual’s personality, interests, and attitudes change. The value system of the individual is usually altered with time. A choice that appears attractive one day may not be so attractive the next. The influence of another individual may have modified this judgement.

A more detailed treatment of individual differences may be found in Leona E. Tyler, The Psychology of Human Differences, New York, Appleton-Century-Crofts, Inc., 1956. 

\subsection{Influence of Classical Greek Philosophy}\index{Influence of Classical Greek Philosophy}

The concept of nature as a standard by which to make judgments was a basic presupposition in Greek philosophy. Specifically, ``almost all'' classical philosophers accepted that a good human life is a life in accordance with nature.[1 ](Notions and concepts of human nature from China, Japan, or India are not taken up in the present discussion.).

On this subject, the approach of Aristotle - sometimes considered to be a teleological approach - came to be dominant by late classical and medieval times. This approach understands human nature in terms of final and formal causes. In other words, nature itself (or a nature-creating divinity) has intentions and goals, similar somehow to human intentions and goals, and one of those goals is humanity living naturally. Such understandings of human nature see this nature as an ``idea'', or ``form'' of a human.[2] By this account, human nature really causes humans to become what they become, and so it exists somehow independently of individual humans. This in turn has sometimes been understood as also showing a special connection between human nature and divinity.

However, the existence of this invariable human nature is a subject of much historical debate, continuing into modern times. Against this idea of a fixed human nature, the relative malleability of man has been argued especially strongly in recent centuries—firstly by early modernists such as Thomas Hobbes and Jean-Jacques Rousseau. In Rousseau's Emile, or On Education, Rousseau wrote: ``We do not know what our nature permits us to be.''[3] Since the early 19th century, thinkers such as Hegel, Marx, Kierkegaard, Nietzsche, Sartre, structuralists, and postmodernists have also sometimes argued against a fixed or innate human nature.

Charles Darwin's theory of evolution has changed the nature of the discussion, confirming the fact that mankind's ancestors were not like mankind today. Still more recent scientific perspectives - such as behaviorism, determinism, and the chemical model within modern psychiatry and psychology - claim to be neutral regarding human nature. (As in much of modern science, such disciplines seek to explain with little or no recourse to metaphysical causation.)[4] They can be offered to explain human nature's origins and underlying mechanisms, or to demonstrate capacities for change and diversity which would arguably violate the concept of a fixed human nature.

Article: Ancient Greek philosophy Classical Greek philosophy

Philosophy in classical Greece is the ultimate origin of the western conception of the nature of a thing. According to Aristotle, the philosophical study of human nature itself originated with Socrates, who turned philosophy from study of the heavens to study of the human things.[5] 

Socrates is said to have studied the question of how a person should best live, but he left no written works. It is clear from the works of his students Plato and Xenophon, and also by what was said about him by Aristotle (Plato's student), that Socrates was a rationalist and believed that the best life and the life most suited to human nature involved reasoning. The Socratic school was the dominant surviving influence in philosophical discussion in the Middle Ages, amongst Islamic, Christian, and Jewish philosophers.

The human soul in the works of Plato and Aristotle has a divided nature, divided in a specifically human way. One part is specifically human and rational, and divided into a part which is rational on its own, and a spirited part which can understand reason. Other parts of the soul are home to desires or passions like those found in animals. In both Aristotle and Plato, spiritedness (thumos) is distinguished from the other passions (epithumiai).[6] The proper function of the "rational" was to rule the other parts of the soul, helped by spiritedness. By this account, using one's reason is the best way to live, and philosophers are the highest types of humans.

Aristotle—Plato's most famous student—made some of the most famous and influential statements about human nature. In his works, apart from using a similar scheme of a divided human soul, some clear statements about human nature are made:

\begin{enumerate}
\item Man is a conjugal animal, meaning an animal which is born to couple when an adult, thus building a household (oikos) and, in more successful cases, a clan or small village still run upon patriarchal lines.[7]
\item Man is a political animal, meaning an animal with an innate propensity to develop more complex communities the size of a city or town, with a division of labor and law-making. This type of community is different in kind from a large family and requires the special use of human reason.[8]
\item Man is a mimetic animal. Man loves to use his imagination (and not only to make laws and run town councils). He says, ``we enjoy looking at accurate likenesses of things which are themselves painful to see, obscene beasts, for instance, and corpses.'' And the ``reason why we enjoy seeing likenesses is that, as we look, we learn and infer what each is, for instance, `that is so and so.'''[9]
\end{enumerate}

For Aristotle, reason is not only what is most special about humanity compared to other animals, but it is also what we were meant to achieve at our best. Much of Aristotle's description of human nature is still influential today. However, the particular teleological idea that humans are ``meant'' or intended to be something has become much less popular in modern times.[10]

For the Socratics, human nature, and all natures, are metaphysical concepts. Aristotle developed the standard presentation of this approach with his theory of four causes. Every living thing exhibits four aspects or ``causes:'' matter, form, effect, and end. For example, an oak tree is made of plant cells (matter), grew from an acorn (effect), exhibits the nature of oak trees (form), and grows into a fully mature oak tree (end). Human nature is an example of a formal cause, according to Aristotle. Likewise, to become a fully actualized human being (including fully actualizing the mind) is our end. Aristotle (Nicomachean Ethics, Book X) suggests that the human intellect () is ``smallest in bulk'' but the most significant part of the human psyche and should be cultivated above all else. The cultivation of learning and intellectual growth of the philosopher, which is thereby also the happiest and least painful life.

\subsection{Effectiveness and Efficiency in Individual Behavior}\index{Effectiveness and Efficiency in Individual Behavior}

Individuals engage in activity. Sometimes the activity appears to be more successful than other times. The individual may accomplish what he set out to accomplish while another time he may not be so successful. Sometimes he accomplishes what he set out to accomplish but he still regrets making the decision to undertake that activity. The cost of accomplishment was too high. Sometimes he is unsuccessful, and this does not bother him. The investment in the activity was not very significant. A measure of the success or lack of success in human behavior is needed to describe these situations. Such a measure will be made in terms of the effectiveness and the efficiency of individual behavior3. This measure will then subsequently e applied to describe the relative success of cooperative activity.

Efficiency has a very precise meaning in the physical sciences and in engineering. The efficiency of a machine or of a process is measured as a percentage and is the ratio of the output of the machine or process divided by the input. This efficiency is always less than one hundred percent because the output is always less than the input. The process or machine consumes some energy in the transformation. Thus, while one hundred percent efficiency would be ideal, it is never attained. The efficiency of cooperative activity will also be measured as the ratio of the output over the input. However, this will be a collective ratio of all the participants, and for the cooperative system to survive, it will be shown that this efficiency must exceed rather than be less than one hundred percent, although the same degree of quantification in measurement is rarely possible.

The efficiency of individual activity is related to this traditional definition of efficiency, but it is personal and subjective. The efficiency of individual activity is dependent upon the cost incurred in undertaking the activity in comparison to the satisfaction achieved. This cost is not a monetary cost so much as a cost of individual utility or satisfaction. The input is a measure of the physiological and/or psychological investment. The output is then a measure of the satisfaction obtained from the activity.

    1 These measures are introduced in Chapter I. Barnard, The Functions of the Executive, Cambridge, Harvard University Press, 1938.