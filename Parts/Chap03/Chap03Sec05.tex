\section{Systems Engineering Organization}\index{Systems Engineering Organization}

An initial step in managing the systems engineering process is to develop an early implementation plan and an organizational structure that will be responsive to program requirements. Systems engineering planning starts with the identification of a customer need and the definition of requirements for a program (project) to design, develop, produce, and deliver a system that will be responsive and affordable. Although every program is somewhat different, its overall planning is usually promulgated through a program management plan (PMP) or equivalent. From this top-level plan, a systems engineering management plan (SEMP), or systems engineering plan (SEP), is derived to guide implementation of the technical activities described in the prior chapters of this textbook.

A SEMP, which should be prepared during the conceptual design phase, provides the necessary guidance for the many design and management plans required for a given program. Included within the SEMP is the identification of systems engineering program tasks, a program work breakdown structure (WBS), task schedule and cost requirements, and the needed organizational structure for program implementation. In developing an organizational approach, it is essential that an environment be established that will allow for the effective and efficient coordination and integration of the various engineering and supporting disciplines that contribute to the overall system design process. Appropriate leadership must also be in place to promote good communications across organizational lines and to foster a truly interdisciplinary approach to system design and development.

The primary objective of systems engineering management is to facilitate the timely integration of numerous design considerations (refer to ) into a functioning system that will be of high value to the customer. Accordingly, the purpose in this chapter is to provide the reader with an understanding of the basic systems engineering planning and organizational requirements for a typical program through consideration of the following factors
\begin{itemize}
\item Systems engineering program planning needs
\item The development of a systems engineering management plan (SEMP) - statement of work (SOW), systems engineering program tasks, work breakdown structure (WBS), scheduling of tasks, projecting costs for program tasks, interfacing with other planning activities
\item Organization for systems engineering - developing the organizational structure, customer/producer/supplier relationships, customer organization and functions, producer organization and functions, supplier organization and functions, and staffing the organization (human resource requirements).
\end{itemize}

The purpose of organization, considered in the systems engineering context, is to fulfill the requirements described in the SEMP and in the system specification (Type A). The goals of the organization are directed toward this end and pertain to two levels of activity: (1) the goals specified by the TPMs, discussed in, which quantify the DDPs making specific the characteristics that must be incorporated into the design of a given system; and (2) the goals of the organization relative to accomplishing the necessary activities (tasks) to ensure that the first objective is attained. A basic question is, How effective and efficient is the organization functioning in the accomplishment of the first goal?

In this instance, the question can be addressed to the applicable systems engineering organization relative to (1) its impact on the ultimate system/product design configuration itself; and (2) its effectiveness and efficiency in performing the 11 tasks described in.2. It is not uncommon to address only the second goal, believing that the organization is doing well in accomplishing the identified tasks when, at the same time, the organization (and its personnel) has not impacted the design at all. Thus, when assessing the organizational capabilities later, the second goal must be evaluated in terms of the first.

Regarding the issue of organizational structure, which is the point of emphasis here, the objective is to develop a complete systems engineering capability that will enable the accomplishment of the 11 program tasks (or functions of an equivalent nature) in an effective manner. This, in turn, requires (1) the establishment of a desired set of metrics against which each of the applicable tasks can be assessed; (2) the development of the necessary processes for the performance of these tasks; and (3) implementation of a data collection and information capability that will enable management to “track” performance and determine just how well the organization is functioning overall. The organizational objective is to establish a disciplined, well-defined approach for the performance of these tasks, establish measurable goals that can be quantitatively expressed and controlled, and initiate a provision for continuous process improvement.

Having identified the systems engineering tasks to be accomplished, along with the associated ``metrics,'' management needs to assess the current status of the organization. At the same time, it may be appropriate to compare the results with other comparable entities. A benchmarking approach can be applied to aid in the development of future goals. Benchmarking can be defined as ``an ongoing activity of comparing one’s own process, product, or service against the best known similar activity, so that challenging but attainable goals can be set, and a realistic course of action implemented to efficiently become and remain the best of the best in a reasonable period of time.'' The ultimate questions are, Where are we today? How do we compare with others relative to the product, process, and/or organization? Where would we like to be in the future?

The organizational goals and objectives must first be responsive to the established system operational requirements and the TPMs for the system being developed (refer to ). Then, additional goals can be established for the systems engineering organization itself. Through the use of benchmarking, future goals can be identified and an organizational growth plan can be developed, which, when implemented, can aid in realizing the desired objectives.

\subsection{Transdisciplinarity Reaching Beyond Disciplines to Find Connections}\index{Transdisciplinarity Reaching Beyond Disciplines to Find Connections}

Considering the foregoing, it is worth clarifying the differences among intradisciplinary, multidisciplinary, interdisciplinary, and transdisciplinary research. Intradisciplinary research involves problems that can be successfully tackled from within a single discipline such as engineering or medicine. Multidisciplinary research involves problems that require cooperation among individuals from different disciplines. Interdisciplinary research involves cooperation among disciplines leading to enrichment of one or more contributing disciplines and occasionally resulting in new discipline. Transdisciplinary research involves looking beyond traditional disciplines to find new connections among disciplines that facilitate knowledge unification. Table 1 compares and contrasts these various forms of research initiatives. 

We live in an era in which the world is becoming increasingly more connected or, as Tom Friedman puts it, ``flat.'' The inevitable consequence of this interconnectedness trend is that problems are becoming much too complex to successfully solve by applying methods from within a single discipline. This recognition is most evident in the growing trend toward multidisciplinary and interdisciplinary collaboration among traditionally independent disciplines. As such collaboration intensifies, existing disciplines are being enriched and occasionally new disciplines are beginning to emerge. At the same time, the knowledge gaps among the disciplines are beginning to surface. What appears to be lacking is a new way of thinking that strives to harmonize traditional disciplines by reaching beyond their traditional boundaries to fill the knowledge gaps not addressed by them. Transdisciplinary thinking promises to reach beyond disciplinary boundaries to identify and overcome knowledge voids and incompatibilities in the quest for knowledge unification. Achieving these objectives is key to fostering new relationships among traditionally independent disciplines and, in so doing, begin to address problems of national and global significance. This paper discusses the aims of transdisciplinarity, the road to transdisciplinarity, successes resulting from transdisciplinary thinking, and recommendations for a research and education agenda embracing trandsdisciplinary thinking. 

Today there is a growing recognition that such collaboration, while essential, is neither straightforward nor easy. This is because the gulf between independent disciplines needs to be bridged before such collaboration can start paying real dividends. Bridging independent disciplines typically requires extending them, reconciling their differences, and unifying the knowledge associated with them in new and novel ways. This recognition inevitably leads to the notion of transdisciplinarity, the quest for and discovery of new connections among disciplines leading to novel approaches for unifying their knowledge. It is not surprising, therefore, that there is a growing interest in transdisciplinary thinking to address problems that appear to be intractable when viewed from the perspective of a single discipline. 

Transactions of the SDPS MARCH 2007, Vol. 11, No. 1, pp. 1-11 

Transdisciplinarity is a global perception of the ultimate connection of multiple (possibly all) disciplines (Nicolescu, 1997). From this perspective, not only science, but all human activities appear as a unitary whole, and part of the unity of the universe. According to Rodriguez Delgado, an eminent Spaniard systemist, unity and diversity (within transdisciplinarity) are not viewed as opposing, but complementary perspectives. 

Despite its obvious allure, transdisciplinarity organization faces many challenges. To begin with, there is no single, agreed upon definition of transdisciplinarity. Academic and societal viewpoints differ. This lack of a common definition is further exacerbated with the formation of new societies, each promoting their own language for discourse. Fortunately, the academic and business communities remain undaunted as evidenced by the growing interest worldwide in infusing transdisciplinary concepts and projects into their educational and research agendas. For example, when it comes to public health, the National Academies (National Academies, 2002) recommends moving from research dominated by a single discipline or a small number of disciplines to transdisciplinary research. They define transdisciplinary research as involving broadly constituted teams of researchers that work across disciplines to develop significant research questions. In these recommendations, transdisciplinary research implies the conception of research questions that transcend individual disciplines and specialized knowledge bases because they are intended to solve applied public health research questions that are, by definition, beyond the purview of any single discipline. In transdisciplinary research, different specialties combine their expertise (and that of community members) to collectively define the health problem and their solutions. The National Academies sum up their position by pointing out that the one qualitatively different and unique aspect of the transdisciplinary process is the holistic blending of expert and community inputs to produce greater integration across disciplines. 

It is worth recognizing that transdisciplinarity originates from the increasing demand for relevance and applicability of academic research to societal challenges (Nicolescu, 2002). Not surprisingly, the two popular definitions of transdisciplinary research today center around academic research and societal challenges. The academic research-oriented definition characterizes transdisciplinarity as ``a special form of interdisciplinarity in which boundaries between and beyond disciplines are transcended and disciplines as well as non-scientific sources are integrated.'' The societal challenge-oriented definition characterizes transdisciplinarity as ``a new form of learning and problem-solving involving cooperation among different parts of society (including academia) to meet complex societal challenges. Solutions devised are a result of collaboration and mutual learning among multiple stakeholders.'' As can be seen from the preceding two definitions, there is no standard definition of transdisciplinarity. What is common to both, however, is the unity of knowledge. Journal of Integrated Design and Process Science MARCH 2007, Vol. 11, No. 1, pp. 2
