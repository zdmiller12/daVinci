\chapterimage{leafOnTable.jpg} % Chapter heading image
\chapter{ECONOMICS AND ENTERPRISE SYSTEMS}\label{chap:4}

Various but basic principles of economics apply around the world and have applied over thousands of years of recorded history. They apply in many very different kinds of economies - capitalist, socialist, feudal, or whatever - and among a wide variety of peoples, cultures, and governments. 

We can begin understanding economics by first being clear as to what economics means. To know what economics is, we must first know what an economy is. Perhaps most of us think of an economy as a system for the production and distribution of the goods and services we use in everyday life. That is true as far as it goes, but it does not go far enough.

The Garden of Eden included a system for the production and distribution of goods and services, but it was not an economy, because everything was available in unlimited abundance. Without scarcity, there is no need to economize - and therefore no economics was yet extant.

\section{The Nature of Human Wants}\index{The Nature of Human Wants}

Economic considerations embrace many of the subtleties and complexities characteristic of people. Economics is derived from the behavior of humans individually and collectively, particularly as their behavior relates to the satisfaction of wants.

Human wants have always exceeded the means of satisfying them. The relative scarcity of good and services has been and will continue to be an economic dilemma that all must face. Some wants cannot be satisfied at all, other can be partially satisfied, and only a few wants can be fully satisfied. Added to the elusiveness in satisfying wants are changes that occur.  As certain wants are satisfied, additional wants develop.

Some human wants are more predictable than others. The demand for food, clothing, and shelter needed for bare physical existence is more stable and predictable than the demand for those things that satisfy people’s emotional needs. The number of calories of energy needed to sustain life may be determined fairly accurately. Clothing and shelter requirements may be predicted within narrow limits from climatic conditions. Once people are assured of physical existence, they demand satisfactions of a less predictable nature resulting from being members and products of a society rather than only biological organisms.

The wants of people are motivated largely by emotional drives and tensions and, to a lesser extent, by logical reasoning processes. A part of human wants can be satisfied by physical goods and services, but people are rarely satisfied by physical things alone. In food, sufficient calories to meet physical needs will rarely satisfy. People want the food they eat to satisfy both energy needs and emotional needs. In consequence, people are concerned with the flavor of food, its consistency, the china and silverware with which it is served, the person or persons who serve it, the people in whose company it is eaten, and the atmosphere of the room in which it is served. Similarly, there are many desires associated with clothing and shelter in addition to those required merely to meet physical needs.

Much or little progress has been made in obtaining knowledge on which to base predictions of human behavior, depending on one’s viewpoint. The idea that human reactions will someday be well enough understood to be predictable is accepted by many people; even though this has been the objective of the thinkers of the world since the beginning of time, it appears the progress in psychology has been meager compared with the rapid progress made in the physical sciences. Even though human behavior can be neither predicted nor explain, it must be considered by those who are concerned with economic decision analysis.

What does ``scarce'' mean?  It means that what everybody wants ads up to more than there is. This may seem like a simple thing, but its implications are often grossly misunderstood, even by highly educated people. For example, a feature article in the New York Times laid out the economic woes and worries of middle-class Americans – one of the most affluent groups of human beings ever to inhabit this planet. Although this story included a structure of a middle-class American family in their own swimming pool, main headline read: ``The American Middle, Just Getting By''

In short, middle-class Americans’ desires exceed what they can comfortably afford, even though wat they already have would be considered unbelievable prosperity by people in many other countries around the world – or even by earlier generations of Americans. Yet both they and the reporter regarded them as “just getting by” and a Harvard sociologist was quoted as saying “how budget-constrained these people really are.”  But it is not something as man-made as a budget which constrains them: Reality constrains them. There has never been enough to satisfy everyone completely. That is the real constraint. That is what scarcity means.

To all people – from academia and journalism, as well as the middle-class people themselves – is apparently seemed strange somehow that there should be such a thing as scarcity and that this should imply a need for both productive efforts on their part and personal responsibility in spending the resulting income. Yet nothing has been more pervasive in the history of the human race than scarcity and all the requirements for economizing that go with scarcity.

Regardless of our policies, practices, or institutions - whether they are wise or unwise, noble or ignoble - there is simply not enough to go around to satisfy all our desires to the fullest. ``Unmet needs'' are inherent in these circumstances, whether we have a capitalist, socialist, feudal, or other kind of economy. These various kinds of economics are just differently institutional ways of making trade-offs that are inescapable in any economy.

Economics is not just about dealing with the existing output of goods and services as consumers. It is also, and more fundamentally, about producing that output from scarce resources in the first place - turning inputs into output.

In other words, economics studies the consequences of decisions that are made about the use of land, labor, capital and other resources that go into producing the volume of output which determines a country’s standard of living. Those decisions and their consequences can be more important than the resources and countries like Japan and Switzerland with relatively few natural resources but high standards of living. The values of natural resources per capita in Uruguay and Venezuela are several times what they are in Japan and Switzerland, but real income per capita in Japan and Switzerland is more than double that of Uruguay and several times that of Venezuela.

Not only scarcity but also ``alternative uses'' are at the heart of economics. If each resource had only one use, economics would be much simpler. But water can be used to produce ice or steam by itself or innumerable mixtures and compounds in combination with other things. Similarly, from petroleum comes not only gasoline and heating oil, but also plastics, asphalt and Vaseline. Iron ore can be used to produce steel products ranging from paper clips to automobiles to the frameworks of skyscrapers.

How much of each resource should be allocated to each of its many uses?  Every economy has to answer that question, and each one does, in one way or another, efficiently or inefficiently. Doing so efficiently is what economics is about. Different kinds of economics are essentially different ways of making decisions about the allocation of scarce resources - and those decisions have repercussions on the life of the whole society.

During the days of the Soviet Union, for example, that country’s industries used more electricity than American industries used, even though Soviet industries produced a smaller amount of output than American industries produced. Such inefficiencies in turning inputs into outputs translated into a lower standard of living, in a country richly endowed with natural resources - perhaps more richly endowed than any other country in the world. Russia is, for example, one of the few industrial nations that produces more oil than it consumes. But an abundance of resources does not automatically create an abundance of goods.

Efficiency in production - the rate at which inputs are turned into output - is not just some technicality that economists talk about. It affects the standard of living of whole societies. When visualizing this process, it helps to think about the real things - the iron ore, petroleum, wood and other inputs that go into the production process and the furniture, food and automobiles that come out the other end - rather than think of economic decisions as being simply decisions about money. Although the word ``economics'' suggests money to some people, for a society as a whole money is just an artificial device to get real things done. Otherwise, the government could make us all rich by simply printing more money. It is not money but the volume of goods and services which determines whether a country is poverty stricken or prosperous.
    
%------------------------------------------------

\section{Concepts of Value and Utility}\index{Concepts of Value and Utility}

In economics, the term value designates the worth that a person attaches to a good or service. Thus, the value of an item is not inherent in the item but is inherent in the regard a person or people have for it. Value should not be confused with the cost or the price of an item in economic analysis. There may be little relation between the value a person ascribes to an article and the cost of providing it or the price that is asked for it.

\subsection{The Power to Satisfy Wants}\index{The Power to Satisfy Wants}

An accepted economic definition of the term utility is the power to satisfy human wants. The utility that an item has for an individual is determined subjectively. Thus, the utility of an item, like its value, is not inherent in the article itself but is inherent in the regard that a person has for it. Value and utility are closely related in the economic sense. The utility than an object has for a person is the satisfaction he derives from its use. Value is an appraisal of utility in terms of a medium of exchange.

In ordinary circumstances a large variety of goods and services is available to an individual. The utility that available items may have in the mind of a prospective user may be expected to be such that his desire for them will range from abhorrence, through indifference, to intense desire. His evaluation of the utility of various items is not ordinarily constant but may be expected to change with time. Each person also possesses either goods or services that he may offer to exchange. These have the utility for the person himself that he regards them to have. These same goods and possible services may also be desired by others, who may ascribe to them very different utilities. The possibility for exchange exists when each of two persons possesses goods or services desired by the other.

Most things that have utility for an individual are physically manifested. This is readily apparent in such physical objects as a house, an automobile, or a pair of shoes. The situation may be true as well with less tangible things. One can enjoy a television program because light waves impinge on the retina and sound waves strike the ear. Even friendship is realized through the senses and, therefore, has its physical aspects.
    
\subsection{The Creation of Utility}\index{The Creation of Utility}

The creation of utility is achieved through a change in the physical environment. For example, the consumer utility of raw steak can be increased by altering its physical condition by an appropriate application of heat. In the area of producer utilities, the machining of a bar of steel to produce a shaft for a rolling mill is an example of creating activity is to determine how the physical environment may be altered to create the most utility for the least cost.

Carl Menger, an early Austrian Economist, published Principles of Economics that challenged the fundamental premises of the classical economists, from Adam Smith through David Ricardo to John Stuart Mill. Menger argued that the labor theory of value (apparently adopted by Alan) was flawed in presuming that the value of goods was determined by the relative quantities of labor that had been expended in their production. Others argued that value if added justified pricing changes.

Menger formulated a subjective theory of value, reasoning that value originates in the mind of an evaluator (this cannot be Alan; it must be Wolt). Labor, like raw materials and other resources, derives value from the value of the goods it can produce. From this starting point, Menger outlined a theory of the value of goods and factors of production, and a theory of the limits of exchange and the formation of prices.

Böhm-Bawerk, mentor of Ludwig von Mises, came across Menger’s book soon after its publication. Both immediately saw the significance of this new subjective approach for the development of economic theory.

\subsection{The Subjective Theory of Value}\index{The Subjective Theory of Value}

Carl Menger, an early Austrian Economist, published Principles of Economics that challenged the fundamental premises of the classical economists, from Adam Smith through David Ricardo to John Stuart Mill. Menger argued that the labor theory of value (apparently adopted by Alan) was flawed in presuming that the value of goods was determined by the relative quantities of labor that had been expended in their production.

Menger formulated a subjective theory of value, reasoning that value originates in the mind of an evaluator (this cannot be Alan; it must be Wolt). Labor, like raw materials and other resources, derives value from the value of the goods it can produce. From this starting point, Menger outlined a theory of the value of goods and factors of production, and a theory of the limits of exchange and the formation of prices.

Böhm-Bawerk, mentor of Ludwig von Mises, came across Menger’s book soon after its publication. Both immediately saw the significance of this new subjective approach for the development of economic theory.

%------------------------------------------------

\section{Consumer and Producer Goods}\index{Consumer and Producer Goods}

Two classes of goods are recognized by economists: consumer goods and producer goods. Consumer goods are products and services that directly satisfy human wants. Examples of consumer goods are television sets, houses, shoes, books, and health services. Producer goods also satisfy human wants but do so indirectly as a part of the production or the construction process. Broadly speaking, the ultimate end of all production and construction activity is to supply goods and services that people may use or consume to satisfy their wants.

Producer goods are, in the long run, used as a means to an end, namely, that of producing goods and services for human use and consumption. Examples of this class of goods are lathes, bulldozers, ship, and digital computers. Producer goods are an intermediate step in an effort to supply human wants. Such goods are not desired for themselves, but because they may be instrumental in producing something that people can use or consume.

But in the 1870s, Menger boldly applied its implications to the determination of value. He noted that since goods are external to the human person and recognized subjectively as possessing qualities that allow for need satisfaction, they could be differentiated between goods of different order. In Principles, he described first-order goods as being goods that we consume to satisfy needs. These are consumption goods.

Second-order goods are goods required to produce the first-order good, so that while a car may be a first-order good satisfying a felt need for transportation, the second-order goods would include the glass, rubber, chrome, and all the other inputs which make up the car. The third-order goods are all of the goods that are required to produce the second-order goods, and so on, with more complex forms of production being characterized with more distant orders of production.

Nonetheless, the values of all the goods of whatever order are derived from the initial subjective desire on the part of the individual to satisfy a felt need, so that rubber has value not in itself or in the work effort going into its production, but because of the initial human desire for transportation, leading to a human preference for cars with tires. This understanding of goods contrasted greatly with the Classical economist’s notion that the value of economic inputs is based on their technical usefulness in production.
    
\subsection{Utility of Consumer Goods}\index{Utility of Consumer Goods}

People will consider two kinds of utility. One kind embraces the utility of goods and services that they intend to consume personally for the satisfaction they get out of them. Thus, it seems reasonable to believe that the utility people ascribe to goods and services that are consumed directly is in large measure a result of subjective, nonlogical, mental processes.

Marketing analysts apparently find emotional appeals more effective than factual information. An analysis of advertising and sales practices used in selling consumer goods will reveal that they appeal primarily to the senses rather than to reason. Perhaps this is as it should be. If the enjoyment of consumer goods almost exclusively on how one feels about them rather than what one reasons about them, it seems logical to make sales presentations for those things to which customers ascribe utility.

It is not uncommon for a salesperson to call on a prospective customer, describe and explain a certain item, state its price, offer it for sale, and have the offer rejected. This is concrete evidence that the item does not possess sufficient utility at the moment to induce the prospective customer to buy it. In such a situation, the salesperson may be able to induce the prospect to buy it. In such a situation, the salesperson may be able to induce the prospect to listen to further sales talk, during which the prospect may decide to buy on the basis of the original offer. This is evidence that the item now possesses sufficient utility to induce the prospective customer to buy. Because there was no change in the item or the price at which it was offered, there must have been a change in the customer’s attitude or regard for it. The pertinent fact is that a proposition at first undesirable now has become desirable as a result of a change in the customer, not in the proposition.

What about the change?  Several reasons could be advanced. Usually, it would be said that the salesperson persuaded the customer to buy; that is, the salesperson induced the customer to believe something, namely, that the item had sufficient utility to warrant its purchase. There are many aspects to persuasion. It may amount only to calling attention to the availability of an item. A person cannot purchase an item he or she does not know exists. A part of the sales function is to call attention to the things available for sale.

It is observable that persuasive ability is much in demand and is often of inestimable beneficial consequences to all concerned. Persuasion as it applies to the sale of goods is of economic importance to industry. A manufacturer must dispose of the goods he produces. He can increase the salability of his products by building into them greater customer appeal in terms of greater usefulness, greater durability, or greater beauty, or he may elect to accompany his products to market with greater persuasive effort in the form of advertising and sales promotion.
    
\subsection{Utility of Producer Good}\index{Utility of Producer Good}

The second kind of utility that an object or service may have for a person is its utility as a means to an end. Producer goods are not consumed for the satisfaction that can be directly derived from them but as a means of producing consumer goods, usually by facilitating alteration of the physical environment.

Once the kind and amount of consumer goods to be produced has been determined, the kinds and amounts of facilities and producer goods necessary to produce them may be calculated with a high degree of certainty. The energy, ash and other contents of coal, for instance, can be determined very accurately and are the bases of evaluating the utility of the coal. The extent to which producer utility may be considered by logical processes is limed only by technical knowledge and the ability to reason.

Although the utility of the consumer goods is primarily determined subjectively, the utility of the producer goods as a means to an end may be, and usually is, in large measure approached objectively. In this connection consider the satisfaction of the human want for harmonic sounds as in music. Suppose it has been decided that the desire for music can be met by 100,000 compact discs. Then the organization of the artists, the technicians, and the equipment necessary to produce the discs become predominantly objective in character. The amount of material that must be processed to form one disc is calculable to a high degree of accuracy. If a concern has been making discs for some time, it will know the various operations that are to be performed and the unit time for performing them. From these data, the kind and amount of producer service, the amount and kind of labor, and the number of various types of machines are determinable within rather narrow limits. Whereas the determination of the kinds and amounts of consumer goods needed at any one time may depend upon the most subjective of human considerations, the problems associated with their production are quire objective by comparison.
    
%------------------------------------------------

\section{Economic Thinking Over Time}\index{Economic Thinking Over Time}

Economics is not simply a topic on which to express opinions or vent emotions. It is a systematic study of cause and effect, showing what happens when you do specific things in specific ways. In economics analysis, the methods used by a Marxist economist like Oskar Lange did not differ in any fundamental way from the methods used by a conservative economist like Milton Friedman. It is these basic economic principles that this book is about.

One of the ways of understanding the consequences of economic decisions is to look at them in terms of the incentives they create, rather than simply the goals they pursue. This means that consequences matter more than intentions – and not just the immediate consequences, but also the longer run repercussions 

When economists analyze prices, wages, profits, or the international balance of trade, for example, it is from the standpoint of how decisions in various parts of the economy affect the allocation of scarce resources in a way that raises or lowers the material standard of living of the people as a whole.

\subsection{History and the Role of Economics}\index{History and the Role of Economics}

While there are controversies in economics, as there are in science, this does not mean that the basic principles of chemistry or physics are just a matter of opinion. Einstein’s analysis of physics, for example, was not just Einstein’s opinion, as the world discovered at Hiroshima and Nagasaki. Economic reactions may not be as spectacular or as tragic, as of a given day, but the worldwide depression of the 1930’s plunged millions of people into poverty, even in the richest countries, producing malnutrition in countries with surplus food, probably causing more deaths around the world than those at Hiroshima and Nagasaki.

Conversely, when India and China – historically, two of the poorest nations on earth – began in the late twentieth century to make fundamental changes in their economic policies, the economies began growing dramatically. It has been estimated that 20 million people in India rose out of destitution in a decade. In China, the number of people living on a dollar a day or less fell from 374 million – one third of the country’s population in 1990 – to 128 million by 2004, now just 10 percent of a growing population. In other words, nearly a quarter of a billion Chinese were now better off as a result of a change in economic policy.

Things like this are what make the study of economics important – and not just a matter of opinions or emotions. Economics is a tool of cause and effect analysis, a body of tested knowledge – and principles derived from that knowledge.

Most of us hate to even think of having to make such choices. Indeed, as we have already seen, some middle-class Americans are distressed at having to make much milder choices and trade-offs. But life does not ask us what we want. It presents us with options. Economics is one of the way of trying to make the most of those options.

The History of Economics. People have been talking about economic issues, and some writing about them, for thousands of years, so it is not possible to put a specific date on when the study of economics began as a separate field. Modern economics is often dated from 1776, when Adam Smith wrote his classic, The Wealth of Nations, but there were substantial books devoted to economics at least a century earlier, and there was a contemporary school of French economists called the Physiocrats, some of whose members Smith met while traveling in France, years before he wrote his own treatise on economics. What was different about The Wealth of Nations was that it became the foundation for a whole school of economists who continued and developed its ideas over the next two generations, including such leading figures as David Ricardo (1772-1823) and John Stuart Mill (1806-1873), and the influence of Adam Smith has to some extent persisted on to the present day. No such claim could be made for any previous economist, despite many people who had written knowledgeably and insightfully on the subject in earlier times.

More than two thousand years ago, Xenophon, a student of Socrates, analyzed economics policies in ancient Athens. In the Middle Ages, religious conceptions of a ``fair'' or ``just'' price, and a ban on usury, let Thomas Aquinas to analyze the economic implications of those doctrines and the exceptions that might therefore be morally acceptable. For example, Aquinas argued that selling something for more than was paid for it could be done ``lawfully'' when the seller has ``improved the thing in some way,'' or as compensation for risk, or because of having incurred costs of transportation. Another way of saying the same thing is that much that looks like sheer taking advantage of other people is often in fact compensation for various costs and risks incurred in the process of bringing goods to consumers or lending money to those who seek to borrow.

However far economists have moved beyond the medieval notion of a fair and just price, that concept still lingers in the background of much present-day thinking among people who speak of things being sold for more or less their ``real'' value and individuals being paid more or less than they are ``really'' worth, as well as in such emotionally powerful but empirically undefined notions as price ``gouging.''

From isolated individual writing about economics there evolved, over time, coherent schools of thought, people writing within a common framework of assumptions - the medieval scholastics, of whom Thomas Aquinas was a prominent example, the mercantilists, the classical economists, the Keynesians, the ``Chicago School,'' and others. Individuals coalesced into various schools of thought even before economists became a profession in the nineteenth century.

Classical Economics. Within a decade after Sir James Steuart’s multi-volume mercantilist treatise, Adam Smith’s The Wealth of Nations was published and dealt a historic blow against mercantilist theories and the whole mercantilist conception of the world. Smith conceived of the nation as all the people living in it. Thus, you could not enrich a nation by keeping wages down in order to export. “No society can surely be flourishing and happy, of which the far greater part of the members is poor and miserable,” Smith said. He also rejected the notion of economic activity as a zero-sum process, in which one nation loses what another nation gains. To him, all nations could advance at the same time in terms of the prosperity of their respective peoples, even though military power – a major concern of the mercantilists – was of course relative and a zero-sum competition.

In short, the mercantilists were preoccupied with the transfer of wealth, whether by export surpluses, imperialism, or slavery – all of which benefit some at the expense of others. Adam Smith was concerned with the creation of wealth, which is not a zero-sum process. Smith rejected government intervention in the economy to help merchants – the source of the name ``mercantilism'' – and instead advocated free markets along the lines of the French economists, the Physiocrats, who had coined the term laissez faire. Smith repeatedly excoriated special-interest legislation to help ``merchants and manufacturers,'' whom he characterized as people whose political activities were designed to deceive and oppress the public. In the context of the times, laissez faire was a doctrine that opposed government favors to business.

The most fundamental difference between Adam Smith and the mercantilists was that Smith did not regard gold as being wealth. The very title of his book - The Wealth of Nations - raised the fundamental question of what wealth consisted of. Smith argued that wealth consisted of the goods and services which determined the standard of living of the people - the whole people, who to Smith constituted the nation. Smith rejected both imperialism and slavery – on economic grounds as well as moral grounds, saying that the ``great fleets and armies'' necessary for imperialism ``acquire nothing which can compensate the expense of maintaining them.'' The Wealth of Nations closed by urging Britain to give up dreams of empire. As for slavery, Smith considered it economically inefficient, as well as morally repugnant, and dismissed with contempt the idea that enslaved Africans were inferior to people of European ancestry.

Although Adam Smith is today often regarded as a “conservative” figure, he in fact attacked many of the dominant ideas and interests of his own times. Moreover, the idea of a spontaneously self-equilibrating system – the market economy – first developed by the Physiocrats and later made part of the tradition of classical economics by Adam Smith, represented a radically new departure, not only in analysis of social causation but also in seeing a reduced role for political, intellectual, or other elites as guides or controllers of the masses.

For centuries, landmark intellectual figures from Plato onward had discussed what policies wise leaders might impose for the benefit of society in various ways. But, in the economy, Smith argued that governments were giving ``a most unnecessary attention'' to things that would work out better if left alone to be sorted out by individuals interacting with one another and making their own mutual accommodations. Government intervention in the economy, which mercantilist Sir James Steuart saw as the role of a wise ``statesman,'' Smith saw as the notions and actions of ``crafty'' politicians, who created more problems than they solved.

While The Wealth of Nations was not the first systematic treatise on economics, it became the foundation of a tradition known as classical economics, which built upon Smith’s work over the next century. Not all earlier treatises were mercantilist by any means. Books by Richard Cantillon in the 1730s and by Ferdinando Galiani in 1751, for example, presented sophisticated economic analyses, and Fancois Quesnay’s Tableau Economique in 1758, contained insights that inspired the transient but significant school of economists called the Physiocrats. But, as already noted, these earlier pioneers created no enduring school of leading economists in later generations who based themselves on their work, as Adam Smith did.

\subsection{Entreprenuership and Profit}\index{Entreprenuership and Profit}

In the most fundamental humans are all, with each of our actions, always and invariably profit-seeking entrepreneurs. Whenever we act, we employ some physical means (things valued as goods) — at a minimum our body and its standing room, but in most cases also various other, ``external'' things — so as to divert the ``natural'' course of events (the course of events we expect to happen if we were to act differently) to reach some more highly valued anticipated future state of affairs instead. With every action we aim at substituting a more favorable future for a less favorable one that would result if we were to act differently. In this sense, with every action we seek to increase our satisfaction and attain a psychic profit.

But every action is threatened also with the possibility of loss. For every action refers to the future and the future is uncertain or at best only partially known. Every actor, in deciding on a course of action, compares the value of two anticipated states of affairs: the state he wants to effect through his action but that has not yet been realized, and another state that would result if he were to act differently but cannot come into existence, because he acts the way he does. This makes every action a risky enterprise. An actor can always fail and suffer a loss. He may not be able to effect the anticipated future state of affairs - that is, the actor’s technical knowledge, his ``know how'' may be deficient or it may be temporarily ``superseded,'' due to some unforeseen external contingencies. Or else, even if he has successfully produced the desired state of physical affairs, he may still consider his action a failure and suffer a loss, if this state of affairs provides him with less satisfaction than what he could have attained had he chosen otherwise (some earlier-on rejected alternative course of action) - that is, the actor’s speculative knowledge - his knowledge of the temporal change and fluctuation of values and valuations - may be deficient.

Since all our actions display entrepreneurship and are aimed at being successful and yielding the actor a profit, there can be nothing wrong with entrepreneurship and profit. Wrong, in any meaningful sense of the term, are only failure and loss, and accordingly, in all our actions, we always try to avoid them.

The question of justice, i.e., whether a specific action and the profit or loss resulting from it is ethically right or wrong, arises only in connection with conflicts.

Since every action requires the employment of specific physical means — a body, standing room, external objects — a conflict between different actors must arise, whenever two actors try to use the same physical means for the attainment of different purposes. The source of conflict is always and invariably the same: the scarcity of physical means. Two actors cannot at the same time use the same physical means – the same bodies, spaces and objects - for alternative purposes. If they try to do so, they must clash. Therefore, in order to avoid conflict or resolve it if it occurs, an action-able principle and criterion of justice is required, i.e., a principle regulating the just or ``proper'' vs. the unjust or ``improper'' use and control (ownership) of scarce physical means.

Logically, what is required to avoid all conflict is clear: It is only necessary that every good be always and at all times owned privately, i.e., controlled exclusively by some specified individual (or individual partnership or association), and that it be always recognizable which good is owned and by whom, and which is not. The plans and purposes of various profit-seeking actor-entrepreneurs may then be as different as can be, and yet no conflict will arise so long as their respective actions involve only and exclusively the use of their own, private property.

Yet how can this state of affairs: the complete and unambiguously clear privatization of all goods, be practically accomplished? How can physical things become private property in the first place; and how can conflict be avoided from the beginning of mankind on?

A single — praxeo-logical — solution to this problem exists and has been essentially known to mankind since its beginnings — even if it has only been slowly and gradually elaborated and logically re-constructed. To avoid conflict from the start, it is necessary that private property be founded through acts of original appropriation. Property must be established through acts (instead of mere words or declarations), because only through actions, taking place in time and space, can an objective — inter-subjectively ascertainable — link be established between a particular person and a particular thing. And only the first appropriator of a previously un-appropriated thing can acquire this thing as his property without conflict. For, by definition, as the first appropriator he cannot have run into conflict with anyone in appropriating the good in question, as everyone else appeared on the scene only later.

This importantly implies that while every person is the exclusive owner of its own physical body as his primary means of action, no person can ever be the owner of any other person’s body. For we can use another person’s body only indirectly, i.e., in using our directly appropriated and controlled own body first. Thus, direct appropriation temporally and logically precedes indirect appropriation; and accordingly, any non-consensual use of another person’s body is an unjust misappropriation of something already directly appropriated by someone else.

All just property, then, goes back directly or indirectly, through a chain of mutually beneficial — and thus likewise conflict-free — property-title transfers, to original appropriators and acts of original appropriation. Mutatis mutandis, all claims to and uses made of things by a person who had neither appropriated or produced these things, nor acquired them through a conflict-free exchange from some previous owner, are unjust. And by implication: All profits gained or losses suffered by an actor-entrepreneur with justly acquired means are just profits (or losses); and all profits and losses accruing to him through the use of unjustly acquired means are unjust.

This analysis applies in full also to the case of the entrepreneur in the term’s narrower definition, as a capitalist-entrepreneur.

The capitalist entrepreneur acts with a specific goal in mind: to attain a monetary profit. He saves or borrows saved money, he hires labor, and he buys or rents raw materials, capital goods and land. He then proceeds to produce his product or service, whatever it may be, and he hopes to sell this product for a monetary profit. For the capitalist, ``profit appears as a surplus of money received over money expended and loss as a surplus of money expended over money received. Profit and loss can be expressed in definite amounts of money.'' (Mises 1966, p. 289)

As all action, a capitalist enterprise is risky. The cost of production - the money expended - does not determine the revenue received. In fact, if the cost of production determined price and revenue, no capitalist would ever fail. Rather, it is anticipated prices and revenues that determine what production costs the capitalist can possibly afford.

Yet, the capitalist does not know what future prices will be paid or what quantity of his product will be bought at such prices. This depends exclusively on the buyers of his product, and the capitalist has no control over them. The capitalist must speculate what the future demand will be. If he is correct and the expected future prices do correspond to the later fixed market prices, he will earn a profit. On the other hand, while no capitalist aims at making losses - because losses imply that he must ultimately give up his function as a capitalist and become either a hired employee of another capitalist or a self-sufficient producer-consumer - every capitalist can err with his speculation and the actually realized prices fall below his expectations and his accordingly assumed production cost, in which case he does not earn a profit but incurs a loss.

While it is possible to determine exactly how much money a capitalist has gained or lost in the course of time, his money profit or loss do not imply much if anything about the capitalist’s state of happiness, i.e., about his psychic profit or loss. For the capitalist, money is rarely if ever the ultimate goal (safe, may be, for Scrooge McDuck, and only under a gold standard). In practically all cases, money is a means to further action, motivated by still more distant and ultimate goals. The capitalist may want to use it to continue or expand his role as a profit-seeking capitalist. He may use it as cash to be held for not yet determined future employments. He may want to spend it on consumer goods and personal consumption. Or he may wish to use it for philanthropic or charitable causes, etc.

What can be unambiguously stated about a capitalist’s profit or loss is this: His profit or loss are the quantitative expression of the size of his contribution to the well-being of his fellow men, i.e., the buyers and consumers of his product, who have surrendered their money in exchange for his (by the buyers) more highly valued product. The capitalist’s profit indicates that he has successfully transformed socially less highly valued and appraised means of action into socially more highly valued and appraised ones and thus increased and enhanced social welfare. Mutatis mutandis, the capitalist’s loss indicates that he has used some more valuable inputs for the production of a less valuable output and so wasted scarce physical means and impoverished society.

Money profits are not just good for the capitalist, then, they are also good for his fellow men. The higher a capitalist’s profit, the greater has been his contribution to social welfare. Likewise, money losses are bad not only for the capitalist, but they are bad also for his fellow men, whose welfare has been impaired by his error.

The question of justice: of the ethically ``right'' or ``wrong'' of the actions of a capitalist-entrepreneur, arises, as in the case of all actions, again only in connection with conflicts, i.e., with rivalrous ownership claims and disputes regarding specific physical means of action. And the answer for the capitalist here is the same as for everyone, in any one of his actions.

The capitalist’s actions and profits are just, if he has originally appropriated or produced his production factors or has acquired them - either bought or rented them - in a mutually beneficial exchange from a previous owner, if all his employees are hired freely at mutually agreeable terms, and if he does not physically damage the property of others in the production process. Otherwise, if some or all of the capitalist’s production factors are neither appropriated or produced by him, nor bought or rented by him from a previous owner (but derived instead from the ex-propriation of another person’s previous property), if he employs non-consensual, ``forced'' labor in his production, or if he causes physical damage to others’ property during production, his actions and resulting profits are unjust.

In that case, the unjustly harmed person, the slave, or any person in possession of proof of his own un-relinquished older title to some or all of the capitalist’s means of production, has a just claim against him and can insist on restitution — exactly as the matter would be judged and handled outside the business world, in all civil affairs.

%------------------------------------------------

\section{Principles of Austrian Economics}\index{Principles of Austrian Economics}

Austrian Economics emanates from the Ludwig Von Mises Institute adjacent to the campus of Auburn University. Based in the axiom of Human Action, Mises realized that the axiom of human action is an a priori synthetic judgment. The axiom of human action says that humans act. This might sound trivial at first glance. At second glance, however, it becomes obvious that the axiom of human action has far-reaching implications.

The axiom of action meets the requirements of an a priori synthetic judgment. First, one cannot observe that humans act in the first place. For doing so, one needs an understanding of what human action is. This knowledge cannot be acquired through experience, because it comes from reason and not from experience.

Second, one cannot deny that humans act, for doing so would result in an intellectual contradiction. Saying ``humans cannot act'' is itself a form of human action, and it would thus contradict the statement’s truth claim.

Mises also realized that, by using formal logic, other truth claims can be deduced from the irrefutably true - propositions that can be validated without taking recourse to experience. Take, for instance, the concept of causality – the idea that each effect has a cause. It is logically implied in the axiom of human action. As Mises put it,

Acting requires and presupposes the category of causality. Only humans see the world in the light of causality is fitted to act. In this sense we may say that causality is a category of action. The category means and ends and presupposes the category cause and effect. In a world without causality and regularity of phenomenon there would be no field for human reasoning and human action. Such a world would be a chaos in which man would be at a loss to find any orientation and guidance. Man is not even capable of imagining the conditions of such a chaotic universe. Where man does not see any causal relation, he cannot act.

Science is a systematic and logical approach to discovering how things in the universe work. It is also the body of knowledge accumulated through the discoveries about all the things in the universe. This is another aspect arising from the fact that science is conducted in a social context. The goal is noble, discovery of reliable knowledge about the world enabling appropriate action. The environment of the activity is fundamentally social, with factors such as who is respected, who is close to power, how persuasively or forcefully participants put forward their views etc., etc. Dr. Tim Ferris.

Yes, science is conducted in a social context which means that there is more transpiring than just science, e.g., who is respected and for what, etc. Similarly, many system design and architecting decisions are taken to placate sponsors or to meet deadlines rather than to maximize system effectiveness. The ``socialness'' of it all highlights the importance of a Standard of Care or at least a lesser Code of Ethics in human activities. Jack Ring.
    
\subsection{Praxeology and Austrian Economics}\index{Praxeology and Austrian Economics}

The Walrasian and the Austrian approaches often come to similar conclusions when it comes to the desirability of markets, but they come to these conclusions using quite different paths. The advantage of the Austrian approach is precisely in the path it takes to come to its conclusions - it maintains several key principles that most of us would have a hard time disagreeing with:

\begin{itemize}
\item Value is in the mind of an individual. It is then by definition subjective and directly unobservable to others.
\item Value is not a physical quantity. Thus, interpersonal comparisons of utility or value are inappropriate.
\item All economic activity is a consequence of individual humans acting on their values.
\end{itemize}

The Walrasian approach often assumes away these principles for the sake of mathematical tractability. This is where the crucial problem arises. To what extent can we believe the conclusions of a model that assumes away the fundamental features of reality as we understand it?  This is actually the most common criticism of the neoclassical defense of markets.

You have probably heard many people blaming the current economic crisis on ``market failure.'' Some would say that markets ``failed'' because real markets are different from the economists’ ``perfect'' models. The logic is as follows: since we can’t trust these market models, neither can we trust the actual markets. There is an error in this logic.

Economics of Exchange. A buyer will purchase an article when he has money available and when he believes that the article has equal or greater utility for him than the amount required to purchase it. Conversely, a seller will sell an article when he believes that the amount of money to be received for the article has greater utility than the article has for him. Thus, an exchange will not take place unless at the time of exchange both parties believe that they will benefit. Exchanges are made when they are thought to result in mutual benefit. This is possible because the objects of exchange are not valued equally by the parties to the exchange.

As an illustration of the economic aspect of exchange, consider the following example. Two workers, upon opening their lunch boxes, discover that one contains a piece of apple pie and the other a piece of cherry pie. Suppose further that Ms. A evaluates her apple pie to have 20 units of utility for her and the cherry pie to have 30 units. Suppose also that Mr. B evaluates his cherry pie to have 20 units of utility for him and the apple pie 40 units. If Ms. A consumes her apple pie and Mr. B consumes his cherry pie, the total utility realized is 20 plus 20, or 40 units. But if the two workers exchange pieces of pie, the resulting utility will be increased to 30 plus 40, or 70 units. The utility in the system can be increased by 30 unites through exchange.

Exchange is possible because consumer utilities are evaluated by the consumer almost entirely, if not entirely, by subjective consideration. Thus, if the workers in the example believe that the exchange has resulted in a gain of net satisfaction to them, no one can deny it. Conversely, at the time of exchange, unless each person valued what he or she had to give less than what he or she was about to receive, an exchange could not occur. Thus, we conclude that an exchange of consumer utilities results in a gain for both parties because the utilities are subjectively evaluated by the participants.

Assuming that each party to an exchange of producer utilities correctly evaluates the objects of exchange in relation to his situation, what makes it possible for each person to gain?  The answer is that the participants are in different economic environments. This fact may be illustrated by an example of a retailer who buys lawn mowers from a manufacturer. For example, at a certain volume of activity the manufacturer finds that he can produce and distribute mowers at a total cost of \$90 per unit and that the retailer buys several mowers at a price of \$110 each. The retailer then finds that by expending an average of \$30 per unit in selling effort, he can sell several mowers to homeowners for \$190 each. Both participants profit by the exchange. The reason the manufacturer profits is that his environment is such that he can sell to the retailer for \$110 several mowers that he cannot sell elsewhere at a higher price, and that he can manufacture lawn mowers for \$90 each. The reason the retailer profits is that his environment is such that he can sell mowers at \$190 each by applying \$30 of selling effort on a mower of certain characteristics which he can buy for \$90 each from the manufacturer in question, but not for less elsewhere.

One may ask: Why does the manufacturer enter the merchandising field and thus increase his profit? or Why doesn’t the retailer enter the manufacturing field?  The answer to these questions is that neither the manufacturer nor the retailer can do so unless each changes the relevant environment. The retailer, for example, lacks physical plant equipment and an organization of engineers and workers competent to build lawn mowers. Also, he may be unable to secure credit necessary to engage in manufacturing, although he may easily secure credit in greater amounts for merchandising activities. It is quite possible that he cannot alter is environment so that he can build mowers for less than \$90. Similar reasoning applies to the manufacturer. Exchange consists essentially of physical activity designed to transfer the control of things from one person to another. Thus, even in exchange, utility is created by altering the physical environment.

Each party in an exchange should seek to give something that has little utility for him but that will have greater utility for the receiver. In this manner, each exchange can result in the greatest gain for each party. Nearly everyone has been a party to such a favorable exchange. When a car becomes stuck in snow, only a slight push may be required to dislodge it. The slight effort involved in the dislodging push may have little utility for the person giving it, so little that he expects no more compensation than a friendly nod. Conversely, it might have great utility for the person whose car was dislodged, so great that he may offer a substantial tip. This is known as the ``range of mutual benefit in exchange.''

The aim of much sales and other research is to find products that not only will have great utility for the buyer but that can be supplied at a low cost, that is, have low utility for the seller. The difference between the utility that a specific good or service has for the buyer and the utility it has for the seller represents the profit or net benefit that is available to divide between buyer and seller. It is called the range of mutual benefit in exchange.

(Figure Here)

Given these insights, the contest in which one would use economic models is quite different from the typical Walrasian approach. In this case, one would say that because markets exist, we may, for illustrative purposes, assume that individuals know the relevant economic characteristics of other individuals in the society. In the typical Walrasian approach, the complete information assumption is a precondition for the existence of efficient markets, while in the Austrian approach, the existence of markets is a precondition for the existence of prices that transform subjective and otherwise unobservable valuations of goods produced and owned by a multitude of individuals into objective and observable metrics.

For many neoclassical economists, the market is a tool (only one of the tools) for allocating production and consumption efficiently. Efficiency here is the state of the world where any change would just make things worse. In this theory, such an “optimal” solution can be reached using means other than the market because of lax assumptions about value and knowledge. More specifically, for a person to determine the optimal allocation of resources in an economy outside of the market process, that person needs to know people’s values, skills, potentials, etc. Thus, in such a model, one needs to assume that these qualities exist as objectively measurable and knowable magnitudes.

Austrians, on the other hand, don’t claim that there is anything like this “optimal” allocation of resources, either within or outside of the market. What they do claim is that, if people want to develop an advanced economy, the market is the way to do this. The path to developing such an economy, the market is the way to do this. The path to developing such an economy is through constant guidance of resource allocation by people’s values reflected in the market prices. In the market, someone will always be dissatisfied with something, but this is not a bad thing. This dissatisfaction is a motive for action and for the improvement of one’s well-being. It is the driving force of the economy.

There are important advantages in being familiar with the Austrian theory. This theory helps one keep in mind fundamental principles such as the subjectivity of value and the incompleteness of information that form the basis for human action. This approach makes it easier to spot errors in one’s economic thinking. One of the common errors is treating economic models as normative standards for reality rather than loose metaphors and illustrations of the logical conclusions resulting from prior theoretical analysis. This error creates a temptation to ``fix'' the reality to fit the model. Often the fix only makes things worse, because it was not the reality that needed fixing. It was, in fact, the economist’s model that did not capture the key features of reality.

Economics is about using scarce resources and available means to achieve the best possible ends. In general, achieving an end is called consumption and applying a means towards an end is called production.

There are four broad categories of means, or factors of production, which are involved in achieving our ends. The first is labor, which refers to our own exertions, whether mental or physical. The second is land, which refers to any of the natural resource existing in nature. The third is time. A certain amount of each of these is required in any production process.

Together, these three factors are called the original factors of production because they exist in nature prior to any human production. The fourth kind of factor is that which is produced, not because it directly brings satisfaction, but because it can be used in a different production process. That fourth factor is capital goods.

Since, all things being equal, people will tend to prefer present over future consumption, it is necessary that a longer production process result in a superior set of consumer goods than a shorter one - enough so to induce people to wait to reap its benefits. The hunter-gatherer will choose hunting over gathering only if he finds the meat he will gain in the future, after presently constructing and using a spear, sufficiently more enticing than the leisure he could enjoy in the present.

An unavoidable feature of capital is that it wears out and therefore must be replaced. An economy that relies on capital must expend work simply to maintain its capital. In an economy that relies on capital, therefore, people must be continually willing to forgo present consumption to maintain their standard of living. ``Since the time spent producing a good could have been consumed immediately as leisure, all production requires that one forgo present consumption for future consumption.''

On the other hand, it is not necessary to save again; this is only necessary for the economy to grow, not to stay the same. The original savings are still around, embodied in the capital goods on which the economy relies. An economy with an abundance of capital goods has a long history of saving and thrift behind it, and as a result has a production process that is long, many-staged, and very productive.

One aspect of any production process not yet mentioned is risk. Since the decision of what to produce takes place in the present whereas consumption is not available until the future, it is always possible that a person could choose incorrectly, and later realize that what he decided to produce was not the best use of his time and resources.

On the free market, goods can be valued in terms of prices, which say what sum of money might be exchanged for them. Prices tend toward the level at which demand equals supply and all the available stock is sold. If the price is higher than this, a seller has the incentive to bid lower to ensure that they sell their stock, and if it is lower, buyers have the incentive to bid higher to make sure they can get the goods they desire.

Consumer goods directly satisfy our desires, so the fact that they are demanded needs no explanation. Their demand and the available supply determine their prices based on the law of supply and demand.

\subsection{The Austrian School and the Entrepreneur}\index{The Austrian School and the Entrepreneur}

Carl Menger is considered the founder of the Austrian School of economics. He described the entrepreneur as a coordinating agent who is both a capitalist and a manager. The entrepreneur owns resources and decides how they will be used. Menger emphasized that entrepreneurs bear uncertainty and take purposeful, decisive action according to the knowledge they have. John Bates Clark and Frank A. Fetter are economists who followed in Menger's approach to economics. Clark believed the entrepreneur must also be the owner of a business. Fetter saw uncertainty-bearing as the key entrepreneurial function. He asserted that an entrepreneur organized and directed production while possessing superior foresight. It is clear that the Austrian School did not start with an emphasis on the entrepreneur as an opportunity discoverer.

While those in the Austrian tradition have always seen the entrepreneur as having a central role in economic affairs, two different strands emerged within the Austrian tradition that led to different conceptions of the entrepreneur. Friedrich von Wieser and F.A. Hayek branched off in emphasizing knowledge, discovery and market process. Wieser saw the entrepreneur as owner, manager, leader, innovator, organizer, and speculator. Hayek emphasized that knowledge is dispersed to individuals throughout the economy. He argued that market competition makes the best use of this dispersed knowledge and brings it to light. Influenced by this strand of thought, Kirzner argues that a competitive market is superior because it best generates entrepreneurial discoveries.

Economists Eugen Böhm-Bawerk, Ludwig von Mises, and Murray Rothbard are considered to compose a different branch of thought than Wieser and Hayek. They emphasize monetary calculation and decision-making under uncertainty. Mises emphasized that the entrepreneur has an anticipative understanding of an uncertain future. Rothbard critiqued Kirzner for not emphasizing the role of entrepreneur as uncertainty-bearer. He also questioned Kirzner's notion that the entrepreneur need not own any resources to perform his function. Rothbard asked, "In what sense can an entrepreneur even make profits if he owns no capital to make profits on?"

%------------------------------------------------

\section{Linking Physical and Economic Factors}\index{Linking Physical and Economic Factors}

Society is confronted with two important interconnected environments, the physical and the economic. Their success of humans in altering the physical environment to produce products and services depends upon a knowledge of physical laws. However, the worth of these products and services lies in their utility measured in economic items. There are numerous examples of structures, machines, processes, and systems that exhibit excellent physical design but have little economic merit.

\subsection{The Physical and Economic Interface}\index{The Physical and Economic Interface}

Want satisfaction in the economic environment and engineering proposals in the physical environment are linked by the production or the construction process. Figure 4.1 illustrates the relationship between engineering proposals, production or construction, and want satisfaction.

Figure 4.1 Relating Physical and Economic Factors

In dealing with the physical environment engineers have a body of physical laws upon which to base their reasoning. Such laws as Boyle’s law, Ohm’s law, and Newton’s laws of motion were developed primarily by collecting and comparing numerous similar instances and by the use of an inductive process. These laws may then be applied by deduction to specific instances. They are supplemented by many formulas and known facts, all of which enable the engineer to come to conclusions that match the facts of the physical environment within narrow limits. Much is known with certainty about the physical environment.

Much less, particularly of a quantitative nature, is known about the economic environment. Since economics is involved with the actions of people, it is apparent that economic laws must be based upon their behavior. Economic laws can be no more exact than the description of the behavior of people acting singly and collectively.

The usual function of engineering is to manipulate the elements of one environment, the physical, to create value in a second environment, the economic. However, engineers sometimes have a tendency to disregard economic feasibility and are often applied in practice by the necessity for meeting situations in which action must be based on estimates and judgment. Yet today’s engineering graduates are increasingly finding themselves in positions in which their responsibility is extended to include economic considerations.

Engineers can readily extend their inherent ability of analysis to become proficient in the analysis of the economic aspects of engineering application. Furthermore, the engineer who aspires to a creative position in engineering will find proficiency in economic analysis helpful. The large percentage of engineers who will eventually be engaged in managerial activities will find such proficiency a necessity.

Initiative for the use of engineering rests, for the most part, upon those who will concern themselves with social and economic consequences. To maintain the initiative, engineers must operate successfully in both the physical and economic sectors of the total environment. It is the objective of engineering economy to prepare engineers to cope effectively with the bi-environmental nature of engineering application.

\subsection{Physical and Economic Efficiency}\index{Physical and Economic Efficiency}

Both individuals and organizations possess limited resources. This makes it necessary to produce the greatest output for a given input – that is, to operate at high efficiency. Thus, the search is not merely for a good opportunity for the employment of limited resources, but for the best opportunity.

People are continually seeking to satisfy their wants. They give up certain utilities to gain others that they value more. This is essentially an economic process, in which the objective is the maximization of economic efficiency.

Engineering is primarily a producer activity that comes into being to satisfy human wants. Its objective is to get the greatest end result per unit of resource expenditure. This is essentially a physical process in which the objective is the maximization of physical efficiency, which may be stated as

				(1.1)

If interpreted broadly enough, physical efficiency is a measure of the success of engineering activity in the physical environment. However, the engineer must be concerned with two levels of efficiency. On the first level is physical efficiency expressed as outputs divided by inputs of such physical units as Btu’s, kilowatts, and foot-pounds. When such physical units are involved, efficiency will always be less than unity, or less than 100\%

On the second level are economic efficiencies. These are expressed in terms of economic units of output divided by economic units of input, each expressed in terms of a medium of exchange such as money. Economic efficiency may be stated as

				(1.2)

It is well known that physical efficiencies over 100\% are not possible. However, economic efficiencies can exceed 100\% and must do so for economic ventures to be successful.

Physical efficiency is related to economic efficiency. For example, a power plant may be profitable in economic terms even through its physical efficiency in converting units of energy in coal to electrical energy may be relatively low. As an example, in the conversion of energy in a certain plant, assume that the physical efficiency is only 36\%. Assuming that output Btu’s in the form of electrical energy have an economic worth of \$14.65 per million and that input Btu’s in the form of coal have an economic cost of \$1.80 per million, then

Since physical processes are of necessity carried out at efficiencies less than 100\% and economic ventures are feasible only if they attain efficiencies greater than 100\%, it is clear that in feasible economic ventures the economic worth per unit of physical output must always be greater than the economic cost per unit of physical input. Consequently, economic efficiency must depend more upon the worth and cost per unit of physical outputs and inputs than upon physical efficiency. Physical efficiency is always significant, but only to the extent that it contributes to economic efficiency.
In the final evaluation of most ventures, even those in which engineering plays a leading role, economic efficiencies must take precedence over physical efficiencies. This is because the function of engineering is to create utility in the economic environment by altering elements of the physical environment.

%------------------------------------------------

\section{Summary and Extensions}\index{Summary and Extensions}

%------------------------------------------------

%% SEA CHAPTER 4 - PRELIMINARY SYSTEM DESIGN
% SEA Question Location in \label{sea-Chapter#-Problem#}
% NEEDS UPDATING
\begin{exercises}
    \begin{exercise}
    \label{sea-4-1}
        Select a system of your choice and develop operational functional flow block diagrams (FFBDs) to the third level. Select one of the functional blocks and develop maintenance functional flows to the second level.Show how the maintenance functional flow diagrams evolve from the operational flows.
    \end{exercise}
    \begin{solution}
    \end{solution}
    
    \begin{exercise}
    \label{sea-4-2}
        Describe how specific resource requirements (i.e., hardware, software, people, facilities, data,and elements of support) are derived from the functional analysis.
    \end{exercise}
    \begin{solution}
    \end{solution}
    
    \begin{exercise}
    \label{sea-4-3}
        Describe the steps involved in transitioning from the functional analysis to a “packaging scheme”for the system.Provide an example.
    \end{exercise}
    \begin{solution}
    \end{solution}
    
    \begin{exercise}
    \label{sea-4-4}
        Refer to the allocation in Figure 4.6. Explain how the quantitative factors (i.e.,TPMs) at the unit level and assembly level were derived.
    \end{exercise}
    \begin{solution}
    \end{solution}
    
    \begin{exercise}
    \label{sea-4-5}
        What steps would you take in accomplishing the allocation of requirements for a system-of-systems (SOS?) configuration? Describe the overall process.
    \end{exercise}
    \begin{solution}
    \end{solution}
    
    \begin{exercise}
    \label{sea-4-6}
        Select a system of your choice and assign some top-level TPMs. Allocate these requirements as appropriate to the second and third levels.
    \end{exercise}
    \begin{solution}
    \end{solution}
    
    \begin{exercise}
    \label{sea-4-7}
        Refer to Figure 4.7. How would you define the design-to requirements for the common unit?"
    \end{exercise}
    \begin{solution}
    \end{solution}
    
    \begin{exercise}
    \label{sea-4-8}
        Describe what is meant by interoperability? Why is it important?
    \end{exercise}
    \begin{solution}
    \end{solution}
    
    \begin{exercise}
    \label{sea-4-9}
        Describe what is meant by environmental sustainability? Identify some specific objectives in the design for such. 
    \end{exercise}
    \begin{solution}
    \end{solution}
    
    \begin{exercise}
    \label{sea-4-10}
        Why is the design for security important? Identify some specific design objectives. 
    \end{exercise}
    \begin{solution}
    \end{solution}
    
    \begin{exercise}
    \label{sea-4-11}
        Why is the design for security important? Identify some specific design objectives. 
    \end{exercise}
    \begin{solution}
    \end{solution}
    
    \begin{exercise}
    \label{sea-4-12}
        Refer to Figure 4.1. Why is the development of a specification tree important? What inherent characteristics should be included?
    \end{exercise}
    \begin{solution}
    \end{solution}
    
    \begin{exercise}
    \label{sea-4-13}
        Refer to Figure 4.2. How are metrics established for the function shown? Give an example.
    \end{exercise}
    \begin{solution}
    \end{solution}
    
    \begin{exercise}
    \label{sea-4-14}
        Refer to Figure 4.4. Describe some of the interfaces or interactions that must occur to ensure a completely integrated approach in the development of the hardware, software, and human system requirements. Be specific.
    \end{exercise}
    \begin{solution}
    \end{solution}
    
    \begin{exercise}
    \label{sea-4-15}
        Define CAD, CAM, CAS, Macro-CAD, and their interrelationships. Include an illustration showing interfaces, information/data flow, etc.
    \end{exercise}
    \begin{solution}
    \end{solution}
    
    \begin{exercise}
    \label{sea-4-16}
        Assume that you are a design engineer and are looking for some analytical models/tools to aid you in the synthesis, analysis, and evaluation process. Develop the criteria that you would apply in selecting the most appropriate tools for your application.
    \end{exercise}
    \begin{solution}
    \end{solution}
    
    \begin{exercise}
    \label{sea-4-17}
        Assume that you have selected an analytical model for a specific application. Explain how you would validate that the model is adequate for the application in question.
    \end{exercise}
    \begin{solution}
    \end{solution}
    
    \begin{exercise}
    \label{sea-4-19}
        List some of the benefits that can be derived from the use of computer-based models. Identify some of the concerns associated with the application of such.
    \end{exercise}
    \begin{solution}
    \end{solution}
    
    \begin{exercise}
    \label{sea-4-20}
        Refer to Figure 4.8. Identify some of the objectives in selecting the appropriate methods/tools for accomplishment of the tasks in the right-hand columns.
    \end{exercise}
    \begin{solution}
    \end{solution}
    
    \begin{exercise}
    \label{sea-4-21}
        Refer to Figure 4.12. Identify some of the objectives in designing a “tool set” as shown by the seven blocks in the figure.
    \end{exercise}
    \begin{solution}
    \end{solution}
    
    \begin{exercise}
    \label{sea-4-22}
        Briefly identify some inputs and outputs for the system design review, the equipment/software design review, and the critical design review.
    \end{exercise}
    \begin{solution}
    \end{solution}
    
    \begin{exercise}
    \label{sea-4-23}
        On what basis are formal design reviews scheduled? What are some of the benefits derived from conducting formal design reviews?
    \end{exercise}
    \begin{solution}
    \end{solution}
    
    \begin{exercise}
    \label{sea-4-24}
        How are supplier requirements determined? How are these requirements passed on to the supplier?
    \end{exercise}
    \begin{solution}
    \end{solution}
    
    \begin{exercise}
    \label{sea-4-25}
        Design is a team effort. Explain why! How should the design team be structured? How does systems engineering fit into the process?
    \end{exercise}
    \begin{solution}
    \end{solution}
    
    \begin{exercise}
    \label{sea-4-26}
        What is the desired output of the Preliminary System Design Phase?
    \end{exercise}
    \begin{solution}
    \end{solution}
\end{exercises}
% SKIPPED