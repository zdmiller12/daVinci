% LISTS (prematurely coded)

The several ways to think of and define a system include:
\begin{enumerate}
	\item A system is composed of parts.
	\item All the parts of a system must be related (directly or indirectly), else there are really two more distinct systems.
	\item A system is encapsulated; it has a boundary.
\end{enumerate}

\begin{enumerate}
	\item Some systems have a very clear physical boundary and are loosely coupled from the rest of the universe. For example, the solar system, the Earth, an airplane, and animal
	\item Sometimes we are interested in a particular ``property of interest'' - the earth’s energy balance, the profitability of a business, the viability of a community, the operational effectiveness of a military system. In such cases it is usually convenient to define the boundary of the ``system of interest'' as the physical boundary of the ``system''
\end{enumerate}

18 century concept from the ``Encyclopedie de D. DIDEROT'':
\begin{enumerate}
	\item SYSTEM (metaphysical) is nothing else than the provision of the various parts of an art or a science in a state where they support themselves all mutually, and where the last ones are explained by the first ones. Those which return reason of the others are called principles; and the system is more perfect if the principles are small in number: it is even wishing to reduce them only one. Because just as in a clock, there is a principal spring on which all the others depend, in all the systems there is also a first principle to which the various parts are subordinate to the other which make it up
	\item SYSTEM (philosophy) generally means assembly or sequence of principles and conclusions; or everything and the whole of a theory of which the various parts are dependent between them, follow and depend the ones on the others
	\item SYSTEM (astronomy) is the assumption of a certain arrangement of the various part which make the universe; according to this assumption the astronomers explain all the phenomena or appearances of the celestial bodies
\end{enumerate}

System Elements. Systems are composed of components, attributes, and relationships. These are described as follows:
\begin{enumerate}
	\item Components are the parts of a system
	\item Attributes are the properties (characteristics, configuration, qualities, powers, constraints, and state) of the components and of the system as a whole
	\item Relationships between pairs of linked components are the result of engineering the attributes of both components so that the pair operates together effectively in contributing to the system’s purpose(s)
\end{enumerate}

A system is a set of interrelated components functioning together toward some common objective(s) or purpose(s). The set of components meets the following requirements:
\begin{enumerate}
	\item The properties and behavior of each component of the set have an effect on the properties and behavior of the set as a whole
	\item The properties and behavior of each component of the set depend on the properties and behavior of at least one other component in the set
	\item Each possible subset of components meets the two requirements listed above; the components cannot be divided into independent subsets
\end{enumerate}

Scientific System Classifications. Science systems and the application of science systems thinking has been grouped into three categories based on the techniques used to tackle a system:
\begin{enumerate}
	\item Hard systems – involving simulations, often using computers and the techniques of operation research/management science. Useful for problems that can justifiably be quantified. However, it cannot easily consider unquantifiable variables (opinions, culture, politics, etc.), and may treat people as being passive, rather than having complex motivations
	\item Soft systems – For systems that cannot easily be quantified, especially those involving people holding multiple and conflicting frames of reference. Useful for understanding motivations, viewpoints, and interactions and addressing qualitative as well as quantitative dimensions of problem situations. Soft systems are a field that utilizes foundation methodological work developed by Peter Checkland, Brian Wilson and their colleagues at Lancaster University Morphological analysis is a complementary method for structuring and analyzing non-quantifiable problem complexes
	\item Evolutionary systems – Bela H. Banathy developed a methodology that is applicable to the design of complex social systems. This technique integrates critical systems inquiry with soft systems methodology. Evolutionary systems, like dynamic systems are understood as open, complex systems, but with the capacity to evolve over time. Banathy uniquely integrated the interdisciplinary perspectives of systems research (including chaos, complexity, cybernetics), cultural anthropology, evolutionary theory, and others
\end{enumerate}

The systems thinking approach incorporates several tenets:
\begin{enumerate}
	\item Interdependence of objects and their attributes – independent elements can never constitute a system.
	\item Holism – emergent properties not possible to detect by analysis should be possible to define by a holistic approach.
	\item Goal seeking – systemic interaction must result in some goal or final state.
	\item Inputs and Outputs – in a closed system inputs are determined once and constant; in an open system additional inputs are admitted from the environment.
	\item Transformation of inputs into outputs – this is the process by which the goals are obtained.
	\item Entropy – the amount of disorder or randomness present in any system.
	\item Regulation – a method of feedback is necessary for the system to operate predictably.
	\item Hierarchy – complex wholes are made up of smaller subsystems.
	\item Differentiation – specialized units perform specialized functions.
	\item Equifinality – alternative ways of attaining the same objectives (convergence).
	\item Multifinality – attaining alternative objectives from the same inputs (divergence).
\end{enumerate}

A treatise on systems thinking ought to address many issues including:
\begin{enumerate}
	\item Encapsulation of a system in space and/or time
	\item Active and passive systems (or structures)
	\item Transformation by an activity system of inputs into outputs
	\item Persistent and transient systems
	\item Evolution, the effects of time passing, the life histories of systems and their parts.
	\item Design and designers.
\end{enumerate}

\begin{enumerate}
	\item System classifications, similarities, and dissimilarities
	\item The fundamental distinction between natural and human-made systems
	\item The elements of a system and the position of the system in the hierarchy of systems
	\item The domain of systems science, with consideration of cybernetics, general systems theory, and systemology
	\item Technology as the progenitor for the creation of technical systems, recognizing its impact on the natural world
	\item The transition from the machine or industrial age to the Systems Age, with recognition of its impact upon people and society
	\item System complexity and scope and the demands these factors make on engineering in the Systems Age
	\item The range of contemporary definitions of systems engineering used within the profession.
\end{enumerate}

\section*{Resources for Further Exploration}
\begin{itemize}
	\item \href{http://innovbfa.viabloga.com/files/Herbert_Simon___theories_of_bounded_rationality___1972.pdf}{Theories of Bounded Rationality by Herbert A. Simon}
	\item \href{https://medium.com/disruptive-design/tools-for-systems-thinkers-the-6-fundamental-concepts-of-systems-thinking-379cdac3dc6a}{Tools of a Systems Thinker by Leyla Acaroglu}
	\item \href{http://www.afscet.asso.fr/resSystemica/Crete02/Dyer.pdf}{Beyond Systems Design as we know it? by Gordon Dyer}
\end{itemize}

\begin{enumerate}
	\item The boundary of a system is a decision made by an observer, or a group of observers.
	\item A system can be nested inside another system.
	\item A system can overlap with another system.
	\item A system is bounded in time.
	\item A system is bounded in space, though the parts are not necessarily co-located.
	\item A system receives input from, and sends output into, the wider environment.
	\item Science systems thinkers consider that:
	\item A system is a dynamic and complex whole, interacting with a structured functional unit. Energy, material, and information flow among the different elements that compose the system;
	\item A system is a community situated within an environment. Energy, material, and information flow from and to the surrounding environment via semi-permeable membranes or boundaries;
	\item Systems are often composed of entities seeking equilibrium but can exhibit oscillating, chaotic, or exponential behavior.
\end{enumerate}

\begin{enumerate}
	\item More attention to design can alleviate problems revealed during operations because operational outcomes are inherent and linked to the physics of design
	\item The on-line linking of individual designer’s decisions to evaluation at the system level and to stakeholder value is becoming more technically feasible. How?
\end{enumerate}

\begin{itemize}
	\item WHY “To Make the World Better for People” (Retitle source here)
	\item Human-made entities should be designed to satisfy human needs, wants, and objectives effectively, while minimizing system life-cycle cost as well as the intangible costs of ecological and societal impacts on the natural world
	\item HOW ``Adopt and Practice Systems Thinking and Engineering''
	\item A technologically based interdisciplinary process for bringing systems, products, structures, and services (human-made entities) into being
\end{itemize}

\begin{enumerate}
	\item Man is a conjugal animal, meaning an animal which is born to couple when an adult, thus building a household (oikos) and, in more successful cases, a clan or small village still run upon patriarchal lines.
	\item Man is a political animal, meaning an animal with an innate propensity to develop more complex communities the size of a city or town, with a division of labor and law-making. This type of community is different in a kind from a large family and requires the special use of human reason.
	\item Man is mimetic animal. Man loves to use his imagination (and not only to make laws and run town councils). He says ``we enjoy looking at accurate likenesses of things which are themselves painful to see, obscene beasts, for instance, and corpses.''  And the ``reason why we enjoy seeing likenesses is that, as we look, we learn and infer what each is, for instance, ‘that is so and so.’''
\end{enumerate}

\begin{itemize}
	\item Systemology and synthesis. The science of systems or their formation is called systemology. Problems and problem complexities faced by humankind do not organize themselves along disciplinary lines. New arrangements of scientific and professional efforts based on the common attributes and characteristics of needs and problems should contribute to the progress. More attention should be paid to human action and praxeology applications at the macro-level to help understand both the economic and non-economic dimensions of the world in which we live.
	\item The formation of interdisciplines began in the middle of the last century and that has brought about an evolutionary synthesis of knowledge. This has occurred not only within science, but between science and technology and between science and humanities. The forward progress of systemology in the study of large-scale complex systems requires a synthesis of science and the humanities in addition to a synthesis of science and technology.
	\item When synthesizing human-made systems, unintended effects can be minimized and the natural system can sometimes be improved by engineering the larger human-modified system instead of engineering only the human-made. If system evaluation is applied beyond the human-made, then the boundary of the target system (meant to include both natural and human-made systems) should be adopted as the boundary of the human-modified domain.
	\item Systems are as pervasive as the universe in which they exist. They are as grand as the universe itself or as infinitesimal as the atom. Systems appeared first in natural forms, but with the advent of human beings, a variety of human-made systems have come into existence. In recent decades, we have begun to understand structure and characteristics of natural and human-made systems in a scientific way.
\end{itemize}

\begin{enumerate}
	\item System classifications, similarities, and dissimilarities
	\item The fundamental distinction between natural and human-made systems
	\item The elements of a system and the position of the system in the hierarchy of systems
	\item The domain of systems science, with consideration of cybernetics, general systems theory, and systemology
	\item Technology as the progenitor for the creation of technical systems, recognizing its impact on the natural world
	\item The transition from the machine or industrial age to the Systems Age, with recognition of its impact upon people and society
	\item System complexity and scope and the demands these factors make on engineering in the Systems Age
	\item The range of contemporary definitions of systems engineering used within the profession.
\end{enumerate}

\begin{enumerate}
	\item Everyone has the right to judge for themselves whether `a reusable conclusion’ has been arrived at, \textit{but}
	\item This must be done after listening to all the evidence, including what, how, and where scientists obtained the evidence
\end{enumerate}

\begin{enumerate}
	\item The level of static structure or frameworks, ranging from the pattern of the atom to the anatomy of an animal to a map of the earth to the geography of the universe.
	\item The level of the simple dynamic system, or clockworks, adding predetermined, necessary motions, such as the pulley, the steam engine, and the solar system.
	\item The level of the thermostat or cybernetic system, adding the transmission and interpretation of information.
	\item The level of the cell, the open system where life begins to be evident, adding self-maintenance of structure in the midst of a through put of material.
	\item The level of the plant, adding a genetic-societal structure with differentiated and mutually dependent parts, “blueprinted” growth, and primitive information receptors.
	\item The level of the animal, adding mobility, teleological behavior, and self-awareness using specialized information receptors, a nervous system, and a brain with a knowledge structure.
	\item The level of the human, adding self-consciousness, the ability to produce, absorb, and interpret symbols; and understanding of time, relationship, and history.
	\item The level of social organization, adding roles, communication channels, the content and meaning of messages, value systems, transcription of image into historical record, art, music, poetry, and complex human emotion.
	\item The level of the transcendental system, adding the ultimates and absolutes and unknowables.
\end{enumerate}

\begin{enumerate}
	\item The principle of fallibility: perhaps I am wrong and perhaps you are right. But we could easily both be wrong.
	\item The principle of rational discussion: we want to try, as impersonally as possible, to weight up our reasons for and against a theory; a theory that is definite and criticizable.
	\item The principle of approximation to the truth: we can nearly always come closer to the truth in a discussion which avoids personal attacks. It can help us to achieve a better understanding; even in those cases where we do not reach an agreement.
\end{enumerate}

\begin{enumerate}
	\item Basic Assumptions include underlying philosophy, for example is there an outside intelligence operating, or is the system closed depending only on internal known laws. Are the laws of nature constant everywhere?  Were conditions in the past the same as they are now?  What initial conditions are assumed?  Science does not and cannot take place in a vacuum – an underlying world-view or philosophy is presupposed.
	\item The hypothesis is a proposed explanation for an observed phenomenon. The simpler the explanation that fits the facts, the better, known as Occam’s razor. Science assumes we live in a universe. That is, the laws of physics are the same everywhere and, furthermore, they do not vary erratically – nature is predictable, and the universe is rational.
	\item If the hypothesis does not explain the known facts or the new data, it is important to carefully examine the initial conditions to if one or more of them is incorrect or suspect. Wrong assumptions a long time ago that have not been challenged cause the weight of tradition to prevail – until there are overwhelming reasons for changing the prevailing scientific paradigm.
	‘Science is the only self-correcting human institution, but it is also a process that progresses only by showing itself to be wrong.’ – Alan Sandage
	\item Data inputs include measurements and observations. These are systematized and subjected to statistical scrutiny whenever possible.
	\item When a theory has been found that seems to the known facts, the theory is then extended into the unknown to make predictions. These predictions are then tested by seeking additional data, exceptions, or confirmations.
	\item A new scientific theory or model remains in vogue until new facts are found that contradict the model or whenever a better theory comes along.
\end{enumerate}

\begin{enumerate}
	\item Form a hypothesis – a tentative description of what’s been observed, and make predictions based on that hypothesis.
	\item Test the hypothesis and predictions in an experiment that can be reproduced.
	\item Analyze the data and draw conclusions; accept or reject the hypothesis or modify the hypothesis if necessary.
\end{enumerate}

Some key underpinnings to the scientific method are:
\begin{itemize}
	\item The hypothesis must be testable and falsifiable, according to North Carolina State University. Falsifiable means that there must be a possible negative answer to the hypothesis.
	\item Research must involve deductive reasoning and inductive reasoning. Deductive reasoning is the process of using true premises to reach a logical true conclusion while inductive reasoning takes the opposite approach.
	\item An experiment should include a dependent variable (which does not change) and an independent variable (which does change).
	\item An experiment should include an experimental group and a control group. The control group is what the experimental group is compared against.
\end{itemize}

\begin{enumerate}
	\item Man is a conjugal animal, meaning an animal which is born to couple when an adult, thus building a household (oikos) and, in more successful cases, a clan or small village still run upon patriarchal lines.[7]
	\item Man is a political animal, meaning an animal with an innate propensity to develop more complex communities the size of a city or town, with a division of labor and law-making. This type of community is different in kind from a large family and requires the special use of human reason.[8]
	\item Man is a mimetic animal. Man loves to use his imagination (and not only to make laws and run town councils). He says, ``we enjoy looking at accurate likenesses of things which are themselves painful to see, obscene beasts, for instance, and corpses.'' And the ``reason why we enjoy seeing likenesses is that, as we look, we learn and infer what each is, for instance, `that is so and so.'''[9]
\end{enumerate}

\begin{itemize}
	\item Establishing specific organizational goals and objectives
	\item Outsourcing requirements and the identification of suppliers
	\item Providing day-to-day program leadership and direction
	\item Implementing a program evaluation and feedback capability
	\item Conducting a risk analysis and management function
\end{itemize}

\begin{itemize}
	\item Systems engineering program planning needs
	\item The development of a systems engineering management plan (SEMP) - statement of work (SOW), systems engineering program tasks, work breakdown structure (WBS), scheduling of tasks, projecting costs for program tasks, interfacing with other planning activities
	\item Organization for systems engineering - developing the organizational structure, customer/producer/supplier relationships, customer organization and functions, producer organization and functions, supplier organization and functions, and staffing the organization (human resource requirements).
\end{itemize}

\begin{itemize}
	\item Value is in the mind of an individual. It is then by definition subjective and directly unobservable to others.
	\item Value is not a physical quantity. Thus, interpersonal comparisons of utility or value are inappropriate.
	\item All economic activity is a consequence of individual humans acting on their values.
\end{itemize}

\begin{enumerate}
	\item Engineering technical expertise (e.g., aeronautical engineers, civil engineers, electrical engineers, mechanical engineers, software engineers, reliability engineers, logistics engineers, and environmental engineers)
	\item Engineering technical support (e.g., technicians, component part specialists, computer programmers, model builders, drafting personnel, test technicians, and data analysts)
	\item Nontechnical support (e.g., marketing, purchasing and procurement, contracts, budgeting and accounting, industrial relations, manufacturing personnel, and logistics supply chain specialists). 
\end{enumerate}

\begin{enumerate}
	\item The systems movement as a system of ideas including
	\item The systems science community as some individuals, some organizations, and some publications; and
	\item Ten frames to guide thinking and discussion about changes in society, economics and technology in the 21st century (based on Ing (2011)); and
	\item John N. Warfield’s ``A Challenge for Systems Engineers: To Evolve towards Systems Science,'' published in INCOSE Insight (2007).
\end{enumerate}

\begin{itemize}
	\item Soft systems methodology (SSM): in the field of organizational studies is an approach to organizational process modelling, and it can be used both for general problem solving and in the management of change. It was developed in England by academics at the University of Lancaster Systems Department through a ten-year Action Research programme.
	\item System development methodology (SDM) in the field of IT development is a general term applied to a variety of structured, organized processes for developing information technology and embedded software systems.
\end{itemize}

\begin{enumerate}
	\item The essence of the relationships between SS and SE, the stable relation and potential synergies between them.
	\item The changes needed in SE and SS to address emerging challenges and opportunities in the systems arena.
\end{enumerate}

\begin{enumerate}
	\item System classifications, similarities, and dissimilarities
	\item The fundamental distinction between natural and human-made systems
	\item The elements of a system and the position of the system in the hierarchy of systems
	\item The domain of systems science, with consideration of cybernetics, general systems Theory, and systemology
	\item Technology as the progenitor for the creation of technical systems, recognizing its impact on the natural world
	\item The transition from the machine or industrial age to the Systems Age, with recognition of its impact upon people and society
	\item System complexity and scope and the demands these factors make on engineering in the Systems Age
	\item The range of contemporary definitions of systems engineering used within the profession
\end{enumerate}

\begin{enumerate}
	\item The economy in which we work will be strongly influenced by the global marketplace for engineering services, evidenced by the outsourcing of engineering jobs, a growing need for interdisciplinary and system-based approaches, demands for new paradigms of customization, and an increasingly international talent pool.
	\item The steady integration of technology in our public infrastructures and lives will call for more involvement by engineers in the setting of public policy and for participation in the civic arena.
	\item The external forces in society, the economy, and the professional environment will all challenge the stability of the engineering workforce and affect our ability to attract the most talented individuals to an engineering career.
\end{enumerate}

\begin{enumerate}
	\item There are instances where an academic administrative unit will be the home for more than one-degree program or major; e.g., Systems Engineering and Industrial Engineering. The department name may or may not subsume the names of all degree programs.
	\item There are instances where the institution will offer both a SE Centric (SEC) and a Domain Centric SE (DCSE) program; e.g., Systems Engineering and Manufacturing Systems Engineering. The DCSE program may be administered in an interdepartmental mode, whereas the SEC program will usually be administered within a department.
	\item In those instances where an institution offers a SEC program at the basic and advanced levels, all are usually administered within a department. This is also true for DCSE programs, except that the SE component may not exist at all degree levels.
\end{enumerate}

\begin{enumerate}
	\item Provide a focal point for the dissemination of systems engineering knowledge
	\item Promote collaboration in systems engineering education and research
	\item Assure the establishment of professional standards for integrity in the practice of systems engineering
	\item Improve the professional status of persons engaged in the practice of systems engineering
	\item Ecourage governmental and industrial support for research and educational programs that will improve the systems engineering process and its practice. 
\end{enumerate}

\begin{enumerate}
	\item The ISSS and INCOSE agree to a relationship for mutual benefit, to be reconfirmed every three years. The purpose of the relationship is to further the practices and knowledge jointly in systems sciences and systems engineering.
	\item INCOSE members are interested in gaining foundational knowledge in systems science concepts, methods and tools that may be applied in the practice of systems engineering.
	\item ISSS members are interested in seeing systems theories applied in practice, and further developing approaches on practical problems in systems engineering, based on the rich legacy of research in the systems sciences and on the feedback from systems engineering experience in applying the theories.
\end{enumerate}

\begin{itemize}
	\item “An interdisciplinary approach and means to enable the realization of successful systems.”
	\item “An interdisciplinary approach encompassing the entire technical effort to evolve into and verify an integrated and life-cycle balanced set of systems people, product, and process solutions that satisfy customer needs. Systems engineering encompasses (a) the technical efforts related to the development, manufacturing, verification, deployment, operations, support, disposal of, and user training for, system products and processes; (b) the definition and management of the system configuration; (c) the translation of the system definition into work breakdown structures; and (d) development of information for management decision making.”
	\item “The application of scientific and engineering efforts to (a) transform an operational need into a description of system performance parameters and a system configuration through the use of an iterative process of definition, synthesis, analysis, design, test, and evaluation; (b) integrate related technical parameters and ensure compatibility of all physical, functional, and program interfaces in a manner that optimizes the total system definition and design; and (c) integrate reliability, maintainability, safety, survivability, human engineering, and other such factors into the total engineering effort to meet cost, schedule, supportability, and technical performance objectives.”
	\item “An interdisciplinary collaborative approach to derive, evolve, and verify a life-cycle balanced system solution which satisfies customer expectations and meets public acceptability.”
	\item “An approach to translate operational needs and requirements into operationally suitable blocks of systems. The approach shall consist of a top-down, iterative process of requirements analysis, functional analysis and allocation, design synthesis and verification, and system analysis and control. Systems engineering shall permeate design, manufacturing, test and evaluation, and support of the product. Systems engineering principles shall influence the balance between performance, risk, cost, and schedule.”
\end{itemize}

\begin{enumerate}
	\item A top-down approach that views the system as a whole. Although engineering activities in the past have adequately covered the design of various system components (representing a bottom-up approach), the necessary overview and understanding of how these components effectively perform together is frequently overlooked.
	\item A life-cycle orientation that addresses all phases to include system design and development, production and/or construction, distribution, operation, maintenance and support, retirement, phase-out, and disposal. Emphasis in the past has been placed primarily on design and system acquisition activities, with little (if any) consideration given to their impact on production, operations, maintenance, support, and disposal. If one is to adequately identify risks associated with the up-front decision-making process, then such decisions must be based on life-cycle considerations.
	\item A better and more complete effort is required regarding the initial definition of system requirements, relating these requirements to specific design criteria, and the follow-on analysis effort to ensure the effectiveness of early decision making in the design process. The true system requirements need to be well defined and specified and the traceability of these requirements from the system level downward needs to be visible. In the past, the early “front-end” analysis as applied to many new systems has been minimal. The lack of defining an early “baseline” has resulted in greater individual design efforts downstream.
	\item An interdisciplinary or team approach throughout the system design and development process to ensure that all design objectives are addressed in an effective and efficient manner. This requires a complete understanding of many different design disciplines and their interrelationships, together with the methods, techniques, and tools that can be applied to facilitate implementation of the system engineering process.
\end{enumerate}

\begin{enumerate}
	\item Improving methods for determining the scope of needs to be met by the system. All aspects of a systems engineering project are profoundly affected by the scope of needs, so this determination should be accomplished first. Initial consideration of a broad set of needs often yields a consolidated solution that addresses multiple needs in a more cost-effective manner than a separate solution for each need
	\item Improving methods for defining product and system requirements as they relate to verified customer needs and external mandates. This should be done early in the design phase, along with a determination of performance, effectiveness, and specification of essential system characteristics
	\item Addressing the total system with all its elements from a life-cycle perspective, and from the product to its elements of support and renewal. This means defining the system in functional terms before identifying hardware, software, people, facilities, information, or combinations thereof
	\item Considering interactions in the overall system hierarchy. This includes relationships between pairs of system components, between higher and lower levels within the system hierarchy, and between sibling systems or subsystems
	\item Organizing and integrating the necessary engineering and related disciplines into a top-down system-engineering effort in a concurrent and timely manner
	\item Establishing a disciplined approach with appropriate review, evaluation, and feedback provisions to ensure orderly and efficient progress from the initial identification of needs through phase-out and disposal
\end{enumerate}

\begin{enumerate}
	\item Engineered systems have functional purposes in response to identified needs and have the ability to achieve stated operational objectives
	\item Engineered systems are brought into being and operate over a life cycle, beginning with identification of needs and ending with phase-out and disposal
	\item Engineered systems have design momentum that steadily increases throughout design, production, and deployment, and then decreases throughout phase-out, retirement, and disposal
	\item Engineered systems are composed of a harmonized <emphasis>combination of resources, such as facilities, equipment, materials, people, information, software, and money
	\item Engineered systems are composed of subsystems and related components that interact with each other to produce a desired system response or behavior
	\item Engineered systems are part of a hiearchy and are influenced by external factors from larger systems of which they are a part and from sibling systems from which they are composed
	\item Engineered systems are embedded into the natural world and interact with it in desirable as well as undesirable ways
\end{enumerate}

\begin{itemize}
	\item A set of things working together as parts of a mechanism or an interconnecting network
	\item A set of organs in the body with a common structure or function
	\item A group of related hardware units or software programs or both, especially when dedicated to a single application
	\item A major range of strata that corresponds to a period in time, subdivided into series
	\item A group of celestial objects connected by their mutual attractive forces, especially moving in orbits about a center
	The entities of a human constructed system can include people, hardware, software, facilities, policies, and documents; that is, all the things required to produce system-level results. These results include system-level qualities, properties, characteristics, functions, behavior and performance. The value added by the system as a whole, beyond that contributed independently by the parts, is primarily created by the relationships among the parts. In other words, the ``value-add'' of the system emerges in the synergy created when the parts come together.
\end{itemize}

\begin{itemize}
	\item Developing design requirements for subsystems and major system elements from system-level requirements
	\item Preparing <emphasis>development, product, process, and material specifications applicable to subsystems
	\item Accomplishing functional analysis and allocation to and below the subsystem level
	\item Establishing detailed design requirements and developing plans for their handoff to engineering domain specialists
	\item Identifying and utilizing appropriate engineering design tools and technologies
	\item Conducting trade-off studies to achieve design and operational effectiveness
	\item Conducting design reviews at predetermined points in time
\end{itemize}

\begin{itemize}
	\item System specification (Type A): includes the technical, performance, operational, and support characteristics for the system as an entity; the results of a feasibility analysis, operational requirements, and the maintenance and support concept; the appropriate TPM requirements at the system level; a functional description of the system; design requirements for the system; and an allocation of design requirements to the subsystem level (refer to 
	\item Development specification (Type B): includes the technical requirements (qualitative and quantitative) for any new item below the system level where research, design, and development are needed. This may cover an item of equipment, assembly, computer program, facility, critical item of support, data item, and so on. Each specification must include the performance, effectiveness, and support characteristics that are required in the evolving of design from the system level and down.
	\item Product specification (Type C): includes the technical requirements (qualitative and quantitative) for any item below the system level that is currently in inventory and can be procured “off the shelf.” This may cover any commercial off-the-shelf (COTS) equipment, software module, component, item of support, or equivalent
	\item Process specification (Type D): includes the technical requirements (qualitative and quantitative) associated with a process and/or a service performed on any element of a system or in the accomplishment of some functional requirement. This may include a manufacturing process (e.g., machining, bending, and welding), a logistics process (e.g., materials handling and transportation), an information handling process, and so on
	\item Material specification (Type E): includes the technical requirements that pertain to raw materials (e.g., metals, ore, and sand), liquids (e.g., paints and chemical compounds), semifabricated materials (e.g., electrical cable and piping), and so on.</para></listitem></orderedlist>
\end{itemize}

\begin{enumerate}
	\item One of the systems already exists, is operational, and the design is basically ``fixed,'' while the other system is new and in the early stages of design and development. Assuming that System ABC is operational, the design characteristics for the unit identified as being ``Common'' in the figure and required as a functioning element of ABC are essentially ``fixed.'' This, in turn, may have a significant impact on the overall effectiveness of System XYZ. In order to meet the overall requirements for XYZ, more stringent design input factors may have to be placed on the design of new Units C and D for that system. Or the common unit must be modified to be compatible with higher-level XYZ requirements, which (in turn) would likely impact the operational effectiveness of System ABC. Care must be taken to ensure that the overall requirements for both System ABC and System XYZ will be met. This can be accomplished by establishing a fixed requirement for the ``common unit'' and by modifying the design requirements for Assemblies 1, 2, and 3 within that unit to meet allocated requirements from both ABC and XYZ
	\item Both System ABC and System XYZ are new and are being developed concurrently. The allocation process illustrated in is initially accomplished for each of the systems, ensuring that the overall requirements at the system level are established. The design-to characteristics for the ``common unit'' are compared and a single set of requirements is identified for the unit through the accomplishment of trade-off analyses, and the requirements for each of the various system units are then modified as necessary across-the-board while ensuring that the top system-level requirements are maintained. This may constitute an iterative process of analysis, feedback, and so on
\end{enumerate}

\begin{enumerate}
	\item Reliability analysis: reliability models and block diagrams; failure mode, effects, and criticality analysis (FMECA); fault-tree analysis (FTA); reliability prediction (refer to 
	\item Maintainability analysis: maintainability models; reliability-centered maintenance (RCM); level-of-repair analysis (LORA); maintenance task analysis (MTA); total productive maintenance (TPM); maintainability prediction (refer to 
	\item Human factors analysis: operator task analysis (OTA); operational sequence diagrams (OSDs); safety/hazard analysis; personnel training requirements (refer to 
	\item Maintenance and logistic support: supply chain and supportability analysis leading to the definition of maintenance and support requirements—spares/repair parts and associated inventories, test and support equipment, transportation and handling equipment, maintenance personnel, facilities, technical data, information (refer to 
	\item Producibility, disposability, and sustainability analysis (refer to 
	\item Affordability analysis: life-cycle and total ownership cost (refer to 
\end{enumerate}

\begin{enumerate}
	\item Identification of the prime and alternate or secondary missions of the system. What is the system to accomplish? How will the system accomplish its objectives? The mission may be defined through one or a set of scenarios or operational profiles. It is important that the <emphasis>dynamics</emphasis> of system operating conditions be identified to the extent possible.
	\item Performance and physical parameters: Definition of the operating characteristics or functions of the system (e.g., size, weight, speed, range, accuracy, flow rate, capacity, transmit, receive, throughput, etc.). What are the critical system performance parameters? How are they related to the mission scenario(s)?
	\item Operational deployment or distribution: Identification of the quantity of equipment, software, personnel, facilities, and so on and the expected geographical location to include transportation and mobility requirements. How much equipment and associated software is to be distributed, and where is it to be located and for how long? When does the system become fully operational?
	\item Operational life cycle (horizon): Anticipated time that the system will be in operational use (expected period of sustainment). What is the total inventory profile throughout the system life cycle? Who will be operating the system and for what period of time?
	\item Utilization requirements: Anticipated usage of the system and its elements (e.g., hours of operation per day, percentage of total capacity, operational cycles per month, facility loading). How is the system to be used by the customer, operator, or operating authority in the field?
	\item Effectiveness factors: System requirements specified as figures-of-merit (FOMs) such as cost/system effectiveness, operational availability (Ao), readiness rate, dependability, logistic support effectiveness, mean time between maintenance (MTBM), failure rate (), maintenance downtime (MDT), facility utilization (in percent), operator skill levels and task accomplishment requirements, and personnel efficiency. Given that the system will perform, how effective or efficient is it? How are these factors related to the mission scenario(s)?
	\item Environmental factors: Definition of the environment in which the system is expected to operate (e.g., temperature, humidity, arctic or tropics, mountainous or flat terrain, airborne, ground, or shipboard). This should include a range of values as applicable and should cover all transportation, handling, and storage modes. How will the system be handled in transit? To what will the system be subjected during its operational use, and for how long? A complete environmental profile should be developed.
\end{enumerate}

\begin{enumerate}
	\item It provides a formalized check (audit) of the proposed system/subsystem design with respect to specification requirements. Major problem areas are discussed and corrective action is taken as required
	\item It provides a common baseline for all project personnel. The design engineer is provided the opportunity to explain and justify his or her design approach, and representatives from the various supporting organizations (e.g., maintainability, logistic support, and marketing) are provided the opportunity to learn of the design engineer’s problems. This serves as an excellent communication medium and creates a better understanding among design and support personnel
	\item It provides a means for solving interface problems and promotes the assurance that all system elements will be compatible, internally and externally
	\item It provides a formalized record of what design decisions were made and the reasons for making them. Analyses, predictions, and trade-off study reports are noted and are available to support design decisions. Compromises to performance, reliability, maintainability, human factors, cost, and/or logistic support are documented and included in the trade-off study reports
	\item It promotes a higher probability of mature design, as well as the incorporation of the latest techniques (where appropriate). Group review may identify new ideas, possibly resulting in simplified processes and ultimate cost savings
\end{enumerate}

\begin{enumerate}
	\item Identifying and translating a problem or deficiency into a definition of need for a system that will provide a preferred solution
	\item Accomplishing advanced system planning and architecting in response to the identified need
	\item Developing system operational requirements describing the functions that the system must perform to accomplish its intended purpose(s) or mission(s);
	\item Conducting exploratory studies leading to the definition of a technical approach for system design
	\item Proposing a maintenance concept for the sustaining support of the system throughout its planned life cycle
	\item Identifying and prioritizing technical performance measures (TPMs) and related criteria for design
	\item Accomplishing a system-level functional analysis and allocating requirements to various subsystems and components
	\item Performing systems analysis and producing trade-off studies
	\item Developing a system specification
	\item Conducting a conceptual design review
\end{enumerate}

\begin{enumerate}
	\item Identification of the items to be reviewed
	\item A selected date for the review
	\item The location or facility where the review is to be conducted
	\item An agenda for the review (including a definition of the basic objectives)
	\item A design review board representing the organizational elements and disciplines affected by the review. Basic design functions, reliability, maintainability, human factors, quality control, manufacturing, sustainability, and logistic support representation are included. Individual organizational responsibilities should be identified. Depending on the type of review, the customer and/or individual equipment suppliers may be included
	\item Equipment (hardware) and/or software requirements for the review. Engineering models, prototypes and/or mock-ups may be required to facilitate the review process
	\item Design data requirements for the review. This may include all applicable specifications, lists, drawings, predictions and analyses, logistic data, computer data, and special reports
	\item Funding requirements. Planning is necessary in identifying sources and a means for providing the funds for conducting the review
	\item Reporting requirements and the mechanism for accomplishing the necessary follow-up actions stemming from design review recommendations. Responsibilities and action-item time limits must be established
\end{enumerate}

\begin{itemize}
	\item Determining the requirements for system test, evaluation, and validation
	\item Describing the categories of system test and evaluation
	\item Planning for system test and evaluation
	\item Preparing for system test and evaluation
	\item Conducting the system test, collecting and analyzing the test data, comparing the results with the initially specified requirements, preparing a test report
	\item Incorporating system modifications as required
\end{itemize}

\begin{itemize}
	\item Developing design requirements for all lower-level components of the system
	\item Implementing the necessary technical activities to fulfill all design objectives
	\item Integrating system elements and activities
	\item Selecting and utilizing design tools and aids
	\item Preparing design data and documentation
	\item Developing engineering and prototype models
	\item Implementing a design review, evaluation, and feedback capability
	\item Incorporating design changes as appropriate
\end{itemize}

\begin{enumerate}
	\item Select a standard component that is commercially available and for which there are a number of viable suppliers; for example, a commercial off-the-shelf (COTS) item, or equivalent. The objective is, of course, to gain the advantage of competition (at reduced cost) and to provide the assurance that the appropriate maintenance and support will be readily available in the future and throughout the system life cycle when required, or
	\item Modify an existing commercially available off-the-shelf item by providing a mounting for the purposes of installation, adding an adapter cable for the purposes of compatibility, providing a software interface module, and so on. Care must be taken to ensure that the proposed modification is relatively simple and inexpensive and doesn’t result in the introduction of a lot of additional problems in the process, or
	\item Design and develop a new and unique component to meet a specific functional requirement. This approach will require that the component selected be properly integrated into the overall system design and development process in a timely and effective manner
\end{enumerate}

\begin{enumerate}
	\item Provide the design engineer with the opportunity of experimenting with different facility layouts, packaging schemes, panel displays, cable runs, and so on, before the preparation of final design data. A mock-up or engineering model of the proposed river crossing bridge can be developed to better visualize the overall structure, its location, interfaces with the communities on each side of the river, and so on
	\item Provide the reliability–maintainability–human factors engineer with the opportunity to accomplish a more effective review of a proposed design configuration for the incorporation of supportability characteristics. Problem areas readily become evident
	\item Provide the maintainability-human factors engineer with a tool for use in the accomplishment of predictions and detailed task analyses. It is often possible to simulate operator and maintenance tasks to acquire task sequence and time data
	\item Provide the design engineer with an excellent tool for conveying his or her final design approach during a formal design review.</para></listitem>
	\item Serve as an excellent marketing tool
	\item Can be employed to facilitate the training of system operator and maintenance personnel
	\item Can be utilized by production and industrial engineering personnel in developing fabrication and assembly processes and procedures and in the design of factory tooling and associated test fixtures
	\item Can serve as a tool at a later stage in the system life cycle for the verification of a modification kit design prior to the preparation of formal data and the development of kit hardware, software, and supporting materials
\end{enumerate}